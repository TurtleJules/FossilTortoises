\section{Discussion}

\subsection{Completeness of data set}

completeness of data set/benefits of additional sampling (SACs)
- how much of the "actual" data is represented by our data set?




\subsection{Population structure?}


\subsection{Time-scale analysis}

--> what does model support depend on? what does a relatively low model support mean?


% LIMITATIONS
%\todo{add at some point that phylogenetics were not considered because not enough data for fossil species --> or only in discussion? full-evidence analysis would be nice as in Slater et al., 2017 (baleen whales) --> cite Lapparent de Broin, 2006 + 2001: phylogenetic relationships between genera are not definitely established, }


- study would benefit from more sampling! (SAC, model supports paleoTS)

- include more data, not only literature but actually measure shells/other skeletal elements --> maybe gather further data to more reliably estimate body sizes

- include shapes/geometric morphometrics --> volume/body mass?

- paleoTS --> is designed to deal with incomplete data (FOSSIL)

- unbiased random walk on continents --> CL fluctuates more than on islands --> "giant" forms completetely disappeared in comparison to insular species

- what can influence distribution: climate, selective pressures, diet, intra-specific competition (Madagascar)

- I guess: climate affects body size reduction, but extinctions were human-driven
--> Aldabra/Galapagos (whaling industry)
--> mammal megafauna was hunted to extinction by humans