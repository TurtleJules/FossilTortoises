\section{Discussion}

\subsection{Data coverage}
The sample-based accumulation curve shows that the body size data set covers the generic level well, but not the species level.
Since there are less genera than species, it is to be expected that genera reach an asymptote earlier than species \citep{Gotelli2001}.
Although the accumulation curve for the entire data set does not completely plateau, considering the large area covered \citep{Thompson2002} and the high number of rare genera in the dataset, it can be considered well enough sampled for our purposes \citep{Gotelli2001}.
Conducting the rest of the analyses on the generic level is favourable here, because many fossils cannot be identified at the species level and species that are rare in the fossil record may not reflect their actual abundance, since only a small number of individuals are actually preserved \citep{Jass2014}.
According to \cite{Rhodin2015} 121 species of testudinidae have been recognized in the fossil record since the beginning of the Pleistocene. For 117 species, body size data could be obtained for this study, therefore it can be assumed, that the data set sufficiently resembles the actual fossil record of testudinidae on a global scale.
For smaller-scale analyses, for example, on individual continents, further body size data should be collected.
For insular species, the oldest fossils with body size data available stem from the Upper Miocene, which matches the colonization of islands by tortoises ? \citep{.}

% maybe add colonization of islands and oldest insular tortoises --> when did they occur? in my dataset earliest record is in upper miocene! find citation

%________________________________________________________

% DISCUSSION: as could be expected, since there are less genera than species \citep{Gotelli2001}. At a large geographic scale, it can be expected, that an asymptote is not reached \citep{Thompson2002}. Fig. \ref{fig:SACGen} corresponds to the shape one would expect, when there are many genera that are rare and only a few abundant ones. For the different continents, only Europe and Eurasia show some sort of "typical" SAC shape.
%%%____
% DISCUSSION ? 
%

%better anyway, since many fossils cannot be identified at the species level, many species are rare (because they have been misnamed as new species --> revision necessary!) and  rare fossils in fossil record may not reflect actual abundance \citep{Jass2014}



%completeness of data set/benefits of additional sampling (SACs)
%- how much of the "actual" data is represented by our data set?

%- how many species/genera are there, how many are present in my data set?
%lapparent de broin, 2001 + rhodin et al., 2015
%--> check turtles of the world checklist, how many extant genera + species?

%island records start at Tortonian --> does that fit?? 
%--> island colonization

%Rhodin et al., 2015: 121 testudinid taxa existed and about 69 (almost 60 \%) of those went extinct since Plio-Pleistocene boundary and present
%my data set: 117 testudinid species since beginnig of pleistocene (51 of those insular)

\subsection{Distribution of testudinid body size}

Distribution of testudinid body size is rather uniform across a spatial and temporal scale.
The body size distribution is right-skewed on a large scale, as well as for modern, fossil and continental species, which has been reported frequently for animal body size distributions \citep{Blackburn1994,Kozlowski2002}.
Only insular taxa show a body size distribution that is skewed to the left, with a higher frequency of larger-bodied species. This left-skewed distribution is largely driven by fossil insular species, as modern insular species are not skewed and show a rather flat, symmetrical distribution.
The bimodality of the overall body size distribution and the consistency across the continents is similar to what \cite{Lyons2008} report for Quaternary mammals. However, since tortoise body size is only sampled well enough for Europe and Eurasia, these results have to be considered with caution.
When looking at continental tortoises on a temporal scale, the second mode of large body sizes disappears for modern tortoises, which is due to the extinction of large continental taxa, similar to what has been observed in the mammalian megafauna during the Quaternary \citep{Lyons2008}. For insular species the bimodal body size distribution is constant over time, which could be expected, since large insular forms are still present, in spite of their diminished diversity and abundance compared to former times \citep{Rhodin2015}.
Whether or not these results can be considered as complying the island rule, depends on the biogeographic history of giant tortoises and whether they evolved to be large on islands or prior to island colonizations. Many authors agree that tortoises were already large when they colonized the islands \citep{Itescu2014,Cheke2016, Gerlach2006}, which would contradict the island rule, as has also been argued in a large-scale study on the intra-specific, inter-specific as well as clade level \citep{Itescu2014}. 
%Cheke et al 2016. co-evolution with plants they feed on --> are exactly thee same height
Modern tortoises are smaller-bodied than their fossil conspecifics, which coincides with earlier findings for animals in general \citep{Blackburn1994} and reptiles as a clade \citep{Smith2016}.




%Lyons 2008:
%Here we are using a macroecological approach to examine patterns in mammalian body size distributions at multiple spatial and temporal scales
%and as a result, mam- malian body size distributions have been well studied and found to be re- markably consistent across continents (Smith et al. 2004) and across re- cent time (Lyons et al. 2004).
%The overall shapes and ranges of the body size distribu- tions are similar on each of the four main continents, Eurasia, Africa, North America, and South America (fi




%Histograms
%general body size distribution similar to published results from other clades and constant for modern vs. fossil as well as on a spatial scale (bimodal and right-skewed for all continents, ALTHOUGH only Europe and Eurasia sampled well enough)


%Island forms larger than continental forms: overall as well as modern, fossil (but fossil continental also larger than modern continental)
%--> conforms to island rule?
%(discuss Jaffe et al., Itescu et al., Millien et al.??)

%As has been reported for many animals, the body size distribution of testudinidae is right-skewed. 
%The bimodal distribution looks similar to those of mammals reported by \cite{Lyons2008}. Values for skewness and kurtosis are also similar \todo{check again!!} (in this context it has to be noted, that \cite{Lyons2008} reported excess kurtosis, which is kurtosis - 3).
%Other than for the mammalian communities these authors investigated, the second mode of larger body size is still apparent in testudinidae, because giant tortoises did not go extinct completely, although their abundance has decreased considerably.

%- compare with Lyons and Smith, 2010 (mammals though, but)
%- is there a study about population structure in testudinidae?
%- kozlowski and gawelczyk, 2002 --> body size distributions usually right-skewed

%\citep{Blackburn1994, Lyons2008}

\subsection{Time-scale analysis}

The time scale analysis showed that overall there is no change in testudinid body size. However, if only considering continental taxa, unbiased random walk is the favoured model, which is a special case of directional evolution, where the probability of descendants being larger or smaller than their ancestors is the same.
This change is most likely due to the extinction of the giant continental forms, which is also apparent in the frequency curves. 
%There is a large drop in mean body size at the Gelasian/Lower Pleistocene border. --> Results!
For insular species, however, stasis is again the favoured evolutionary model, which is unsurprising since there still are giant forms on islands today. Also, model support is the highest for insular species, fitting only stasis.
On a continental level, for Europe stasis fits best for the complete data set as well as for continental and insular species.
Eurasia, however, reflects the overall trend, with body size of continental genera being best described by a unbiased random walk, although model support is weaker than for Europe.
Yet, all model supports for Eurasia (complete, continental and insular) are better than for the overall data set and the continetal taxa, which suggest that Eurasia somehow drives this trend.
It would be interesting to see which model fits best, if more Asian samples were being included.

In the literature, stasis is often encountered in large-scale analysis \citep{Smith2016, Hunt2006,Hunt2007,Hunt2015}. On a broad scale, stasis may be favoured when evolutionary changes are too small to be noted \citep{Hunt2015}.
Unbiased random walk has also been reported for several animal groups at the clade leven \citep{Smith2016}.
The unbiased random walk for continental testudinids is certainly influenced by the complete loss of giant forms in recent times. Additionaly, a shift towards smaller body size has been suggested for certain tortoise species \citep{Klein2000,Steele2005,Franz2005,Speth2002}.
 

%Suggest no trend in body size across time, at least not in general. But changes on continental level as well as for Eurasia, although Europe suggests stasis on all levels --> something up with Asia!
%But Asia is different from Europe, so what does mixing both do? Reliable? maybe not, since model support is rather weak!
%but somehow seems to drive continental trend...

%stasis prevalent in the fossil record. probably due to large scale!

%island stasis not unexpected, since the only giants still left occur solely on islands --> go into discussion why?

\vspace{2 cm}
human extirpation --> direct: hunting, indirect: habitat fragmentation (especially on continents maybe? since on islands range is smaller anyway), introduced predators (island forms had thinner shells! what about continental ones... probably too since they grew large and needed to reduce weight --> armoured animals)\\

- how does it fit with climate? \\

- how do points in time fit? \\

%- discuss Rhodin et al. 2015

--> Cione et al.: discuss mammal and tortoise extinction (in south america) --> the broken zig-zag\\


%--> what does model support depend on? what does a relatively low model support mean?


% LIMITATIONS
%\todo{add at some point that phylogenetics were not considered because not enough data for fossil species --> or only in discussion? full-evidence analysis would be nice as in Slater et al., 2017 (baleen whales) --> cite Lapparent de Broin, 2006 + 2001: phylogenetic relationships between genera are not definitely established, }


%- study would benefit from more sampling! (SAC, model supports paleoTS)

%- include more data, not only literature but actually measure shells/other skeletal elements --> maybe gather further data to more reliably estimate body sizes

- include shapes/geometric morphometrics --> volume/body mass?\\

- paleoTS --> is designed to deal with incomplete data (FOSSIL)\\

%- unbiased random walk on continents --> CL fluctuates more than on islands --> "giant" forms completetely disappeared in comparison to insular species

- what can influence distribution: climate, selective pressures, diet, intra-specific competition (Madagascar)\\

%- I guess: climate affects body size reduction, but extinctions were human-driven
reasons for extinction:
--> Aldabra/Galapagos (whaling industry)
--> mammal megafauna was hunted to extinction by humans




%____________________________________________________

Aldabra tortoise: no evidence of size increase, probably originate from giant continental tortoises (large tortoises are able to float: bouyancy and fasting endurance)

\vspace{2 cm}
%- Meiolania --> study that they were extirpated by humans?

%___________________________
\subsection{Conclusion}


This study would definitely benefit from further sampling, ideally by directly measuring fossil specimens from museum collections.

beyond literature research but actual measurements of museum collections, especially on the continents other than Europe. Possibly, a nearly complete (as complete as it can be, based on fossils) dataset could be achieved, if estimates based on single scutes/shell fragments could be established.

Giant tortoises seem to occupy only a small temperature range, since they are in danger of overheating \citep{.}, but also cannot cope with cold winters \citep{.}. However, obviously appropriate climates are still available, since giant tortoises are still present in some places, which means climate alone is very unlikely to have driven the extinction, at least not on tropical islands.
I would assume, that climate drove some decrease in body size (as exemplified by \textit{Gopherus agassizi} \citep{.} and \textit{Chersina angulata}), although in the last example the size decrease has been labeled human-induced (since large individuals are hunted more frequently, shifting the body size distribution to the smaller scale).


%- studies on individual species which seem to decrease in size: gopherus und chersina angulata --> stone age

%- smith et al., 2016: on higher taxa level, most groups seem to conform to the null model (changes occur on species level, display trends, but no tendencies)

\vspace{2 cm}
outlook\\

- rerun analyses with more data\\

- fit more and more dynamic models \citep{Hunt2015}\\


%Smith et al., 2016
%As drawn, there is no upward tendency, as increases and decreases in size are equally frequent. This is the pattern expected in the null case with no evolutionary forces acting at the scale of the lineage. Notice, however, that there is nevertheless a trend—a trend in the maximum. The distinction is that “tendency” refers to the pattern of change at the lower level, in this case the species level, whereas “trend” refers to a change in a summary statistic at the clade level, in this case an increase in the clademaximum.Other summary statistics of interest include the mean, median, and minimum. Thus, there is not necessarily a connection between tendency and trend: One can have a trend without a tendency. For body size evolution in particular, the most commonly observed and documentable kind of trend—a rise in the maximum—does not by itself tell us much about underlying tendencies. Maxima are expected to increase even if no tendencies, no evolutionary forces, are present. Thus, we cannot infer the existence of a selective advantage of large size merely from an increase in the maximum. The

would be interesting to investigate body size trends for individual genera or even species, for this further sampling is needed. use ancestor-descendant data! \\

terrestrial tortoises: about 90 \% of all species that occurred since beginning of Pleistocene are either extinct or endangered --> need to get a better understanding of tortoise extinction to protect them accordingly/to improve conservation efforts/to do effective conservation work
