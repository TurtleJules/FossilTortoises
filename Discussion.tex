\section{Discussion}

\subsection{Completeness of data set}

completeness of data set/benefits of additional sampling (SACs)
- how much of the "actual" data is represented by our data set?

- how many species/genera are there, how many are present in my data set?
lapparent de broin, 2001 + rhodin et al., 2015
--> check turtles of the world checklist, how many extant genera + species?


\subsection{Population structure?}

As has been reported for many animals, the body size distribution of testudinidae is right-skewed. 
The bimodal distribution looks similar to those of mammals reported by \cite{Lyons2008}. Values for skewness and kurtosis are also similar \todo{check again!!} (in this context it has to be noted, that \cite{Lyons2008} reported excess kurtosis, which is kurtosis - 3).
Other than for the mammalien communities these authors investigated, the second mode of larger body size is still apparent in testudinidae, because giant tortoises did not go extinct completely, although their abundance has decreased considerably.

- compare with Lyons and Smith, 2010 (mammals though, but)
- is there a study about population structure in testudinidae?
- kozlowski and gawelczyk, 2002 --> body size distributions usually right-skewed

\subsection{Time-scale analysis}

--> what does model support depend on? what does a relatively low model support mean?


% LIMITATIONS
%\todo{add at some point that phylogenetics were not considered because not enough data for fossil species --> or only in discussion? full-evidence analysis would be nice as in Slater et al., 2017 (baleen whales) --> cite Lapparent de Broin, 2006 + 2001: phylogenetic relationships between genera are not definitely established, }


- study would benefit from more sampling! (SAC, model supports paleoTS)

- include more data, not only literature but actually measure shells/other skeletal elements --> maybe gather further data to more reliably estimate body sizes

- include shapes/geometric morphometrics --> volume/body mass?

- paleoTS --> is designed to deal with incomplete data (FOSSIL)

- unbiased random walk on continents --> CL fluctuates more than on islands --> "giant" forms completetely disappeared in comparison to insular species

- what can influence distribution: climate, selective pressures, diet, intra-specific competition (Madagascar)

- I guess: climate affects body size reduction, but extinctions were human-driven
--> Aldabra/Galapagos (whaling industry)
--> mammal megafauna was hunted to extinction by humans




%____________________________________________________

Aldabra tortoise: no evidence of size increase, probably originate from giant continental tortoises (large tortoises are able to float: bouyancy and fasting endurance)

- Meiolania --> study that they were extirpated by humans?

%___________________________
\subsection{Conclusion}


This study would definitely benefit from further sampling, ideally by directly measuring fossil specimens in museum collections.

beyond literature research but actual measurements of museum collections, especially on the continents other than Europe. Possibly, a nearly complete (as complete as it can be, based on fossils) dataset could be achieved, if estimates based on single scutes/shell fragments could be established.

Giant tortoises seem to occupy only a small temperature range, since they are in danger of overheating (...), but also cannot cope with cold winters (...). However, obviously appropriate climates are still available, since giant tortoises are still present in some places, which means climate alone cannot have driven the extinction, at least not on tropical islands.
I would assume, that climate drove some decrease in body size (as exemplified by Gopherus agassizi (...) and Cherins angulata), although in the last example the size decrease has been labeled human-induced (since large individuals are hunted more frequently, shifting the body size distribution to the smaller scale).


- studies on individual species which seem to decrease in size: gopherus und chersina angulata --> stone age

- smith et al., 2016: on higher taxa level, most groups seem to conform to the null model (changes occur on species level, display trends, but no tendencies)

%Smith et al., 2016
%As drawn, there is no upward tendency, as increases and decreases in size are equally frequent. This is the pattern expected in the null case with no evolutionary forces acting at the scale of the lineage. Notice, however, that there is nevertheless a trend—a trend in the maximum. The distinction is that “tendency” refers to the pattern of change at the lower level, in this case the species level, whereas “trend” refers to a change in a summary statistic at the clade level, in this case an increase in the clademaximum.Other summary statistics of interest include the mean, median, and minimum. Thus, there is not necessarily a connection between tendency and trend: One can have a trend without a tendency. For body size evolution in particular, the most commonly observed and documentable kind of trend—a rise in the maximum—does not by itself tell us much about underlying tendencies. Maxima are expected to increase even if no tendencies, no evolutionary forces, are present. Thus, we cannot infer the existence of a selective advantage of large size merely from an increase in the maximum. The

would be interesting to investigate body size trends for individual genera or even species, for this further sampling is needed. use ancestor-descendant data! 
some pap
