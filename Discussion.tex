\section{Discussion}

For this study, a data set comprising 58 extant and 98 fossil species of \T has been assembled, including body size measurements from 616 specimens.
%Then, maybe a brief summary to start the discussion: recapture your results!
The analyses revealed that \T have a bimodally distributed body size, which is generally right-skewed except for insular species, where a negative skewness indicates a higher abundance of large-bodied testudinids.
It could be confirmed that modern testudinids are significantly smaller than fossil tortoises. Moreover, continental taxa are significantly smaller than insular forms, which has been observed for both modern and fossil testudinids.
Surprisingly, the time scale analysis identified stasis as the best-fitting model for the complete data set as well as most subgroups. Only for continental testudinids, globally and in Eurasia, an unbiased random walk was the favoured evolutionary model.


%First, how  body size data is distributed within \T in general. For the description and exploration of body size distributions, I follow the methods described in \cite{Lyons2008}.
%Secondly, since many large-bodied species have gone extinct, I would expect extant tortoises to be smaller than fossil ones. Thirdly, as extant tortoises reach larger body sizes on islands than on the mainland, if this pattern can be observed in fossil testudinids as well.
%Finally, the main goal of this study is to identify general body size trends in tortoises (Testudinidae) on a global scale across the last 20 million years. 



\subsection{Data coverage}
Sample-based accumulation curves show that genera have been sufficiently sampled, i.e. that further sampling will probably not result in recording more genera. However, this is not the case for species.
%The sample-based accumulation curve shows that the body size data set covers the generic level well, but not the species level.
Since there are fewer genera than species in a clade, it is to be expected that genera reach an asymptote earlier than species \citep{Gotelli2001}.
Although the accumulation curve for the entire data set does not completely converge to an asymptote, considering the large area covered \citep{Thompson2002} and the high number of rare genera in the dataset (which are to be expected in a fossil dataset), it can be considered well enough sampled for the present study \citep{Gotelli2001}.
%Conducting the remaining analyses on the generic level is advantageous here
The remaining analyses are conducted on the generic level because generic level identifications in the fossil record are more robust than species level identification and genera are better sampled in my data set \citep{Jass2014}.
According to \cite{Rhodin2015} 121 species of testudinidae have been recognized in the fossil record since the beginning of the Pleistocene. For 117 species from that time period, body size data could be obtained for this study. Therefore it can be assumed, that at least for the time period since the Pleistocene the data set sufficiently resembles the actual fossil record of testudinidae on a global scale. %The number of sampled genera does not vary largely across all time bins, hence I assume
However, to be able to investigate body size patterns on smaller spatial or temporal scales further body size data should be collected.
%For insular species, the oldest fossils with body size data available stem from the Upper Miocene, which matches the colonization of islands by tortoises ? \citep{.}

% maybe add colonization of islands and oldest insular tortoises --> when did they occur? in my dataset earliest record is in upper miocene! find citation

%________________________________________________________

% DISCUSSION: as could be expected, since there are less genera than species \citep{Gotelli2001}. At a large geographic scale, it can be expected, that an asymptote is not reached \citep{Thompson2002}. Fig. \ref{fig:SACGen} corresponds to the shape one would expect, when there are many genera that are rare and only a few abundant ones. For the different continents, only Europe and Eurasia show some sort of "typical" SAC shape.
%%%____
% DISCUSSION ? 
%

%better anyway, since many fossils cannot be identified at the species level, many species are rare (because they have been misnamed as new species --> revision necessary!) and  rare fossils in fossil record may not reflect actual abundance \citep{Jass2014}



%completeness of data set/benefits of additional sampling (SACs)
%- how much of the "actual" data is represented by our data set?

%- how many species/genera are there, how many are present in my data set?
%lapparent de broin, 2001 + rhodin et al., 2015
%--> check turtles of the world checklist, how many extant genera + species?

%island records start at Tortonian --> does that fit?? 
%--> island colonization

%Rhodin et al., 2015: 121 testudinid taxa existed and about 69 (almost 60 \%) of those went extinct since Plio-Pleistocene boundary and present
%my data set: 117 testudinid species since beginnig of pleistocene (51 of those insular)

\subsection{Distribution of testudinid body size}

The distribution of testudinid body size is rather homogenous across spatial and temporal scales.
Body size distribution is right-skewed on a large scale, as well as for modern, fossil and continental species. Similar patterns have been observed in tortoises \citep{Angielczyk2015,Jaffe2011a} and frequently throughout the animal kingdom \citep{Blackburn1994a,Kozlowski2002}.

Only insular testudinids show a body size distribution with a negative skewness, which indicates a higher frequency of larger-bodied species. This left-skewed distribution seems to be largely driven by fossil insular species, as modern insular species are not skewed and show a rather flat, symmetrical distribution. Interestingly, \cite{Angielczyk2015} found a strongly left-skewed body size distribution for insular species when investigating the entire clade of Testudines. This observation is probably driven by their well-sampled data set on \T, which comprises all extant and a few extinct species. 
The bimodality of overall body size distribution and the consistency across the continents in testudinids observed in the present study is similar to what has been reported for Quaternary mammals, which were observed to have a constant bimodal distribution across alls continents but Australia \citep{Lyons2008,Smith2004}.
However, since tortoise body size is only sampled well enough for Europe and Eurasia, these results have to be considered with caution.

When looking at continental tortoises on a temporal scale, the second peak representing large body sizes disappears for modern tortoises, which is probably due to the extinction of large continental taxa and coincides with other findings \citep{Itescu2014}. This may be partially due to the fact that almost half of the extant body size data for this study was obtained from the mentioned publication. Nonetheless, also with the higher number of data records for the present study, the results are consistent with \cite{Itescu2014}.
Moreover, the disappearance of larger taxa in recent times is similar to what has been observed in the mammalian megafauna during the Quaternary \citep{Lyons2008}.
For insular species, however, the results of the present study deviate from the observations made by \cite{Lyons2008}. While the bimodal body size distribution of testudinids is constant over time on islands, it is also left-skewed, whereas for mammals the opposite has been observed \citep{Lyons2008}. This contradiction is easily explained by the fact that many mammals show a decrease in body size in insular environments due to a lower predation risk as stated by the island rule \citep{Foster1964}. Insular tortoises, on the contrary, reach larger sizes than continental testudinids, which is consistent with other findings \cite{Itescu2014,Jaffe2011a,Angielczyk2015}.
Whether or not these results can be considered as complying the island rule depends on the biogeographic history of giant tortoises and whether they evolved to be large on islands or prior to island colonizations. Many authors agree that tortoises were already large when they colonized the islands, which would contradict the island rule (\citeauthor{Itescu2014}, \citeyear{Itescu2014}; \citeauthor{Cheke2016}, \citeyear{Cheke2016}; \citeauthor{Gerlach2006}, \citeyear{Gerlach2006}; \citeauthor{Caccone1999}, \citeyear{Caccone1999}; but see \citeauthor{Jaffe2011a}, \citeyear{Jaffe2011a}).
Colonization of oceanic islands via sea dispersal has been argued to be the only plausible explanation for the presence of tortoises on islands \citep{Cheke2016}. Since a larger size improves bouyancy and fasting endurance in tortoises, it seems logical that tortoises first evolved large size and then spread to islands \citep{Patterson1973,Gerlach2006,Cheke2016,Pritchard1996,Jaffe2011a}.
%Meilaniidae have been found on islands, and insular forms are smaller than continental species \
Modern tortoises are significantly smaller-bodied than their fossil conspecifics, which coincides with earlier findings for animals in general \citep{Blackburn1994a} and reptiles as a clade \citep{Smith2016}. This significant decrease in testudinid body size takes place during the Pleistocene and seems to coincide with the spread of humanity across the globe \citep{Rhodin2015}.
One thing to consider, however, is that smaller individuals are less likely to be preserved in the fossil record than larger ones which may introduce a taphonomic bias \citep{Lyons2008}. Further, specifically for body size, data on larger species or individuals is more likely to be available in the literature, because they are considered more interesting or significant finds. On the contrary, small species are sometimes only mentioned as being present at the site without providing measurements or photographs.





%Lyons 2008:
%Here we are using a macroecological approach to examine patterns in mammalian body size distributions at multiple spatial and temporal scales
%and as a result, mam- malian body size distributions have been well studied and found to be re- markably consistent across continents (Smith et al. 2004) and across re- cent time (Lyons et al. 2004).
%The overall shapes and ranges of the body size distribu- tions are similar on each of the four main continents, Eurasia, Africa, North America, and South America (fi




%Histograms
%general body size distribution similar to published results from other clades and constant for modern vs. fossil as well as on a spatial scale (bimodal and right-skewed for all continents, ALTHOUGH only Europe and Eurasia sampled well enough)


%Island forms larger than continental forms: overall as well as modern, fossil (but fossil continental also larger than modern continental)
%--> conforms to island rule?
%(discuss Jaffe et al., Itescu et al., Millien et al.??)

%As has been reported for many animals, the body size distribution of testudinidae is right-skewed. 
%The bimodal distribution looks similar to those of mammals reported by \cite{Lyons2008}. Values for skewness and kurtosis are also similar \todo{check again!!} (in this context it has to be noted, that \cite{Lyons2008} reported excess kurtosis, which is kurtosis - 3).
%Other than for the mammalian communities these authors investigated, the second mode of larger body size is still apparent in testudinidae, because giant tortoises did not go extinct completely, although their abundance has decreased considerably.

%- compare with Lyons and Smith, 2010 (mammals though, but)
%- is there a study about population structure in testudinidae?
%- kozlowski and gawelczyk, 2002 --> body size distributions usually right-skewed

%\citep{Blackburn1994, Lyons2008}

\subsection{Time-scale analysis}

The time scale analysis showed that stasis is the evolutionary model that best fits overall testudinid body size evolution, which contradicts my initial hypothesis.
Only for continental taxa on a global scale and Eurasian continental taxa, the favoured evolutionary model is an unbiased random walk.
Stasis, which describes fluctuations around a mean which in the end do not results in a net change, has often been observed in the fossil record, both on the species and on higher taxonomic levels \citep{Smith2016,Hunt2006,Pimiento2015, Hunt2015}. 
%Underlying reasons for stasis can be 
Unbiased random walk is a special case of directional evolution, where a trait is equally likely to increase or decrease over time \citep{Hunt2004} and has also been observed in the fossil record \citep{Hunt2006,Smith2016,Hunt2004}.
Stasis may have different underlying reasons, i. e. distinctive ecological conditions in a wide-spread taxon, resilience to environmental change and slow evolutionary rates \citep{Pimiento2015,Hunt2015,Sheldon1996,Benton2001}.
Further, stasis may be observed when evolutionary changes occur quickly within a limited amount of time whereas the remaining time there is no change in mean trait, therefore obscuring evolutionary changes \citep{Hunt2004}.
For all time scale analyses in this study, it looks as though body size increases first and then decreases again, however, most of these changes do not seem pronounced enough to deviate from stasis. Only for continental testudinids, globally and in Eurasia, the sharp drop at the end of the Pleistocene where mean body size as well as body size range decrease strongly, evolutionary change seems to be pronounced enough to be described as a random walk \citep{Hunt2004, Hunt2015}.
Since the comparison of body size across time bins has shown that body size does not differ significantly over the time period from the Miocene until the Pleistocene, stasis seems plausible for the overall data set.
This constancy over time could be a results of the bimodal body size distribution, which might suggets that there is more than one optimal body size for testudinids. \cite{Angielczyk2015} and \cite{Jaffe2011a} demonstrated for Testudines that optimal body size can differ based on habitat, which included mainland and island habitats. Since my results suggest a difference between these two habitats, namely that continental taxa are smaller than insular taxa, that may also be the case for testudinids. 
The unbiased random walk for continental testudinids seems to be influenced by the complete loss of giant forms in recent times, because this extinction leads to a change in mean body size as well as body size range.
Additionally, within-lineage changes, referred to as tendencies, towards smaller body size have been suggested for certain continental tortoise species \citep{Klein2000,Steele2005,Franz2005,Speth2002}.
Thus, the size-biased extinction of giant continental species coupled with tendencies towards body size on the species level, seems to result in an evolutionary trajectory best desribed by stasis.
Alternatively, on a global scale and on continents, body size ranges do not change considerably, although the decline during the Pleistocene still results in modern taxa being significantly smaller than fossil taxa.


%unteren satz besser ausführen!!
%That stasis is the favoured model for insular taxa as well as insular and continental taxa combined, although modern tortoises have significantly smaller body sizes compared to fossil tortoises, might be because the range of body sizes is still considerably large for insular taxa, whereas the range of body sizes in extant continental taxa is profoundly smaller than for fossil continental taxa.
%Therefore, the within-lineage tendencies towards smaller body size might not be visible as a trend at the clade level as long as body size range does not decrease significantly.
 



%The time scale analysis showed that overall there is no net change in testudinid body size through time. However, when only considering continental taxa, unbiased random walk is the favoured model. This is a special case of directional evolution, where the probability of descendants being larger or smaller than their ancestors is the same \citep{Hunt2010,Smith2016}.
%This change is most likely due to the extinction of the giant continental forms, which is also apparent in the frequency curves. 
%There is a large drop in mean body size at the Gelasian/Lower Pleistocene border. --> Results!
%For insular species, however, stasis is again the favoured evolutionary model, which seems plausible since there still are giant forms on islands today. Also, model support is the highest for insular species, fitting only stasis.
%On a continental level, for Europe stasis fits best for the complete data set as well as for continental and insular species.
%Eurasia, however, reflects the overall trend, with body size of continental genera being best described by a unbiased random walk, although model support is weaker than for Europe.
%Yet, all model supports for Eurasia (complete, continental and insular data sets) are better than for the overall data set and the continental taxa, which suggest that Eurasia somehow drives this trend.
%It would be interesting to see which model fits best, if more Asian samples were being included.

%In the literature, stasis is often encountered in large-scale analyses \citep{Smith2016, Hunt2006,Hunt2007,Hunt2015}. On a broad scale, stasis may be favoured when evolutionary changes are too small to be noted \citep{Hunt2015}.
%oberen Satz weiter ausführen
%Unbiased random walk has also been reported for several animal groups at the clade level \citep{Smith2016,Pimiento2015}. %mehr ausführen!!

%

%what does unbiased random walk tell me about possible reasons? selection pressure? or not because it is not driven in one direction?
\subsection{Causes for extinction}
There are numerous accounts of tortoise exploitation by humans from all over the world \citep{Blasco2008,Blasco2011,Blasco2016,Pritchard2013,Speth2002,Thompson2014,Steadman2017,Franz2001,Avery2004,Karl2012,Archer2014,Mudar2007,Munro2010,Peres2006,Sampson1998,Sampson2000}. Accordingly, extinction patterns in tortoises are associated with the spread of hominin and humans. For example, humans spread on the continents first and only later reached islands, which is why many large island species were overexploited during the Holocene leading to their extinction \citep{Rhodin2015}. 
In many archeological sites where tortoise remains are found, cut or burn marks are visible, indicating human consumption \citep{Archer2014,Biton2017,Blasco2008,Blasco2016,Munro2010}. But besides direct anthropogenic threats like hunting, human presence was also associated with issues like habitat fragmentation or, especially on islands, introduced predators or competitors which may have further accelerated tortoise extinction \citep{Sterli2015}. 
Tortoises are frequently found associated with dwarf forms of proboscideans on islands \citep{Hooijer1951,Vlachos2014}, which were found to have been overexploited by humans and only able to survive in regions inaccessible to humans \citep{Surovell2005}.
The significant decrease in tortoise body size that was observed during and especially at the end of the Pleistocene also coincides with the time of human spread and may be comparable to the exploitation of the mammalian megfauna, for which human influence has been suggested to be the main cause \citep{Barnosky2004, Sandom2014}.
Further, for the other clade of terrestrial tortoises, the Australian Meiolaniidae, evidence suggests that human exploitation lead to their extinction \citep{White2010}.
However, there are also records of mass mortalities of giant insular tortoises in Mauritius and Réunion associated with volcanic activity, before humans had even reached the islands \citep{Cheke2016}. 
Moreover, some extant species have been heavily exploited by humans in the past but did not go extinction, which suggest that human exploitation may not be the sole reason for tortoise extinction \citep{Stiner1999,Steele2005}. 
However, the aforementioned exploitations were associated with medium sized tortoises and did lead to a decrease in body size over time \citep{Stiner1999,Steele2005}.
Considering that large tortoises yield more nutritional value and are more likely to be collected by human \citep{Rhodin2015}, the extirpation of giant tortoises might be attributable to humans, possibly in conjunction with climate change \citep{Cione2003}.

%DISCUSS human exploitation (Froyd 2014) Holman1984 Moodie1979 Hansen2010
%human extirpation --> direct: hunting, indirect: habitat fragmentation (especially on continents maybe? since on islands range is smaller anyway), introduced predators (island forms had thinner shells! what about continental ones... probably too since they grew large and needed to reduce weight --> armoured animals)\\
%reasons for extinction:
%--> Aldabra/Galapagos (whaling industry)
%--> mammal megafauna was hunted to extinction by humans
%--> Meiolania were extirpated by humans!! \citep{White2010}
Giant tortoises seem to occupy only a small temperature range, as they are in danger of overheating \citep{Swingland1979,Schleich1981}, but also seem to be unable to cope with cold winters \citep{Hibbard1960}. In the timescale analyses a drop in body size at the Miocene/Pliocene border is visible, where climate started to cool down and lead to a change in vegetation cover shifting towards more open habitats \citep{Domingo2009}. When giant tortoises feel temperatures get too hot, they will move into the shade, but a change in vegetation may have robbed them of suitable hiding places \citep{Sturbaum1982,Hunter2013, Cheke2016, Schleich1981,Lujan2014}. This may have contributed to their extinction in some places, for example on the Aldabra Atoll many tortoises have been observed to die from heat exposure \citep{Swingland1979, Swingland1979a}.



%For this reason, it seems likely that human extirpation combined with changing climatic conditions has caused giant tortoises to go extinct on the continents.



%DISCUSS climate
%--> Cione et al.: discuss mammal and tortoise extinction (in south america) --> the broken zig-zag\\

%Cheke et al., 2016: mass mortality event on Mauritius and Reunion before first traces of humans on these islands

%DISCUSS timing --> human spread or climate changes?!

%__________________________________________
%DISCUSS implications for conservation
%_________________________________

%terrestrial tortoises: about 90 \% of all species that occurred since beginning of Pleistocene are either extinct or endangered --> need to get a better understanding of tortoise extinction to protect them accordingly/to improve conservation efforts/to do effective conservation work

\subsection{Conclusion}
The results of my study show that modern \T are significantly smaller than fossil testudinids and reach larger sizes on inslands compared to the mainlands.
Additionally, the evolutionary mode that best describes body size evolution in testudinids is stasis.
However, for continental taxa, which have descreased in body size over time due to extinction of all giant contintental testudinids, an unbiased random walk has been identified as evolutionary mode.
The significant size difference between modern and fossil tortoises on a global scale and within-lineage tendencies on the species level are not reflected as a trend in overall testudinid body size, which may be due to the wide-spread distribution of testudinids comprising different ecological conditions.
Discrepancies between continental and insular habitat may contribute to a stationary evolutionary trajectory.
The results suggest that the extinction of giant continental fossil tortoises seems to drive evolutionary patterns of continental and insular species. Loss of biodiversity is not reflected in these patterns, if the size range does not change significantly.
Possible reasons for the extinction of giant tortoises are complex and require further investigation. On the one hand, direct and indirect anthropogenic influence was massive and may have affected tortoises in the same way as the mammalian megafauna \citep{Barnosky2004,Sandom2014}. On the other hand, giant tortoises seem to have a narrow optimal temperature range and climatic fluctuations might have affected tortoise populations \citep{Cione2003}.
%while at the same time being ecosystem engineers
%include conservation efforts somehow!
%- reasons for extinction complex! on the one hand anthropogenic influence massive (and has been identified as main reason for mammalian megafauna), on the other hand, especially giant tortoises very specialized and sensitive to environmental change (thermoregulation, vegetation cover, burrowing ability, distribution range/inter-specific competition, diet, predation risks --> decreased armour/thin shells, especially in giant forms due to decreased weight, migration), but on the other hand might be ecosystem engineers?! (--> introduced species, Froyd 2014)
This study could certainly benefit from further sampling, ideally by directly measuring fossil specimens from museum collections.
With a larger data set, smaller-scale analyses could be conducted, for example for separate continents or individual lineages.
Phylogenetically informed analysis is necessary to ultimately conclude about the evolution of testudinid body size. However, phylogeny of this clade is not well resolved and currently under heavy revision \citep{LapparentdeBroin2006}. The obtained results are therefore less accurate, but still reliable because they are free of potentially false phylogenetic model constructions.

%It would also be interesting to include phylogenetic relationships, although many species and genera would have to be revisited and revised before a complete phylogeny can be created.


%--> Try model that allows for different optimal body sizes as in \cite{Jaffe2011a}

%In addition to carapace length, the different shell shape and thickness could be considered, since there are certain differences among tortoise species, especially in insular testudinids, where different
%- further research regarding extinction necessary: quantitative analysis on anthropogenic or human influence?
%- include shell shapes in addition to CL? and compare on geographic scale in relation to environment (diet, migration)



%outlook:
%- further sampling
%- smaller-scale analysis --> ancestor-descendant lineages, maybe even with species from their continental origin to their island colonization (do they get larger? trend visible?)
%- full-evidence analysis (similar to Slater et al., 2017) --> including phylogenetics (monophyletic group)




%--> what does model support depend on? what does a relatively low model support mean?


% LIMITATIONS
%\todo{add at some point that phylogenetics were not considered because not enough data for fossil species --> or only in discussion? full-evidence analysis would be nice as in Slater et al., 2017 (baleen whales) --> cite Lapparent de Broin, 2006 + 2001: phylogenetic relationships between genera are not definitely established, }


%- study would benefit from more sampling! (SAC, model supports paleoTS)

%- include more data, not only literature but actually measure shells/other skeletal elements --> maybe gather further data to more reliably estimate body sizes

%- include shapes/geometric morphometrics --> volume/body mass?\\

%- paleoTS --> is designed to deal with incomplete data (FOSSIL)\\

%- unbiased random walk on continents --> CL fluctuates more than on islands --> "giant" forms completetely disappeared in comparison to insular species

%- what can influence distribution: climate, selective pressures, diet, intra-specific competition (Madagascar)\\

%- I guess: climate affects body size reduction, but extinctions were human-driven





%____________________________________________________

%Aldabra tortoise: no evidence of size increase, probably originate from giant continental tortoises (large tortoises are able to float: bouyancy and fasting endurance)

%\vspace{2 cm}
%- Meiolania --> study that they were extirpated by humans?


%This study would definitely benefit from further sampling, ideally by directly measuring fossil specimens from museum collections.

%beyond literature research but actual measurements of museum collections, especially on the continents other than Europe. Possibly, a nearly complete (as complete as it can be, based on fossils) dataset could be achieved, if estimates based on single scutes/shell fragments could be established.



%I would assume, that climate drove some decrease in body size (as exemplified by \textit{Gopherus agassizi} \citep{.} and \textit{Chersina angulata}), although in the last example the size decrease has been labeled human-induced (since large individuals are hunted more frequently, shifting the body size distribution to the smaller scale).


%- studies on individual species which seem to decrease in size: gopherus und chersina angulata --> stone age

%- smith et al., 2016: on higher taxa level, most groups seem to conform to the null model (changes occur on species level, display trends, but no tendencies)

%\vspace{2 cm}
%outlook\\

%- rerun analyses with more data\\

%- fit more and more dynamic models \citep{Hunt2015}\\


%Smith et al., 2016
%As drawn, there is no upward tendency, as increases and decreases in size are equally frequent. This is the pattern expected in the null case with no evolutionary forces acting at the scale of the lineage. Notice, however, that there is nevertheless a trend—a trend in the maximum. The distinction is that “tendency” refers to the pattern of change at the lower level, in this case the species level, whereas “trend” refers to a change in a summary statistic at the clade level, in this case an increase in the clademaximum.Other summary statistics of interest include the mean, median, and minimum. Thus, there is not necessarily a connection between tendency and trend: One can have a trend without a tendency. For body size evolution in particular, the most commonly observed and documentable kind of trend—a rise in the maximum—does not by itself tell us much about underlying tendencies. Maxima are expected to increase even if no tendencies, no evolutionary forces, are present. Thus, we cannot infer the existence of a selective advantage of large size merely from an increase in the maximum. The

%would be interesting to investigate body size trends for individual genera or even species, for this further sampling is needed. use ancestor-descendant data! \\


