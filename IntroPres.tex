\section{Introduction}


\subsection{Body size trends in Neogene tortoises}

\begin{frame}{Terrestrial Tortoises}
\begin{picture}(300,250)
\put(0,90){
	%\missingfigure%\includegraphics[height=5 cm]{}
}
\put(150,210){
%\fbox{
\begin{minipage}[t]{0.5\linewidth}
\begin{itemize}[<+->]
\p 2 clades: Meiolaniidae \textbf{P}  and Testudinidae \textbf{P}
\p Meiolaniidae: extinct, used to be present in Australia + South America
\p Testudinidae: comprise all extant terrestrial tortoises (America, Europe, Africa, Asia)
\p Testudinidae: probably Asian origin, oldet fossils: North America + Europe
\p today's most famous examples: giant tortoises (Galapagos + Aldabra) --> \textbf{map?}
\p throughout Earth's history: many giant forms on the continents as well
\end{itemize}
\end{minipage}}
%}
\end{picture}
\end{frame}

\begin{frame}{Megafauna}
\begin{picture}(300,250)
\put(0,90){
	%\missingfigure%\includegraphics[height=5 cm]{}
}
\put(150,210){
	%\fbox{
	\begin{minipage}[t]{0.5\linewidth}
	\begin{itemize}[<+->]
	\p animals with body mass $>$ 44 kg = megafauna
	\p mammalian megafauna: popular, famous, well investgated
	\p megafauna extinctions! --> humans or climate change
	\p ...
	\end{itemize}
	\end{minipage}}
%}
\end{picture}
\end{frame}

\begin{frame}{Pleistocene Extinctions}
\begin{picture}(300,250)
\put(0,90){
	%\missingfigure%\includegraphics[height=5 cm]{}
}
\put(150,210){
	%\fbox{
	\begin{minipage}[t]{0.5\linewidth}
	\begin{itemize}[<+->]
	\p human influence
	\p climate change
	\p ...
	\p ...
	\end{itemize}
	\end{minipage}}

\end{picture}
\end{frame}

%overview at the end of the introduction!
\begin{frame}
\begin{enumerate}[<+->]
\p Body size distribution of Testudinidae?
\bigskip
\p Body size differences on spatial/temporal scale?
\bigskip
\p General body size trends?
%--> or rather continuous gene flow or all 3 species arose from one shared ancestor?
%\bigskip
%\p Reasons for extinction?
%\bigskip
%\textcolor{gray}{....}
\end{enumerate}
\end{frame}
%}