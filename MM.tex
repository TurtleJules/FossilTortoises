\section{Material \& Methods}

\subsection{Data collection}
I collected data on body size of fossil testudinids from the Upper Miocene until recent times. The body size data set includes 26 genera, comprising over 70 fossil and extant species. The majority of the data was obtained from the primary literature (\ref{TabS1}). To find relevant publications, I relied mostly on the references listed in FosFarBase (CITE), PDBD (cite), and "Fossil Turtle Checklist (CITE).
Furthermore, the FosFarBase provided fossil occurences of testudinids all over the world, including their exact localities and age (\ref{TabS2}), which were used to get an overview over the availability of body size data. 
For extant taxa, I measured dry material (n = 67) from the collection of the Museum für Naturkunde zu Berlin (MFN). In addition, body size data from the literature was included (\ref{TabS3}).

\subsection{Body size estimation}
Body size is reported as straight carapace length (SCL). Where SCL was not available from the primary literature, it was estimated either from plastron length (PL), femur length or humerus length (\ref{TabS1}). For carapace length estimations based on plastron length, the measurements from the MFN collection material was used to calculate the ratio between SCL and PL. Since the SC/PL ratio was similar for all species and genera, a single general ratio was calculated for all testudinids and hence used for the SCL estimations unless stated otherwise. For estimations based on humeri or femora, the ratios provided by Franz et al. (2001) were used.

TO DO: check Franz \& Quitmyer, 2005 again!! (CL regression)

\subsection{Analyses}
All subsequent analyses were performed with R (version ...), including the packages dplyr (cite) to prepare the data for the analysis (???) and ggplot2 (cite) to create figures. Species Accumulation Curves were created with the R package vegan (Cite) to see if sample size sufficed. This was repeated on genus level, since genera of fossil testudinids are relatively well resolved by now whereas determination on the species level is still somewhat obscure in many cases, as quite some species were based on scarce material. Since the data set relies on literature, occurrences increase with each added reference and reach a maximum, when no new species/genera are added.



- only used samples > 11.000 mya?!

\subsection{Body size trends over time}
To investigate trends in body size over time, the R package paleoTS (cite) was used. Data were split into time bins according to the stratigraphic timescale (\ref{Tab1}). 