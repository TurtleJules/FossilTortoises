\begin{landscape}

\tiny{
\begin{longtable}[]{@{}llllrllrlll@{}}
	\caption[Body size data set of fossil \T]{Body size data set of fossil testudinids. Contains information on locality, taxonomy (Genus and Species name), carapace length [mm], age and geographic distribution. Additionaly, it is stated whether carapace length was directly measured (m: exact measurements provided in reference, mf: measured from scaled figure, mo: estimated by original authors) or estimated (e: estimated from fragmentary carapace/plastron, ev: estimated from verbal description, ep: estimated from plastron length, ef: estimated from femur length, eh: estimated from humerus length, ec: estimated from claw phalanges). Further, it is stated on which continent the fossil record was recovered and whether it was continental (n: no) or insular (y: yes). Finally, the references from which the data were obtained are listed.}
	\phantomsection
	\label{tab:DataFossil}\tabularnewline
	\toprule
	& Locality & Genus & Taxon & CL & estimated & Stages & Age & Insular &
	Continent & Reference\tabularnewline
	\midrule
	\endfirsthead
	\multicolumn{10}{c}%
	{\tablename\ \thetable\ -- \textit{continued from previous page}}\tabularnewline
	\toprule
	& Locality & Genus & Taxon & CL & estimated & Stages & Age & Insular &
	Continent & Reference\tabularnewline
	\midrule
	\endhead
	1 & Laetoli, Tanzania & Aldabrachelys & ``Aldabrachelys'' laetoliensis &
	1000.00 & mo & Piacencian & 2.70300 & n & Africa & Meylan and
	Auffenberg, 1986\tabularnewline
	2 & Sal Island & Centrochelys & Centrochelys atlantica & 400.00 & mo &
	Lower Pleistocene & 1.30000 & y & Africa & Lopez-Jurado et al.,
	1998\tabularnewline
	3 & Ahl al Oughlam (near Casablanca) & Centrochelys & Centrochelys
	marocana & 2050.00 & mo & Gelasian & 2.50000 & n & Africa & Lapparent de
	Broin F.de, 2002a: A giant tortoise from the Late Pliocene of Lesvos
	Island (Greece) and its possible relationships. Annales Geologiques des
	Pays Helleniques, 1e Serie, t.XXXIX, fasc. A: 99-130\tabularnewline
	4 & Kanapoi & Geochelone & Geochelone crassa & 865.00 & mf & Zanclean &
	4.14500 & n & Africa & Harris et al., 2003\tabularnewline
	5 & Djebel Krechem & Geochelone & Geochelone sp. & 1446.00 & eh &
	Tortonian & 8.47600 & n & Africa & Geraads, 1989\tabularnewline
	6 & Pellatal Phosphate Member, E Quarry
	Langebaanweg & Geochelone & Geochelone stromeri & 350.00 & m & Zanclean
	& 4.46600 & n & Africa & Meylan and Auffenberg, 1986\tabularnewline
	7 & Pellatal Phosphate Member, E Quarry
	Langebaanweg & Geochelone & Geochelone stromeri & 425.00 & m & Zanclean
	& 4.46600 & n & Africa & Meylan and Auffenberg, 1986\tabularnewline
	8 & South Africa & Homopus & Homopus fenestratus & 90.00 & mo &
	Piacencian & 3.05650 & n & Africa & Rhodin et al., 2015\tabularnewline
	9 & Rusinga Island, Lake Victoria, Kenya & Impregnochelys &
	Impregnochelys pachytectis & 620.00 & m & Burdigalian/Aquitanian &
	19.50000 & n & Africa & Meylan and Auffenberg, 1986\tabularnewline
	10 & Arrisdrift & Mesocherus & Mesocherus orangeus & 160.00 & mo &
	Burdigalian/Aquitanian & 17.25000 & n & Africa & Lapparent de Broin
	F.de, 2003: Miocene Chelonians from southern Namibia. in: B. Senut \& M.
	Pickford coord., Faunas from the southern Namibia. Memoir Geol. Surv.
	Namibia 19: 67-102aus den Diamantfeldern Deutsch Südwestafrica. in: Die
	Diamantenwüste Südwest-Afrikas, Erich Kaiser (ed.) 2: 139-141, D.
	Reimer, Berlin\tabularnewline
	11 & Arrisdrift & Mesocherus & Mesocherus orangeus & 180.00 & mo &
	Burdigalian/Aquitanian & 17.25000 & n & Africa & Lapparent de Broin
	F.de, 2003: Miocene Chelonians from southern Namibia. in: B. Senut \& M.
	Pickford coord., Faunas from the southern Namibia. Memoir Geol. Surv.
	Namibia 19: 67-102aus den Diamantfeldern Deutsch Südwestafrica. in: Die
	Diamantenwüste Südwest-Afrikas, Erich Kaiser (ed.) 2: 139-141, D.
	Reimer, Berlin\tabularnewline
	12 & Arrisdrift & Mesocherus & Mesocherus orangeus & 180.00 & mo &
	Burdigalian/Aquitanian & 17.25000 & n & Africa & Lapparent de Broin
	F.de, 2003: Miocene Chelonians from southern Namibia. in: B. Senut \& M.
	Pickford coord., Faunas from the southern Namibia. Memoir Geol. Surv.
	Namibia 19: 67-102aus den Diamantfeldern Deutsch Südwestafrica. in: Die
	Diamantenwüste Südwest-Afrikas, Erich Kaiser (ed.) 2: 139-141, D.
	Reimer, Berlin\tabularnewline
	13 & Arrisdrift & Mesocherus & Mesocherus orangeus & 180.00 & mo &
	Burdigalian/Aquitanian & 17.25000 & n & Africa & Lapparent de Broin
	F.de, 2003: Miocene Chelonians from southern Namibia. in: B. Senut \& M.
	Pickford coord., Faunas from the southern Namibia. Memoir Geol. Surv.
	Namibia 19: 67-102aus den Diamantfeldern Deutsch Südwestafrica. in: Die
	Diamantenwüste Südwest-Afrikas, Erich Kaiser (ed.) 2: 139-141, D.
	Reimer, Berlin\tabularnewline
	14 & Arrisdrift & Mesocherus & Mesocherus orangeus & 180.00 & mo &
	Burdigalian/Aquitanian & 17.25000 & n & Africa & Lapparent de Broin
	F.de, 2003: Miocene Chelonians from southern Namibia. in: B. Senut \& M.
	Pickford coord., Faunas from the southern Namibia. Memoir Geol. Surv.
	Namibia 19: 67-102aus den Diamantfeldern Deutsch Südwestafrica. in: Die
	Diamantenwüste Südwest-Afrikas, Erich Kaiser (ed.) 2: 139-141, D.
	Reimer, Berlin\tabularnewline
	15 & Arrisdrift & Mesocherus & Mesocherus orangeus & 180.00 & mo &
	Burdigalian/Aquitanian & 17.25000 & n & Africa & Lapparent de Broin
	F.de, 2003: Miocene Chelonians from southern Namibia. in: B. Senut \& M.
	Pickford coord., Faunas from the southern Namibia. Memoir Geol. Surv.
	Namibia 19: 67-102aus den Diamantfeldern Deutsch Südwestafrica. in: Die
	Diamantenwüste Südwest-Afrikas, Erich Kaiser (ed.) 2: 139-141, D.
	Reimer, Berlin\tabularnewline
	16 & Arrisdrift & Mesocherus & Mesocherus orangeus & 200.00 & mo &
	Burdigalian/Aquitanian & 17.25000 & n & Africa & Lapparent de Broin
	F.de, 2003: Miocene Chelonians from southern Namibia. in: B. Senut \& M.
	Pickford coord., Faunas from the southern Namibia. Memoir Geol. Surv.
	Namibia 19: 67-102aus den Diamantfeldern Deutsch Südwestafrica. in: Die
	Diamantenwüste Südwest-Afrikas, Erich Kaiser (ed.) 2: 139-141, D.
	Reimer, Berlin\tabularnewline
	17 & Arrisdrift & Namibchersus & Namibchersus aff. namaquensis & 1100.00
	& mo & Burdigalian/Aquitanian & 17.25000 & n & Africa & Lapparent de
	Broin F.de, 2003: Miocene Chelonians from southern Namibia. in: B. Senut
	\& M. Pickford coord., Faunas from the southern Namibia. Memoir Geol.
	Surv. Namibia 19: 67-102aus den Diamantfeldern Deutsch Südwestafrica.
	in: Die Diamantenwüste Südwest-Afrikas, Erich Kaiser (ed.) 2: 139-141,
	D. Reimer, Berlin\tabularnewline
	18 & Arrisdrift & Namibchersus & Namibchersus aff. namaquensis & 440.00
	& mo & Burdigalian/Aquitanian & 17.25000 & n & Africa & Lapparent de
	Broin F.de, 2003: Miocene Chelonians from southern Namibia. in: B. Senut
	\& M. Pickford coord., Faunas from the southern Namibia. Memoir Geol.
	Surv. Namibia 19: 67-102aus den Diamantfeldern Deutsch Südwestafrica.
	in: Die Diamantenwüste Südwest-Afrikas, Erich Kaiser (ed.) 2: 139-141,
	D. Reimer, Berlin\tabularnewline
	19 & Arrisdrift & Namibchersus & Namibchersus aff. namaquensis & 550.00
	& mo & Burdigalian/Aquitanian & 17.25000 & n & Africa & Lapparent de
	Broin F.de, 2003: Miocene Chelonians from southern Namibia. in: B. Senut
	\& M. Pickford coord., Faunas from the southern Namibia. Memoir Geol.
	Surv. Namibia 19: 67-102aus den Diamantfeldern Deutsch Südwestafrica.
	in: Die Diamantenwüste Südwest-Afrikas, Erich Kaiser (ed.) 2: 139-141,
	D. Reimer, Berlin\tabularnewline
	20 & Auchas & Namibchersus & Namibchersus namaquensis & 254.00 & m &
	Burdigalian/Aquitanian & 18.00000 & n & Africa & Lapparent de Broin
	F.de, 2003: Miocene Chelonians from southern Namibia. in: B. Senut \& M.
	Pickford coord., Faunas from the southern Namibia. Memoir Geol. Surv.
	Namibia 19: 67-102aus den Diamantfeldern Deutsch Südwestafrica. in: Die
	Diamantenwüste Südwest-Afrikas, Erich Kaiser (ed.) 2: 139-141, D.
	Reimer, Berlin\tabularnewline
	21 & Elisabethfeld (= Elisabeth Bay) area, northern Sperrgebiet &
	Namibchersus & Namibchersus namaquensis & 264.00 & m &
	Burdigalian/Aquitanian & 19.50000 & n & Africa & Lapparent de Broin
	F.de, 2003: Miocene Chelonians from southern Namibia. in: B. Senut \& M.
	Pickford coord., Faunas from the southern Namibia. Memoir Geol. Surv.
	Namibia 19: 67-102aus den Diamantfeldern Deutsch Südwestafrica. in: Die
	Diamantenwüste Südwest-Afrikas, Erich Kaiser (ed.) 2: 139-141, D.
	Reimer, Berlin\tabularnewline
	22 & Elisabethfeld (= Elisabeth Bay) area, northern Sperrgebiet &
	Namibchersus & Namibchersus namaquensis & 300.00 & m &
	Burdigalian/Aquitanian & 19.50000 & n & Africa & Lapparent de Broin
	F.de, 2003: Miocene Chelonians from southern Namibia. in: B. Senut \& M.
	Pickford coord., Faunas from the southern Namibia. Memoir Geol. Surv.
	Namibia 19: 67-102aus den Diamantfeldern Deutsch Südwestafrica. in: Die
	Diamantenwüste Südwest-Afrikas, Erich Kaiser (ed.) 2: 139-141, D.
	Reimer, Berlin\tabularnewline
	23 & Auchas & Namibchersus & Namibchersus namaquensis & 470.00 & m &
	Burdigalian/Aquitanian & 18.00000 & n & Africa & Lapparent de Broin
	F.de, 2003: Miocene Chelonians from southern Namibia. in: B. Senut \& M.
	Pickford coord., Faunas from the southern Namibia. Memoir Geol. Surv.
	Namibia 19: 67-102aus den Diamantfeldern Deutsch Südwestafrica. in: Die
	Diamantenwüste Südwest-Afrikas, Erich Kaiser (ed.) 2: 139-141, D.
	Reimer, Berlin\tabularnewline
	24 & Auchas & Namibchersus & Namibchersus namaquensis & 470.00 & m &
	Burdigalian/Aquitanian & 18.00000 & n & Africa & Lapparent de Broin
	F.de, 2003: Miocene Chelonians from southern Namibia. in: B. Senut \& M.
	Pickford coord., Faunas from the southern Namibia. Memoir Geol. Surv.
	Namibia 19: 67-102aus den Diamantfeldern Deutsch Südwestafrica. in: Die
	Diamantenwüste Südwest-Afrikas, Erich Kaiser (ed.) 2: 139-141, D.
	Reimer, Berlin\tabularnewline
	25 & Auchas & Namibchersus & Namibchersus namaquensis & 815.00 & m &
	Burdigalian/Aquitanian & 18.00000 & n & Africa & Lapparent de Broin
	F.de, 2003: Miocene Chelonians from southern Namibia. in: B. Senut \& M.
	Pickford coord., Faunas from the southern Namibia. Memoir Geol. Surv.
	Namibia 19: 67-102aus den Diamantfeldern Deutsch Südwestafrica. in: Die
	Diamantenwüste Südwest-Afrikas, Erich Kaiser (ed.) 2: 139-141, D.
	Reimer, Berlin\tabularnewline
	26 & Drimolon, Sterkfontein, Krugersdorp District, Gauteng Province &
	Psammobates & Psammobates antiquorum & 107.80 & m & Lower Pleistocene &
	1.80000 & n & Africa & Broadley, 1997\tabularnewline
	27 & Ahl al Oughlam (near Casablanca) & Testudo & Testudo aff.
	kenitrensis & 142.00 & mf & Gelasian & 2.50000 & n & Africa & Gmira s.,
	2013\tabularnewline
	28 & Kénitra, Guilloux quarry, near Rabat & Testudo & Testudo
	kenitrensis & 132.00 & mo & Middle Pleistocene & 0.45350 & n & Africa &
	Gmira S., 1993: Une nouvelle espèce de tortue Testudininei (Testudo
	kenitrensis n. sp.) de l'Inter Amirien-Tensiftien de Kénitra (Maroc).
	Comptes rendus de l'Académie des Sciences de Paris -II 316:
	701-707\tabularnewline
	29 & Ahl al Oughlam (near Casablanca) & Testudo & Testudo oughlamensis &
	120.00 & mo & Gelasian & 2.50000 & n & Africa & Gmira s.,
	2013\tabularnewline
	30 & Ahl al Oughlam (near Casablanca) & Testudo & Testudo sp. & 184.00 &
	mf & Gelasian & 2.50000 & n & Africa & Gmira s., 2013\tabularnewline
	31 & Ahl al Oughlam (near Casablanca) & Testudo & Testudo sp. & 200.00 &
	mf & Gelasian & 2.50000 & n & Africa & Gmira s., 2013\tabularnewline
	32 & Tha Chang area, Nakhon Ratchasima
	Province & Aldabrachelys & Aldabrachelys ? sp. & 1500.00 & mo &
	Piacencian & 3.00000 & n & Asia & Claude J., Naksri W., Boonchai N.,
	Buffetaut E., Duangkrayom J., Laojumpon C., Jintasakul P., Lauprasert
	K., Martin J., Sutheethorn V., Tong H., 2011: Neogene reptiles of
	northeastern Thailand and their paleogeographical significance. Annales
	de Paléontologie (2011)
	\url{doi:10.1016/j.annpal.2011.08.002}\tabularnewline
	33 & Tha Chang area, Nakhon Ratchasima
	Province & Aldabrachelys & Aldabrachelys ? sp. & 1500.00 & mo &
	Piacencian & 3.00000 & n & Asia & Claude J., Naksri W., Boonchai N.,
	Buffetaut E., Duangkrayom J., Laojumpon C., Jintasakul P., Lauprasert
	K., Martin J., Sutheethorn V., Tong H., 2011: Neogene reptiles of
	northeastern Thailand and their paleogeographical significance. Annales
	de Paléontologie (2011)
	\url{doi:10.1016/j.annpal.2011.08.002}\tabularnewline
	34 & Tha Chang area, Chaloem Pra Kiat district, Nakhon Ratchasima
	Province & Aldabrachelys & Aldabrachelys ? sp. & 1500.00 & mo &
	Piacencian & 3.00000 & n & Asia & Claude J., Naksri W., Boonchai N.,
	Buffetaut E., Duangkrayom J., Laojumpon C., Jintasakul P., Lauprasert
	K., Martin J., Sutheethorn V., Tong H., 2011: Neogene reptiles of
	northeastern Thailand and their paleogeographical significance. Annales
	de Paléontologie (2011)
	\url{doi:10.1016/j.annpal.2011.08.002}\tabularnewline
	35 & Tha Chang area, Chaloem Pra Kiat district, Nakhon Ratchasima
	Province & Aldabrachelys & Aldabrachelys ? sp. & 1500.00 & mo &
	Piacencian & 3.00000 & n & Asia & Claude J., Naksri W., Boonchai N.,
	Buffetaut E., Duangkrayom J., Laojumpon C., Jintasakul P., Lauprasert
	K., Martin J., Sutheethorn V., Tong H., 2011: Neogene reptiles of
	northeastern Thailand and their paleogeographical significance. Annales
	de Paléontologie (2011)
	\url{doi:10.1016/j.annpal.2011.08.002}\tabularnewline
	36 & Altan-Teli main fossiliferous bed (Dzereg valley) & Ergilemys &
	Ergilemys oskarkuhni & 198.00 & m & Zanclean & 3.95000 & n & Asia &
	Mlynarski, 1968: Notes on tortoises (Testudinidae) from the tertiary of
	Mongolia. Paleontologica Polonica, 19: 85-99\tabularnewline
	37 & Altan-Teli main fossiliferous bed (Dzereg valley) & Ergilemys &
	Ergilemys oskarkuhni & 220.00 & m & Zanclean & 3.95000 & n & Asia &
	Mlynarski, 1968: Notes on tortoises (Testudinidae) from the tertiary of
	Mongolia. Paleontologica Polonica, 19: 85-99\tabularnewline
	38 & Guangxi & gen. & gen. indet. & 900.00 & mo & Lower Pleistocene &
	1.68450 & n & Asia & Rhodin et al., 2015\tabularnewline
	39 & Ghaba & Geochelone & Geochelone sp. & 800.00 & ev &
	Burdigalian/Aquitanian & 16.50000 & n & Asia & Roger,
	1994\tabularnewline
	40 & Lang Rongrien Rockshelter, Krabi, Thailand & Indotestudo &
	Indotestudo elongata & 270.00 & m & Upper Pleistocene & 0.03700 & n &
	Asia & Mudar and Anderson, 2007\tabularnewline
	41 & Punjab & Manouria & Manouria punjabiensis & 900.00 & mo & Gelasian
	& 2.19050 & n & Asia & Rhodin et al., 2015\tabularnewline
	42 & Sulawesi (Celebes), Indonesia & Megalochelys & Megalochelys atlas &
	1400.00 & mo & Gelasian & 2.00000 & y & Asia & Hooijer,
	1951\tabularnewline
	43 & Northwest of Naipli & Megalochelys & Megalochelys atlas & 1600.00 &
	mo & Piacencian & 3.09400 & n & Asia & Badam, 1981\tabularnewline
	44 & Northwest of Naipli & Megalochelys & Megalochelys atlas & 1600.00 &
	mo & Piacencian & 3.09400 & n & Asia & Badam, 1981\tabularnewline
	45 & Northwest of Naipli & Megalochelys & Megalochelys atlas & 1600.00 &
	mo & Piacencian & 3.09400 & n & Asia & Badam, 1981\tabularnewline
	46 & Northwest of Naipli & Megalochelys & Megalochelys atlas & 1600.00 &
	mo & Piacencian & 3.09400 & n & Asia & Badam, 1981\tabularnewline
	47 & Sulawesi (Celebes), Indonesia & Megalochelys & Megalochelys atlas &
	1650.00 & mo & Gelasian & 2.00000 & y & Asia & Setiyabudi,
	2009\tabularnewline
	48 & Pauk Twonship & Megalochelys & Megalochelys atlas & 1800.00 & m &
	Messinian & 5.42300 & n & Asia & Hirayama, R., Sonoda, T., Takai, M.,
	Htike, T., Thein, Z. M. M., \& Takahashi, A. (2015).~Megalochelys:
	gigantic tortoise from the Neogene of Myanmar~(No. e1185). PeerJ
	PrePrints.\tabularnewline
	49 & Siwalik & Megalochelys & Megalochelys atlas & 2000.00 & mo &
	Gelasian & 2.19050 & n & Asia & Setiyabudi, 2009\tabularnewline
	50 & Pauk Twonship & Megalochelys & Megalochelys atlas & 2100.00 & mo &
	Messinian & 5.42300 & n & Asia & Hirayama, R., Sonoda, T., Takai, M.,
	Htike, T., Thein, Z. M. M., \& Takahashi, A. (2015).~Megalochelys:
	gigantic tortoise from the Neogene of Myanmar~(No. e1185). PeerJ
	PrePrints.\tabularnewline
	51 & Tres Hermanas, Manila, Luzon & Megalochelys & Megalochelys sondaari
	& 1000.00 & ec & Lower Pleistocene & 1.35000 & y & Asia & Karl, H., \&
	Staesche, U. (2007). Fossile Riesen-Landschildkroten von den Philippinen
	und ihre palaogeographische Bedeutung. Geologisches Jahrbuch Reihe B,
	98, 171.\tabularnewline
	52 & Tres Hermanas, Manila, Luzon & Megalochelys & Megalochelys sondaari
	& 818.00 & ec & Lower Pleistocene & 1.35000 & y & Asia & Karl, H., \&
	Staesche, U. (2007). Fossile Riesen-Landschildkroten von den Philippinen
	und ihre palaogeographische Bedeutung. Geologisches Jahrbuch Reihe B,
	98, 171.\tabularnewline
	53 & Flores & Megalochelys & Megalochelys sp. & 1200.00 & ev & Lower
	Pleistocene & 0.90000 & y & Asia & Setiyabudi, 2009\tabularnewline
	54 & Bumiayu, Java Island & Megalochelys & Megalochelys sp. & 191.40 & m
	& Lower Pleistocene & 1.68450 & y & Asia & Setiyabudi,
	2009\tabularnewline
	55 & Java Island & Megalochelys & Megalochelys sp. & 2000.00 & m & Lower
	Pleistocene & 1.68450 & y & Asia & Hirayama, R., Sonoda, T., Takai, M.,
	Htike, T., Thein, Z. M. M., \& Takahashi, A. (2015).~Megalochelys:
	gigantic tortoise from the Neogene of Myanmar~(No. e1185). PeerJ
	PrePrints.\tabularnewline
	56 & Zhejiang & Testudo & Testudo changshanesis & 330.00 & mo & Lower
	Pleistocene & 1.68450 & n & Asia & Rhodin et al., 2015\tabularnewline
	57 & Khatlon & Testudo & Testudo ranovi & 200.00 & mo & Gelasian &
	2.19050 & n & Asia & Rhodin et al., 2015\tabularnewline
	58 & Gerogia (Caucasus) & Testudo & Testudo transcaucasia & 150.00 & mo
	& Gelasian & 2.19050 & n & Asia & Rhodin et al., 2015\tabularnewline
	59 & Sawmill Sink, Abaco & Chelonoidis & Chelonoidis alburyorum & 424.00
	& m & Piacencian & 3.20150 & y & America & Franz, R., \& Franz, S. E.
	(2009). A new fossil land tortoise in the genus Chelonoidis (Testudines:
	Testudinidae) from the Northern Bahamas: with an osteological assessment
	of other neotropical tortoises. University of Florida.\tabularnewline
	60 & Sawmill Sink, Abaco & Chelonoidis & Chelonoidis alburyorum & 428.00
	& m & Piacencian & 3.20150 & y & America & Franz, R., \& Franz, S. E.
	(2009). A new fossil land tortoise in the genus Chelonoidis (Testudines:
	Testudinidae) from the Northern Bahamas: with an osteological assessment
	of other neotropical tortoises. University of Florida.\tabularnewline
	61 & Sawmill Sink, Abaco & Chelonoidis & Chelonoidis alburyorum & 453.00
	& m & Piacencian & 3.20150 & y & America & Franz, R., \& Franz, S. E.
	(2009). A new fossil land tortoise in the genus Chelonoidis (Testudines:
	Testudinidae) from the Northern Bahamas: with an osteological assessment
	of other neotropical tortoises. University of Florida.\tabularnewline
	62 & Sawmill Sink, Abaco & Chelonoidis & Chelonoidis alburyorum & 466.00
	& m & Piacencian & 3.20150 & y & America & Franz, R., \& Franz, S. E.
	(2009). A new fossil land tortoise in the genus Chelonoidis (Testudines:
	Testudinidae) from the Northern Bahamas: with an osteological assessment
	of other neotropical tortoises. University of Florida.\tabularnewline
	63 & Santa Clara & Chelonoidis & Chelonoidis cubensis & 1139.00 & ef &
	Middle Pleistocene & 0.39350 & y & America & Williams, E. E. (1950).
	Testudo cubensis and the evolution of western hemisphere tortoises.
	Bulletin of the AMNH; v. 95, article 1.\tabularnewline
	64 & Cueva del Papayo, Pedernales & Chelonoidis & Chelonoidis marcanoi &
	530.00 & eh & Upper Pleistocene & 0.06900 & y & America & Turvey, S. T.,
	Almonte, J., Hansford, J., Scofield, R. P., Brocca, J. L., \& Chapman,
	S. D. (2017). A new species of extinct Late Quaternary giant tortoise
	from Hispaniola.~Zootaxa,~4277(1), 1-16.\tabularnewline
	65 & Cueva del Papayo, Pedernales & Chelonoidis & Chelonoidis marcanoi &
	614.00 & eh & Upper Pleistocene & 0.06900 & y & America & Turvey, S. T.,
	Almonte, J., Hansford, J., Scofield, R. P., Brocca, J. L., \& Chapman,
	S. D. (2017). A new species of extinct Late Quaternary giant tortoise
	from Hispaniola.~Zootaxa,~4277(1), 1-16.\tabularnewline
	66 & Cueva del Papayo, Pedernales & Chelonoidis & Chelonoidis marcanoi &
	767.00 & eh & Upper Pleistocene & 0.06900 & y & America & Turvey, S. T.,
	Almonte, J., Hansford, J., Scofield, R. P., Brocca, J. L., \& Chapman,
	S. D. (2017). A new species of extinct Late Quaternary giant tortoise
	from Hispaniola.~Zootaxa,~4277(1), 1-16.\tabularnewline
	67 & Cueva del Papayo, Pedernales & Chelonoidis & Chelonoidis marcanoi &
	778.00 & eh & Upper Pleistocene & 0.06900 & y & America & Turvey, S. T.,
	Almonte, J., Hansford, J., Scofield, R. P., Brocca, J. L., \& Chapman,
	S. D. (2017). A new species of extinct Late Quaternary giant tortoise
	from Hispaniola.~Zootaxa,~4277(1), 1-16.\tabularnewline
	68 & Mona Island & Chelonoidis & Chelonoidis monensis & 500.00 & m &
	Upper Pleistocene & 0.06450 & y & America & Williams,
	1952\tabularnewline
	69 & Sombrero Island & Chelonoidis & Chelonoidis sombrerensis & 990.00 &
	m & Upper Pleistocene & 0.06900 & y & America & Carlson, L. A. (2000).
	Aftermath of a feast: Human colonization of the southern Bahamian
	archipelago and its effects on the indigenous fauna.\tabularnewline
	70 & Navassa Island & Chelonoidis & Chelonoidis sp. & 400.00 & mo &
	Upper Pleistocene & 0.06900 & y & America & Auffenberg, W. (1967). Notes
	on West Indian tortoises. Herpetologica, 23(1), 34-44.\tabularnewline
	71 & San Pedro, Curaçao & Chelonoidis & Chelonoidis sp. & 600.00 & mo &
	Lower Pleistocene & 1.35700 & y & America & Hooijer, D. A. (1963).
	Geochelone from the Pleistocene of Curaçao, Netherlands Antilles.
	Copeia, 3, 579-580.\tabularnewline
	72 & Bayaguana, Los Haitises, San Cristobal & Chelonoidis & Chelonoidis
	sp. & 600.00 & mo & Upper Pleistocene & 0.06900 & y & America & Franz,
	R., \& Woods, C. A. (1983). A fossil tortoise from Hispaniola. Journal
	of Herpetology, 17(1), 79-81.\tabularnewline
	73 & San Pedro, Curaçao & Chelonoidis & Chelonoidis sp. & 750.00 & mo &
	Lower Pleistocene & 1.35700 & y & America & Hooijer, D. A. (1963).
	Geochelone from the Pleistocene of Curaçao, Netherlands Antilles.
	Copeia, 3, 579-580.\tabularnewline
	74 & San Pedro, Curaçao & Chelonoidis & Chelonoidis sp. & 800.00 & mo &
	Lower Pleistocene & 1.35700 & y & America & Hooijer, D. A. (1963).
	Geochelone from the Pleistocene of Curaçao, Netherlands Antilles.
	Copeia, 3, 579-580.\tabularnewline
	75 & Cedazo local fauna, Aguascalientes, Mexico & Geochelone &
	Geochelone sp. & 340.00 & mo & Lower Pleistocene & 1.05000 & n & America
	& Mooser, 1972\tabularnewline
	76 & Cedazo local fauna, Aguascalientes, Mexico & Gopherus & Gopherus
	berlandieri & 195.00 & m & Lower Pleistocene & 1.05000 & n & America &
	Mooser, 1972\tabularnewline
	77 & Cedazo local fauna, Aguascalientes, Mexico & Gopherus & Gopherus
	berlandieri & 256.30 & m & Lower Pleistocene & 1.05000 & n & America &
	Mooser, 1972\tabularnewline
	78 & Cedazo local fauna, Aguascalientes, Mexico & Gopherus & Gopherus
	flavomarginatus & 450.00 & m & Lower Pleistocene & 1.05000 & n & America
	& Mooser, 1972\tabularnewline
	79 & Smith's Parrish, No. 3Verdmont Valley Close & Hesperotestudo &
	Hesperotestudo bermudae & 270.00 & m & Middle Pleistocene & 0.31000 & y
	& America & Meylan and Sterrer, 2000\tabularnewline
	80 & Smith's Parrish, No. 3Verdmont Valley Close & Hesperotestudo &
	Hesperotestudo bermudae & 500.00 & m & Middle Pleistocene & 0.31000 & y
	& America & Olson and Meylan, 2009\tabularnewline
	81 & Río Tomayate, Apopa Municipality & Hesperotestudo & Hesperotestudo
	sp. & 1500.00 & mo & Lower Pleistocene & 0.96600 & n & America &
	Cisneros, 2005\tabularnewline
	82 & Belomechetskaya & Ergilemys & Ergilemys sp. & 1000.00 & m &
	Langhian & 14.00000 & n & Europe & FosFarBase\tabularnewline
	83 & Dmanisi & Testudo & Testudo graeca & 195.00 & mf & Lower
	Pleistocene & 1.77000 & n & Europe & Blain H.A., Agustí H.,
	Lordkipanidze D., Rook L., Delfino M., 2014: Paleoclimatic and
	paleoenvironmental context of the Early Pleistocene hominins from
	Dmanisi (Georgia, Lesser Caucasus) inferred from the herpetofaunal
	assemblage. Quaternary Science Reviews 105: 136-150\tabularnewline
	84 & Prottes & ``Hadrianus'' & ``Hadrianus sp.'' & 1000.00 & m &
	Tortonian & 8.30000 & n & Europe & Bachmayer F., M?ynarski M., 1985: Die
	Landschildkröten (Testudinidae) aus den Schotter-Ablagerungen (Pontien)
	von Prottes, Niederösterreich.. Annalen des Naturhistorischen Museums in
	Wien 87 A: 65-77\tabularnewline
	85 & Adeje, Tenerife & Centrochelys & Centrochelys burchardi & 500.00 &
	mo & Middle Pleistocene & 0.43500 & y & Europe & Ahl, E. (1925). Über
	eine ausgestorbene Riesenschildkröte der Insel Teneriffa. Zeitschrift
	der Deutschen Geologischen Gesellschaft, 575-580.\tabularnewline
	86 & Callao de Fañabé, Tenerife & Centrochelys & Centrochelys burchardi
	& 650.00 & mo & Middle Pleistocene & 0.43500 & y & Europe & Hutterer et
	al., 1998\tabularnewline
	87 & Adeje, Tenerife & Centrochelys & Centrochelys burchardi & 800.00 &
	m & Middle Pleistocene & 0.43500 & y & Europe & Ahl, E. (1925). Über
	eine ausgestorbene Riesenschildkröte der Insel Teneriffa. Zeitschrift
	der Deutschen Geologischen Gesellschaft, 575-580.\tabularnewline
	88 & Callao de Fañabé, Tenerife & Centrochelys & Centrochelys burchardi
	& 940.00 & mo & Middle Pleistocene & 0.43500 & y & Europe & Hutterer et
	al., 1998\tabularnewline
	89 & Corrida, Malta & Centrochelys & Centrochelys robusta & 1100.00 & mo
	& Zanclean & 4.91700 & y & Europe & Adams A.L., 1877: On gigantic
	land-tortoises and a small freshwater species from the ossiferous
	caverns of Malta, together with a list of their fossil fauna; and a note
	on Chelonian remains from the rock-cavities of Gibraltar. Quarterly
	Journal of the Geological Society of London 33: 177-191\tabularnewline
	90 & Ghar Dalam & Centrochelys & Centrochelys robusta & 1200.00 & ev &
	Lower Pleistocene & 1.30000 & y & Europe & Hunt and Schembri,
	1999\tabularnewline
	91 & Ghar Dalam & Centrochelys & Centrochelys robusta & 600.00 & ev &
	Lower Pleistocene & 1.30000 & y & Europe & Hunt and Schembri,
	1999\tabularnewline
	92 & Mnaidra Gap, Malta & Centrochelys & Centrochelys robusta & 790.00 &
	ef & Zanclean & 4.91700 & y & Europe & Adams A.L., 1877: On gigantic
	land-tortoises and a small freshwater species from the ossiferous
	caverns of Malta, together with a list of their fossil fauna; and a note
	on Chelonian remains from the rock-cavities of Gibraltar. Quarterly
	Journal of the Geological Society of London 33: 177-191\tabularnewline
	93 & Corrida, Malta & Centrochelys & Centrochelys robusta & 850.00 & mo
	& Zanclean & 4.91700 & y & Europe & Adams A.L., 1877: On gigantic
	land-tortoises and a small freshwater species from the ossiferous
	caverns of Malta, together with a list of their fossil fauna; and a note
	on Chelonian remains from the rock-cavities of Gibraltar. Quarterly
	Journal of the Geological Society of London 33: 177-191\tabularnewline
	94 & Zebbug and Gahr Dalam Cave deposits & Centrochelys & Centrochelys
	robusta & 850.00 & mo & Upper Pleistocene & 0.06600 & y & Europe &
	Lapparent de Broin F.de, 2002a: A giant tortoise from the Late Pliocene
	of Lesvos Island (Greece) and its possible relationships. Annales
	Geologiques des Pays Helleniques, 1e Serie, t.XXXIX, fasc. A:
	99-130\tabularnewline
	95 & Ghar Dalam & Centrochelys & Centrochelys robusta & 850.00 & ev &
	Lower Pleistocene & 1.30000 & y & Europe & Hunt and Schembri,
	1999\tabularnewline
	96 & Barranco de las Ballenas, Las Palmas, Gran Canaria & Centrochelys &
	Centrochelys vulcanica & 610.00 & mo & Piacencian & 3.09400 & y & Europe
	& Hutterer et al., 1998\tabularnewline
	97 & Pujo d'es Fum, Formentera, Balearic Islands & Cheirogaster &
	Cheirogaster cf.~gymnesica & 789.00 & mo & Lower Pleistocene & 1.80000 &
	y & Europe & Filella-Subira et al., 1999\tabularnewline
	98 & Punta Nati near Ciutadella, Minorca & Cheirogaster & Cheirogaster
	gymnesica & 739.00 & ef & Zanclean & 4.45000 & y & Europe & Mercadal B.,
	Pretus Real L., 1980: Nuevo yacimiento de Testudo gymnesicus Bate, 1914
	en la isla de Menorca. Boletín de la Sociedad de Historia Natural de
	Baleares 24: 15-21\tabularnewline
	99 & Hostalets de Piérola, Barcelone province, Cataluña & Cheirogaster & Cheirogaster richardi & 1155.00 & mo & Tortonian
	& 10.40000 & n & Europe & Pérez-García A., Vlachos E., 2014: New generic
	proposal for the European Neogene large testudinids (Cryptodira) and the
	?rst phylogenetic hypothesis for the medium and large representatives of
	the European Cenozoic record. Zoological Journal of the Linnean Society
	172: 653-719\tabularnewline
	100 & La Ciesma 1, Aragón & Cheirogaster & Cheirogaster sp. & 1000.00 &
	mo & Serravallian & 12.20000 & n & Europe & Murelaga X., Azanza B.,
	Astibia H., 2006: Restos de quelonios del Mioceno medio del área de
	Tarazona de Aragón (Cuenca del Ebro, Aragón, España). Estudios
	Geológicos 62(1): 205-212\tabularnewline
	101 & El Lugarejo (Arévalo), Ávilla, Castilla & Cheirogaster &
	Cheirogaster sp. & 1170.00 & m & Tortonian & 10.25000 & n & Europe &
	Jiménez Fuentes, E., Acosta, P., \& Fincias San Martín, B. (1986). Un
	nuevo ejemplar de tortuga gigante del Mioceno de Arévalo (Ávila). Studia
	Geologica. Salmanticensia, 23, 11.\tabularnewline
	102 & Chañe, Segovia & Cheirogaster & Cheirogaster sp. & 1500.00 & e &
	Serravallian & 13.80000 & n & Europe & Jiménez Fuentes E., 2000:
	Tortugas gigantes fósiles de la provincia de Segovia (Castilla y León,
	España). Nueva localidad: Chañe. Studia Geologica Salamanticensia 36:
	109-115\tabularnewline
	103 & Crevillente 2 & Cheirogaster & Cheirogaster sp. & 1540.00 & ef &
	Tortonian & 8.30000 & n & Europe & Jiménez E., Montoya P., 2002:
	Quelonios del Mioceno superior de Crevillente 2 (Alicante, Espana).
	Studia Geologica Salamanticensia, 38: 87-103 or Pickford M., Morales J.,
	1994: Biostratigraphy and palaeobiogeography of East Africa and the
	Iberian peninsula. Palaeogeography, Palaeoclimatology, Palaeoecology
	112: 297-322\tabularnewline
	104 & Rock-Cavities, Gibraltar Peninsula & Cheirogaster & Cheirogaster
	sp. & 925.00 & ef & Lower Pleistocene & 0.96500 & y & Europe & Adams
	A.L., 1877: On gigantic land-tortoises and a small freshwater species
	from the ossiferous caverns of Malta, together with a list of their
	fossil fauna; and a note on Chelonian remains from the rock-cavities of
	Gibraltar. Quarterly Journal of the Geological Society of London 33:
	177-191\tabularnewline
	105 & Soave, Zoppega 2 cave, Verona & Eurotestudo & Eurotestudo aff.
	hermanni & 179.30 & mf & Middle Pleistocene & 0.74000 & n & Europe &
	Lapparent de Broin F. de, Bour R., Perälä J., 2006: Morphological
	definition of Eurotestudo (Testudinidae, Chelonii): First part. Annales
	de Paléontologie 92(3): 255-304\tabularnewline
	106 & Soave, Zoppega 2 cave, Verona & Eurotestudo & Eurotestudo aff.
	hermanni & 194.70 & mf & Middle Pleistocene & 0.74000 & n & Europe &
	Lapparent de Broin F. de, Bour R., Perälä J., 2006: Morphological
	definition of Eurotestudo (Testudinidae, Chelonii): First part. Annales
	de Paléontologie 92(3): 255-304\tabularnewline
	107 & Monte Tuttavista VII mustelide, Sardinia & Eurotestudo &
	Eurotestudo cf.~hermanni & 150.00 & mo & Gelasian & 2.00000 & y & Europe
	& Abbazzi L., Angelone C., Arca M., Barisone G., Bedetti C., Delfino M.,
	Kotsakis T., Marcolini F., Palombo M.R., Pavia M., Piras P., Rook L.,
	Torre D., Tuveri C., Valli A.M.F., Wilkens B., 2004: Plio-Pleistocene
	fossil vertebrates of Monte Tuttavista (Orosei, Eastern Sardinia,
	Italy), an overview. Rivista Italiana di Paleontologia e Stratigraphia
	110(3): 681-706\tabularnewline
	108 & Le Ville, Upper Valdarno & Eurotestudo & Eurotestudo globosa &
	263.00 & m & Lower Pleistocene & 1.80000 & n & Europe & Portis A., 1890:
	I Rettili pliocenici del Valdarno superiore e di alcune altre località
	plioceniche di Toscana. Le Monnier Suc., Firenze (1--32) or Lapparent de
	Broin F. de, Bour R., Perälä J., 2006: Morphological definition of
	Eurotestudo (Testudinidae, Chelonii): First part. Annales de
	Paléontologie 92(3): 255-304 or Lapparent de Broin F. de, Bour R.,
	Perälä J., 2006a: Morphological definition of Eurotestudo (Testudinidae,
	Chelonii): second part. Annales de Paléontologie 92(4):
	325-357\tabularnewline
	109 & Cueva de la Victoria-1 (CV-1), Carthagène, Murcia & Eurotestudo &
	Eurotestudo hermanni & 126.00 & mf & Lower Pleistocene & 1.15000 & n &
	Europe & Pérez-García, 2012\tabularnewline
	110 & Saint-Estève-Janson, l'Escale Cave (Bouches du Rhône) &
	Eurotestudo & Eurotestudo hermanni & 170.50 & mf & Middle Pleistocene &
	0.60000 & n & Europe & Lapparent de Broin F. de, Bour R., Perälä J.,
	2006: Morphological definition of Eurotestudo (Testudinidae, Chelonii):
	First part. Annales de Paléontologie 92(3): 255-304\tabularnewline
	111 & Cova del Rinoceront, eastern Garraf Massif, Castelldelfs & Eurotestudo & Eurotestudo hermanni & 187.00 & mf & Upper
	Pleistocene & 0.11050 & n & Europe & Daura J., Sanz M., Julià R.,
	García-Fernández D., Fornós J.J., Vaquero M., Allué E., López-García
	J.M., Blain H.A., Ortiz J.E., Torres T., Albert J.M., et al., 2015: Cova
	del Rinoceront (Castelldefels, Barcelona): a terrestrial record for the
	Last Interglacial period (MIS 5) in the Mediterranean coast of the
	Iberian Peninsula. Quaternay Science Reviews 114: 203-227\tabularnewline
	112 & Saint-Estève-Janson, l'Escale Cave (Bouches du Rhône) &
	Eurotestudo & Eurotestudo hermanni & 237.60 & mf & Middle Pleistocene &
	0.60000 & n & Europe & Lapparent de Broin F. de, Bour R., Perälä J.,
	2006: Morphological definition of Eurotestudo (Testudinidae, Chelonii):
	First part. Annales de Paléontologie 92(3): 255-304\tabularnewline
	113 & Sierra de Quibas, Abanilla, Murcia & Eurotestudo & Eurotestudo
	hermanni & 284.00 & mf & Lower Pleistocene & 1.35000 & n & Europe &
	Pérez-García et al., 2015\tabularnewline
	114 & Tarazona de Aragón & gen. & gen. indet. & 1000.00 & mo & Langhian
	& 14.70000 & n & Europe & Murelaga X., Azanza B., Astibia H., 2006:
	Restos de quelonios del Mioceno medio del área de Tarazona de Aragón
	(Cuenca del Ebro, Aragón, España). Estudios Geológicos 62(1):
	205-212\tabularnewline
	115 & La Ciesma 1, Aragón & gen. & gen. indet. & 270.00 & mo &
	Serravallian & 12.20000 & n & Europe & Murelaga X., Azanza B., Astibia
	H., 2006: Restos de quelonios del Mioceno medio del área de Tarazona de
	Aragón (Cuenca del Ebro, Aragón, España). Estudios Geológicos 62(1):
	205-212\tabularnewline
	116 & Monteagudo, Aragón & gen. & gen. indet. & 270.00 & mo &
	Burdigalian/Aquitanian & 16.40000 & n & Europe & Murelaga X., Azanza B.,
	Astibia H., 2006: Restos de quelonios del Mioceno medio del área de
	Tarazona de Aragón (Cuenca del Ebro, Aragón, España). Estudios
	Geológicos 62(1): 205-212\tabularnewline
	117 & Kohfidisch & gen. & gen. indet. & 440.00 & m & Tortonian & 8.75000
	& n & Europe & Bachmayer F., M?ynarski M., 1983: Die Fauna der
	pontischen Höhlen- und Spaltenfüllungen bei Kohfidisch, Burgenland
	(Österreich)-Schildkröten (Emydidae und Testudinidae). Annalen des
	Naturhistorischen Museums in Wien 85/A: 107-128\tabularnewline
	118 & Kohfidisch & gen. & gen. indet. & 660.00 & m & Tortonian & 8.75000
	& n & Europe & Bachmayer F., M?ynarski M., 1983: Die Fauna der
	pontischen Höhlen- und Spaltenfüllungen bei Kohfidisch, Burgenland
	(Österreich)-Schildkröten (Emydidae und Testudinidae). Annalen des
	Naturhistorischen Museums in Wien 85/A: 107-128\tabularnewline
	119 & Zubbio di Cozzo San Pietro & gen. & gen. indet. & 813.00 & ef &
	Upper Pleistocene & 0.01250 & y & Europe & Delfino et al.,
	2015\tabularnewline
	120 & Kohfidisch & gen. & gen. indet. & 880.00 & m & Tortonian & 8.75000
	& n & Europe & Bachmayer F., M?ynarski M., 1983: Die Fauna der
	pontischen Höhlen- und Spaltenfüllungen bei Kohfidisch, Burgenland
	(Österreich)-Schildkröten (Emydidae und Testudinidae). Annalen des
	Naturhistorischen Museums in Wien 85/A: 107-128\tabularnewline
	121 & Jambol & Geochelone & Geochelone s. l. & 1750.00 & mo & Zanclean &
	4.46600 & n & Europe & Stojanoc, A. (2009). Erster Nachweis einer
	Riesenlandschildkröte (Geochelone sl Gray, 1872) aus Bulgarien.~Revue de
	Paléobiologie,~28, 457-470.\tabularnewline
	122 & Kirchdorf an der Iller & Geochelone & Geochelone sp. & 1000.00 & m
	& Burdigalian/Aquitanian & 16.65000 & n & Europe &
	FosFarBase\tabularnewline
	123 & Hohenhöwen, Engen, Hegau, southwestern Germany & Paleotestudo &
	Paleotestudo antiqua & 145.00 & mf & Serravallian & 13.00000 & n &
	Europe & Corsini J.A., Böhme M., Joyce W.G., 2014: Reappraisal of
	Testudo antiqua (Testudines, Testudinidae) from the Miocene of
	Hohenhöwen, Germany. Journal of Paleontology
	88(5):948-966\tabularnewline
	124 & Hohenhöwen, Engen, Hegau, southwestern Germany & Paleotestudo &
	Paleotestudo antiqua & 152.00 & m & Serravallian & 13.00000 & n & Europe
	& Schleich H.H., 1981: Jungtertiäre Schildkröten Süddeutschlands unter
	besonderer Berücksichtigung der Fundstelle Sandelzhausen. Courier
	Forschungsinstitut Senckenberg 48: 372pp., Frankfurt\tabularnewline
	125 & Hohenhöwen, Engen, Hegau, southwestern Germany & Paleotestudo &
	Paleotestudo antiqua & 159.50 & m & Serravallian & 13.00000 & n & Europe
	& Corsini J.A., Böhme M., Joyce W.G., 2014: Reappraisal of Testudo
	antiqua (Testudines, Testudinidae) from the Miocene of Hohenhöwen,
	Germany. Journal of Paleontology 88(5):948-966\textbar{}Schleich H.H.,
	1981: Jungtertiäre Schildkröten Süddeutschlands unter besonderer
	Berücksichtigung der Fundstelle Sandelzhausen. Courier
	Forschungsinstitut Senckenberg 48: 372pp., Frankfurt\tabularnewline
	126 & Hohenhöwen, Engen, Hegau, southwestern Germany & Paleotestudo &
	Paleotestudo antiqua & 180.00 & m & Serravallian & 13.00000 & n & Europe
	& Corsini J.A., Böhme M., Joyce W.G., 2014: Reappraisal of Testudo
	antiqua (Testudines, Testudinidae) from the Miocene of Hohenhöwen,
	Germany. Journal of Paleontology 88(5):948-966\textbar{}Schleich H.H.,
	1981: Jungtertiäre Schildkröten Süddeutschlands unter besonderer
	Berücksichtigung der Fundstelle Sandelzhausen. Courier
	Forschungsinstitut Senckenberg 48: 372pp., Frankfurt\tabularnewline
	127 & Gammelsdorf & Paleotestudo & Paleotestudo antiqua & 183.70 & m &
	Serravallian & 12.15000 & n & Europe & Schleich H.H., 1981: Jungtertiäre
	Schildkröten Süddeutschlands unter besonderer Berücksichtigung der
	Fundstelle Sandelzhausen. Courier Forschungsinstitut Senckenberg 48:
	372pp., Frankfurt\tabularnewline
	128 & Hohenhöwen, Engen, Hegau, southwestern Germany & Paleotestudo &
	Paleotestudo antiqua & 185.00 & mf & Serravallian & 13.00000 & n &
	Europe & Corsini J.A., Böhme M., Joyce W.G., 2014: Reappraisal of
	Testudo antiqua (Testudines, Testudinidae) from the Miocene of
	Hohenhöwen, Germany. Journal of Paleontology
	88(5):948-966\tabularnewline
	129 & Sansan, Gers (lake) & Paleotestudo & Paleotestudo antiqua & 191.00
	& mf & Serravallian & 13.60000 & n & Europe & Pérez-García A., 2016:
	Analysis of the Iberian Aragonian record of Paleotestudo, and refutation
	of the validity of the Spanish
	\texttt{Testudo\ catalaunica'\ and\ the\ French}Paleotestudo
	canetotiana'. Spanish Journal of Palaeontology 31(2):
	321-340\tabularnewline
	130 & Hohenhöwen, Engen, Hegau, southwestern Germany & Paleotestudo &
	Paleotestudo antiqua & 195.00 & m & Serravallian & 13.00000 & n & Europe
	& Corsini J.A., Böhme M., Joyce W.G., 2014: Reappraisal of Testudo
	antiqua (Testudines, Testudinidae) from the Miocene of Hohenhöwen,
	Germany. Journal of Paleontology 88(5):948-966\textbar{}Schleich H.H.,
	1981: Jungtertiäre Schildkröten Süddeutschlands unter besonderer
	Berücksichtigung der Fundstelle Sandelzhausen. Courier
	Forschungsinstitut Senckenberg 48: 372pp., Frankfurt\tabularnewline
	131 & Hohenhöwen, Engen, Hegau, southwestern Germany & Paleotestudo &
	Paleotestudo antiqua & 195.00 & mf & Serravallian & 13.00000 & n &
	Europe & Corsini J.A., Böhme M., Joyce W.G., 2014: Reappraisal of
	Testudo antiqua (Testudines, Testudinidae) from the Miocene of
	Hohenhöwen, Germany. Journal of Paleontology
	88(5):948-966\tabularnewline
	132 & Gammelsdorf & Paleotestudo & Paleotestudo antiqua & 203.00 & m &
	Serravallian & 12.15000 & n & Europe & Schleich H.H., 1981: Jungtertiäre
	Schildkröten Süddeutschlands unter besonderer Berücksichtigung der
	Fundstelle Sandelzhausen. Courier Forschungsinstitut Senckenberg 48:
	372pp., Frankfurt\tabularnewline
	133 & Hohenhöwen, Engen, Hegau, southwestern Germany & Paleotestudo &
	Paleotestudo antiqua & 206.00 & mf & Serravallian & 13.00000 & n &
	Europe & Corsini J.A., Böhme M., Joyce W.G., 2014: Reappraisal of
	Testudo antiqua (Testudines, Testudinidae) from the Miocene of
	Hohenhöwen, Germany. Journal of Paleontology
	88(5):948-966\tabularnewline
	134 & Sansan, Gers (lake) & Paleotestudo & Paleotestudo antiqua & 213.00
	& mf & Serravallian & 13.60000 & n & Europe & Lapparent de Broin F. de,
	Bour R., Perälä J., 2006: Morphological definition of Eurotestudo
	(Testudinidae, Chelonii): First part. Annales de Paléontologie 92(3):
	255-304\tabularnewline
	135 & Hohenhöwen, Engen, Hegau, southwestern Germany & Paleotestudo &
	Paleotestudo antiqua & 220.00 & mf & Serravallian & 13.00000 & n &
	Europe & Corsini J.A., Böhme M., Joyce W.G., 2014: Reappraisal of
	Testudo antiqua (Testudines, Testudinidae) from the Miocene of
	Hohenhöwen, Germany. Journal of Paleontology
	88(5):948-966\tabularnewline
	136 & Hohenhöwen, Engen, Hegau, southwestern Germany & Paleotestudo &
	Paleotestudo antiqua & 229.00 & mf & Serravallian & 13.00000 & n &
	Europe & Corsini J.A., Böhme M., Joyce W.G., 2014: Reappraisal of
	Testudo antiqua (Testudines, Testudinidae) from the Miocene of
	Hohenhöwen, Germany. Journal of Paleontology
	88(5):948-966\tabularnewline
	137 & Sansan, Gers (lake) & Paleotestudo & Paleotestudo antiqua & 234.00
	& mf & Serravallian & 13.60000 & n & Europe & Pérez-García A., 2016:
	Analysis of the Iberian Aragonian record of Paleotestudo, and refutation
	of the validity of the Spanish
	\texttt{Testudo\ catalaunica'\ and\ the\ French}Paleotestudo
	canetotiana'. Spanish Journal of Palaeontology 31(2):
	321-340\tabularnewline
	138 & Hohenhöwen, Engen, Hegau, southwestern Germany & Paleotestudo &
	Paleotestudo antiqua & 240.00 & m & Serravallian & 13.00000 & n & Europe
	& Schleich H.H., 1981: Jungtertiäre Schildkröten Süddeutschlands unter
	besonderer Berücksichtigung der Fundstelle Sandelzhausen. Courier
	Forschungsinstitut Senckenberg 48: 372pp., Frankfurt\tabularnewline
	139 & Sansan, Gers (lake) & Paleotestudo & Paleotestudo antiqua & 240.00
	& mf & Serravallian & 13.60000 & n & Europe & Pérez-García A., 2016:
	Analysis of the Iberian Aragonian record of Paleotestudo, and refutation
	of the validity of the Spanish
	\texttt{Testudo\ catalaunica'\ and\ the\ French}Paleotestudo
	canetotiana'. Spanish Journal of Palaeontology 31(2):
	321-340\tabularnewline
	140 & Barajas, Madrid & Paleotestudo & Paleotestudo antiqua & 275.00 &
	mf & Langhian & 15.00000 & n & Europe & Pérez-García A., 2016: Analysis
	of the Iberian Aragonian record of Paleotestudo, and refutation of the
	validity of the Spanish
	\texttt{Testudo\ catalaunica'\ and\ the\ French}Paleotestudo
	canetotiana'. Spanish Journal of Palaeontology 31(2):
	321-340\tabularnewline
	141 & Illescas, Toledo & Paleotestudo & Paleotestudo antiqua & 283.80 &
	mf & Serravallian & 12.50000 & n & Europe & Pérez-García A., 2016:
	Analysis of the Iberian Aragonian record of Paleotestudo, and refutation
	of the validity of the Spanish
	\texttt{Testudo\ catalaunica'\ and\ the\ French}Paleotestudo
	canetotiana'. Spanish Journal of Palaeontology 31(2):
	321-340\tabularnewline
	142 & Can Mas near El Papiol, Barcelone province, Cataluña & Paleotestudo & Paleotestudo cf.~antiqua & 113.00
	& mf & Burdigalian/Aquitanian & 17.30000 & n & Europe & Pérez-García A.,
	2016: Analysis of the Iberian Aragonian record of Paleotestudo, and
	refutation of the validity of the Spanish
	\texttt{Testudo\ catalaunica'\ and\ the\ French}Paleotestudo
	canetotiana'. Spanish Journal of Palaeontology 31(2):
	321-340\tabularnewline
	143 & El Buste, Aragón & Paleotestudo & Paleotestudo cf.~sp. & 270.00 &
	mo & Serravallian & 12.40000 & n & Europe & Murelaga X., Azanza B.,
	Astibia H., 2006: Restos de quelonios del Mioceno medio del área de
	Tarazona de Aragón (Cuenca del Ebro, Aragón, España). Estudios
	Geológicos 62(1): 205-212\tabularnewline
	144 & Tarazona de Aragón & Paleotestudo & Paleotestudo cf.~sp. & 270.00
	& mo & Langhian & 14.70000 & n & Europe & Murelaga X., Azanza B.,
	Astibia H., 2006: Restos de quelonios del Mioceno medio del área de
	Tarazona de Aragón (Cuenca del Ebro, Aragón, España). Estudios
	Geológicos 62(1): 205-212\tabularnewline
	145 & Cerro de los Batallones, Madrid & Paleotestudo & Paleotestudo sp.
	& 170.00 & mf & Tortonian & 9.50000 & n & Europe & Pérez-García and
	Murelaga, 2013\tabularnewline
	146 & Teiritzberg (T1 = 001/D/C), Korneuburg Basin, Lower Austria &
	Paleotestudo & Paleotestudo sp. & 179.30 & m & Burdigalian/Aquitanian &
	16.55000 & n & Europe & Gemel R., 2002b: Weitere Schildkrötenreste aus
	dem Karpatium des Korneuburger Beckens (Untermiozän; Niederösterreich).
	in: Sovis W. \& Schmid B.: Das Karpat des Korneuburger Beckens, Teil 2,
	Beitr.Paläont. 27: 373-393, Wien\tabularnewline
	147 & Cerro de los Batallones, Madrid & Paleotestudo & Paleotestudo sp.
	& 261.00 & mf & Tortonian & 9.50000 & n & Europe & Pérez-García and
	Murelaga, 2013\tabularnewline
	148 & Cerro de los Batallones, Madrid & Paleotestudo & Paleotestudo sp.
	& 270.00 & mf & Tortonian & 9.50000 & n & Europe & Pérez-García and
	Murelaga, 2013\tabularnewline
	149 & Torrente Melacce, Cinigiano (GR) & Testudo & Testudo amiatae &
	140.00 & mo & Messinian & 5.81500 & n & Europe & Chesi, F. (2009). Il
	registro fossile italiano dei cheloni (Doctoral dissertation, PhD Thesis
	in Earth Sciences, Università di Firenze).\tabularnewline
	150 & Milia, Grevena, W Macedonia & Testudo & Testudo brevitesta &
	165.00 & mf & Piacencian & 2.60000 & n & Europe & Vlachos E., Tsoukala
	E., 2016: The diverse fossil chelonians from Milia (Late Pliocene,
	Grevena, Greece) with a new species of Testudo Linnaeus, 1758
	(Testudines: Testudinidae). Papers in Palaeontology 2(1):
	71-86\tabularnewline
	151 & Milia, Grevena, W Macedonia & Testudo & Testudo brevitesta &
	300.00 & mf & Piacencian & 2.60000 & n & Europe & Vlachos E., Tsoukala
	E., 2016: The diverse fossil chelonians from Milia (Late Pliocene,
	Grevena, Greece) with a new species of Testudo Linnaeus, 1758
	(Testudines: Testudinidae). Papers in Palaeontology 2(1):
	71-86\tabularnewline
	152 & Kohfidisch & Testudo & Testudo burgenlandica & 112.00 & m &
	Tortonian & 8.75000 & n & Europe & Karl, ??? (Einige Bemerkungen über
	die fossilen Schildkröten des Bundeslandes Salzburg)\tabularnewline
	153 & Kohfidisch & Testudo & Testudo burgenlandica & 275.00 & m &
	Tortonian & 8.75000 & n & Europe & Bachmayer F., M?ynarski M., 1983: Die
	Fauna der pontischen Höhlen- und Spaltenfüllungen bei Kohfidisch,
	Burgenland (Österreich)-Schildkröten (Emydidae und Testudinidae).
	Annalen des Naturhistorischen Museums in Wien 85/A:
	107-128\tabularnewline
	154 & Sant Quirze de Terrassa/de Galliners (del Vallès), Barcelona &
	Testudo & Testudo catalaunica & 107.00 & m & Tortonian & 11.50000 & n &
	Europe & Luján et al., 2016\tabularnewline
	155 & Castell de Barbera & Testudo & Testudo catalaunica & 165.00 & m &
	Tortonian & 11.50000 & n & Europe & Luján et al., 2016\tabularnewline
	156 & Sant Quirze de Terrassa/de Galliners (del Vallès), Barcelona &
	Testudo & Testudo catalaunica & 175.00 & m & Tortonian & 11.50000 & n &
	Europe & Luján et al., 2016\tabularnewline
	157 & Sant Quirze de Terrassa/de Galliners (del Vallès), Barcelona &
	Testudo & Testudo catalaunica & 181.00 & m & Tortonian & 11.50000 & n &
	Europe & Luján et al., 2016\tabularnewline
	158 & Abocador de Can Mata (els Hostalets de Pierola), Cataluña & Testudo & Testudo catalaunica & 232.00
	& m & Serravallian & 12.35000 & n & Europe & Luján et al.,
	2016\tabularnewline
	159 & Megalo Emvolon 1 (MEV), 20 km SW Thessaloniki & Testudo & Testudo
	cf.~graeca & 185.00 & m & Zanclean & 3.90000 & n & Europe & Bachmayer
	F., M?ynarski M., Symeonidis N., 1980: Fossile Schiildkröten aus dem
	Pliozän von Megalo Emvolo (Karaburun) bei Saloniki (Griechenland) A.
	Eine fossile Maurische Landschildkröte (Testudo cf.~graeca LINNE) B.
	Fossile Reste von Riesenschildkröten. Annales Géologiques des Pays
	Hellénique 31: 267-276\tabularnewline
	160 & Prottes & Testudo & Testudo cf.~promarginata & 250.00 & m &
	Tortonian & 8.30000 & n & Europe & Bachmayer F., M?ynarski M., 1985: Die
	Landschildkröten (Testudinidae) aus den Schotter-Ablagerungen (Pontien)
	von Prottes, Niederösterreich.. Annalen des Naturhistorischen Museums in
	Wien 87 A: 65-77\tabularnewline
	161 & Prottes & Testudo & Testudo cf.~promarginata & 250.00 & m &
	Tortonian & 8.30000 & n & Europe & Bachmayer F., M?ynarski M., 1985: Die
	Landschildkröten (Testudinidae) aus den Schotter-Ablagerungen (Pontien)
	von Prottes, Niederösterreich.. Annalen des Naturhistorischen Museums in
	Wien 87 A: 65-77\tabularnewline
	162 & Prottes & Testudo & Testudo cf.~promarginata & 250.00 & m &
	Tortonian & 8.30000 & n & Europe & Bachmayer F., M?ynarski M., 1985: Die
	Landschildkröten (Testudinidae) aus den Schotter-Ablagerungen (Pontien)
	von Prottes, Niederösterreich.. Annalen des Naturhistorischen Museums in
	Wien 87 A: 65-77\tabularnewline
	163 & Prottes & Testudo & Testudo cf.~promarginata & 250.00 & m &
	Tortonian & 8.30000 & n & Europe & Bachmayer F., M?ynarski M., 1985: Die
	Landschildkröten (Testudinidae) aus den Schotter-Ablagerungen (Pontien)
	von Prottes, Niederösterreich.. Annalen des Naturhistorischen Museums in
	Wien 87 A: 65-77\tabularnewline
	164 & Prottes & Testudo & Testudo cf.~promarginata & 250.00 & m &
	Tortonian & 8.30000 & n & Europe & Bachmayer F., M?ynarski M., 1985: Die
	Landschildkröten (Testudinidae) aus den Schotter-Ablagerungen (Pontien)
	von Prottes, Niederösterreich.. Annalen des Naturhistorischen Museums in
	Wien 87 A: 65-77\tabularnewline
	165 & Prottes & Testudo & Testudo cf.~promarginata & 250.00 & m &
	Tortonian & 8.30000 & n & Europe & Bachmayer F., M?ynarski M., 1985: Die
	Landschildkröten (Testudinidae) aus den Schotter-Ablagerungen (Pontien)
	von Prottes, Niederösterreich.. Annalen des Naturhistorischen Museums in
	Wien 87 A: 65-77\tabularnewline
	166 & Prottes & Testudo & Testudo cf.~promarginata & 250.00 & m &
	Tortonian & 8.30000 & n & Europe & Bachmayer F., M?ynarski M., 1985: Die
	Landschildkröten (Testudinidae) aus den Schotter-Ablagerungen (Pontien)
	von Prottes, Niederösterreich.. Annalen des Naturhistorischen Museums in
	Wien 87 A: 65-77\tabularnewline
	167 & Prottes & Testudo & Testudo cf.~promarginata & 250.00 & m &
	Tortonian & 8.30000 & n & Europe & Bachmayer F., M?ynarski M., 1985: Die
	Landschildkröten (Testudinidae) aus den Schotter-Ablagerungen (Pontien)
	von Prottes, Niederösterreich.. Annalen des Naturhistorischen Museums in
	Wien 87 A: 65-77\tabularnewline
	168 & Prottes & Testudo & Testudo cf.~promarginata & 250.00 & m &
	Tortonian & 8.30000 & n & Europe & Bachmayer F., M?ynarski M., 1985: Die
	Landschildkröten (Testudinidae) aus den Schotter-Ablagerungen (Pontien)
	von Prottes, Niederösterreich.. Annalen des Naturhistorischen Museums in
	Wien 87 A: 65-77\tabularnewline
	169 & Prottes & Testudo & Testudo cf.~promarginata & 250.00 & m &
	Tortonian & 8.30000 & n & Europe & Bachmayer F., M?ynarski M., 1985: Die
	Landschildkröten (Testudinidae) aus den Schotter-Ablagerungen (Pontien)
	von Prottes, Niederösterreich.. Annalen des Naturhistorischen Museums in
	Wien 87 A: 65-77\tabularnewline
	170 & Prottes & Testudo & Testudo cf.~promarginata & 250.00 & m &
	Tortonian & 8.30000 & n & Europe & Bachmayer F., M?ynarski M., 1985: Die
	Landschildkröten (Testudinidae) aus den Schotter-Ablagerungen (Pontien)
	von Prottes, Niederösterreich.. Annalen des Naturhistorischen Museums in
	Wien 87 A: 65-77\tabularnewline
	171 & Prottes & Testudo & Testudo cf.~promarginata & 250.00 & m &
	Tortonian & 8.30000 & n & Europe & Bachmayer F., M?ynarski M., 1985: Die
	Landschildkröten (Testudinidae) aus den Schotter-Ablagerungen (Pontien)
	von Prottes, Niederösterreich.. Annalen des Naturhistorischen Museums in
	Wien 87 A: 65-77\tabularnewline
	172 & Prottes & Testudo & Testudo cf.~promarginata & 250.00 & m &
	Tortonian & 8.30000 & n & Europe & Bachmayer F., M?ynarski M., 1985: Die
	Landschildkröten (Testudinidae) aus den Schotter-Ablagerungen (Pontien)
	von Prottes, Niederösterreich.. Annalen des Naturhistorischen Museums in
	Wien 87 A: 65-77\tabularnewline
	173 & Prottes & Testudo & Testudo cf.~promarginata & 250.00 & m &
	Tortonian & 8.30000 & n & Europe & Bachmayer F., M?ynarski M., 1985: Die
	Landschildkröten (Testudinidae) aus den Schotter-Ablagerungen (Pontien)
	von Prottes, Niederösterreich.. Annalen des Naturhistorischen Museums in
	Wien 87 A: 65-77\tabularnewline
	174 & Prottes & Testudo & Testudo cf.~promarginata & 250.00 & m &
	Tortonian & 8.30000 & n & Europe & Bachmayer F., M?ynarski M., 1985: Die
	Landschildkröten (Testudinidae) aus den Schotter-Ablagerungen (Pontien)
	von Prottes, Niederösterreich.. Annalen des Naturhistorischen Museums in
	Wien 87 A: 65-77\tabularnewline
	175 & Prottes & Testudo & Testudo cf.~promarginata & 250.00 & m &
	Tortonian & 8.30000 & n & Europe & Bachmayer F., M?ynarski M., 1985: Die
	Landschildkröten (Testudinidae) aus den Schotter-Ablagerungen (Pontien)
	von Prottes, Niederösterreich.. Annalen des Naturhistorischen Museums in
	Wien 87 A: 65-77\tabularnewline
	176 & Prottes & Testudo & Testudo cf.~promarginata & 250.00 & m &
	Tortonian & 8.30000 & n & Europe & Bachmayer F., M?ynarski M., 1985: Die
	Landschildkröten (Testudinidae) aus den Schotter-Ablagerungen (Pontien)
	von Prottes, Niederösterreich.. Annalen des Naturhistorischen Museums in
	Wien 87 A: 65-77\tabularnewline
	177 & Prottes & Testudo & Testudo cf.~promarginata & 250.00 & m &
	Tortonian & 8.30000 & n & Europe & Bachmayer F., M?ynarski M., 1985: Die
	Landschildkröten (Testudinidae) aus den Schotter-Ablagerungen (Pontien)
	von Prottes, Niederösterreich.. Annalen des Naturhistorischen Museums in
	Wien 87 A: 65-77\tabularnewline
	178 & Prottes & Testudo & Testudo cf.~promarginata & 250.00 & m &
	Tortonian & 8.30000 & n & Europe & Bachmayer F., M?ynarski M., 1985: Die
	Landschildkröten (Testudinidae) aus den Schotter-Ablagerungen (Pontien)
	von Prottes, Niederösterreich.. Annalen des Naturhistorischen Museums in
	Wien 87 A: 65-77\tabularnewline
	179 & Prottes & Testudo & Testudo cf.~promarginata & 250.00 & m &
	Tortonian & 8.30000 & n & Europe & Bachmayer F., M?ynarski M., 1985: Die
	Landschildkröten (Testudinidae) aus den Schotter-Ablagerungen (Pontien)
	von Prottes, Niederösterreich.. Annalen des Naturhistorischen Museums in
	Wien 87 A: 65-77\tabularnewline
	180 & Prottes & Testudo & Testudo cf.~promarginata & 250.00 & m &
	Tortonian & 8.30000 & n & Europe & Bachmayer F., M?ynarski M., 1985: Die
	Landschildkröten (Testudinidae) aus den Schotter-Ablagerungen (Pontien)
	von Prottes, Niederösterreich.. Annalen des Naturhistorischen Museums in
	Wien 87 A: 65-77\tabularnewline
	181 & Prottes & Testudo & Testudo cf.~promarginata & 250.00 & m &
	Tortonian & 8.30000 & n & Europe & Bachmayer F., M?ynarski M., 1985: Die
	Landschildkröten (Testudinidae) aus den Schotter-Ablagerungen (Pontien)
	von Prottes, Niederösterreich.. Annalen des Naturhistorischen Museums in
	Wien 87 A: 65-77\tabularnewline
	182 & Prottes & Testudo & Testudo cf.~promarginata & 250.00 & m &
	Tortonian & 8.30000 & n & Europe & Bachmayer F., M?ynarski M., 1985: Die
	Landschildkröten (Testudinidae) aus den Schotter-Ablagerungen (Pontien)
	von Prottes, Niederösterreich.. Annalen des Naturhistorischen Museums in
	Wien 87 A: 65-77\tabularnewline
	183 & Prottes & Testudo & Testudo cf.~promarginata & 250.00 & m &
	Tortonian & 8.30000 & n & Europe & Bachmayer F., M?ynarski M., 1985: Die
	Landschildkröten (Testudinidae) aus den Schotter-Ablagerungen (Pontien)
	von Prottes, Niederösterreich.. Annalen des Naturhistorischen Museums in
	Wien 87 A: 65-77\tabularnewline
	184 & Prottes & Testudo & Testudo cf.~promarginata & 250.00 & m &
	Tortonian & 8.30000 & n & Europe & Bachmayer F., M?ynarski M., 1985: Die
	Landschildkröten (Testudinidae) aus den Schotter-Ablagerungen (Pontien)
	von Prottes, Niederösterreich.. Annalen des Naturhistorischen Museums in
	Wien 87 A: 65-77\tabularnewline
	185 & Pylea, eastern part of Thessaloniki, western Chalkidiki peninsula
	& Testudo & Testudo graeca & 167.00 & m & Messinian & 5.50000 & n &
	Europe & Vlachos E., Kotsakis T., Delfino M., 2015: The chelonians from
	the Latest Miocene--Earliest Pliocene localities of Allatini and Pylea
	(East Thessaloniki, Macedonia, Greece). Comptes Rendus Palevol 14:
	187-205\tabularnewline
	186 & Allatini, eastern part of Thessaloniki, western Chalkidiki
	peninsula & Testudo & Testudo graeca & 200.00 & mf & Messinian & 5.50000
	& n & Europe & Vlachos E., Kotsakis T., Delfino M., 2015: The chelonians
	from the Latest Miocene--Earliest Pliocene localities of Allatini and
	Pylea (East Thessaloniki, Macedonia, Greece). Comptes Rendus Palevol 14:
	187-205\tabularnewline
	187 & Platania, Drama basin & Testudo & Testudo graeca & 210.00 & mf &
	Tortonian & 8.45000 & n & Europe & Vlachos and Tsoukala,
	2014\tabularnewline
	188 & Sima del Elefante TE14, Sierra de Atapuerca, Burgos & Eurotestudo
	& Testudo hermanni & 133.10 & mf & Lower Pleistocene & 1.22000 & n &
	Europe & Blasco R., Blain H.A., Rosell J., Díez J.C., Huguet R.,
	Rodríguez J., Arsuga J.L., Bermúdez de Castro J.M., Carbonell E., 2011:
	Earliest evidence for human consumption of tortoises in the European
	Early Pleistocene from Sima del Elefante, Sierra de Atapuerca, Spain.
	Journal of Human Evolution
	\url{doi:10.1016/jhevol.2011.06.002}\tabularnewline
	189 & Obermaintor, Ebensfeld (Lichtenfels), Franken & Testudo & Testudo
	hermanni & 220.00 & mf & Lower Pleistocene & 1.30000 & n & Europe & Karl
	\& Tichy, 2002\tabularnewline
	190 & Leithagebirge between Au and Loretto & Testudo & Testudo
	kalksburgensis & 225.00 & mo & Burdigalian/Aquitanian & 18.00000 & n &
	Europe & Siebenrock F., 1914: Testudo kalksburgensis Toula aus dem
	Leithagebirge. Jahrbuch der Kaiserlich-Königlichen Geologischen
	Reichsanstalt 64(1-2): 357-362\tabularnewline
	191 & Eggenburg-Schindergraben, Lower Austria & Testudo & Testudo
	kalksburgensis & 230.00 & m & Burdigalian/Aquitanian & 19.96500 & n &
	Europe & Gemel R., 2002b: Weitere Schildkrötenreste aus dem Karpatium
	des Korneuburger Beckens (Untermiozän; Niederösterreich). in: Sovis W.
	\& Schmid B.: Das Karpat des Korneuburger Beckens, Teil 2,
	Beitr.Paläont. 27: 373-393, Wien\tabularnewline
	192 & Wien-Kalksburg & Testudo & Testudo kalksburgensis & 275.00 & m &
	Langhian & 14.50000 & n & Europe & Bachmayer M.F., M?ynarski M., 1981:
	Testudo kalksburgensis Toula, 1896, eine valide Schildkrötenart aus den
	miozänen Strandbildungen von Kalksburg bei Wien. Sitzungsberichte der
	Akademie der Wissenschaften mathematisch-naturwissenschaftliche Klasse
	190: 111-119\tabularnewline
	193 & Cova de Gràcia, Park Güell, Barcelona & Testudo & Testudo
	lunellensis & 176.00 & mo & Middle Pleistocene & 0.45350 & n & Europe &
	Delfino M., Luján À.H., Carmona R., Alba D.M., 2012: Revision of the
	extinct Pleistocene tortoise Testudo lunellensis Almera and Bo?ll, 1903
	from Cova de Gràcia (Barcelona, Spain). Amphibia-Reptilia 33:
	215-225\tabularnewline
	194 & Caverna de Gràcia, Güell park, Barcelona & Testudo & Testudo
	lunellensis & 194.00 & mf & Middle Pleistocene & 0.45000 & n & Europe &
	Lapparent de Broin F. de, Bour R., Perälä J., 2006: Morphological
	definition of Eurotestudo (Testudinidae, Chelonii): First part. Annales
	de Paléontologie 92(3): 255-304\tabularnewline
	195 & Cova de Gràcia, Park Güell, Barcelona & Testudo & Testudo
	lunellensis & 231.00 & ev & Middle Pleistocene & 0.45350 & n & Europe &
	Delfino M., Luján À.H., Carmona R., Alba D.M., 2012: Revision of the
	extinct Pleistocene tortoise Testudo lunellensis Almera and Bo?ll, 1903
	from Cova de Gràcia (Barcelona, Spain). Amphibia-Reptilia 33:
	215-225\tabularnewline
	196 & Caverna de Gràcia, Güell park, Barcelona & Testudo & Testudo
	lunellensis & 260.70 & mf & Middle Pleistocene & 0.45000 & n & Europe &
	Lapparent de Broin F. de, Bour R., Perälä J., 2006: Morphological
	definition of Eurotestudo (Testudinidae, Chelonii): First part. Annales
	de Paléontologie 92(3): 255-304\tabularnewline
	197 & Lakonia & Testudo & Testudo marginata & 210.00 & m & Lower
	Pleistocene & 1.72000 & n & Europe & Schleich H.H., 1982a: Testudo
	marginata Schoepff aus plio/pleistozänen Ablagerungen SE-Lakoniens
	(Peloponnes, Griechenland). Paläontologische Zeitschrift
	56:259-264\tabularnewline
	198 & Zourida-Höhle & Testudo & Testudo marginata & 290.00 & m & Lower
	Pleistocene & 1.30000 & y & Europe & Bachmayer, Brinkerink and
	Symeonidis, 1975\tabularnewline
	199 & Gerani-Höhle an der Nordküste Kretamin der Nähe von Rethymnon &
	Testudo & Testudo marginata & 310.00 & m & Lower Pleistocene & 1.30000 &
	y & Europe & Bachmayer, Brinkerink and Symeonidis, 1975\tabularnewline
	200 & Capo Mannu near San Vero Milis, base of D4 dune, Sardinia &
	Testudo & Testudo pecorinii & 225.00 & m & Piacencian & 3.09400 & y &
	Europe & Abbazzi L., Carboni S., Delfino M., Gallai G., Lecca L., Rook
	L., 2008: Fossil vertebrates (Mammalia and Reptilia) from Capo Mannu
	Formation (Late Pliocene, Sardinia, Italy), with description of a new
	Testudo (Chelonii, Testudinidae) species. Rivista Italiana di
	Paleontologia e Stratigrafia 114: 119-132\tabularnewline
	201 & Saint-Gérand-le-Puy, Allier & Testudo & Testudo promarginata &
	230.00 & mf & Burdigalian/Aquitanian & 21.50000 & n & Europe &
	Pérez-García A., 2016: Analysis of the Iberian Aragonian record of
	Paleotestudo, and refutation of the validity of the Spanish
	\texttt{Testudo\ catalaunica'\ and\ the\ French}Paleotestudo
	canetotiana'. Spanish Journal of Palaeontology 31(2):
	321-340\tabularnewline
	202 & Saint-Gérand-le-Puy, Allier & Testudo & Testudo promarginata &
	304.70 & mf & Burdigalian/Aquitanian & 21.50000 & n & Europe &
	Pérez-García A., 2016: Analysis of the Iberian Aragonian record of
	Paleotestudo, and refutation of the validity of the Spanish
	\texttt{Testudo\ catalaunica'\ and\ the\ French}Paleotestudo
	canetotiana'. Spanish Journal of Palaeontology 31(2):
	321-340\tabularnewline
	203 & Neuville-aux-Bois, Loiret & Testudo & Testudo promarginata &
	310.00 & mf & Burdigalian/Aquitanian & 18.00000 & n & Europe &
	Pérez-García A., 2016: Analysis of the Iberian Aragonian record of
	Paleotestudo, and refutation of the validity of the Spanish
	\texttt{Testudo\ catalaunica'\ and\ the\ French}Paleotestudo
	canetotiana'. Spanish Journal of Palaeontology 31(2):
	321-340\tabularnewline
	204 & Sandelzhausen & Testudo & Testudo rectogularis & 213.00 & mo &
	Burdigalian/Aquitanian & 16.37000 & n & Europe & Schleich H.H., 1981:
	Jungtertiäre Schildkröten Süddeutschlands unter besonderer
	Berücksichtigung der Fundstelle Sandelzhausen. Courier
	Forschungsinstitut Senckenberg 48: 372pp., Frankfurt\tabularnewline
	205 & Nikiti 2, Chalkidiki, Macedonia & Testudo & Testudo s. s. & 189.00
	& m & Tortonian & 8.00000 & n & Europe & Garcia et al.,
	2016\tabularnewline
	206 & Liossati, Kiourka & Testudo & Testudo sp. & 1200.00 & mf &
	Zanclean & 3.96000 & n & Europe & Bachmayer, F., \& Symeonidis, N.
	(1977). Eine neue „Pikermi ``Fundstelle im Gebiet von Liossati (Kiourka)
	nördlich von Athen (Griechenland)(Beschreibung einer Riesenschildkröte).
	In Annales Géologiques des Pays Helléniques (Vol. 28,
	pp.~8-16).\tabularnewline
	207 & Santa-Vittoria d'Alba & Testudo & Testudo sp. & 200.00 & mf &
	Messinian & 6.16500 & n & Europe & Chesi, F. (2009). Il registro fossile
	italiano dei cheloni (Doctoral dissertation, PhD Thesis in Earth
	Sciences, Università di Firenze).\tabularnewline
	208 & Holzmannsdorfberg bei St.~Marein & Testudo & Testudo sp. & 232.10
	& m & Tortonian & 10.75000 & n & Europe & Gross M., 2002: Aus der
	paläontologischen Sammlung des Landesmuseums Joanneum - Die fossilen
	Schildkröten (Testudines). Joannea Geol. Paläont. 4: 5-68,
	Taf.1-22\tabularnewline
	209 & Prottes & Testudo & Testudo sp. & 245.00 & m & Tortonian & 8.30000
	& n & Europe & Bachmayer F., M?ynarski M., 1985: Die Landschildkröten
	(Testudinidae) aus den Schotter-Ablagerungen (Pontien) von Prottes,
	Niederösterreich.. Annalen des Naturhistorischen Museums in Wien 87 A:
	65-77\tabularnewline
	210 & Prottes & Testudo & Testudo sp. & 245.00 & m & Tortonian & 8.30000
	& n & Europe & Bachmayer F., M?ynarski M., 1985: Die Landschildkröten
	(Testudinidae) aus den Schotter-Ablagerungen (Pontien) von Prottes,
	Niederösterreich.. Annalen des Naturhistorischen Museums in Wien 87 A:
	65-77\tabularnewline
	211 & Prottes & Testudo & Testudo sp. & 245.00 & m & Tortonian & 8.30000
	& n & Europe & Bachmayer F., M?ynarski M., 1985: Die Landschildkröten
	(Testudinidae) aus den Schotter-Ablagerungen (Pontien) von Prottes,
	Niederösterreich.. Annalen des Naturhistorischen Museums in Wien 87 A:
	65-77\tabularnewline
	212 & Prottes & Testudo & Testudo sp. & 245.00 & m & Tortonian & 8.30000
	& n & Europe & Bachmayer F., M?ynarski M., 1985: Die Landschildkröten
	(Testudinidae) aus den Schotter-Ablagerungen (Pontien) von Prottes,
	Niederösterreich.. Annalen des Naturhistorischen Museums in Wien 87 A:
	65-77\tabularnewline
	213 & Prottes & Testudo & Testudo sp. & 245.00 & m & Tortonian & 8.30000
	& n & Europe & Bachmayer F., M?ynarski M., 1985: Die Landschildkröten
	(Testudinidae) aus den Schotter-Ablagerungen (Pontien) von Prottes,
	Niederösterreich.. Annalen des Naturhistorischen Museums in Wien 87 A:
	65-77\tabularnewline
	214 & Prottes & Testudo & Testudo sp. & 245.00 & m & Tortonian & 8.30000
	& n & Europe & Bachmayer F., M?ynarski M., 1985: Die Landschildkröten
	(Testudinidae) aus den Schotter-Ablagerungen (Pontien) von Prottes,
	Niederösterreich.. Annalen des Naturhistorischen Museums in Wien 87 A:
	65-77\tabularnewline
	215 & Prottes & Testudo & Testudo sp. & 245.00 & m & Tortonian & 8.30000
	& n & Europe & Bachmayer F., M?ynarski M., 1985: Die Landschildkröten
	(Testudinidae) aus den Schotter-Ablagerungen (Pontien) von Prottes,
	Niederösterreich.. Annalen des Naturhistorischen Museums in Wien 87 A:
	65-77\tabularnewline
	216 & Prottes & Testudo & Testudo sp. & 245.00 & m & Tortonian & 8.30000
	& n & Europe & Bachmayer F., M?ynarski M., 1985: Die Landschildkröten
	(Testudinidae) aus den Schotter-Ablagerungen (Pontien) von Prottes,
	Niederösterreich.. Annalen des Naturhistorischen Museums in Wien 87 A:
	65-77\tabularnewline
	217 & Prottes & Testudo & Testudo sp. & 245.00 & m & Tortonian & 8.30000
	& n & Europe & Bachmayer F., M?ynarski M., 1985: Die Landschildkröten
	(Testudinidae) aus den Schotter-Ablagerungen (Pontien) von Prottes,
	Niederösterreich.. Annalen des Naturhistorischen Museums in Wien 87 A:
	65-77\tabularnewline
	218 & Prottes & Testudo & Testudo sp. & 245.00 & m & Tortonian & 8.30000
	& n & Europe & Bachmayer F., M?ynarski M., 1985: Die Landschildkröten
	(Testudinidae) aus den Schotter-Ablagerungen (Pontien) von Prottes,
	Niederösterreich.. Annalen des Naturhistorischen Museums in Wien 87 A:
	65-77\tabularnewline
	219 & Prottes & Testudo & Testudo sp. & 245.00 & m & Tortonian & 8.30000
	& n & Europe & Bachmayer F., M?ynarski M., 1985: Die Landschildkröten
	(Testudinidae) aus den Schotter-Ablagerungen (Pontien) von Prottes,
	Niederösterreich.. Annalen des Naturhistorischen Museums in Wien 87 A:
	65-77\tabularnewline
	220 & Prottes & Testudo & Testudo sp. & 245.00 & m & Tortonian & 8.30000
	& n & Europe & Bachmayer F., M?ynarski M., 1985: Die Landschildkröten
	(Testudinidae) aus den Schotter-Ablagerungen (Pontien) von Prottes,
	Niederösterreich.. Annalen des Naturhistorischen Museums in Wien 87 A:
	65-77\tabularnewline
	221 & Prottes & Testudo & Testudo sp. & 245.00 & m & Tortonian & 8.30000
	& n & Europe & Bachmayer F., M?ynarski M., 1985: Die Landschildkröten
	(Testudinidae) aus den Schotter-Ablagerungen (Pontien) von Prottes,
	Niederösterreich.. Annalen des Naturhistorischen Museums in Wien 87 A:
	65-77\tabularnewline
	222 & Prottes & Testudo & Testudo sp. & 245.00 & m & Tortonian & 8.30000
	& n & Europe & Bachmayer F., M?ynarski M., 1985: Die Landschildkröten
	(Testudinidae) aus den Schotter-Ablagerungen (Pontien) von Prottes,
	Niederösterreich.. Annalen des Naturhistorischen Museums in Wien 87 A:
	65-77\tabularnewline
	223 & Prottes & Testudo & Testudo sp. & 245.00 & m & Tortonian & 8.30000
	& n & Europe & Bachmayer F., M?ynarski M., 1985: Die Landschildkröten
	(Testudinidae) aus den Schotter-Ablagerungen (Pontien) von Prottes,
	Niederösterreich.. Annalen des Naturhistorischen Museums in Wien 87 A:
	65-77\tabularnewline
	224 & Prottes & Testudo & Testudo sp. & 245.00 & m & Tortonian & 8.30000
	& n & Europe & Bachmayer F., M?ynarski M., 1985: Die Landschildkröten
	(Testudinidae) aus den Schotter-Ablagerungen (Pontien) von Prottes,
	Niederösterreich.. Annalen des Naturhistorischen Museums in Wien 87 A:
	65-77\tabularnewline
	225 & Prottes & Testudo & Testudo sp. & 245.00 & m & Tortonian & 8.30000
	& n & Europe & Bachmayer F., M?ynarski M., 1985: Die Landschildkröten
	(Testudinidae) aus den Schotter-Ablagerungen (Pontien) von Prottes,
	Niederösterreich.. Annalen des Naturhistorischen Museums in Wien 87 A:
	65-77\tabularnewline
	226 & Prottes & Testudo & Testudo sp. & 245.00 & m & Tortonian & 8.30000
	& n & Europe & Bachmayer F., M?ynarski M., 1985: Die Landschildkröten
	(Testudinidae) aus den Schotter-Ablagerungen (Pontien) von Prottes,
	Niederösterreich.. Annalen des Naturhistorischen Museums in Wien 87 A:
	65-77\tabularnewline
	227 & Prottes & Testudo & Testudo sp. & 245.00 & m & Tortonian & 8.30000
	& n & Europe & Bachmayer F., M?ynarski M., 1985: Die Landschildkröten
	(Testudinidae) aus den Schotter-Ablagerungen (Pontien) von Prottes,
	Niederösterreich.. Annalen des Naturhistorischen Museums in Wien 87 A:
	65-77\tabularnewline
	228 & Prottes & Testudo & Testudo sp. & 245.00 & m & Tortonian & 8.30000
	& n & Europe & Bachmayer F., M?ynarski M., 1985: Die Landschildkröten
	(Testudinidae) aus den Schotter-Ablagerungen (Pontien) von Prottes,
	Niederösterreich.. Annalen des Naturhistorischen Museums in Wien 87 A:
	65-77\tabularnewline
	229 & Prottes & Testudo & Testudo sp. & 245.00 & m & Tortonian & 8.30000
	& n & Europe & Bachmayer F., M?ynarski M., 1985: Die Landschildkröten
	(Testudinidae) aus den Schotter-Ablagerungen (Pontien) von Prottes,
	Niederösterreich.. Annalen des Naturhistorischen Museums in Wien 87 A:
	65-77\tabularnewline
	230 & Prottes & Testudo & Testudo sp. & 245.00 & m & Tortonian & 8.30000
	& n & Europe & Bachmayer F., M?ynarski M., 1985: Die Landschildkröten
	(Testudinidae) aus den Schotter-Ablagerungen (Pontien) von Prottes,
	Niederösterreich.. Annalen des Naturhistorischen Museums in Wien 87 A:
	65-77\tabularnewline
	231 & Prottes & Testudo & Testudo sp. & 245.00 & m & Tortonian & 8.30000
	& n & Europe & Bachmayer F., M?ynarski M., 1985: Die Landschildkröten
	(Testudinidae) aus den Schotter-Ablagerungen (Pontien) von Prottes,
	Niederösterreich.. Annalen des Naturhistorischen Museums in Wien 87 A:
	65-77\tabularnewline
	232 & Prottes & Testudo & Testudo sp. & 245.00 & m & Tortonian & 8.30000
	& n & Europe & Bachmayer F., M?ynarski M., 1985: Die Landschildkröten
	(Testudinidae) aus den Schotter-Ablagerungen (Pontien) von Prottes,
	Niederösterreich.. Annalen des Naturhistorischen Museums in Wien 87 A:
	65-77\tabularnewline
	233 & Prottes & Testudo & Testudo sp. & 245.00 & m & Tortonian & 8.30000
	& n & Europe & Bachmayer F., M?ynarski M., 1985: Die Landschildkröten
	(Testudinidae) aus den Schotter-Ablagerungen (Pontien) von Prottes,
	Niederösterreich.. Annalen des Naturhistorischen Museums in Wien 87 A:
	65-77\tabularnewline
	234 & Prottes & Testudo & Testudo sp. & 245.00 & m & Tortonian & 8.30000
	& n & Europe & Bachmayer F., M?ynarski M., 1985: Die Landschildkröten
	(Testudinidae) aus den Schotter-Ablagerungen (Pontien) von Prottes,
	Niederösterreich.. Annalen des Naturhistorischen Museums in Wien 87 A:
	65-77\tabularnewline
	235 & Prottes & Testudo & Testudo sp. & 245.00 & m & Tortonian & 8.30000
	& n & Europe & Bachmayer F., M?ynarski M., 1985: Die Landschildkröten
	(Testudinidae) aus den Schotter-Ablagerungen (Pontien) von Prottes,
	Niederösterreich.. Annalen des Naturhistorischen Museums in Wien 87 A:
	65-77\tabularnewline
	236 & Prottes & Testudo & Testudo sp. & 245.00 & m & Tortonian & 8.30000
	& n & Europe & Bachmayer F., M?ynarski M., 1985: Die Landschildkröten
	(Testudinidae) aus den Schotter-Ablagerungen (Pontien) von Prottes,
	Niederösterreich.. Annalen des Naturhistorischen Museums in Wien 87 A:
	65-77\tabularnewline
	237 & Prottes & Testudo & Testudo sp. & 245.00 & m & Tortonian & 8.30000
	& n & Europe & Bachmayer F., M?ynarski M., 1985: Die Landschildkröten
	(Testudinidae) aus den Schotter-Ablagerungen (Pontien) von Prottes,
	Niederösterreich.. Annalen des Naturhistorischen Museums in Wien 87 A:
	65-77\tabularnewline
	238 & Prottes & Testudo & Testudo sp. & 245.00 & m & Tortonian & 8.30000
	& n & Europe & Bachmayer F., M?ynarski M., 1985: Die Landschildkröten
	(Testudinidae) aus den Schotter-Ablagerungen (Pontien) von Prottes,
	Niederösterreich.. Annalen des Naturhistorischen Museums in Wien 87 A:
	65-77\tabularnewline
	239 & Prottes & Testudo & Testudo sp. & 245.00 & m & Tortonian & 8.30000
	& n & Europe & Bachmayer F., M?ynarski M., 1985: Die Landschildkröten
	(Testudinidae) aus den Schotter-Ablagerungen (Pontien) von Prottes,
	Niederösterreich.. Annalen des Naturhistorischen Museums in Wien 87 A:
	65-77\tabularnewline
	240 & Prottes & Testudo & Testudo sp. & 245.00 & m & Tortonian & 8.30000
	& n & Europe & Bachmayer F., M?ynarski M., 1985: Die Landschildkröten
	(Testudinidae) aus den Schotter-Ablagerungen (Pontien) von Prottes,
	Niederösterreich.. Annalen des Naturhistorischen Museums in Wien 87 A:
	65-77\tabularnewline
	241 & Prottes & Testudo & Testudo sp. & 245.00 & m & Tortonian & 8.30000
	& n & Europe & Bachmayer F., M?ynarski M., 1985: Die Landschildkröten
	(Testudinidae) aus den Schotter-Ablagerungen (Pontien) von Prottes,
	Niederösterreich.. Annalen des Naturhistorischen Museums in Wien 87 A:
	65-77\tabularnewline
	242 & Prottes & Testudo & Testudo sp. & 245.00 & m & Tortonian & 8.30000
	& n & Europe & Bachmayer F., M?ynarski M., 1985: Die Landschildkröten
	(Testudinidae) aus den Schotter-Ablagerungen (Pontien) von Prottes,
	Niederösterreich.. Annalen des Naturhistorischen Museums in Wien 87 A:
	65-77\tabularnewline
	243 & Prottes & Testudo & Testudo sp. & 245.00 & m & Tortonian & 8.30000
	& n & Europe & Bachmayer F., M?ynarski M., 1985: Die Landschildkröten
	(Testudinidae) aus den Schotter-Ablagerungen (Pontien) von Prottes,
	Niederösterreich.. Annalen des Naturhistorischen Museums in Wien 87 A:
	65-77\tabularnewline
	244 & Prottes & Testudo & Testudo sp. & 245.00 & m & Tortonian & 8.30000
	& n & Europe & Bachmayer F., M?ynarski M., 1985: Die Landschildkröten
	(Testudinidae) aus den Schotter-Ablagerungen (Pontien) von Prottes,
	Niederösterreich.. Annalen des Naturhistorischen Museums in Wien 87 A:
	65-77\tabularnewline
	245 & Megalo Emvolon 1 (MEV), 20 km SW Thessaloniki & Testudo & Testudo
	sp. & 2500.00 & mf & Zanclean & 3.90000 & n & Europe & Bachmayer F.,
	M?ynarski M., Symeonidis N., 1980: Fossile Schiildkröten aus dem Pliozän
	von Megalo Emvolo (Karaburun) bei Saloniki (Griechenland) A. Eine
	fossile Maurische Landschildkröte (Testudo cf.~graeca LINNE) B. Fossile
	Reste von Riesenschildkröten. Annales Géologiques des Pays Hellénique
	31: 267-276\tabularnewline
	246 & Megalo Emvolon 1 (MEV), 20 km SW Thessaloniki & Testudo & Testudo
	sp. & 2500.00 & mf & Zanclean & 3.90000 & n & Europe & Bachmayer F.,
	M?ynarski M., Symeonidis N., 1980: Fossile Schiildkröten aus dem Pliozän
	von Megalo Emvolo (Karaburun) bei Saloniki (Griechenland) A. Eine
	fossile Maurische Landschildkröte (Testudo cf.~graeca LINNE) B. Fossile
	Reste von Riesenschildkröten. Annales Géologiques des Pays Hellénique
	31: 267-276\tabularnewline
	247 & Megalo Emvolon 1 (MEV), 20 km SW Thessaloniki & Testudo & Testudo
	sp. & 2500.00 & mf & Zanclean & 3.90000 & n & Europe & Bachmayer F.,
	M?ynarski M., Symeonidis N., 1980: Fossile Schiildkröten aus dem Pliozän
	von Megalo Emvolo (Karaburun) bei Saloniki (Griechenland) A. Eine
	fossile Maurische Landschildkröte (Testudo cf.~graeca LINNE) B. Fossile
	Reste von Riesenschildkröten. Annales Géologiques des Pays Hellénique
	31: 267-276\tabularnewline
	248 & Megalo Emvolon 1 (MEV), 20 km SW Thessaloniki & Testudo & Testudo
	sp. & 2500.00 & mf & Zanclean & 3.90000 & n & Europe & Bachmayer F.,
	M?ynarski M., Symeonidis N., 1980: Fossile Schiildkröten aus dem Pliozän
	von Megalo Emvolo (Karaburun) bei Saloniki (Griechenland) A. Eine
	fossile Maurische Landschildkröte (Testudo cf.~graeca LINNE) B. Fossile
	Reste von Riesenschildkröten. Annales Géologiques des Pays Hellénique
	31: 267-276\tabularnewline
	249 & W??e 1 & Testudo & Testudo sp. & 500.00 & mo & Zanclean & 3.90000
	& n & Europe & M?ynarski M., 1955: Zolwie z pliocenu Polski {[}Tortoises
	from the Pliocene of Poland{]}. Acta Geologica Polonica 5: 161--214
	(Engl. Consp., 46--62) or M?ynarski M., 1956a: On a new species of
	emydid-tortoise from the Pliocene of Poland. Acta Palaeontologica
	Polonica 1(2): 153-164 or Lapparent de Broin F. de, Bour R., Perälä J.,
	2006: Morphological definition of Eurotestudo (Testudinidae, Chelonii):
	First part. Annales de Paléontologie 92(3): 255-304 or Lapparent de
	Broin F. de, Bour R., Perälä J., 2006a: Morphological definition of
	Eurotestudo (Testudinidae, Chelonii): second part. Annales de
	Paléontologie 92(4): 325-357\tabularnewline
	250 & Altenstadt, 7 km S Illertissen & Testudo & Testudo steinheimensis
	& 111.00 & m & Serravallian & 12.15000 & n & Europe & Staesche K., 1931:
	Die Schildkröten des Steinheimer Beckens. A. Testudinidae.
	Palaeontographica, suppl. Bd.8(2): 1-17\tabularnewline
	251 & Steinheim a. Albuch & Testudo & Testudo steinheimensis & 227.70 &
	mf & Serravallian & 13.00000 & n & Europe & Schleich H.H., 1981:
	Jungtertiäre Schildkröten Süddeutschlands unter besonderer
	Berücksichtigung der Fundstelle Sandelzhausen. Courier
	Forschungsinstitut Senckenberg 48: 372pp., Frankfurt\tabularnewline
	252 & Lesbos Island, F-Site & Titanochelon & Titanochelon aff. schafferi
	& 1860.00 & m & Gelasian & 2.00000 & y & Europe & Lapparent de Broin
	F.de, 2002a: A giant tortoise from the Late Pliocene of Lesvos Island
	(Greece) and its possible relationships. Annales Geologiques des Pays
	Helleniques, 1e Serie, t.XXXIX, fasc. A: 99-130\tabularnewline
	253 & Epanomi (EPN II), western Chalkidiki Peninsula, Thessaloniki area
	& Titanochelon & Titanochelon bacharidisi & 1164.00 & m & Zanclean &
	3.95000 & n & Europe & Vlachos E., Tsoukala E., Corsini J., 2014:
	Cheirogaster bacharidisi, sp. nov., a new species of a giant tortoise
	from the Pliocene of Thessaloniki (Macedonia, Greece). Journal of
	Vertebrate Paleontology 34(3): 560-575\tabularnewline
	254 & Epanomi (EPN I), western Chalkidiki Peninsula, Thessaloniki area &
	Titanochelon & Titanochelon bacharidisi & 1196.00 & m & Zanclean &
	3.95000 & n & Europe & Vlachos E., Tsoukala E., Corsini J., 2014:
	Cheirogaster bacharidisi, sp. nov., a new species of a giant tortoise
	from the Pliocene of Thessaloniki (Macedonia, Greece). Journal of
	Vertebrate Paleontology 34(3): 560-575\tabularnewline
	255 & Nea Michaniona, western Chalkidiki Peninsula, Thessaloniki area &
	Titanochelon & Titanochelon bacharidisi & 900.00 & mo & Zanclean &
	3.95000 & n & Europe & Vlachos E., Tsoukala E., Corsini J., 2014:
	Cheirogaster bacharidisi, sp. nov., a new species of a giant tortoise
	from the Pliocene of Thessaloniki (Macedonia, Greece). Journal of
	Vertebrate Paleontology 34(3): 560-575\tabularnewline
	256 & Nea Kallikratia, western Chalkidiki Peninsula, Thessaloniki area &
	Titanochelon & Titanochelon bacharidisi & 900.00 & mo & Zanclean &
	3.95000 & n & Europe & Vlachos E., Tsoukala E., Corsini J., 2014:
	Cheirogaster bacharidisi, sp. nov., a new species of a giant tortoise
	from the Pliocene of Thessaloniki (Macedonia, Greece). Journal of
	Vertebrate Paleontology 34(3): 560-575\tabularnewline
	257 & Nea Michaniona, western Chalkidiki Peninsula, Thessaloniki area &
	Titanochelon & Titanochelon bacharidisi & 900.00 & mo & Zanclean &
	3.95000 & n & Europe & Vlachos E., Tsoukala E., Corsini J., 2014:
	Cheirogaster bacharidisi, sp. nov., a new species of a giant tortoise
	from the Pliocene of Thessaloniki (Macedonia, Greece). Journal of
	Vertebrate Paleontology 34(3): 560-575\tabularnewline
	258 & Nea Kallikratia, western Chalkidiki Peninsula, Thessaloniki area &
	Titanochelon & Titanochelon bacharidisi & 900.00 & mo & Zanclean &
	3.95000 & n & Europe & Vlachos E., Tsoukala E., Corsini J., 2014:
	Cheirogaster bacharidisi, sp. nov., a new species of a giant tortoise
	from the Pliocene of Thessaloniki (Macedonia, Greece). Journal of
	Vertebrate Paleontology 34(3): 560-575\tabularnewline
	259 & Vallecas, Madrid & Titanochelon & Titanochelon bolivari & 1100.00
	& mo & Langhian & 15.00000 & n & Europe & Pérez-García A., Vlachos E.,
	2014: New generic proposal for the European Neogene large testudinids
	(Cryptodira) and the ?rst phylogenetic hypothesis for the medium and
	large representatives of the European Cenozoic record. Zoological
	Journal of the Linnean Society 172: 653-719\tabularnewline
	260 & Puerto de la Cadena, Murcia & Titanochelon & Titanochelon bolivari
	& 1150.00 & m & Messinian & 6.28900 & n & Europe & Jiménez et al.,
	2001\tabularnewline
	261 & Alcalá de Henares, Cerro del Viso, Madrid & Titanochelon & Titanochelon bolivari &
	1250.00 & mo & Langhian & 15.00000 & n & Europe & Pérez-García A.,
	Vlachos E., 2014: New generic proposal for the European Neogene large
	testudinids (Cryptodira) and the ?rst phylogenetic hypothesis for the
	medium and large representatives of the European Cenozoic record.
	Zoological Journal of the Linnean Society 172: 653-719\tabularnewline
	262 & Cerro de los Batallones, Madrid & Titanochelon & Titanochelon
	bolivari & 1300.00 & mf & Tortonian & 9.50000 & n & Europe &
	Pérez-García and Murelaga, 2013\tabularnewline
	263 & Cerro del Otero, Palencia & Titanochelon & Titanochelon bolivari &
	1353.00 & mo & Serravallian & 12.50000 & n & Europe & Pérez-García A.,
	Vlachos E., 2014: New generic proposal for the European Neogene large
	testudinids (Cryptodira) and the ?rst phylogenetic hypothesis for the
	medium and large representatives of the European Cenozoic record.
	Zoological Journal of the Linnean Society 172: 653-719\tabularnewline
	264 & Charneco do Lumiar & Titanochelon & Titanochelon cf.~bolivari &
	1300.00 & ev & Langhian & 14.89500 & n & Europe & Pérez-García et al.,
	2016: Westernmost records of extinct large tortoises\tabularnewline
	265 & Aveiras de Baixo, Azambuja & Titanochelon & Titanochelon
	cf.~bolivari & 1500.00 & mf & Tortonian & 9.43300 & n & Europe &
	Pérez-García et al., 2016: Westernmost records of extinct large
	tortoises\tabularnewline
	266 & Quinta da Farinheira & Titanochelon & Titanochelon cf.~bolivari &
	1600.00 & ef & Langhian & 14.89500 & n & Europe & Pérez-García et al.,
	2016: Westernmost records of extinct large tortoises\tabularnewline
	267 & Sandelzhausen unterer Geröllmergel (B) & Titanochelon &
	Titanochelon cf.~perpiniana & 1001.00 & mo & Burdigalian/Aquitanian &
	16.37000 & n & Europe & Schleich H.H., 1981: Jungtertiäre Schildkröten
	Süddeutschlands unter besonderer Berücksichtigung der Fundstelle
	Sandelzhausen. Courier Forschungsinstitut Senckenberg 48: 372pp.,
	Frankfurt\tabularnewline
	268 & Cala Es Pous near Ciutadella, Minorca & Titanochelon &
	Titanochelon gymnesica & 1300.00 & ef & Lower Pleistocene & 1.30000 & y
	& Europe & Bate, D. M. (1914). II.---On Remains of a Gigantic Land
	Tortoise (Testudo Gymnesious, N. Sp.) from the Pleistocene of
	Menorca.~Geological Magazine,~1(3), 100-107.\tabularnewline
	269 & Serrat-d'en-Vacquer near Perpignan, Pyrénées-Orientales &
	Titanochelon & Titanochelon perpiniana & 1140.00 & m & Zanclean &
	3.90000 & n & Europe & Lapparent de Broin F.de, 2002a: A giant tortoise
	from the Late Pliocene of Lesvos Island (Greece) and its possible
	relationships. Annales Geologiques des Pays Helleniques, 1e Serie,
	t.XXXIX, fasc. A: 99-130\tabularnewline
	270 & Samos 1 & Titanochelon & Titanochelon schafferi & 1850.00 & m &
	Messinian & 6.25000 & y & Europe & Lapparent de Broin F.de, 2002a: A
	giant tortoise from the Late Pliocene of Lesvos Island (Greece) and its
	possible relationships. Annales Geologiques des Pays Helleniques, 1e
	Serie, t.XXXIX, fasc. A: 99-130\tabularnewline
	271 & Pikermi & Titanochelon & Titanochelon schafferi & 2500.00 & mo &
	Zanclean & 4.46600 & n & Europe & Bachmayer, F. (1967).~Eine
	Riesenschildkröte aus den altpliozänen Schichten von Pikermi,
	Griechenland.\tabularnewline
	272 & Fonelas P-1, Guadix Basin & Titanochelon & Titanochelon sp. &
	1420.00 & mo & Gelasian & 1.85000 & n & Europe & Pérez-García, A.,
	Vlachos, E., \& Arribas, A. (2017). The last giant continental tortoise
	of Europe: A survivor in the Spanish Pleistocene site of Fonelas P-1.
	Palaeogeography, Palaeoclimatology, Palaeoecology.\tabularnewline
	273 & Cova de Ca Na Reia, Eivissa, Ibiza & Titanochelon & Titanochelon
	sp. & 520.00 & mo & Piacencian & 2.60000 & y & Europe & Bour R., 1985:
	Una nova tortuga terrestre del Pleistocè d'Eivissa: la tortuga de la
	Cova de Ca Na Reia. Endins 10-11: 57-62\tabularnewline
	274 & Plum Point, Calvert County, Maryland & Caudochelys & Caudochelys
	ducateli & 339.90 & m & Langhian & 15.00000 & n & America & Collins \&
	Lynns, 1936\tabularnewline
	275 & Rexroad local fauna (Fox Canyon locality 3), Meade County, Kansas
	& Caudochelys & Caudochelys rexroadensis & 781.00 & m & Zanclean &
	4.55000 & n & America & Oelrich T.M., 1952: A New Testudo from the Upper
	Pliocene of Kansas with Additional Notes on Associated Rexroad Mammals.
	Transactions of the Kansas Academy of Science 55(3):
	300-311\tabularnewline
	276 & Rexroad local fauna (Fox Canyon locality 3), Meade County, Kansas
	& Caudochelys & Caudochelys rexroadensis & 830.00 & m & Zanclean &
	4.55000 & n & America & Oelrich T.M., 1952: A New Testudo from the Upper
	Pliocene of Kansas with Additional Notes on Associated Rexroad Mammals.
	Transactions of the Kansas Academy of Science 55(3):
	300-311\tabularnewline
	277 & Garvin Gullv, 2 mi. north of Navasota, Jl J . Grimes County,
	Texas & Caudochelys & Caudochelys williamsi &
	334.00 & m & Burdigalian/Aquitanian & 17.75000 & n & America &
	Auffenberg, 1964\tabularnewline
	278 & Gilliland local fauna, Burnett Ranch, Knox
	County, Texas & Geochelone & Geochelone sp. & 170.00 & mf & Middle
	Pleistocene & 0.70000 & n & America & Preston, 1966\tabularnewline
	279 & Santee, Knox County, Nebraska & Geochelone & Geochelone sp. &
	176.00 & e & Zanclean & 5.00000 & n & America & Parmley D., 1992a:
	Turtles from the late Hemphillian (latest Miocene) of Knox County,
	Nebraska. The Texas Journal of Science 44(3): 339--348 or Parmley D.,
	Holman J.A., 1995: Hemphillian (late Miocene) snakes from Nebraska, with
	comments on Arikareean through Blancan snakes of midcontinental North
	America. Journal of Vertebrate Paleontology 15: 79-95\tabularnewline
	280 & Orange Lake 2 miles south, Marion County, Florida & Geochelone &
	Geochelone sp. & 350.00 & ef & Upper Pleistocene & 0.06900 & n & America
	& Holman J.A., 1959b: A Pleistocene herpetofauna near Orange Lake,
	Florida. Herpetologica 15(3): 121-125\tabularnewline
	281 & Ricardo Fauna, Mojave Desert, Kern County, California & Geochelone
	& Geochelone sp. & 500.00 & m & Tortonian & 10.10000 & n & America &
	Whistler D.P., Tedford R.H., Takeuchi G.T., Wang X., Tseng Z., Perkins
	M.E., 2009: Revised Miocene biostratigraphy and biochronology of the
	Dove Spring Formation, Mojave Desert, California. in: Albright L.B.
	(ed.) Papers on Geology, Vertebrate Paleontology, and Biostratigraphy in
	Honor of Michael O. Woodburne. Museum of Northern Arizona Bulletin 65:
	331-362\tabularnewline
	282 & Banana Hole, New Providence Island & Geochelone & Geochelone sp. &
	600.00 & mo & Upper Pleistocene & 0.01250 & y & America & Olson,
	1982\tabularnewline
	283 & Lee Creek Mine, Yorktown Sample, Beaufort County, North Carolina &
	Geochelone & Geochelone sp. & 880.00 & m & Zanclean & 4.50000 & n &
	America & Zug G.R., 2001: Turtles of the Lee Creek Mine (Pliocene: North
	Carolina). Smithsonian Contributions to Paleobiology 90: 203-218 or Ray
	C.E., Bohaska D.J., 2001: Geology and Paleontology of the Lee Creek
	Mine, North Carolina, III. Smithsonian Contributions to Paleobiology 90,
	365 pages, 127 figures, 45 plates, 32 tables\tabularnewline
	284 & Thomas Farm Local Fauna, Gilchrist County, Florida & Geochelone &
	Geochelone tedwhitei & 370.00 & m & Burdigalian/Aquitanian & 18.50000 &
	n & America & Williams E., 1953: A new fossil tortoise from the Thomas
	Farm Miocene of Florida. Bulletin of the Museum of Comparative Zoology
	107(11): 537-552 or Williams M.J., 2009: Miocene herpetofaunas from the
	central Gulf Coast USA: their paleoecology, biogeography, and
	biostratigraphy. Thesis Louisiana State University and Agricultural and
	Mechanical College, Department of Geology and Geophysics,
	152pp\tabularnewline
	285 & Thomas Farm Local Fauna, Gilchrist County, Florida & Geochelone &
	Geochelone tedwhitei & 440.00 & m & Burdigalian/Aquitanian & 18.50000 &
	n & America & Williams E., 1953: A new fossil tortoise from the Thomas
	Farm Miocene of Florida. Bulletin of the Museum of Comparative Zoology
	107(11): 537-552 or Williams M.J., 2009: Miocene herpetofaunas from the
	central Gulf Coast USA: their paleoecology, biogeography, and
	biostratigraphy. Thesis Louisiana State University and Agricultural and
	Mechanical College, Department of Geology and Geophysics,
	152pp\tabularnewline
	286 & Ricardo Fauna, Mojave Desert, Kern County, California & Gopherus &
	Gopherus ? sp. & 500.00 & m & Tortonian & 10.10000 & n & America &
	Whistler D.P., Tedford R.H., Takeuchi G.T., Wang X., Tseng Z., Perkins
	M.E., 2009: Revised Miocene biostratigraphy and biochronology of the
	Dove Spring Formation, Mojave Desert, California. in: Albright L.B.
	(ed.) Papers on Geology, Vertebrate Paleontology, and Biostratigraphy in
	Honor of Michael O. Woodburne. Museum of Northern Arizona Bulletin 65:
	331-362\tabularnewline
	287 & Iron Canyon Fauna, Mojave Desert, Kern County, California &
	Gopherus & Gopherus ? sp. & 500.00 & m & Serravallian & 11.85000 & n &
	America & Whistler D.P., Tedford R.H., Takeuchi G.T., Wang X., Tseng Z.,
	Perkins M.E., 2009: Revised Miocene biostratigraphy and biochronology of
	the Dove Spring Formation, Mojave Desert, California. in: Albright L.B.
	(ed.) Papers on Geology, Vertebrate Paleontology, and Biostratigraphy in
	Honor of Michael O. Woodburne. Museum of Northern Arizona Bulletin 65:
	331-362\tabularnewline
	288 & Sabertooth Camel Maze, Eddy County, New Mexico
	& Gopherus & Gopherus agassizi & 252.00 & m & Upper Pleistocene &
	0.02550 & n & America & Van Devender T.R., Moodie K.B., Harris A.H.,
	1976: The desert tortoise (Gopherus agassizi) in the Pleistocene of the
	northern Chihuahuan Desert. Herpetologica 32: 298-304\tabularnewline
	289 & Pecos River near Melena and Acme, Chaves
	County, New Mexico & Gopherus & Gopherus agassizi & 445.00 & mo & Middle
	Pleistocene & 0.15600 & n & America & Lucas and Morgan,
	1996\tabularnewline
	290 & North Cita Canyon (Middle Stratum), Randall County, Texas &
	Gopherus & Gopherus canyonensis & 885.50 & m & Piacencian & 2.70000 & n
	& America & Johnston C.S., 1937: Osteology of Bysmachelys canyonensis a
	new turtle from the Pliocene of Texas. Journal of Paleontology 45(4):
	439-447\tabularnewline
	291 & Texas & Gopherus & Gopherus laticaudatus & 375.00 & mo & Middle
	Pleistocene & 0.39635 & n & America & Rhodin et al., 2015\tabularnewline
	292 & Barstow Beds, San Bernardino County, California & Gopherus &
	Gopherus mohavetus & 202.00 & m & Tortonian & 8.47600 & n & America &
	Brattstrom, 1961\tabularnewline
	293 & Cache Peak fauna, Tehachapi Mountains, Kern County, California &
	Gopherus & Gopherus mohavetus & 315.00 & m & Tortonian & 8.47600 & n &
	America & Brattstrom, 1961\tabularnewline
	294 & Barstow Beds, San Bernardino County, California & Gopherus &
	Gopherus mohavetus & 334.50 & m & Tortonian & 8.47600 & n & America &
	Brattstrom, 1961\tabularnewline
	295 & Barstow Beds, San Bernardino County, California & Gopherus &
	Gopherus mohavetus & 360.00 & m & Tortonian & 8.47600 & n & America &
	Brattstrom, 1961\tabularnewline
	296 & Barstow Beds, San Bernardino County, California & Gopherus &
	Gopherus mohavetus & 412.50 & m & Tortonian & 8.47600 & n & America &
	Brattstrom, 1961\tabularnewline
	297 & Texas & Gopherus & Gopherus pertenuis & 1050.00 & mo & Lower
	Pleistocene & 1.68450 & n & America & Rhodin et al., 2015\tabularnewline
	298 & Surprise Cave, Alachua, Florida & Gopherus & Gopherus polyphemus &
	102.44 & mo & Upper Pleistocene & 0.06900 & n & America & Franz and
	Quitmyer, 2005\tabularnewline
	299 & Reddick IA+B, Marion County, Florida & Gopherus & Gopherus
	polyphemus & 155.50 & mo & Upper Pleistocene & 0.06900 & n & America &
	Franz and Quitmyer, 2005\tabularnewline
	300 & Leisey Shell Pit 1A, Hillsborough County, Florida & Gopherus &
	Gopherus polyphemus & 217.90 & mo & Lower Pleistocene & 1.20000 & n &
	America & Franz and Quitmyer, 2005\tabularnewline
	301 & Haile, Alachua County, Florida & Gopherus & Gopherus polyphemus &
	239.80 & mo & Middle Pleistocene & 0.25000 & n & America & Franz and
	Quitmyer, 2005\tabularnewline
	302 & Surprise Cave, Alachua, Florida & Gopherus & Gopherus polyphemus &
	252.56 & mo & Upper Pleistocene & 0.06900 & n & America & Franz and
	Quitmyer, 2005\tabularnewline
	303 & Haile, Alachua County, Florida & Gopherus & Gopherus polyphemus &
	253.70 & mo & Middle Pleistocene & 0.25000 & n & America & Franz and
	Quitmyer, 2005\tabularnewline
	304 & Haile, Alachua County, Florida & Gopherus & Gopherus polyphemus &
	256.44 & mo & Middle Pleistocene & 0.25000 & n & America & Franz and
	Quitmyer, 2005\tabularnewline
	305 & Haile, Alachua County, Florida & Gopherus & Gopherus polyphemus &
	257.80 & mo & Middle Pleistocene & 0.25000 & n & America & Franz and
	Quitmyer, 2005\tabularnewline
	306 & Surprise Cave, Alachua, Florida & Gopherus & Gopherus polyphemus &
	258.30 & mo & Upper Pleistocene & 0.06900 & n & America & Franz and
	Quitmyer, 2005\tabularnewline
	307 & Surprise Cave, Alachua, Florida & Gopherus & Gopherus polyphemus &
	260.11 & mo & Upper Pleistocene & 0.06900 & n & America & Franz and
	Quitmyer, 2005\tabularnewline
	308 & Coleman 2A & Gopherus & Gopherus polyphemus & 260.50 & mo & Middle
	Pleistocene & 0.40000 & n & America & Franz and Quitmyer,
	2005\tabularnewline
	309 & Coleman 2A & Gopherus & Gopherus polyphemus & 260.51 & mo & Middle
	Pleistocene & 0.40000 & n & America & Franz and Quitmyer,
	2005\tabularnewline
	310 & Haile, Alachua County, Florida & Gopherus & Gopherus polyphemus &
	267.00 & mo & Middle Pleistocene & 0.25000 & n & America & Franz and
	Quitmyer, 2005\tabularnewline
	311 & Leisey Shell Pit 1A, Hillsborough County, Florida & Gopherus &
	Gopherus polyphemus & 268.90 & mo & Lower Pleistocene & 1.20000 & n &
	America & Franz and Quitmyer, 2005\tabularnewline
	312 & Haile, Alachua County, Florida & Gopherus & Gopherus polyphemus &
	272.48 & mo & Middle Pleistocene & 0.25000 & n & America & Franz and
	Quitmyer, 2005\tabularnewline
	313 & Coleman 2A & Gopherus & Gopherus polyphemus & 272.57 & mo & Middle
	Pleistocene & 0.40000 & n & America & Franz and Quitmyer,
	2005\tabularnewline
	314 & Surprise Cave, Alachua, Florida & Gopherus & Gopherus polyphemus &
	273.24 & mo & Upper Pleistocene & 0.06900 & n & America & Franz and
	Quitmyer, 2005\tabularnewline
	315 & Haile, Alachua County, Florida & Gopherus & Gopherus polyphemus &
	274.30 & mo & Middle Pleistocene & 0.25000 & n & America & Franz and
	Quitmyer, 2005\tabularnewline
	316 & Leisey Shell Pit 1A, Hillsborough County, Florida & Gopherus &
	Gopherus polyphemus & 276.60 & mo & Lower Pleistocene & 1.20000 & n &
	America & Franz and Quitmyer, 2005\tabularnewline
	317 & Surprise Cave, Alachua, Florida & Gopherus & Gopherus polyphemus &
	278.00 & mo & Upper Pleistocene & 0.06900 & n & America & Franz and
	Quitmyer, 2005\tabularnewline
	318 & Surprise Cave, Alachua, Florida & Gopherus & Gopherus polyphemus &
	279.94 & mo & Upper Pleistocene & 0.06900 & n & America & Franz and
	Quitmyer, 2005\tabularnewline
	319 & Haile, Alachua County, Florida & Gopherus & Gopherus polyphemus &
	283.00 & mo & Middle Pleistocene & 0.25000 & n & America & Franz and
	Quitmyer, 2005\tabularnewline
	320 & Haile, Alachua County, Florida & Gopherus & Gopherus polyphemus &
	283.41 & mo & Middle Pleistocene & 0.25000 & n & America & Franz and
	Quitmyer, 2005\tabularnewline
	321 & Surprise Cave, Alachua, Florida & Gopherus & Gopherus polyphemus &
	284.90 & mo & Upper Pleistocene & 0.06900 & n & America & Franz and
	Quitmyer, 2005\tabularnewline
	322 & Haile, Alachua County, Florida & Gopherus & Gopherus polyphemus &
	285.20 & mo & Middle Pleistocene & 0.25000 & n & America & Franz and
	Quitmyer, 2005\tabularnewline
	323 & Coleman 2A & Gopherus & Gopherus polyphemus & 285.60 & mo & Middle
	Pleistocene & 0.40000 & n & America & Franz and Quitmyer,
	2005\tabularnewline
	324 & Haile, Alachua County, Florida & Gopherus & Gopherus polyphemus &
	292.00 & mo & Middle Pleistocene & 0.25000 & n & America & Franz and
	Quitmyer, 2005\tabularnewline
	325 & Haile, Alachua County, Florida & Gopherus & Gopherus polyphemus &
	292.94 & mo & Middle Pleistocene & 0.25000 & n & America & Franz and
	Quitmyer, 2005\tabularnewline
	326 & Coleman 2A & Gopherus & Gopherus polyphemus & 293.00 & mo & Middle
	Pleistocene & 0.40000 & n & America & Franz and Quitmyer,
	2005\tabularnewline
	327 & Coleman 2A & Gopherus & Gopherus polyphemus & 293.57 & mo & Middle
	Pleistocene & 0.40000 & n & America & Franz and Quitmyer,
	2005\tabularnewline
	328 & Surprise Cave, Alachua, Florida & Gopherus & Gopherus polyphemus &
	294.16 & mo & Upper Pleistocene & 0.06900 & n & America & Franz and
	Quitmyer, 2005\tabularnewline
	329 & Coleman 2A & Gopherus & Gopherus polyphemus & 295.90 & mo & Middle
	Pleistocene & 0.40000 & n & America & Franz and Quitmyer,
	2005\tabularnewline
	330 & Surprise Cave, Alachua, Florida & Gopherus & Gopherus polyphemus &
	301.97 & mo & Upper Pleistocene & 0.06900 & n & America & Franz and
	Quitmyer, 2005\tabularnewline
	331 & Surprise Cave, Alachua, Florida & Gopherus & Gopherus polyphemus &
	302.40 & mo & Upper Pleistocene & 0.06900 & n & America & Franz and
	Quitmyer, 2005\tabularnewline
	332 & Haile, Alachua County, Florida & Gopherus & Gopherus polyphemus &
	302.40 & mo & Middle Pleistocene & 0.25000 & n & America & Franz and
	Quitmyer, 2005\tabularnewline
	333 & Surprise Cave, Alachua, Florida & Gopherus & Gopherus polyphemus &
	304.20 & mo & Upper Pleistocene & 0.06900 & n & America & Franz and
	Quitmyer, 2005\tabularnewline
	334 & Coleman 2A & Gopherus & Gopherus polyphemus & 304.70 & mo & Middle
	Pleistocene & 0.40000 & n & America & Franz and Quitmyer,
	2005\tabularnewline
	335 & Haile, Alachua County, Florida & Gopherus & Gopherus polyphemus &
	306.00 & mo & Middle Pleistocene & 0.25000 & n & America & Franz and
	Quitmyer, 2005\tabularnewline
	336 & Haile, Alachua County, Florida & Gopherus & Gopherus polyphemus &
	306.00 & mo & Middle Pleistocene & 0.25000 & n & America & Franz and
	Quitmyer, 2005\tabularnewline
	337 & Haile, Alachua County, Florida & Gopherus & Gopherus polyphemus &
	306.00 & mo & Middle Pleistocene & 0.25000 & n & America & Franz and
	Quitmyer, 2005\tabularnewline
	338 & Haile, Alachua County, Florida & Gopherus & Gopherus polyphemus &
	306.00 & mo & Middle Pleistocene & 0.25000 & n & America & Franz and
	Quitmyer, 2005\tabularnewline
	339 & Coleman 2A & Gopherus & Gopherus polyphemus & 308.20 & mo & Middle
	Pleistocene & 0.40000 & n & America & Franz and Quitmyer,
	2005\tabularnewline
	340 & Haile, Alachua County, Florida & Gopherus & Gopherus polyphemus &
	314.60 & mo & Middle Pleistocene & 0.25000 & n & America & Franz and
	Quitmyer, 2005\tabularnewline
	341 & Haile, Alachua County, Florida & Gopherus & Gopherus polyphemus &
	322.63 & mo & Middle Pleistocene & 0.25000 & n & America & Franz and
	Quitmyer, 2005\tabularnewline
	342 & Surprise Cave, Alachua, Florida & Gopherus & Gopherus polyphemus &
	324.00 & mo & Upper Pleistocene & 0.06900 & n & America & Franz and
	Quitmyer, 2005\tabularnewline
	343 & Reddick IA+B, Marion County, Florida & Gopherus & Gopherus
	polyphemus & 327.60 & mo & Upper Pleistocene & 0.06900 & n & America &
	Franz and Quitmyer, 2005\tabularnewline
	344 & Surprise Cave, Alachua, Florida & Gopherus & Gopherus polyphemus &
	334.70 & mo & Upper Pleistocene & 0.06900 & n & America & Franz and
	Quitmyer, 2005\tabularnewline
	345 & Haile, Alachua County, Florida & Gopherus & Gopherus polyphemus &
	337.30 & mo & Middle Pleistocene & 0.25000 & n & America & Franz and
	Quitmyer, 2005\tabularnewline
	346 & Coleman 2A & Gopherus & Gopherus polyphemus & 348.70 & mo & Middle
	Pleistocene & 0.40000 & n & America & Franz and Quitmyer,
	2005\tabularnewline
	347 & Surprise Cave, Alachua, Florida & Gopherus & Gopherus polyphemus &
	350.00 & mo & Upper Pleistocene & 0.06900 & n & America & Franz and
	Quitmyer, 2005\tabularnewline
	348 & Coleman 2A & Gopherus & Gopherus polyphemus & 350.83 & mo & Middle
	Pleistocene & 0.40000 & n & America & Franz and Quitmyer,
	2005\tabularnewline
	349 & Little Salt Spring, Florida & Gopherus & Gopherus polyphemus &
	352.00 & mo & Upper Pleistocene & 0.01200 & n & America & Holman \&
	Clausenm, 1984\tabularnewline
	350 & Coleman 2A & Gopherus & Gopherus polyphemus & 353.30 & mo & Middle
	Pleistocene & 0.40000 & n & America & Franz and Quitmyer,
	2005\tabularnewline
	351 & Reddick IA+B, Marion County, Florida & Gopherus & Gopherus
	polyphemus & 391.90 & mo & Upper Pleistocene & 0.06900 & n & America &
	Franz and Quitmyer, 2005\tabularnewline
	352 & Surprise Cave, Alachua, Florida & Gopherus & Gopherus polyphemus &
	431.48 & mo & Upper Pleistocene & 0.06900 & n & America & Franz and
	Quitmyer, 2005\tabularnewline
	353 & Gilliland local fauna, Burnett Ranch, Knox
	County, Texas & Gopherus & Gopherus polyphemus & 539.00 & mf & Middle
	Pleistocene & 0.70000 & n & America & Preston, 1966\tabularnewline
	354 & Melbourne, Brevard County, Florida & Gopherus & Gopherus
	praecedens & 360.00 & mo & Upper Pleistocene & 0.06900 & n & America &
	Franz and Quitmyer, 2005\tabularnewline
	355 & Inglis 1A, Florida & Gopherus & Gopherus sp. & 118.90 & mo &
	Gelasian & 1.90000 & n & America & Franz and Quitmyer,
	2005\tabularnewline
	356 & Inglis 1A, Florida & Gopherus & Gopherus sp. & 143.90 & mo &
	Gelasian & 1.90000 & n & America & Franz and Quitmyer,
	2005\tabularnewline
	357 & Inglis 1A, Florida & Gopherus & Gopherus sp. & 163.50 & mo &
	Gelasian & 1.90000 & n & America & Franz and Quitmyer,
	2005\tabularnewline
	358 & Inglis 1A, Florida & Gopherus & Gopherus sp. & 180.90 & mo &
	Gelasian & 1.90000 & n & America & Franz and Quitmyer,
	2005\tabularnewline
	359 & Inglis 1A, Florida & Gopherus & Gopherus sp. & 181.00 & mo &
	Gelasian & 1.90000 & n & America & Franz and Quitmyer,
	2005\tabularnewline
	360 & Inglis 1A, Florida & Gopherus & Gopherus sp. & 181.00 & mo &
	Gelasian & 1.90000 & n & America & Franz and Quitmyer,
	2005\tabularnewline
	361 & Inglis 1A, Florida & Gopherus & Gopherus sp. & 181.00 & mo &
	Gelasian & 1.90000 & n & America & Franz and Quitmyer,
	2005\tabularnewline
	362 & Inglis 1A, Florida & Gopherus & Gopherus sp. & 181.00 & mo &
	Gelasian & 1.90000 & n & America & Franz and Quitmyer,
	2005\tabularnewline
	363 & Inglis 1A, Florida & Gopherus & Gopherus sp. & 182.30 & mo &
	Gelasian & 1.90000 & n & America & Franz and Quitmyer,
	2005\tabularnewline
	364 & Inglis 1A, Florida & Gopherus & Gopherus sp. & 188.30 & mo &
	Gelasian & 1.90000 & n & America & Franz and Quitmyer,
	2005\tabularnewline
	365 & Inglis 1A, Florida & Gopherus & Gopherus sp. & 188.70 & mo &
	Gelasian & 1.90000 & n & America & Franz and Quitmyer,
	2005\tabularnewline
	366 & Inglis 1A, Florida & Gopherus & Gopherus sp. & 193.30 & mo &
	Gelasian & 1.90000 & n & America & Franz and Quitmyer,
	2005\tabularnewline
	367 & Inglis 1A, Florida & Gopherus & Gopherus sp. & 194.90 & mo &
	Gelasian & 1.90000 & n & America & Franz and Quitmyer,
	2005\tabularnewline
	368 & Inglis 1C, Florida & Gopherus & Gopherus sp. & 202.80 & mo & Lower
	Pleistocene & 1.80000 & n & America & Franz and Quitmyer,
	2005\tabularnewline
	369 & Inglis 1A, Florida & Gopherus & Gopherus sp. & 204.40 & mo &
	Gelasian & 1.90000 & n & America & Franz and Quitmyer,
	2005\tabularnewline
	370 & Inglis 1A, Florida & Gopherus & Gopherus sp. & 209.60 & mo &
	Gelasian & 1.90000 & n & America & Franz and Quitmyer,
	2005\tabularnewline
	371 & Inglis 1A, Florida & Gopherus & Gopherus sp. & 218.80 & mo &
	Gelasian & 1.90000 & n & America & Franz and Quitmyer,
	2005\tabularnewline
	372 & Inglis 1C, Florida & Gopherus & Gopherus sp. & 224.10 & mo & Lower
	Pleistocene & 1.80000 & n & America & Franz and Quitmyer,
	2005\tabularnewline
	373 & Inglis 1C, Florida & Gopherus & Gopherus sp. & 230.10 & mo & Lower
	Pleistocene & 1.80000 & n & America & Franz and Quitmyer,
	2005\tabularnewline
	374 & Inglis 1A, Florida & Gopherus & Gopherus sp. & 236.70 & mo &
	Gelasian & 1.90000 & n & America & Franz and Quitmyer,
	2005\tabularnewline
	375 & Inglis 1C, Florida & Gopherus & Gopherus sp. & 241.90 & mo & Lower
	Pleistocene & 1.80000 & n & America & Franz and Quitmyer,
	2005\tabularnewline
	376 & Inglis 1C, Florida & Gopherus & Gopherus sp. & 245.40 & mo & Lower
	Pleistocene & 1.80000 & n & America & Franz and Quitmyer,
	2005\tabularnewline
	377 & Inglis 1C, Florida & Gopherus & Gopherus sp. & 259.50 & mo & Lower
	Pleistocene & 1.80000 & n & America & Franz and Quitmyer,
	2005\tabularnewline
	378 & McGehee Farm near Newberry, Alachua County, Florida &
	Hesperotestudo & Hesperotestudo alleni & 240.90 & m & Tortonian &
	10.95000 & n & America & Holman J.A., 1972b: Amphibian and Reptiles. in:
	M.F. Skinner \& C.W. Hibbard (eds.) Early Pleistocene periglacial and
	glacial rocks and faunas of north-central Nebraska. Bulletin of the
	American Museum of Natural History 148(1): 55-71\tabularnewline
	379 & Texas & Hesperotestudo & Hesperotestudo campester & 1000.00 & mo &
	Gelasian & 2.19050 & n & America & Rhodin et al., 2015\tabularnewline
	380 & Little Salt Spring, Florida & Hesperotestudo & Hesperotestudo
	crassiscutata & 1250.00 & ev & Upper Pleistocene & 0.01200 & n & America
	& Holman \& Clausenm, 1984\tabularnewline
	381 & Haile, Alachua County, Florida & Hesperotestudo & Hesperotestudo
	crassiscutata & 168.00 & m & Lower Pleistocene & 1.30000 & n & America &
	Auffenberg, W. (1963).~Fossil testudinine turtles of Florida, genera
	Geochelone and Floridemys. University of Florida.\tabularnewline
	382 & Haile, Alachua County, Florida & Hesperotestudo & Hesperotestudo
	crassiscutata & 180.00 & m & Lower Pleistocene & 1.30000 & n & America &
	Auffenberg, W. (1963).~Fossil testudinine turtles of Florida, genera
	Geochelone and Floridemys. University of Florida.\tabularnewline
	383 & Reddick IA+B, Marion County, Florida & Hesperotestudo &
	Hesperotestudo crassiscutata & 180.40 & m & Upper Pleistocene & 0.06900
	& n & America & Auffenberg, W. (1963).~Fossil testudinine turtles of
	Florida, genera Geochelone and Floridemys. University of
	Florida.\tabularnewline
	384 & Little Salt Spring, Florida & Hesperotestudo & Hesperotestudo
	crassiscutata & 188.00 & mo & Upper Pleistocene & 0.01200 & n & America
	& Holman \& Clausenm, 1984\tabularnewline
	385 & Haile, Alachua County, Florida & Hesperotestudo & Hesperotestudo
	crassiscutata & 192.00 & m & Lower Pleistocene & 1.30000 & n & America &
	Auffenberg, W. (1963).~Fossil testudinine turtles of Florida, genera
	Geochelone and Floridemys. University of Florida.\tabularnewline
	386 & Reddick IA+B, Marion County, Florida & Hesperotestudo &
	Hesperotestudo crassiscutata & 282.70 & m & Upper Pleistocene & 0.06900
	& n & America & Auffenberg, W. (1963).~Fossil testudinine turtles of
	Florida, genera Geochelone and Floridemys. University of
	Florida.\tabularnewline
	387 & Reddick IA+B, Marion County, Florida & Hesperotestudo &
	Hesperotestudo crassiscutata & 284.90 & m & Upper Pleistocene & 0.06900
	& n & America & Auffenberg, W. (1963).~Fossil testudinine turtles of
	Florida, genera Geochelone and Floridemys. University of
	Florida.\tabularnewline
	388 & Haile, Alachua County, Florida & Hesperotestudo & Hesperotestudo
	crassiscutata & 327.00 & m & Lower Pleistocene & 1.30000 & n & America &
	Auffenberg, W. (1963).~Fossil testudinine turtles of Florida, genera
	Geochelone and Floridemys. University of Florida.\tabularnewline
	389 & Little Salt Spring, Florida & Hesperotestudo & Hesperotestudo
	crassiscutata & 425.00 & mo & Upper Pleistocene & 0.01200 & n & America
	& Holman \& Clausenm, 1984\tabularnewline
	390 & Leisey Shell Pit 1A, Hillsborough County, Florida & Hesperotestudo
	& Hesperotestudo crassiscutata & 561.00 & m & Lower Pleistocene &
	1.25000 & n & America & Meylan P.A., 1995: Pleistocene amphibians and
	reptiles from the Leisey Shell Pit, Hillsborough County, Florida.
	Florida Museum Natural History 37(Part I, 9): 273-297 or Morgan G.S.,
	Emslie S.D., 2010: Tropical and western in?uences in vertebrate faunas
	from the Pliocene and Pleistocene of Florida. Quaternary International
	217: 143-158 or MacFadden B.J., 1995: Magnetic polarity stratigraphy and
	correlation of the Leisey Shell Pits, Tampa Bay, Hillsborough County,
	Florida. Bulletin Florida Museum Natural History 37(Part I, 3) 107-116
	or Morgan G.S., Hulbert R., 1995: Overview of the geology and vertebrate
	paleontology of the Leisey Shell Pit Local Fauna, Hillsborough County,
	Florida. Bulletin Florida Museum Natural History 37(Part I, 1) 1-92 or
	Hulbert R.C., Morgan G.S., 1989: Stratigraphy, paleoecology, and
	vertebrate fauna of the Leisey Shell Pit Local Fauna, early Pleistocene
	(Irvingtonian) of southwestern Florida. Papers in Florida Paleontology
	2: 1-19\tabularnewline
	391 & Cragin Quarry Local Fauna, Meade County, Kansas & Hesperotestudo &
	Hesperotestudo equicomes & 340.00 & ev & Middle Pleistocene & 0.30000 &
	n & America & Holman J.A., 1972b: Amphibian and Reptiles. in: M.F.
	Skinner \& C.W. Hibbard (eds.) Early Pleistocene periglacial and glacial
	rocks and faunas of north-central Nebraska. Bulletin of the American
	Museum of Natural History 148(1): 55-71\tabularnewline
	392 & Haile, Alachua County, Florida & Hesperotestudo & Hesperotestudo
	incisa & 212.00 & m & Lower Pleistocene & 1.30000 & n & America &
	Auffenberg, W. (1963).~Fossil testudinine turtles of Florida, genera
	Geochelone and Floridemys. University of Florida.\tabularnewline
	393 & Haile, Alachua County, Florida & Hesperotestudo & Hesperotestudo
	incisa & 216.00 & m & Lower Pleistocene & 1.30000 & n & America &
	Auffenberg, W. (1963).~Fossil testudinine turtles of Florida, genera
	Geochelone and Floridemys. University of Florida.\tabularnewline
	394 & Haile, Alachua County, Florida & Hesperotestudo & Hesperotestudo
	incisa & 224.00 & m & Lower Pleistocene & 1.30000 & n & America &
	Auffenberg, W. (1963).~Fossil testudinine turtles of Florida, genera
	Geochelone and Floridemys. University of Florida.\tabularnewline
	395 & Haile, Alachua County, Florida & Hesperotestudo & Hesperotestudo
	incisa & 228.00 & m & Lower Pleistocene & 1.30000 & n & America &
	Auffenberg, W. (1963).~Fossil testudinine turtles of Florida, genera
	Geochelone and Floridemys. University of Florida.\tabularnewline
	396 & Haile, Alachua County, Florida & Hesperotestudo & Hesperotestudo
	incisa & 231.00 & m & Lower Pleistocene & 1.30000 & n & America &
	Auffenberg, W. (1963).~Fossil testudinine turtles of Florida, genera
	Geochelone and Floridemys. University of Florida.\tabularnewline
	397 & Arredondo IIA, Alachua County, Florida & Hesperotestudo &
	Hesperotestudo incisa & 232.76 & m & Upper Pleistocene & 0.06900 & n &
	America & Holman J.A., 1972b: Amphibian and Reptiles. in: M.F. Skinner
	\& C.W. Hibbard (eds.) Early Pleistocene periglacial and glacial rocks
	and faunas of north-central Nebraska. Bulletin of the American Museum of
	Natural History 148(1): 55-71\tabularnewline
	398 & Haile, Alachua County, Florida & Hesperotestudo & Hesperotestudo
	incisa & 241.00 & m & Lower Pleistocene & 1.30000 & n & America &
	Auffenberg, W. (1963).~Fossil testudinine turtles of Florida, genera
	Geochelone and Floridemys. University of Florida.\tabularnewline
	399 & Haile, Alachua County, Florida & Hesperotestudo & Hesperotestudo
	incisa & 290.40 & m & Lower Pleistocene & 1.30000 & n & America &
	Auffenberg, W. (1963).~Fossil testudinine turtles of Florida, genera
	Geochelone and Floridemys. University of Florida.\tabularnewline
	400 & Cita Canyon, UCMP V-3721, Harrell Ranch, Randall County, Texas &
	Hesperotestudo & Hesperotestudo johnstoni & 235.00 & m & Piacencian &
	3.35000 & n & America & Auffenberg W., 1962: A new species of Geochelone
	from the Pleistocene of Texas. Copeia 1962(3): 627-636\tabularnewline
	401 & Leisey Shell Pit 1A, Hillsborough County, Florida & Hesperotestudo
	& Hesperotestudo mlynarskii & 165.00 & m & Lower Pleistocene & 1.25000 &
	n & America & Auffenberg, 1988\tabularnewline
	402 & Leisey Shell Pit 2, Hillsborough County, Florida & Hesperotestudo
	& Hesperotestudo mlynarskii & 203.50 & m & Lower Pleistocene & 1.25000 &
	n & America & Auffenberg, 1988\tabularnewline
	403 & Sand Draw local fauna, Brown County, Nebraska & Hesperotestudo &
	Hesperotestudo oelrichi & 283.80 & m & Piacencian & 3.00000 & n &
	America & Preston R.E., 1979: Late Pleistocene cold-blooded vertebrate
	faunas from the mid-continental United States. I. Reptilia: Testudines,
	Crocodilia. Papers on Paleontology 19, Claude W. Hibbard Memorial Volume
	6: 1-53\tabularnewline
	404 & UCMP V71137, Turlock Lake 10, Stanislaus County, California &
	Hesperotestudo & Hesperotestudo orthopygia & 1200.00 & mo & Messinian &
	5.50000 & n & America & Biewer J., Sankey J., Hutchison H., Garber D.,
	2016: A fossil giant tortoise from the Mehrten Formation of Northern
	California. PaleoBios 33: 1-13 or Brattstrom B.H., 1961: Some new fossil
	tortoises from western North America with remarks on the zoogeography
	and paleoecology of tortoises. Journal of Paleontology 35(3):
	543-560\tabularnewline
	405 & UCMP V81248, Turlock Lake 11, Stanislaus County, California &
	Hesperotestudo & Hesperotestudo orthopygia & 682.00 & mo & Messinian &
	5.50000 & n & America & Biewer J., Sankey J., Hutchison H., Garber D.,
	2016: A fossil giant tortoise from the Mehrten Formation of Northern
	California. PaleoBios 33: 1-13\tabularnewline
	406 & Buis Ranch Local Fauna, Beaver County, Oklahoma & Hesperotestudo &
	Hesperotestudo riggsi & 159.50 & mo & Tortonian & 7.60000 & n & America
	& Oelrich, 1957\tabularnewline
	407 & Buis Ranch Local Fauna, Beaver County, Oklahoma & Hesperotestudo &
	Hesperotestudo riggsi & 159.50 & mo & Tortonian & 7.60000 & n & America
	& Oelrich, 1957\tabularnewline
	408 & Buis Ranch Local Fauna, Beaver County, Oklahoma & Hesperotestudo &
	Hesperotestudo riggsi & 159.50 & mo & Tortonian & 7.60000 & n & America
	& Oelrich, 1957\tabularnewline
	409 & Buis Ranch Local Fauna, Beaver County, Oklahoma & Hesperotestudo &
	Hesperotestudo riggsi & 159.50 & mo & Tortonian & 7.60000 & n & America
	& Oelrich, 1957\tabularnewline
	410 & Sawrock Canyon local fauna, Seward County, Kansas & Hesperotestudo
	& Hesperotestudo riggsi & 176.00 & m & Piacencian & 3.00000 & n &
	America & Hibbard C.W., 1944: A new land tortoise, Testudo riggsi, from
	the Middle Pliocene of Seward County, Kansas. The University of Kansas
	Science Bulletin 30(7): 71-76\tabularnewline
	411 & Sawrock Canyon local fauna, Seward County, Kansas & Hesperotestudo
	& Hesperotestudo riggsi & 185.00 & m & Piacencian & 3.00000 & n &
	America & Hibbard C.W., 1944: A new land tortoise, Testudo riggsi, from
	the Middle Pliocene of Seward County, Kansas. The University of Kansas
	Science Bulletin 30(7): 71-76\tabularnewline
	412 & Rexroad local fauna (Fox Canyon locality 3), Meade County, Kansas
	& Hesperotestudo & Hesperotestudo riggsi & 195.80 & m & Zanclean &
	4.55000 & n & America & Oelrich, 1957\tabularnewline
	413 & Caballo Local Fauna, Palomas Basin, Sierra County, New Mexico &
	Hesperotestudo & Hesperotestudo sp. & 1000.00 & mo & Gelasian & 2.00000
	& n & America & Morgan et al., 2011\tabularnewline
	414 & UCMP V-3952, Ingram Creek site 8, Stanislaus County, California &
	Hesperotestudo & Hesperotestudo sp. & 1200.00 & ev & Tortonian & 9.50000
	& n & America & Biewer J., Sankey J., Hutchison H., Garber D., 2016: A
	fossil giant tortoise from the Mehrten Formation of Northern California.
	PaleoBios 33: 1-13\tabularnewline
	415 & Gilliland local fauna, Burnett Ranch, Knox
	County, Texas & Hesperotestudo & Hesperotestudo sp. & 1500.00 & mo &
	Middle Pleistocene & 0.70000 & n & America & Holman, 1969; Preston,
	1966\tabularnewline
	416 & Cuchillo Negro Creek Local Fauna, Sierra County, New
	Mexico & Hesperotestudo & Hesperotestudo sp. & 176.00 & mf & Piacencian
	& 3.10000 & n & America & Morgan et al., 2011\tabularnewline
	417 & Gilliland local fauna, Burnett Ranch, Knox
	County, Texas & Hesperotestudo & Hesperotestudo sp. & 1800.00 & mo &
	Middle Pleistocene & 0.70000 & n & America & Preston,
	1966\tabularnewline
	418 & Ingleside Local Fauna, San Patricio County, Texas & Hesperotestudo
	& Hesperotestudo sp. & 639.00 & m & Upper Pleistocene & 0.06000 & n &
	America & Auffenberg, 1962: A Redescription of Testudo hexagonata
	Cope\tabularnewline
	419 & Ingleside Local Fauna, San Patricio County, Texas & Hesperotestudo
	& Hesperotestudo sp. & 974.00 & ep & Upper Pleistocene & 0.06000 & n &
	America & Auffenberg, 1962: A Redescription of Testudo hexagonata
	Cope\tabularnewline
	420 & Kansas & Hesperotestudo & Hesperotestudo turgida & 230.00 & mo &
	Lower Pleistocene & 1.68450 & n & America & Rhodin et al.,
	2015\tabularnewline
	421 & Friesenhahn Cave, Bexar County, Texas & Hesperotestudo &
	Hesperotestudo wilsoni & 226.00 & m & Upper Pleistocene & 0.01800 & n &
	America & Milstead W.W., 1956: Fossil turtles of Friesenhahn Cave,
	Texas, with the description of a new species of Testudo. Copeia 1956(3):
	162-171\tabularnewline
	422 & Atascosa county, Texas & Testudo & Testudo sp. & 400.00 & mo &
	Langhian & 14.18100 & n & America & Hay, 1902\tabularnewline
	423 & Libertador San Martín, Entre Rios Province & Chelonoidis & Chelonoidis denticulata &
	616.00 & m & Upper Pleistocene & 0.12000 & n & America & Manzano A.S.,
	Noriega J.I., Joyce W.G., 2009: The tropical tortoise Chelonoidis
	denticulata (Testudines: Testudinidae) from the Late Pleistocene of
	Argentina and its paleoclimatological implications. Journal of
	Paleontology 83(6): 975-980\tabularnewline
	424 & Arroyo Toropí, Corrientes & Chelonoidis & Chelonoidis lutzae &
	830.00 & m & Upper Pleistocene & 0.03850 & n & America & Zacarías et
	al., 2013\tabularnewline
	425 & Quebrada de Ñuapua, Chuquisaca department & Chelonoidis &
	Chelonoidis sp. & 1000.00 & mo & Upper Pleistocene & 0.06900 & n &
	America & De Broin, 1991\tabularnewline
	426 & Beautiful Bone, Alta Guajira Peninsula, Cocinetas basin &
	Chelonoidis & Chelonoidis sp. & 1060.00 & ec & Langhian & 15.90000 & n &
	America & Cadena, 2015\tabularnewline
	427 & Beautiful Bone, Alta Guajira Peninsula, Cocinetas basin &
	Chelonoidis & Chelonoidis sp. & 300.00 & mo & Langhian & 15.90000 & n &
	America & Cadena, 2015\tabularnewline
	428 & Beautiful Bone, Alta Guajira Peninsula, Cocinetas basin &
	Chelonoidis & Chelonoidis sp. & 300.00 & mo & Langhian & 15.90000 & n &
	America & Cadena, 2015\tabularnewline
	429 & Beautiful Bone, Alta Guajira Peninsula, Cocinetas basin &
	Chelonoidis & Chelonoidis sp. & 300.00 & mo & Langhian & 15.90000 & n &
	America & Cadena, 2015\tabularnewline
	430 & Beautiful Bone, Alta Guajira Peninsula, Cocinetas basin &
	Chelonoidis & Chelonoidis sp. & 300.00 & mo & Langhian & 15.90000 & n &
	America & Cadena, 2015\tabularnewline
	431 & San Nicolas, UCMP locality V4536 & Geochelone & Geochelone
	hesterna & 278.00 & m & Tortonian & 8.50000 & n & America & Auffenberg
	W., 1971: A new fossil tortoise, with remarks on the origin of South
	American testudinines. Copeia 1: 106-117\tabularnewline
	\bottomrule
\end{longtable}
}

\end{landscape}