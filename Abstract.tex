\section{Zusammenfassung}

Körpergrößenvariationen entlang räumlicher und zeitlicher Skalen zu beschreiben und zu verstehen ist seit geraumer Zeit ein zentraler Punkt der Evolutionsbiologie.
Die Körpergröße eines Organismus ist vor allem deshalb so interessant, weil sie ein universelles Merkmal aller Lebewesen ist, das von vielen Faktoren beeinflusst wird und mit evolutionären und ökologischen Prozessen im Zusammenhang steht. 
%Man kann sogar bei den meisten Fossilien zumindest grob die Körpergröße bestimmen und damit Vergleiche über lange Zeiträume ziehen.
%Die Körpergröße eines Tieres hängt zu großen Teilen von den verfügbaren Ressourcen und Umweltbedingungen während des Wachstums, der genetischen Veranlagung und dem Abwägen zwischen den Kosten einer frühen gegenüber einer späten Geschlechtsreife, also der Investition in Wachstum oder Reproduktion, ab.
Besonders interessant sind dabei extreme Ausprägungen, die 
%unter unterschiedlichen Bedingungen, zum Beispiel in Inselhabitaten, auftreten können. Ein verringertes Prädationsrisiko beispielsweise führt häufig zum Zwergenwuchs, während ein verringertes Konkurrenzverhältnis oft in Gigantismus resultiert.
%Eine extreme Körpergröße ist bringt 
viele Vor- und Nachteile mit sich bringen.
%, so haben sehr große Tiere meist längere Generationszeiten und einen erhöhten Ressourcenbedarf, allerdings auch ein verringertes Prädationsrisiko, oft eine erweiterte Lebensspanne und eine erhöhte Konkurrenzfähigkeit.
In der Erdgeschichte gab es wiederholt sehr große Tiere, die Megafauna.
Während des Quartär starben zahlreiche dieser Arten aus, darunter viele Säuger. Das Aussterben der Säugetiermegafauna ist sehr gut untersucht und zum Großteil wird der Mensch dafür verantwortlich gemacht. 
%Die Säugetiermegafauna ist sehr gut untersucht und ihre Aussterbeereignisse während des Quartärs wurden zum Großteil auf den Menschen als Hauptverantwortlichen zurückgeführt.
%und die damit verbundenden Aussterbeergeignisse während des Quartärs sind sehr gut untersucht und der Mensch als Hauptverantwortlicher 
Allerdings gab es auch noch andere Vertreter der Megafauna, beispielsweise Riesenschildkröten, von denen nur wenige Arten auf vereinzelten Inseln bis ins heutige Zeitalter überlebt haben. Wieso kontinentale Arten ausstarben und wie ihre Entwicklung verlief ist weitestgehend unklar.
Im Rahmen dieser Arbeit wurden Körpergrößenverteilungen von rezenten und fossilen Testudinidae vom Miozän bis heute untersucht, mit dem Ziel allgemeine Entwicklungstrends bezüglich ihrer Körpergröße zu ermitteln.
Die Analysen ergaben, dass Landschildkröten eine bimodale Körpergrößenverteilung haben, die sowohl räumlich als auch zeitlich weitestgehend konstant bleibt. Rezente Testudinidae sind signifikant kleiner als fossile und auf Inseln vorkommende
Landschildkröten sind signifikant größer als die auf den Kontinenten lebenden.
Über die Zeit jedoch scheinen Testudinidae im Schnitt keine nennenswerte Änderung in der durchschnittlichen Körpergröße aufzuweisen, ihre generelle Körpergrößenevolution kann am besten als Stasis beschrieben werden.
Untersucht man jedoch die kontinentalen Testudinidae separat, ist ein Trend erkennbar, der allerdings insgesamt nicht-direktional verläuft. Diese Beobachtung lässt sich auf das Aussterben der kontinentalen Riesenschildkröten zurückführen, was zu einer starken Verringerung der Körpergröße gegen Ende des Pleistozäns führte.
Diese Abnahme der durchschnittlichen Körpergröße lässt sich auch auf globaler Ebende und bei Inselarten beobachten, allerdings weniger stark ausgeprägt.
Ob nun der Mensch die Verantwortung für das Aussterben der Riesenschildkröten trägt, lässt sich nicht abschließend klären, obwohl einiges darauf hindeutet. Höchstwahrscheinlich handelt es sich aber um ein Zusammenspiel aus anthropogenem Einfluss und klimatischen Veränderungen.

%- no body size trend in global scale
%- continents: random walk
%- islands: stasis

%- modern testudinids are smaller than fossil ones
%- continental testudinids are smaller than insular ones
%--> these proportions are constant over time

%- population strucure is constant on spatial and temporal scale
%- right-skewed like most animals; bimodal; on islands: left-skewed (more large species!)