\section{Introduction}

Body size has been of interest to researchers for a long time (Haldane, ..., Peters, ...), because it is a universal trait, that can be easily measured in most organisms. Moreover, it is linked to and influenced by many ecological and evolutionary processes, although the underlying mechanisms are often complex. 
Some basic questions deal with the issue of optimal body size: is there one for every organism and if so, how is it determined, how does it evolve?
Animals --> magintudes of body size, many have their greatest body size right now, possible maximum has not been reached! BUT reptiles not

--> optimal body size

--> evolutionary models???

--> maximum body size 

--> megafauna 

--> human-induced extinction / climate

--> reptile "megafauna" --> large turtles (archelon) --> giant tortoises!

--> have been abundant on continents and islands in former times. no longer, only on islands.
in earlier publication fossil testudinids were grouped into two genera, based on size: small = testudo, large = geochelone. 
By now, most areas have been revised: America = Hesperotestudo and Chelonoidis (Galapagos), Europe = Cheirogaster or better: Titanochelon, Africa = Aldabrachelys, Chersina, Asia = Geochelone ??

Evolution of tortoises:
- where did they originate, how did they spread?
--> floated to islands
--> relationship with humans --> tortoises/turtles as food - whaling industry
- tortoise extinctions --> only on Galapagos and Aldabra giant tortoises --> which went extinct within the last couple of years? --> which are associated with humans?
what about climate?
--> largest extant continental tortoise: Centrochelys sulcata (80 cm)





%________________________________________________________________________________

- Body size as a trait (read Smith, Smith \& Lyons) and over time
why is it interesting?
is there an optimal body size for every organism? how can it be determined? (--> stasis??)

- evolutionary models (read Gene Hunt's paper and Posada, 2003)
-->  make sense of evolutionary modes:
* stasis
* unbiased random walk
* generalized random walk


- body size in tortoises %(why not use biomass? - not necessary)
- distribution of tortoises (?)
- giant tortoises well suited for drifting on ocean currents (Meylan, 2000)

OR
- mammal megafauna extinctions --> giant tortoises

- human and climatic influence

- purpose of this work: determine body size trends in tortoises and identify evolutionary mode (if possible). what lead to extinction?




read:

body size:

Smith and Lyons

Smith et al., 2016

megafauna extinctions:

Sandom et al., 2014

Schuster and Schüle, 2000

using paleontological data:

Willis et al., 2010

Willis and Fortey, 2000

tortoise/turtle overview.

Rhodin et al., 2015

Lapparent de Broin, 2001

Perez-Garcia/Vlachos/Lujan etc.

%__________________________________

Introduce the field:
-Evolution of traits, in particular body size

Introduce the general topic
- Maximum body size OR body size in relation to extinction events
--> Megfauna extinctions ?! and megafauna evolution?

Introduce the particular topic

--> giant tortoises/tortoise body size/extinction

Defining the scope of the particular topic by

--> introducing the research parameters
- SCL? evolutionary models?

--> summarizing previous research
- studies on declining body size in certain species, extinction/conservation of giant insular tortoises

Preparing for present research by

--> indicating a gap in previous research
- has mostly focused in mammal megafauna

--> indicating a possible extension of previous research
- giant tortoises can also be considered megafauna and also went extinct --> why?

Introducing present research by

--> stating the aim of the research
body size trends? mode of evolution? underlying reasons (humans? climate?)?

OR 
--> describing briefly the work carried out
body size data was collected and compared across geographical and temporal scales to identify possible patterns

--> justifying the research
- evolution of body size is useful for understanding of evolutionary processes in general, useful for many organisms, since all organisms have body size
- if we understand, why giant tortoises went extinct, we can possibly prevent similar exxtinction events in the future --> giant tortoises are still endangered


--> HOW DID GIANT TORTOISES EVOLVE, HOW WAS LARGE BODY SIZE ATTAINED? WHY DID THEY GO EXTINCT (ON THE CONTINENTS AT LEAST)?


--> check mammal megafauna paper! did all of them go extinct? particularly on continents? (Barnosky et al., 2004)
