\section{Introduction}
\subsection{Body size evolution}

%Laurin, 2004
Body size has been of interest to researchers for a long time (Haldane, ..., Peters, ...), because it is a universal trait, that can be easily measured and compared among different organisms, extant and fossil. Moreover, it is linked to and influenced by many ecological and evolutionary processes and the identification of patterns has been.../patterns have been described for many animal groups (...). Body size is influenced by a number of biotic and abiotic factors including habitat, resource availability, competition, climate etc.
Patterns of body size variation across spatial and/or temporal scales have been documented and many biogeographical rules concern body size evolution: Cope's rule, which describes the gradual within-lineage body size increase over time (cite). The island rule according to which large species get smaller on islands while small species often show an increase in body size (). Bergmann's rule and the temperature-size rule state that animals attain larger body sizes at higher latitudes, where temperatures are lower ().
While such patterns have been described in many animal groups and species (), the underlying mechanisms might not be the same in all taxa, which require further investigation \citep{Smith2009}.

%Influence of temperature on body size --> Hunt, 2015

%Smith et al., 2016
%However, when patterns of size increase are examined within lower-level taxa (e.g., phyla, classes), the trends found there are generally more consistent with the null (Table 2)

\subsection{Maximal body size - Megafauna}
A potential optimal body size and extreme body size - either large or small - are of particular interest to many researchers \citep{Smith2009}. A large size comes with a lot of implications regarding resource availability, resilience, fecundity etc. 
Patterns of when and how often maximum body size is achieved is inconsistent across time and different animal groups.
(Also, different patterns in minimum size, which could be due to poorer preservation of smaller forms.)
Giant animal forms are often intrigiung, because they usually present the upper limit of what is possibe physically, physiologically and structurally. 
For example, the large insect (dragonflies etc.) from the Carbonian/Permian border (?) seem absolutely surreal to us as well as the later non-avian dinosaurs, which still fascinate many people. And today's largest animal, the blue whale, is of interest to many people.
Animals exceeding 44 kg in body mass are referred to as megafauna.
Animals are considered megafauna, if their body mass/body length exceeds 44 kg \citep{Barnosky2004} (or sometimes 10 kg \citep{Sandom2014}).
%Large size can have many advantages but also some disadvantages.
%Especially the mammalian megafauna has been in the focus of research, wooly mammoths, giant sloths, waldelefanten?, etc.
It has been suggested, that large animals are more prone to extinction than smaller conterfeits, because they usually need more resources, have a larger range and longer generation times. During the Quarternary, a huge number of large mammal which belonged to the megafauna went extinct.
A number of causes for these extinction events has been discussed, but while a meteorite and diseases have been discussed but were not supported by new findings, two main suggestions have been debated: climate change and anthropogenic influence. 
Some recent studies \citep{Barnosky2004, Sandom2014, Gibbons2004, Schuster2000} suggest that human influence has been the main driver of these extinctions of mammal megafauna.

- extinctions seem to be more size-biased in vertebrates

While the mammalian megafauna as well as their extinction %Pleistocene extinctions 
have been well investigated, the herpetofauna has also suffered the same fate \citep{Blain2016}. In former times, giant tortoises were abundant on the continents as well as on many islands, whereas today giant tortoises are left only on two island regions in the tropics.


%Some basic questions deal with the issue of optimal body size: is there one for every organism and if so, how is it determined, how does it evolve?
%Animals --> magintudes of body size, many have their greatest body size right now, possible maximum has not been reached! BUT reptiles not


% Smith et al., 2016:
%For body size evolution in particular, the most commonly observed and documentable kind of trend—a rise in the maximum—does not by itself tell us much about underlying tendencies. Maxima are expected to increase even if no tendencies, no evolutionary forces, are present. Thus, we cannot infer the existence of a selective advantage of large size merely from an increase in the maximum.
%If a clade first evolves at a size close to the minimum size possible for a particular body plan or ecology and subsequent speciation events are random with respect to size (i.e., descendants are equally likely to be larger or smaller than their ancestors), thenmean size will increase, because the lower bound prevents the variance from increasing in the direction of smaller sizes. This is an example of a passive trend, one with no increasing tendency.
%A positive association between body size and extinction risk has been demonstrated for Pleistocene and living mammals (Lyons et al. 2004, Davidson et al. 2009), marine and fresh- water fishes (Olden et al. 2007), and birds (Boyer 2010). The association between body size and extinction risk is widely interpreted to result from the inverse association of body size with ecologically important traits such as population size, fecundity, and total resource requirements (Brown 1995).



%\subsection{Megfauna extinctions (mammals)}


%\subsection{Ectothermic megfauna}

%--> dinos BUT also giant tortoises


\subsection{Giant Tortoises - Testudinidae}
There have been two groups of terrestrial tortoises, which both contained giant forms: Meiolania, exclusively in Australia and completely extinct nowadays, and the tortoises, family testudinidae, which comprises all extant terrestrial tortoises as well as many extinct/fossil taxa, which also included giant forms and occurred on all continents but Australia.
Testudinidae have been known in the fossil record from the Eocene on, the earliest fossils stemming from Africa.
Body size has actually played a role in testudinid taxonomy in the past. In the beginning, all new tortoise fossils were assigned to the huge genus Testudo, but around the beginning of the 20th century(??), tortoises were grouped into two clades, based on body size: large taxa were all assigned to the Genus Geochelone, while small tortoises were assigned to Testudo.
Luckily, in the past decades, the taxonomy has been revised for many regions, based on morphological and biogeographical clues. In the Americas, all large tortoises are now referred to either Chelonoidis (which all extant Galapagos giant tortoises belong to), Hesperotestudo or Gopherus (?). In Europe, the genus Cheirogaster has been used but is currently being replaced by Titanochelon, although not all species have been revised accordingly (Perez-Garcia irgendwas??), while small species still belong to Testudo, which contains the extant Testudo graeca group. In Asia and Africa, the two current biodiversity hotspots for turtles and tortoises in genera, many different taxa have now been differentiated. In Asia, the genera Geochelone (the old name!), Indotestudo and Manouria are present. In mainland Africa there are two extant genera: Homopus and Centrochelys, the latter consisting of one species, Centrochelys sulcata, which is the largest continental tortoise there is (CL = 80 cm). In the West Indian Islands (Madagascar, Seychellen, Aldabra etc.), there are three gerena: Astrochelys, Pyxis and the giant Aldabrachelys.
Giant tortoises exceeding carapace lengths of 2 m were abundant in former times and have frequently been found in fossil deposits on contintents as well as remote islands. The colonization of islands has been achieved by tortoises due to their resilience, they can cope without food or water for months (which is why they have been exploited by humans in the past --> whaling industry), their bouyancy \citep{Gerlach2006, Patterson1973}. There, giant tortoises were abundant, just like they have been until a couple of hundreds of years ago on the Galapagos islands. On the West Indies, 2 species of Aldabrachelys actually went extinct in the past centuries (??)
However, the abundance of giant tortoises on these remote islands along with their resilience and survivability without resources, made them a very attractive food item for humans, especially whalers.
But also on the mainlands, and also small species have frequently been eaten, since turtles and tortoises are easily captured, do not need a great amount of preparation before they can be cooked and can even be kapt as a "staple" since they stay alive for quite a while withoud food or water (a couple of days/weeks in smaller species ??? and up to several months in large forms) \citep{Thompson2002,Thompson2014}. Some studies suggest that the intense hunting affected tortoise body size. In some areas, tortoise consumption was so common, that tortoise body size has been suggested as a proxy for human population size \citep{Steele2005,Stiner1999,Stiner2000}. Turtles and tortoises are still being eaten, although many are endangered (cite).
Furthermore, tortoises, being extotherms, are also influenced by climate and especially large tortoises are very temperature-sensitive. For one, in the fossil record, large tortoises are considered to be an indicator for mild winters \citep{Hibbard1960}, since they are thought to not have been able to burrow during hibernation as modern Gopherus tortoises do (?? only in some places, I guess?). On the other hand, giant tortoises run a high risk of overheating and display behavioral thermoregulation to keep their body temperature below a dangerous or even lethal value. \citep{Sturbaum1982, Schleich1981} 


 
- mention Meiolania?
- taxonomy
- occurence
- evolution/distribution
- tortoises are temperature sensitive (?), well, giant tortoises seem to be.
- have always been a preferred food source for humans (whaling industry)

\subsection{Body size trends in Testudinidae}

The aim of this work is to identify general body size distributions in tortoises (Testudinidae) on a global scale across the last 20 million years. Further, to investigate the general evolutionary mode of body size in testudinidae. Apart from that, focusing on giant tortoises, which were abundant in former times, but are extinct nowadays apart from a few remaining populations on tropical islands, to discuss where and when they occurred in the fossil record and when they disappeared again and what the underlying reasons could be.

- general body size distribution on a temporal and spatial scale
- when/where did giant tortoises occurr, when/where did they disappear?
- evolutionary mode of giant tortoises
- how did they go extinct?


- unbiased random walk: descendants are equally likely to be larger or smaller than their ancestors











































\vspace{3 cm}


--> optimal body size

--> evolutionary models???

--> maximum body size 

--> megafauna 

--> human-induced extinction / climate

--> reptile "megafauna" --> large turtles (archelon) --> giant tortoises!

--> have been abundant on continents and islands in former times. no longer, only on islands.
in earlier publication fossil testudinids were grouped into two genera, based on size: small = testudo, large = geochelone. 
By now, most areas have been revised: America = Hesperotestudo and Chelonoidis (Galapagos), Europe = Cheirogaster or better: Titanochelon, Africa = Aldabrachelys, Chersina, Asia = Geochelone ??

Evolution of tortoises:
- where did they originate, how did they spread?
--> floated to islands
--> relationship with humans --> tortoises/turtles as food - whaling industry
- tortoise extinctions --> only on Galapagos and Aldabra giant tortoises --> which went extinct within the last couple of years? --> which are associated with humans?
what about climate?
--> largest extant continental tortoise: Centrochelys sulcata (80 cm)





%________________________________________________________________________________

- Body size as a trait (read Smith, Smith \& Lyons) and over time
why is it interesting?
is there an optimal body size for every organism? how can it be determined? (--> stasis??)

- evolutionary models (read Gene Hunt's paper and Posada, 2003)
-->  make sense of evolutionary modes:
* stasis
* unbiased random walk
* generalized random walk


- body size in tortoises %(why not use biomass? - not necessary)
- distribution of tortoises (?)
- giant tortoises well suited for drifting on ocean currents (Meylan, 2000)

OR
- mammal megafauna extinctions --> giant tortoises

- human and climatic influence

- purpose of this work: determine body size trends in tortoises and identify evolutionary mode (if possible). what lead to extinction?




read:

body size:

Smith and Lyons

Smith et al., 2016

megafauna extinctions:

Sandom et al., 2014

Schuster and Schüle, 2000

using paleontological data:

Willis et al., 2010

Willis and Fortey, 2000

tortoise/turtle overview.

Rhodin et al., 2015

Lapparent de Broin, 2001

Perez-Garcia/Vlachos/Lujan etc.

%__________________________________

Introduce the field:
-Evolution of traits, in particular body size

Introduce the general topic
- Maximum body size OR body size in relation to extinction events
--> Megfauna extinctions ?! and megafauna evolution?

Introduce the particular topic

--> giant tortoises/tortoise body size/extinction

Defining the scope of the particular topic by

--> introducing the research parameters
- SCL? evolutionary models?

--> summarizing previous research
- studies on declining body size in certain species, extinction/conservation of giant insular tortoises

Preparing for present research by

--> indicating a gap in previous research
- has mostly focused in mammal megafauna

--> indicating a possible extension of previous research
- giant tortoises can also be considered megafauna and also went extinct --> why?

Introducing present research by

--> stating the aim of the research
body size trends? mode of evolution? underlying reasons (humans? climate?)?

OR 
--> describing briefly the work carried out
body size data was collected and compared across geographical and temporal scales to identify possible patterns

--> justifying the research
- evolution of body size is useful for understanding of evolutionary processes in general, useful for many organisms, since all organisms have body size
- if we understand, why giant tortoises went extinct, we can possibly prevent similar exxtinction events in the future --> giant tortoises are still endangered


--> HOW DID GIANT TORTOISES EVOLVE, HOW WAS LARGE BODY SIZE ATTAINED? WHY DID THEY GO EXTINCT (ON THE CONTINENTS AT LEAST)?


--> check mammal megafauna paper! did all of them go extinct? particularly on continents? (Barnosky et al., 2004)
