\section{Introduction}
\subsection{Body size evolution}

%Laurin, 2004
The body size of organisms has been of interest to researchers for a long time (Haldane, ..., Peters, ...). It is a universal trait, that can be easily measured and compared among different organisms, extant and fossil\todo{describe more precisely what this means and why it's a cool trait to compare modern and fossil species}. A number of biotic and abiotic factors influence body size including habitat, resource availability, competition, climate and many more and it is linked to many ecological and evolutionary processes (...)\todo{example?!}.
Patterns of body size variation across spatial and/or temporal scales have been described for many animal groups and are summarised as the following\todo{better term} rules.
%and many biogeographical rules concern body size evolution: 
Cope's rule, which describes the gradual within-lineage body size increase over time (cite). The island rule according to which large species get smaller on islands while small species often show an increase in body size (). Bergmann's rule and the temperature-size rule state that animals attain larger body sizes at higher latitudes, where temperatures are lower ().\todo{rewrite and put into context}
While such patterns have been well documented
%described in many animal groups and species 
(), the underlying mechanisms might not be the same in all taxa and require further investigation \citep{Smith2009}.

%Influence of temperature on body size --> Hunt, 2015

%Smith et al., 2016
%However, when patterns of size increase are examined within lower-level taxa (e.g., phyla, classes), the trends found there are generally more consistent with the null (Table 2)

%\todo{include evolutionary models here!!}


\subsection{Maximal body size - Megafauna}
Many taxa have a right-skewed body size distribution, which means that smaller body sizes are more frequent in species than large ones. This raises the question of why organisms are a certain size, if there is an optimal body size for each organisms and why some species attain larger sizes than others and what 
structural, physical and physiological properties determine and limit maximum body size
%the maximum possible body size depends on, structurally, physically and physiologically 
\citep{Smith2009}.
%are distribution yA potential optimal body size and extreme body size - either large or small - are of particular interest to many researchers . 
A large body size is associated with advanteges ....
On the other hand, there are certain disadvantages...\todo{describe in more detail and highlight dis-/advantages}
a lot of implications regarding resource availability, resilience, fecundity, 
%but it has also been debated whether larger body sizes may entail a higher risk of extinction (...).
\todo{focus more on maximum body size, auf MAXIMUM body size hinführen}
Patterns of when and how often maximum body size is achieved are inconsistent across time and different animal groups ().
%(Also, different patterns in minimum size, which could be due to poorer preservation of smaller forms.)
Some famous examples are the large insects from the Carboniferous/Permian period, the giant non-avian dinosaurs from the Jurassic/Triassic, the giant mammals from the Quatenary or today's largest animal, the blue whale.
%Giant animal forms are often intrigiung, because they usually present the upper limit of what is possibe physically, physiologically and structurally. 
%For example, the large insect (dragonflies etc.) from the Carbonian/Permian border (?) seem absolutely surreal to us as well as the later non-avian dinosaurs, which still fascinate many people. And today's largest animal, the blue whale, is of interest to many people.
Animals exceeding 44 kg in body mass are referred to as megafauna \citep{Barnosky2004, Sandom2014}.
%Animals are considered megafauna, if their body mass/body length exceeds 44 kg \citep{Barnosky2004} (or sometimes 10 kg \citep{Sandom2014}).
%Large size can have many advantages but also some disadvantages.
%Especially the mammalian megafauna has been in the focus of research, wooly mammoths, giant sloths, waldelefanten?, etc.
It has been suggested, that large animals are more prone to extinction than smaller ones. Reasons for this may be that large species usually need more resources, have a larger range and longer generation times. During the Quaternary, a huge number of large mammals which were considered megafauna went extinct.
A number of causes for these extinction events have been discussed, but while a meteorite and diseases were dismissed as possible causes due to lack of evidence (), two possible scenarios are still under discussion: climate change and anthropogenic influence ().
Some recent studies suggest that human influence has been the main driver of these extinctions of mammalian megafauna \citep{Barnosky2004,Sandom2014,Gibbons2004,Schuster2000}.

%- extinctions seem to be more size-biased in vertebrates

While the mammalian megafauna as well as their extinction %Pleistocene extinctions 
have been well investigated, the herpetofauna has also lost a considerable number of species during the Quaternary \citep{Blain2016}. 
For example, many turtle and tortoise species have gone extinct, with a considerable part of large species.
In former times, giant tortoises were abundant on the continents as well as on many islands, whereas today giant tortoises are present only on two island regions in the tropics
% - the Galapagos islands and the West Indian islands
.


%Some basic questions deal with the issue of optimal body size: is there one for every organism and if so, how is it determined, how does it evolve?
%Animals --> magintudes of body size, many have their greatest body size right now, possible maximum has not been reached! BUT reptiles not


% Smith et al., 2016:
%For body size evolution in particular, the most commonly observed and documentable kind of trend—a rise in the maximum—does not by itself tell us much about underlying tendencies. Maxima are expected to increase even if no tendencies, no evolutionary forces, are present. Thus, we cannot infer the existence of a selective advantage of large size merely from an increase in the maximum.
%If a clade first evolves at a size close to the minimum size possible for a particular body plan or ecology and subsequent speciation events are random with respect to size (i.e., descendants are equally likely to be larger or smaller than their ancestors), thenmean size will increase, because the lower bound prevents the variance from increasing in the direction of smaller sizes. This is an example of a passive trend, one with no increasing tendency.
%A positive association between body size and extinction risk has been demonstrated for Pleistocene and living mammals (Lyons et al. 2004, Davidson et al. 2009), marine and fresh- water fishes (Olden et al. 2007), and birds (Boyer 2010). The association between body size and extinction risk is widely interpreted to result from the inverse association of body size with ecologically important traits such as population size, fecundity, and total resource requirements (Brown 1995).



%\subsection{Megfauna extinctions (mammals)}


%\subsection{Ectothermic megfauna}

%--> dinos BUT also giant tortoises


\subsection{Giant Tortoises - Testudinidae}
In the fossil record two groups of terrestrial tortoises have been identified, which both contained giant forms. Firstly, the family Meiolania, which occurred exclusively in Australia and is completely extinct nowadays. And the tortoises, family Testudinidae, which comprises all extant terrestrial tortoises as well as many extinct or fossil taxa. The testudinidae included giant forms and occurred on all continents but Australia.
Testudinidae have been known in the fossil record from the Eocene on, the earliest fossils stemming from Africa (...).
Body size played an important role in the earlier times of testudinid taxonomy. In the beginning, all new tortoise fossils were assigned to the genus Testudo, but around the beginning of the 20th century(??), tortoises were grouped into two clades based on body size. Large taxa were all assigned to the Genus Geochelone, while small tortoises were assigned to Testudo.
Eventually, over the past few decades, the taxonomy has been revised for many regions, based on morphological and biogeographical clues. In the Americas, all large tortoises are now referred to as either Hesperotestudo, Gopherus or Chelonoidis (which all extant Galapagos giant tortoises belong to), (?). In Europe, the genus Cheirogaster has been used but is currently being replaced by Titanochelon, although not all species have been revised accordingly yet (Perez-Garcia irgendwas??). Small species still belong to Testudo, which contains the extant Testudo graeca group ().
In Asia and Africa, the two current biodiversity hotspots for turtles and tortoises in genera, many different taxa have now been differentiated.
In Asia, the genera Geochelone, Indotestudo and Manouria are present. In mainland Africa there are two extant genera: Homopus and Centrochelys, the latter consisting of one species, Centrochelys sulcata, which is the largest continental tortoise there is (CL = 80 cm). In the West Indian Islands (Madagascar, Seychellen, Aldabra etc.), there are three gerena: Astrochelys, Pyxis and the giant Aldabrachelys.

On the contintents as well as remote islands giant tortoises exceeding carapace lengths of 2 m were abundant in former times and have frequently been found in fossil deposits. The presence of large tortoises on islands has been explained by their ability to float \citep{Gerlach2006, Patterson1973} and to survive for months without water of food. 
%The colonization of islands has been achieved by tortoises due to their resilience, they can cope without food or water for months (which is why they have been exploited by humans in the past --> whaling industry), their bouyancy. 

However, the abundance of giant tortoises on these remote islands along with their resilience and survivability without resources, made them a very attractive food item for humans, especially in the whaling industry.
Although the exploitation of giant tortoises on islands is largely recognized, tortoises and turtles, both small and large, were also frequently eaten on the mainland.
%But also on the mainlands, and also small species have frequently been eaten, since turtles and tortoises 
They are easily captured, do not need a great amount of preparation before they can be cooked and can even be kept as a "staple" since they stay alive for quite a while withoud food or water (a couple of days/weeks in smaller species ??? and up to several months in large forms) \citep{Thompson2002,Thompson2014}. Intensive hunting and exploitation has been suggested to affect tortoise body size (...) since humans prefer larger prey and therefore large individuals are more prone to exploitation than small ones. This may have lead to a descreased body size within a tortoise population, where tortoise consumption is common.
For this reason, tortoise body size has been suggested as a proxy for human population size in some areas\citep{Steele2005,Stiner1999,Stiner2000}.
To this day, turtles and tortoises are still being eaten, although many are endangered (cite).
Apart from anthropogenic influences, climate probably also affected tortoises (). 
As ectotherms, turtles and tortoises are inherently more dependent on climate than endotherms(). Especially large tortoises are very temperature-sensitive due to their unique physiology and morphology (?). On the one hand\todo{rewrite}, large tortoises are considered to be an indicator for mild winters in the fossil record \citep{Hibbard1960}, since they are thought to not have been able to burrow during hibernation as modern Gopherus tortoises do (?? only in some places, I guess?)\todo{rewrite and hedge}. On the other hand, giant tortoises run a high risk of overheating and display behavioral thermoregulation to keep their body temperature below a dangerous or even lethal value. \citep{Sturbaum1982, Schleich1981}.
It has not been conclusively ...., if humans influence or climate change or a mixture of both have caused the extinction of giant tortoises ().
Giant 

%There, giant tortoises were abundant, just like they have been until a couple of hundreds of years ago on the Galapagos islands. On the West Indies, 2 species of Aldabrachelys actually went extinct in the past centuries (??)
 
%- mention Meiolania?
%- taxonomy
%- occurence
%- evolution/distribution
%- tortoises are temperature sensitive (?), well, giant tortoises seem to be.
%- have always been a preferred food source for humans (whaling industry)

\subsection{Body size trends in Testudinidae}

The aim of this work is to identify general body size distributions in tortoises (Testudinidae) on a global scale across the last 20 million years. Further, to investigate the general evolutionary mode of body size in testudinidae. 
%For this, the methods of Hunt (...) are used, which fit the mean trait, in this case carapace length, over time to different evolutionary models: either stasis, unbiased random walk or a generalized random walk. ... These mean and have been tested for ....????
Apart from that, focusing on giant tortoises, which were abundant in former times, but are extinct nowadays apart from a few remaining populations on tropical islands, to discuss where and when they occurred in the fossil record and when they disappeared again and what the underlying reasons could be.

- general body size distribution on a temporal and spatial scale
- when/where did giant tortoises occurr, when/where did they disappear?
- evolutionary mode of giant tortoises
- how did they go extinct?


- unbiased random walk: descendants are equally likely to be larger or smaller than their ancestors


