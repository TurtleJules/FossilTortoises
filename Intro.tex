\section{Introduction}
\subsection{Body size evolution}

%Laurin, 2004
The body size of organisms has been of interest to researchers for a long time \citep{Haldane1928,Peters1983}. It is a universal trait that can be easily measured and compared among different organisms \citep{.}. Furthermore, it is readily available for many animals in the fossil record which allows for comparison with extant species \citep{.}. A number of biotic and abiotic factors including habitat, resource availability, competition, climate and many more influence body size which is also linked to ecological and evolutionary processes \citep{Blackburn1994a,Blueweiss1978,Smith2009}.

%, for example \todo{example?!}.
Patterns of body size variation across spatial and/or temporal scales have been described for many animal groups and some have been summarised as ecological rules \citep{Millien2006,Angielczyk2015}.
%and many biogeographical rules concern body size evolution: 
Cope's rule describes the gradual within-lineage body size increase over time \citep{Cope1887}. According to the island rule large species become smaller on islands, because of a reduced predation risk, while small species often show an increase in body size due to reduced competition \citep{Foster1964}. Bergmann's rule \citep{Bergmann1848} states that animals attain a larger body size at higher latitudes, which can be considered a special case of the temperature-size rule, which predicts an increase in body size at lower temperatures \citep{AngillettaJr2004a}.
While such patterns have been well documented
%described in many animal groups and species 
\citep{Millien2006}, the underlying mechanisms might not be the same across all taxa and require further investigation \citep{Smith2009}.


%Smith et al., 2016
%However, when patterns of size increase are examined within lower-level taxa (e.g., phyla, classes), the trends found there are generally more consistent with the null (Table 2)

%\todo{include evolutionary models here!!}


\subsection{Maximum body size - Megafauna}
In many taxa a right-skewed body size distribution is observed, which means that smaller body sizes are more abundant than large ones \citep{Blackburn1994a,Kozlowski2002,Lyons2008}. This raises the question of why organisms are a certain size, if there is an optimal body size for each organisms, why some species attain larger sizes than others and which 
structural, physical and physiological properties determine and limit minimum or maximum body size
%the maximum possible body size depends on, structurally, physically and physiologically 
\citep{Smith2009}.
%are distribution yA potential optimal body size and extreme body size - either large or small - are of particular interest to many researchers . 
A large body size is associated with certain advantages, e. g. decreased predation risk, higher fasting endurance, higher competitive ability, all of which results in a higher fitness \citep{.}.
On the other hand, there are some disadvantages to having a larger body size since usually a larger range size and more resources are needed and generation times are often longer \citep{.}.
%a lot of implications regarding resource availability, resilience, fecundity, 
%but it has also been debated whether larger body sizes may entail a higher risk of extinction (...).
In the history of Earth, animals have repeatedly attained very large body sizes, although patterns of when and how often maximum body size is achieved are inconsistent across time and different animal groups \citep{Smith2016}.
%(Also, different patterns in minimum size, which could be due to poorer preservation of smaller forms.)
Some famous examples of giant forms are the large insects from the Carboniferous/Permian period, the giant non-avian dinosaurs from the Jurassic/Triassic, the giant mammals from the Quatenary or today's largest animal, the blue whale \citep{.}.
%Giant animal forms are often intrigiung, because they usually present the upper limit of what is possibe physically, physiologically and structurally. 
%For example, the large insect (dragonflies etc.) from the Carbonian/Permian border (?) seem absolutely surreal to us as well as the later non-avian dinosaurs, which still fascinate many people. And today's largest animal, the blue whale, is of interest to many people.
Animals with a body mass exceeding 44 kg (or sometimes 10 kg; \citeauthor{Sandom2014}, \citeyear{Sandom2014}) are referred to as megafauna \citep{Barnosky2004, Rhodin2015}.
%Animals are considered megafauna, if their body mass/body length exceeds 44 kg \citep{Barnosky2004} .
%Large size can have many advantages but also some disadvantages.
%Especially the mammalian megafauna has been in the focus of research, wooly mammoths, giant sloths, waldelefanten?, etc.
It has been suggested, that large animals are more prone to extinction than smaller ones \citep{.}. Reasons for this may be that large species usually need more resources, have a larger range and longer generation times \citep{.}. 
During the Quaternary, a huge number of large mammals which were considered megafauna went extinct \citep{.}.
A number of causes for these extinction events have been discussed, but while a meteorite and disease were dismissed as possible causes due to lack of evidence \citep{.}, two possible scenarios are still under discussion: climate change and anthropogenic influence \citep{.}.
Some recent studies suggest that human influence has been the main driver of these extinctions of mammalian megafauna \citep{Barnosky2004,Sandom2014,Gibbons2004,Schuster2000}.

%- extinctions seem to be more size-biased in vertebrates

While the mammalian megafauna as well as their extinction %Pleistocene extinctions 
have been well investigated, the herpetofauna has also lost a considerable number of species during the Quaternary \citep{Blain2016}. 
For example, many turtle and tortoise species have gone extinct, among those quite a few large species which can also be considered megafauna \citep{Rhodin2015,Froyd2014,Pedrono2013}.
In former times, giant tortoises were abundant on the continents as well as on many islands, whereas today giant tortoises are present only on two island regions in the tropics \citep{.}
% - the Galapagos islands and the West Indian islands
.


%Some basic questions deal with the issue of optimal body size: is there one for every organism and if so, how is it determined, how does it evolve?
%Animals --> magintudes of body size, many have their greatest body size right now, possible maximum has not been reached! BUT reptiles not


% Smith et al., 2016:
%For body size evolution in particular, the most commonly observed and documentable kind of trend—a rise in the maximum—does not by itself tell us much about underlying tendencies. Maxima are expected to increase even if no tendencies, no evolutionary forces, are present. Thus, we cannot infer the existence of a selective advantage of large size merely from an increase in the maximum.
%If a clade first evolves at a size close to the minimum size possible for a particular body plan or ecology and subsequent speciation events are random with respect to size (i.e., descendants are equally likely to be larger or smaller than their ancestors), thenmean size will increase, because the lower bound prevents the variance from increasing in the direction of smaller sizes. This is an example of a passive trend, one with no increasing tendency.
%A positive association between body size and extinction risk has been demonstrated for Pleistocene and living mammals (Lyons et al. 2004, Davidson et al. 2009), marine and fresh- water fishes (Olden et al. 2007), and birds (Boyer 2010). The association between body size and extinction risk is widely interpreted to result from the inverse association of body size with ecologically important traits such as population size, fecundity, and total resource requirements (Brown 1995).



%\subsection{Megfauna extinctions (mammals)}


%\subsection{Ectothermic megfauna}

%--> dinos BUT also giant tortoises


\subsection{Giant Tortoises - \T}
%\todo{Smith2009 mammalian megafauna analyzed by kurtosis and skewness, followed, applicable to giant tortoises --> similar results or not?}

In the fossil record two clades of terrestrial tortoises have been identified, which both contained giant forms. One is the family Meiolaniidae, which occurred exclusively in Argentina, Australia and its surrounding Islands and is completely extinct nowadays \citep{Anderson1025, Sterli2015}. The other is the family Testudinidae, which comprises all extant terrestrial tortoises as well as many extinct or fossil taxa \citep{.}. The testudinids included giant forms and occurred on all continents but Australia \citep{.}.
Testudinidae have been known in the fossil record from the Eocene onwards, the earliest fossils being \textit{Hadrianus} which are known from North America and Europe and probably originated in Asia \citep{Cope1872}.
Body size played an important role in the earlier times of testudinid taxonomy \citep{.}. In the beginning, all tortoise fossils were assigned to the genus Testudo, but around the 20th century(??), tortoises were grouped into two clades based on body size. Large taxa were all assigned to the genus \textit{Geochelone}, while small tortoises were assigned to \textit{Testudo}.
Eventually, over the past few decades, the taxonomy has been revised for many regions, based on morphological and biogeographical clues \citep{.}. In the Americas, all tortoises are now referred to as either \textit{Hesperotestudo}, \textit{Gopherus} or \textit{Chelonoidis}, the latter contains all extant Galapagos giant tortoises \citep{.}. In Europe, the genus \textit{Cheirogaster} has been introduced but is currently being replaced by \textit{Titanochelon}, although not all species have been revised accordingly yet \citep{.}. Small species still belong to \textit{Testudo}, which contains the extant \textit{Testudo graeca}-group \citep{.}.
In Asia and Africa, the two current biodiversity hotspots for turtles and tortoises in general \citep{.}, many different taxa have now been differentiated \citep{.}.
In Asia, the genera \textit{Geochelone}, \textit{Indotestudo} and \textit{Manouria} are present \citep{.}. On mainland Africa there are seven extant genera: \textit{Homopus}, \textit{Psammobates}, \textit{Kinixys}, \textit{Malacochersus}, \textit{Chersina}, \textit{Stigmochelys} and \textit{Centrochelys}, the latter consisting of one species, \textit{Centrochelys sulcata}, which is the largest extant continental tortoise with a carapace length of about 80 cm \citep{.}. In the West Indian Islands (Madagascar, Seychelles, Aldabra etc.), there are three extant gerena, \textit{Astrochelys}, \textit{Pyxis} and the giant \textit{Aldabrachelys}, as well as the extinct \textit{Cylindraspis} \citep{.}.

On the contintents as well as remote islands giant tortoises exceeding carapace lengths of 2 m were abundant in former times and have frequently been found in fossil deposits \citep{.}. The presence of large tortoises on islands has been explained by their ability to float and to survive for months without water or food \citep{Gerlach2006, Patterson1973, Cheke2016}. 
%Cheke et al. 2016: Meilania also floated to islands!!

%The colonization of islands has been achieved by tortoises due to their resilience, they can cope without food or water for months (which is why they have been exploited by humans in the past --> whaling industry), their bouyancy. 

However, the abundance of giant tortoises on these remote oceanic islands along with their resilience and survivability without resources made them a very attractive food item for humans, especially in the whaling industry \citep{.}.
In addition to the exploitation of giant tortoises on islands, both small and large tortoises and turtles were also frequently eaten on the mainland \citep{.}.
%But also on the mainlands, and also small species have frequently been eaten, since turtles and tortoises 
Tortoises are easily captured, do not need a great amount of preparation before they can be cooked and can even be kept as a "staple" since they stay alive for quite a while withoud food or water \citep{Thompson2002,Thompson2014}. Intensive hunting and exploitation has been suggested to affect tortoise body size \citep{.} since larger individuals are more easily visible and yield more meat, they are more prone to exploitation than small ones \citep{Rhodin2015}. This may have lead to a decreased body size within a tortoise population, where tortoise consumption is common.
For this reason, tortoise body size has even been suggested as a proxy for human population size in some areas \citep{Steele2005,Stiner1999,Stiner2000}.
To this day, turtles and tortoises are still being eaten in some countries, although many are endangered \citep{.}.
Apart from anthropogenic influences, climate probably also affected tortoises \citep{.}. 
As ectotherms, turtles and tortoises are inherently more dependent on climate than endotherms \citep{.}. Especially large tortoises are very temperature-sensitive due to their unique physiology and morphology \citep{Swingland1979a, Swingland1979b}. In the fossil record, large tortoises are considered to be an indicator for mild winters \citep{Hibbard1960,Schleich1981}, since they are thought to not have been able to dig themselves burrows for hibernation like modern \textit{Gopherus} tortoises do (?? only in some places, I guess?)\citep{Carlson1999, Stojanov2009}. Additionally, giant tortoises run a high risk of overheating and display behavioral thermoregulation to keep their body temperature below a dangerous or even lethal value \citep{Sturbaum1982, Schleich1981}.
It has not been concluded with certainty, if anthropogenic influence or climate change or a mixture of both have caused the extinction of giant tortoises\citep{Rhodin2015}.

% \todo{over which time period??} 

%There, giant tortoises were abundant, just like they have been until a couple of hundreds of years ago on the Galapagos islands. On the West Indies, 2 species of Aldabrachelys actually went extinct in the past centuries (??)
 
%- mention Meiolania?
%- taxonomy
%- occurence
%- evolution/distribution
%- tortoises are temperature sensitive (?), well, giant tortoises seem to be.
%- have always been a preferred food source for humans (whaling industry)

\subsection{Aim of this work}

The aim of this work is to identify general body size distributions in tortoises (Testudinidae) on a global scale across the last 20 million years. Further, to investigate the general evolutionary mode of body size in testudinidae. 
%For this, the methods of Hunt (...) are used, which fit the mean trait, in this case carapace length, over time to different evolutionary models: either stasis, unbiased random walk or a generalized random walk. ... These mean and have been tested for ....????
Apart from that, focusing on giant tortoises, which were abundant in former times, but are extinct nowadays apart from a few remaining populations on tropical islands, to discuss where and when they occurred in the fossil record and which underlying reasons for the extinction of many giant forms have been stated in the literature.
Understanding how and why giant tortoises went extinct in the past, can hold valuable information for conservation work today. Many tortoise species are endangered and extinction rates have been especially high for insular species since the Pleistocene \citep{Rhodin2015}.



The development of body size or any other trait over time can follow different evolutionary trajectories, that describe general trends of trait evolution. If a trait does not significantly change, this observation is described as stasis. If a trait does change, this change can either be directional, referred to as a generalized random walk, or non-directional, which is described as an unbiased random walk. Recently, new analytical tools have been developed, to be able to determine which evolutionary model fits the development of a trait, for example size trends, best over time \citep{Hunt2006,Hunt2015}.
Changes in body size on the clade level can either be due to selective forces acting on the whole clade (trends) or individual species being influenced by different causes (tendencies) \citep{Hunt2006}. A trend can be an increase or decrease in minimum, mean or maximum size and can be caused by differing speciation and extinction rates and therefore occur independently of tendencies \citep{Smith2016}. Maximum size, for example, is usually expected to increase without being selected for \citep{Smith2016}.
%- general body size distribution on a temporal and spatial scale
%- when/where did giant tortoises occurr, when/where did they disappear?
%- evolutionary mode of giant tortoises
%- how did they go extinct?


%- unbiased random walk: descendants are equally likely to be larger or smaller than their ancestors
