\documentclass[toc=bibliographynumbered, liststotoc]{scrartcl} %mit [draft] vor den {} erreicht man, daß statt der Grafiken nur ein Rahmen in Größe der Grafik gezeichnet wird. Zusätzlich wird der Name der einzufügenden Datei in diesen Rahmen geschrieben]{book}
\usepackage[utf8]{inputenc}
\usepackage[T1]{fontenc}
\usepackage{lmodern}
%\usepackage[english,ngerman]{babel} %Deutsch als Hauptsprache, z.b. für Inhaltsverzeichnis etc.
\usepackage[ngerman,english]{babel} % Englisch als Hauptsprache!!!

\usepackage{graphicx}
\usepackage{xspace}
\usepackage{amstext}
\usepackage{amssymb}
\usepackage{setspace}
\doublespacing
\usepackage{longtable}
\usepackage{booktabs}
\usepackage{todonotes}
\usepackage{placeins}

\usepackage[paper=a4paper,margin=2.5cm]{geometry}
\usepackage{helvet}
\renewcommand*{\familydefault}{\sfdefault} %gehört zum Font


\usepackage{here} %erzwingt Platzierung von Bildern ("H" bei Option)
\usepackage{colortbl} % Tabellen: Spalten, Zeilen oder einzelne Zellen einfärben
\usepackage[round]{natbib}	%\citep   \cite
\bibliographystyle{apalike} % beides für Literaturverzeichnis bzw. Zitate

\usepackage{subfig} %lässt einen 2 Bilder oder Tabellen nebeneinander einfügen, mit dem Befehl "\subfloat[Unter-/Überschrift]{...}"
%\usepackage{subcaption}
\usepackage{rotating} %erlaubt das Drehen von Grafiken um 90° ohne angle=90
\usepackage{sidecap} %für Tabellen und Grafiken, die weiter sind als Textweite


% zum besseren Steuern des Anhangs /appendices statt /appendix benutzen

\usepackage[title,titletoc]{appendix}
\usepackage{lscape}


\newcommand{\bigcell}[2]{\begin{tabular}{@{}#1@{}}#2\end{tabular}}


\begin{document}

\begin{titlepage}
%\begin{figure}[t]
\begin{center}
\includegraphics[scale=0.65]{pics/hulogo.pdf}\\[1cm]

%\end{figure}
\textsc{\textbf{\normalsize LEBENSWISSENSCHAFTLICHE FAKULTÄT\\[0.2cm]  INSTITUT FÜR BIOLOGIE\\[1.5cm]
\LARGE MASTERARBEIT\\[0.5cm]
\normalsize ZUM ERWERB DES AKADEMISCHEN GRADES\\[0.5cm] MASTER OF SCIENCE\\[2cm]}
\normalfont %Kapitälchen deaktivieren
\textit{\large "Körpergrößentrends in fossilen Landschildkröten aus dem Neogen"\\[0.5cm]
"Body size trends in Neogene testudinid tortoises" }\\[2cm]
\small vorgelegt von\\[0.5cm] Julia Joos \\[0.2cm] geb. am 18.05.1991 in Freudenstadt \\[1cm] angefertigt in der Arbeitsgruppe Paläozoologie\\[0.2cm]
am Institut für Biologie/Museum für Naturkunde\\[1cm]
Berlin, im Septenber 2017}
\end{center}


\end{titlepage}


\tableofcontents
\newpage
\listoffigures  
\newpage                       
\listoftables                     
\newpage
\section{Introduction}
\subsection{Body size evolution}

%Laurin, 2004
The body size of organisms has been of interest to researchers for a long time (Haldane, ..., Peters, ...). It is a universal trait, that can be easily measured and compared among different organisms, extant and fossil. Moreover, it is influenced by a number of biotic and abiotic factors including habitat, resource availability, competition, climate and many more and it is linked to many ecological and evolutionary processes (...).
Patterns of body size variation across spatial and/or temporal scales have been described for many animal groups and many biogeographical rules concern body size evolution: Cope's rule, which describes the gradual within-lineage body size increase over time (cite). The island rule according to which large species get smaller on islands while small species often show an increase in body size (). Bergmann's rule and the temperature-size rule state that animals attain larger body sizes at higher latitudes, where temperatures are lower ().
While such patterns have been described in many animal groups and species (), the underlying mechanisms might not be the same in all taxa and require further investigation \citep{Smith2009}.

%Influence of temperature on body size --> Hunt, 2015

%Smith et al., 2016
%However, when patterns of size increase are examined within lower-level taxa (e.g., phyla, classes), the trends found there are generally more consistent with the null (Table 2)

%\todo{include evolutionary models here!!}


\subsection{Maximal body size - Megafauna}
Many taxa have a right-skewed body size distribution, which means that smaller body sizes are more frequent in species than large ones. This raises the question of why organisms are a certain size, of there is an optimal body size for each organisms and why some species attain larger sizes than others and what the maximum possible body size depends on, structurally, physically and physiologically \citep{Smith2009}.
%are distribution yA potential optimal body size and extreme body size - either large or small - are of particular interest to many researchers . 
A large body size is associated with a lot of implications regarding resource availability, resilience, fecundity, but it has also been debated whether larger body sizes may entail a higher risk of extinction (...).
Patterns of when and how often maximum body size is achieved is inconsistent across time and different animal groups.
%(Also, different patterns in minimum size, which could be due to poorer preservation of smaller forms.)
Some famous examples are the large insects from the Carboniferous/Permian period, the giant non-avian dinosaurs from the Jurassic/Triassic, the giant mammals from the Quatenary or today's largest animal, the blue whale.
%Giant animal forms are often intrigiung, because they usually present the upper limit of what is possibe physically, physiologically and structurally. 
%For example, the large insect (dragonflies etc.) from the Carbonian/Permian border (?) seem absolutely surreal to us as well as the later non-avian dinosaurs, which still fascinate many people. And today's largest animal, the blue whale, is of interest to many people.
Animals exceeding 44 kg in body mass are referred to as megafauna \citep{Barnosly2004, Sandom2014}.
%Animals are considered megafauna, if their body mass/body length exceeds 44 kg \citep{Barnosky2004} (or sometimes 10 kg \citep{Sandom2014}).
%Large size can have many advantages but also some disadvantages.
%Especially the mammalian megafauna has been in the focus of research, wooly mammoths, giant sloths, waldelefanten?, etc.
It has been suggested, that large animals are more prone to extinction than smaller ones, because large species usually need more resources, have a larger range and longer generation times. During the Quarternary, a huge number of large mammal which belonged to the megafauna went extinct.
A number of causes for these extinction events have been discussed, but while a meteorite and diseases were dismissed as possible causes due to lack of evidenc, two main suggestions have been debated: climate change and anthropogenic influence.
Some recent studies \citep{Barnosky2004,Sandom2014,Gibbons2004,Schuster2000} suggest that human influence has been the main driver of these extinctions of mammal megafauna.

%- extinctions seem to be more size-biased in vertebrates

While the mammalian megafauna as well as their extinction %Pleistocene extinctions 
have been well investigated, the herpetofauna has also lost a considerable number of species during the Quaternary \citep{Blain2016}. In former times, giant tortoises were abundant on the continents as well as on many islands, whereas today giant tortoises are present only on two island regions in the tropics.


%Some basic questions deal with the issue of optimal body size: is there one for every organism and if so, how is it determined, how does it evolve?
%Animals --> magintudes of body size, many have their greatest body size right now, possible maximum has not been reached! BUT reptiles not


% Smith et al., 2016:
%For body size evolution in particular, the most commonly observed and documentable kind of trend—a rise in the maximum—does not by itself tell us much about underlying tendencies. Maxima are expected to increase even if no tendencies, no evolutionary forces, are present. Thus, we cannot infer the existence of a selective advantage of large size merely from an increase in the maximum.
%If a clade first evolves at a size close to the minimum size possible for a particular body plan or ecology and subsequent speciation events are random with respect to size (i.e., descendants are equally likely to be larger or smaller than their ancestors), thenmean size will increase, because the lower bound prevents the variance from increasing in the direction of smaller sizes. This is an example of a passive trend, one with no increasing tendency.
%A positive association between body size and extinction risk has been demonstrated for Pleistocene and living mammals (Lyons et al. 2004, Davidson et al. 2009), marine and fresh- water fishes (Olden et al. 2007), and birds (Boyer 2010). The association between body size and extinction risk is widely interpreted to result from the inverse association of body size with ecologically important traits such as population size, fecundity, and total resource requirements (Brown 1995).



%\subsection{Megfauna extinctions (mammals)}


%\subsection{Ectothermic megfauna}

%--> dinos BUT also giant tortoises


\subsection{Giant Tortoises - Testudinidae}
There have been two groups of terrestrial tortoises, which both contained giant forms: Meiolania, exclusively in Australia and completely extinct nowadays, and the tortoises, family testudinidae, which comprises all extant terrestrial tortoises as well as many extinct/fossil taxa, which also included giant forms and occurred on all continents but Australia.
Testudinidae have been known in the fossil record from the Eocene on, the earliest fossils stemming from Africa (...).
Body size has actually played a role in testudinid taxonomy in the past. In the beginning, all new tortoise fossils were assigned to the genus Testudo, but around the beginning of the 20th century(??), tortoises were grouped into two clades, based on body size: large taxa were all assigned to the Genus Geochelone, while small tortoises were assigned to Testudo.
Luckily, in the past decades, the taxonomy has been revised for many regions, based on morphological and biogeographical clues. In the Americas, all large tortoises are now referred to either Chelonoidis (which all extant Galapagos giant tortoises belong to), Hesperotestudo or Gopherus (?). In Europe, the genus Cheirogaster has been used but is currently being replaced by Titanochelon, although not all species have been revised accordingly (Perez-Garcia irgendwas??), while small species still belong to Testudo, which contains the extant Testudo graeca group. In Asia and Africa, the two current biodiversity hotspots for turtles and tortoises in genera, many different taxa have now been differentiated. In Asia, the genera Geochelone (the old name!), Indotestudo and Manouria are present. In mainland Africa there are two extant genera: Homopus and Centrochelys, the latter consisting of one species, Centrochelys sulcata, which is the largest continental tortoise there is (CL = 80 cm). In the West Indian Islands (Madagascar, Seychellen, Aldabra etc.), there are three gerena: Astrochelys, Pyxis and the giant Aldabrachelys.
Giant tortoises exceeding carapace lengths of 2 m were abundant in former times and have frequently been found in fossil deposits on contintents as well as remote islands. The colonization of islands has been achieved by tortoises due to their resilience, they can cope without food or water for months (which is why they have been exploited by humans in the past --> whaling industry), their bouyancy \citep{Gerlach2006, Patterson1973}. There, giant tortoises were abundant, just like they have been until a couple of hundreds of years ago on the Galapagos islands. On the West Indies, 2 species of Aldabrachelys actually went extinct in the past centuries (??)
However, the abundance of giant tortoises on these remote islands along with their resilience and survivability without resources, made them a very attractive food item for humans, especially whalers.
But also on the mainlands, and also small species have frequently been eaten, since turtles and tortoises are easily captured, do not need a great amount of preparation before they can be cooked and can even be kapt as a "staple" since they stay alive for quite a while withoud food or water (a couple of days/weeks in smaller species ??? and up to several months in large forms) \citep{Thompson2002,Thompson2014}. Some studies suggest that the intense hunting affected tortoise body size. In some areas, tortoise consumption was so common, that tortoise body size has been suggested as a proxy for human population size \citep{Steele2005,Stiner1999,Stiner2000}. Turtles and tortoises are still being eaten, although many are endangered (cite).
Furthermore, tortoises, being extotherms, are also influenced by climate and especially large tortoises are very temperature-sensitive. For one, in the fossil record, large tortoises are considered to be an indicator for mild winters \citep{Hibbard1960}, since they are thought to not have been able to burrow during hibernation as modern Gopherus tortoises do (?? only in some places, I guess?). On the other hand, giant tortoises run a high risk of overheating and display behavioral thermoregulation to keep their body temperature below a dangerous or even lethal value. \citep{Sturbaum1982, Schleich1981} 


 
- mention Meiolania?
- taxonomy
- occurence
- evolution/distribution
- tortoises are temperature sensitive (?), well, giant tortoises seem to be.
- have always been a preferred food source for humans (whaling industry)

\subsection{Body size trends in Testudinidae}

The aim of this work is to identify general body size distributions in tortoises (Testudinidae) on a global scale across the last 20 million years. Further, to investigate the general evolutionary mode of body size in testudinidae. 
For this, the methods of Hunt (...) are used, which fit the mean trait, in this case carapace length, over time to different evolutionary models: either stasis, unbiased random walk or a generalized random walk.
... These mean and have been tested for ....????
Apart from that, focusing on giant tortoises, which were abundant in former times, but are extinct nowadays apart from a few remaining populations on tropical islands, to discuss where and when they occurred in the fossil record and when they disappeared again and what the underlying reasons could be.

- general body size distribution on a temporal and spatial scale
- when/where did giant tortoises occurr, when/where did they disappear?
- evolutionary mode of giant tortoises
- how did they go extinct?


- unbiased random walk: descendants are equally likely to be larger or smaller than their ancestors



\newpage
\section{Material \& Methods}

\subsection{Data collection}
I collected data on body size of fossil testudinids from the Miocene until recent times. The body size data set includes 30 genera, comprising over 100 fossil species. The majority of the data was obtained from the primary literature (Table \ref{TabS1}). To find relevant publications, I relied mostly on the references listed in FosFarBase (CITE), PDBD (cite), and "Fossil Turtle Checklist (CITE).
Furthermore, the FosFarBase provided fossil occurences of testudinids all over the world, including their exact localities and age (Table \ref{TabS2}), which were used to get an overview over the availability of body size data. 
For extant taxa, I measured dry material (n = 67) from the collection of the Museum für Naturkunde zu Berlin (MFN). In addition, body size data from the literature was included (Table \ref{TabS3}).

\subsection{Body size estimation}
Body size is reported as straight carapace length (SCL) in mm. Where SCL was not available from the primary literature, it was estimated either from plastron length (PL) or appendicular elements (Table \ref{TabS1}). For carapace length estimations based on plastron length, the measurements from the MFN collection material was used to calculate the ratio between SCL and PL. Since the SC/PL ratio was similar for all species and genera, a single general ratio was calculated for all testudinids and hence used for the SCL estimations unless stated otherwise (Table \ref{TabS1}). For estimations based on femora and humeri, the ratio provided by Hutterer et al. (1998) and Franz et al. (2001), respectively, were used. A number of publications did not state measurements but instead provided scaled figures of the fossil remains, from which SCL, PL or humeri and femur lengths could be measured.

%TO DO: check Franz \& Quitmyer, 2005 again!! (CL regression)

\subsection{Analyses}
All subsequent analyses were performed with R (version 3.4.1), including the packages dplyr (cite) to prepare the data for the analysis (???) and ggplot2 (cite) to create figures. Sampling Accumulation Curves were created with the R package vegan (Cite) to see if sample size sufficed.
\todo{explain what species accumulation curves are and give reference}
 Since the data set relies on literature, references were used as a sampling unit (x-axis).%and reach a maximum, when no new species/genera are added. 
Since genera were much better sampled than species (Fig. )This was repeated on genus level, since genera of fossil testudinids are relatively well resolved by now whereas determination on the species level is still somewhat obscure in many cases, as some species were based on scarce material. 

% read Species Accumulation Curve papers + Catalinas paper!
% REWRITE

\subsubsection{Distribution and statistics}
Histograms and boxplots of the entire data set and several subgroups (fossil vs. modern, insular vs. continental...) were created to explore the distribution of body size. The Wilcoxon Rank Sum Test (unpaired data) was used to test for differences between two subgroups. To be able to compare different subgroups, a subsample (1000 repeats) of the respective larger subgroup was taken to compare equal sample sizes. 

%- only used samples > 23.000 mya?!

\subsubsection{Body size trends over time}
To investigate trends in body size over time, the R package paleoTS (cite) was used. Data were split into time bins according to stratigraphic stages (Table \ref{tab:bins}, Fig. \ref{fig:bins}), although the two stages of the Lower Miocene are considered one time bin, to increase sample size. To decrease influence of sampling bias and because Sampling Accumulation Curves showed that the genus level was well sampled in contrast to species level, the mean SCL per genus was calculated before the timescale analysis. The paleoTS plots were created, wich display the mean trait over time and can be fitted to different evolutionary models: stasis, which ...., generalized random walk (GRW), which .... or unbiased random walk (URW), which..... . The Akaike Weight Criterion (AICc) indicates which model is best supported --> see Catalina's Paper and Hunt's papers


\begin{figure}[htbp]
	\centering
	\includegraphics{MA_JJ_files/figure-latex/overviewData-1.pdf}
	\caption{Scatterplot of carapace length over time, indicating insular
		(triangle) and continental (circles) and colour indicating continents.
		Lines indicate stratigraphic stages which were used as time bins, the
		dashed line is the border between the two stages of the Lower Miocene,
		which were consideres as one time bin.}
	\label{fig:bins}
\end{figure}

\newpage
\section{Results}

%read Willis et al., 2010 and Willis and Fortey, 2003 on use of paleontological data.

% read Thompson and Withers, 2003: Species Accumulation Curve
\subsection{Sample-based accumulation curves}
The sample-based accumulation curve (SAC) on the generic level shows a relatively low intial slope and a long upward slope to the asymptote, which does not reach full saturation (Fig. \ref{fig:SACGen}).
%Although the SAC does not completely plateau, considering the large area covered and the high number of rare genera in the dataset, it can be considered well enough sampled for our purposes.
In contrast, the species accumulation curves, both per reference and per locality, show only a slight initial increase and, for the same number of references/sampling units, are far from reaching an asymptote (Fig. \ref{fig:SACall} (a), (b)).
% DISCUSSION: as could be expected, since there are less genera than species \citep{Gotelli2001}. At a large geographic scale, it can be expected, that an asymptote is not reached \citep{Thompson2002}. Fig. \ref{fig:SACGen} corresponds to the shape one would expect, when there are many genera that are rare and only a few abundant ones. For the different continents, only Europe and Eurasia show some sort of "typical" SAC shape.
%%%____
% DISCUSSION ? 
%Since there are less genera than species, it is to be expected that genera reach an asymptote earlier than species.
%%%____
Accumulation curves for individual continents show that Europe reflects the trend of the overall dataset, with a long upward slope after the inflection point, whereas the other continents require further sampling (Fig. \ref{fig:SACall} (c) - (i)). For this reason, the timescale analysis was only conducted for Europe and Eurasia, but not for the other continents.

%--> genera sufficiently sampled, species not
%--> for genera, when splitting up continents only Europe and Eurasia are sampled well enough --> ask Johannes again, I feel like Asia is screwing up the results, would stick with Europe.
%--> maybe refer to table 20: overview over fossil genera per time bin??

%_______________________________________________________________________
\begin{figure}[htbp]
	\centering
	\includegraphics[width=0.8\textwidth]{MA_JJ_files/figure-latex/SACGenera-1.pdf}
	\caption[Sample-based accumulation curve on generic level]{Sample-based accumulation curve of fossil genera per reference. Dashed lines represent the confidence inteval.}
	\label{fig:SACGen}
\end{figure}


\FloatBarrier


%read: Smith and Lyons, 2010 (?) and Smith et al., 2016
%--> body size patterns (see also table 22, general statistics)

%data is bimodal (not normally distributed --> fig. ), still visible in pretty much all subgroups that you could split them into.
%as most animal groups/clades (?) right-skewed = smaller body size more frequent, BUT island species are left-skewed with more larger body sizes! (logtransformed data, skewness: negative for Zanclean, Messinian ??, insular, fossil-insular, but not modern-insular!!)
\subsection{Descriptive statistics}

The histograms indicate that testudinid body size is not normally distributed (Fig. \ref{fig:histAll}), which is supported by QQ-Plots for raw as well as log-transformed data (Fig. \ref{fig:NormDis}).

The body size distribution is moderately right-skewed (Table \ref{tab:stats}), with a higher frequency of smaller body sizes.
Body size ranges from a minimum of 80 mm to a maximum of 2500 mm for the entire data set. When comparing body sizes on a temporal scale, the minimum body size per stratigraphic stages excluding modern taxa ranges from 90 mm to 270 mm, while the maximum 1100 mm to 2500 mm. The highest maximum body size was observed in the fossil record from continental Europe (CL = 2500 mm).
%________________________________________________________________________

\begin{figure}[htbp]
	\centering
	\includegraphics[width=0.8\textwidth]{MA_JJ_files/figure-latex/HistAll-1.pdf}
	\caption[CL distribution]{Body size distribution of complete data set. Bimodally distributed and right-skewed.}
	\label{fig:histAll}
\end{figure}

This pattern is also apparent when splitting the data set into fossil and modern taxa (Fig. \ref{fig:HistFMCI} (a)). Considering insularity, body size distribution is right-skewed for continental taxa, but left-skewed for insular species, meaning larger body size is more frequent than smaller body size on islands. Insular taxa are also left-skewed when only considering fossil taxa, but modern insular taxa have a skewness close to 0, indicating a symmetric distribution (Table \ref{tab:stats}).
Kurtosis suggests light tails with no/few outliers (kurtosis < 3) for insular and modern insular species, whereas continental species have a heavy tail (kurtosis > 3; Table \ref{tab:stats}).

\begin{center}
	\begin{figure}[htbp]
		\subfloat[Fossil vs. modern]{\includegraphics[scale=0.45]{MA_JJ_files/figure-latex/HistFosMo-1.pdf}}
		\subfloat[Continental vs. insular]{\includegraphics[scale=0.45]{MA_JJ_files/figure-latex/HistCI-1.pdf}}	
		\caption[Fossil vs. modern, continental vs. insular.]{Histograms for fossil vs. modern and continental vs. insular data.}
		\label{fig:HistFMCI}
	\end{figure}
\end{center}

The histograms show a bimodal distribution, wich is also apparent on most sublevels, except for modern insular species (Fig. \ref{fig:HistRest} (a)).
Body size distributions are similar, right-skewed and bimodal, for the four continents and reflect the overall trend (Fig. \ref{fig:HistRest} (b)).




\FloatBarrier

%__________________________



Mean body size differs significantly across time bins (Kruskal Wallis Test, $\chi^2$ = 71.441, P < 0.01; Fig. \ref{fig:boxBins}). 
The multiple comparison test showed that modern median body size is smaller than body size in the Upper Pleistocene. %(Wilcoxon Rank Sum Test, W = 3853.5, P < 0.01 )
There is no difference in body size within the Pleistocene %(Wilcoxon Rank Sum Test, P > 0.05)
and Pleistocene body size does not differ from body size in the Upper Miocene%(Wilcoxon Rank Sum Test, P > 0.05)
. Serravallian body size is smaller than Langhian body size in the Middle Miocene%(Wilcoxon Rank Sum Test, W = 45, P < 0.01)
, but Langhian body size is not different from Lower Miocene body size%(Wilcoxon Rank Sum Test, W = 311, P = 0.06)
.


%__________________________________________________________________________
\begin{figure}[hbtp]
	\centering
	\includegraphics{MA_JJ_files/figure-latex/BPGBins-1.pdf}
	\caption{Boxplots of mean CL per time bin, including mean and sd CL for
		each genus (as pointrange).}
	\label{fig:boxBins}
\end{figure}

%\todo{random sampling necessary?}
% = 71.441, df = 11, p-value = 6.496e-11


%need statistics for boxplot!! kruskal-wallis-test plus post-hoc test (+ bonferroni correction?) or moving mean (faysal?)
%you can kind of see how the median increases and then decreases again
%(compare in order: Modern < Upper Pleistocene etc., do median and variance??)

% FOR ALL TIME STAGES

%	Kruskal-Wallis rank sum test

%data:  list(M, UPle, MPle, LPle, G, Pia, Z, Mess, Tort, S, L, BA)
%Kruskal-Wallis chi-squared = 71.441, df = 11, p-value = 6.496e-11

%Multiple comparison test after Kruskal-Wallis 
%p.value: 0.05 
%Comparisons
%obs.dif critical.dif difference
%Modern-Upper Pleistocene                  116.987013     93.54915       TRUE
%Modern-Middle Pleistocene                  80.140652     90.54349      FALSE
%Modern-Lower Pleistocene                   66.123604     87.87753      FALSE
%Modern-Gelasian                             1.627566    114.05459      FALSE
%Modern-Piacencian                         113.296537    136.11314      FALSE
%Modern-Zanclean                           205.945804    123.43828       TRUE
%Modern-Messinian                          137.122727    193.24680      FALSE
%Modern-Tortonian                           61.739394     96.96976      FALSE
%Modern-Serravallian                        21.764310    121.34770      FALSE
%Modern-Langhian                           202.487013    164.56067       TRUE
%Modern-Burdigalian/Aquitanian              70.472727    115.73561      FALSE
%Upper Pleistocene-Middle Pleistocene       36.846361    118.78423      FALSE
%Upper Pleistocene-Lower Pleistocene        50.863409    116.76486      FALSE
%Upper Pleistocene-Gelasian                115.359447    137.55006      FALSE
%Upper Pleistocene-Piacencian                3.690476    156.32773      FALSE
%Upper Pleistocene-Zanclean                 88.958791    145.42551      FALSE
%Upper Pleistocene-Messinian                20.135714    207.98052      FALSE
%Upper Pleistocene-Tortonian                55.247619    123.75260      FALSE
%Upper Pleistocene-Serravallian            138.751323    143.65527      FALSE
%Upper Pleistocene-Langhian                 85.500000    181.63641      FALSE
%Upper Pleistocene-Burdigalian/Aquitanian   46.514286    138.94713      FALSE
%Middle Pleistocene-Lower Pleistocene       14.017047    114.37094      FALSE
%Middle Pleistocene-Gelasian                78.513086    135.52379      FALSE
%Middle Pleistocene-Piacencian              33.155885    154.54785      FALSE
%Middle Pleistocene-Zanclean               125.805152    143.51048      FALSE
%Middle Pleistocene-Messinian               56.982075    206.64601      FALSE
%Middle Pleistocene-Tortonian               18.401258    121.49644      FALSE
%Middle Pleistocene-Serravallian           101.904962    141.71632      FALSE
%Middle Pleistocene-Langhian               122.346361    180.10681      FALSE
%Middle Pleistocene-Burdigalian/Aquitanian   9.667925    136.94153      FALSE
%Lower Pleistocene-Gelasian                 64.496038    133.75738      FALSE
%Lower Pleistocene-Piacencian               47.172932    153.00123      FALSE
%Lower Pleistocene-Zanclean                139.822200    141.84356      FALSE
%Lower Pleistocene-Messinian                70.999123    205.49188      FALSE
%Lower Pleistocene-Tortonian                 4.384211    119.52289      FALSE
%Lower Pleistocene-Serravallian             87.887914    140.02804      FALSE
%Lower Pleistocene-Langhian                136.363409    178.78144      FALSE
%Lower Pleistocene-Burdigalian/Aquitanian    4.349123    135.19364      FALSE
%Gelasian-Piacencian                       111.668971    169.39706      FALSE
%Gelasian-Zanclean                         204.318238    159.39129       TRUE
%Gelasian-Messinian                        135.495161    217.97454      FALSE
%Gelasian-Tortonian                         60.111828    139.89893      FALSE
%Gelasian-Serravallian                      23.391876    157.77782      FALSE
%Gelasian-Langhian                         200.859447    192.99946       TRUE
%Gelasian-Burdigalian/Aquitanian            68.845161    153.50345      FALSE
%Piacencian-Zanclean                        92.649267    175.85199      FALSE
%Piacencian-Messinian                       23.826190    230.28513      FALSE
%Piacencian-Tortonian                       51.557143    158.39839      FALSE
%Piacencian-Serravallian                   135.060847    174.39088      FALSE
%Piacencian-Langhian                        89.190476    206.80215      FALSE
%Piacencian-Burdigalian/Aquitanian          42.823810    170.53342      FALSE
%Zanclean-Messinian                         68.823077    223.02794      FALSE
%Zanclean-Tortonian                        144.206410    147.64914      FALSE
%Zanclean-Serravallian                     227.710114    164.68880       TRUE
%Zanclean-Langhian                           3.458791    198.68908      FALSE
%Zanclean-Burdigalian/Aquitanian           135.473077    160.59847      FALSE
%Messinian-Tortonian                        75.383333    209.54137      FALSE
%Messinian-Serravallian                    158.887037    221.87771      FALSE
%Messinian-Langhian                         65.364286    248.16258      FALSE
%Messinian-Burdigalian/Aquitanian           66.650000    218.85882      FALSE
%Tortonian-Serravallian                     83.503704    145.90588      FALSE
%Tortonian-Langhian                        140.747619    183.42158      FALSE
%Tortonian-Burdigalian/Aquitanian            8.733333    141.27276      FALSE
%Serravallian-Langhian                     224.251323    197.39708       TRUE
%Serravallian-Burdigalian/Aquitanian        92.237037    158.99725      FALSE
%Langhian-Burdigalian/Aquitanian           132.014286    193.99761      FALSE


% FOR MODERN - PLEISTOCENE - PLIOCENE - MIOCENE
%Kruskal-Wallis rank sum test

%data:  list(Modern, Plei, Plio, Mio)
%Kruskal-Wallis chi-squared = 37.764, df = 3, p-value = 3.172e-08


%Multiple comparison test after Kruskal-Wallis 
%p.value: 0.05 
%Comparisons
%obs.dif critical.dif difference
%Modern-Pleistocene   110.904114     49.80480       TRUE
%Modern-Pliocene       67.623302     58.35513       TRUE
%Modern-Miocene        64.510137     49.57182       TRUE
%Pleistocene-Pliocene  43.280812     64.36704      FALSE
%Pleistocene-Miocene   46.393977     56.52575      FALSE
%Pliocene-Miocene       3.113165     64.18694      FALSE



%__________________________________________________________________________
\begin{center}
	\begin{figure}[htbp]
		\subfloat[Fossil vs. modern]{\includegraphics[scale=0.45]{MA_JJ_files/figure-latex/BPMF-1.pdf}}
		\subfloat[Continental vs. insular]{\includegraphics[scale=0.45]{MA_JJ_files/figure-latex/BPCI-1.pdf}}
		\caption{Boxplots of CL split into fossil vs. modern (a) and cotinental vs. insular (b)}
		\label{fig:boxFMCI}
	\end{figure}
\end{center}



Comparison of modern and fossil testudinids showed that modern tortoises are significantly smaller than fossil ones (Wilcoxon Rank Sum Test, W = 22318, P < 0.01; Fig. \ref{fig:boxFMCI}). Furthermore, continental testudinids are significantly smaller than insular taxa (Wilcoxon Rank Sum Test, W = 13854, P < 0.01; Fig. \ref{fig:boxFMCI}).
These results can even be considered in combination: modern continental taxa are smaller than fossil continental taxa (Wilcoxon Rank Sum Test, W = 8046, P < 0.01; Fig. \ref{BoxFoMCI}) and modern insular taxa are smaller than fossil insular taxa (Wilcoxon Rank Sum Test, W = 631.5, P < 0.01; Fig. \ref{BoxFoMCI}))

Finally, body size differs among continents (Kruskal Wallis Test, $\chi^2$ = 34.343, P < 0.01; Fig. \ref{fig:boxCon}). The multiple comparison test showed that African testudinids differ significantly from the other three continents in body size. American testudinid body size is comparable to that of Asia, but differs from those of Africa and Europe. Furthermore, Asian and European testidinids are similar in body size. %Since only Europe/Eurasia are well sampled, these relationships could change with further sampling. ??


\begin{figure}[htbp]
	\centering
	\includegraphics[width=0.75\textwidth]{MA_JJ_files/figure-latex/BPCon-1.pdf}
	\caption{Boxplot: body size on different continents, genera summarised}
	\label{fig:boxCon}
\end{figure}



\FloatBarrier
%__________________________________________________________________________

\subsection{paleoTS analysis}\label{paleots-analysis}




\subsubsection{complete dataset}\label{all-continental-and-insular}

Fitting of the three evolutionary models favoured stasis for the entire data set, although model support was only 51\,\% followed by 33\,\% support for the unbiased random walk (Fig. \ref{fig:pTSall}, Table \ref{tab:pTSall}). When solely considering continental genera, the best-fitting model was the unbiased random walk, but again not ideally supported with 55\,\% followed by a modest model support of 30\,\% for generalized random walk (Fig. \ref{fig:pTSC}, Table \ref{tab:pTSCEM}). In contrast, insular genera are best described by stasis, which was very well supported (100\,\%; Fig. \ref{fig:pTSI}, Table \ref{tab:pTSIEM})). 


\begin{figure}[H]
	\centering
	\includegraphics{MA_JJ_files/figure-latex/paleoTSAll-1.pdf}
	\caption{paleoTS plot with genus mean, all}
	\label{fig:pTSall}
\end{figure}

\begin{longtable}[]{@{}lrrrr@{}}
	\caption{Model-fitting results for testudinidae, genera,
		all}
	\label{tab:pTSallEM}\tabularnewline
	\toprule
	& logL & K & AICc & Akaike.wt\tabularnewline
	\midrule
	\endfirsthead
	\toprule
	& logL & K & AICc & Akaike.wt\tabularnewline
	\midrule
	\endhead
	GRW & -81.31790 & 2 & 167.9691 & 0.161\tabularnewline
	URW & -82.05721 & 1 & 166.5144 & 0.332\tabularnewline
	Stasis & -80.16802 & 2 & 165.6694 & 0.507\tabularnewline
	\bottomrule
\end{longtable}

\FloatBarrier
%__________________________________________________________________________

\subsubsection{continental dataset (excluding insular
	species)}\label{continental-excluding-insular-species}


\begin{figure}[H]
	\centering
	\includegraphics{MA_JJ_files/figure-latex/paleoTSC-1.pdf}
	\caption{paleoTS plot with genus mean, continental}
	\label{fig:pTSC}
\end{figure}

\begin{longtable}[]{@{}lrrrr@{}}
	\caption{Model-fitting results for testudinidae, genera,
		continental}
	\label{tab:pTSCEM}\tabularnewline
	\toprule
	& logL & K & AICc & Akaike.wt\tabularnewline
	\midrule
	\endfirsthead
	\toprule
	& logL & K & AICc & Akaike.wt\tabularnewline
	\midrule
	\endhead
	GRW & -82.26287 & 2 & 169.8591 & 0.300\tabularnewline
	URW & -83.12577 & 1 & 168.6515 & 0.548\tabularnewline
	Stasis & -82.93984 & 2 & 171.2130 & 0.152\tabularnewline
	\bottomrule
\end{longtable}


\FloatBarrier
%__________________________________________________________________________

\subsubsection{insular dataset (excluding
	continental)}\label{insular-excluding-continental}




\begin{figure}[H]
	\centering
	\includegraphics{MA_JJ_files/figure-latex/paleoTSI-1.pdf}
	\caption{paleoTS plot with genus mean, insular}
	\label{fig:pTSI}
\end{figure}

\begin{longtable}[]{@{}lrrrr@{}}
	\caption{Model-fitting results for testudinidae, genera,
		insular}
	\label{tab:pTSIEM}\tabularnewline
	\toprule
	& logL & K & AICc & Akaike.wt\tabularnewline
	\midrule
	\endfirsthead
	\toprule
	& logL & K & AICc & Akaike.wt\tabularnewline
	\midrule
	\endhead
	GRW & -68.57344 & 2 & 143.5469 & 0\tabularnewline
	URW & -75.76576 & 1 & 154.1982 & 0\tabularnewline
	Stasis & -60.41581 & 2 & 127.2316 & 1\tabularnewline
	\bottomrule
\end{longtable}

\FloatBarrier
%__________________________________________________________________________

\subsubsection{per continent}\label{per-continent}

\paragraph{Europe, genera}\label{europe-genera}

When repeating the analysis for European taxa only, all three groups -- complete, continental and insular data -- are best described by stasis with a model support between 92\,-\,99\,\% (Fig. \ref{fig:pTSEu}, \ref{fig:pTSEuC}, \ref{fig:pTSEuI}; Tables \ref{tab:pTSEuEM}, \ref{tab:pTSEuCEM}, \ref{tab:pTSEuIEM}).

\begin{figure}[H]
	\centering
	\includegraphics{MA_JJ_files/figure-latex/paleoTSEurope-1.pdf}
	\caption{Genera, Europe}
	\label{fig:pTSEu}
\end{figure}

\begin{longtable}[]{@{}lrrrr@{}}
	\caption{Model-fitting results for testudinidae, genera,
		Europe}
	\label{tab:pTSEuEM}\tabularnewline
	\toprule
	& logL & K & AICc & Akaike.wt\tabularnewline
	\midrule
	\endfirsthead
	\toprule
	& logL & K & AICc & Akaike.wt\tabularnewline
	\midrule
	\endhead
	GRW & -84.14010 & 2 & 173.7802 & 0.006\tabularnewline
	URW & -85.90727 & 1 & 174.2590 & 0.005\tabularnewline
	Stasis & -79.01365 & 2 & 163.5273 & 0.990\tabularnewline
	\bottomrule
\end{longtable}

\FloatBarrier
%__________________________________________________________________________



\paragraph{Eurasia,	genera}\label{eurasia-genera}


For Eurasia, the entire data as well as continental genera are best described by the unbiased random walk, although the model support is weak again. Continental species still have a better support (78\,\%; Fig. \ref{fig:pTSEsC}, Table \ref{tab:pTSEsCEM}) than all Eurasian data with only 56 \% (Fig. \ref{fig:pTSEs}, Table \ref{tab:pTSEsEM}). Insular Eurasian species, however, conform to stasis again, although with lower support values (68\,\%; Fig. \ref{fig:pTSEsI}, Table \ref{tab:pTSEsIEM}).



\begin{figure}[H]
	\centering
	\includegraphics{MA_JJ_files/figure-latex/paleoTSEurasia-1.pdf}
	\caption{paleoTS, genera, Eurasia}
	\label{fig:pTSEs}
\end{figure}

\begin{longtable}[]{@{}lrrrr@{}}
	\caption{Model-fitting results for testudinidae, genera,
		Eurasia}
	\label{tab:pTSEsEM}\tabularnewline
	\toprule
	& logL & K & AICc & Akaike.wt\tabularnewline
	\midrule
	\endfirsthead
	\toprule
	& logL & K & AICc & Akaike.wt\tabularnewline
	\midrule
	\endhead
	GRW & -85.25195 & 2 & 175.8372 & 0.149\tabularnewline
	URW & -85.39072 & 1 & 173.1814 & 0.562\tabularnewline
	Stasis & -84.58890 & 2 & 174.5111 & 0.289\tabularnewline
	\bottomrule
\end{longtable}



\FloatBarrier
%__________________________________________________________________________


\newpage
\section{Discussion}

\subsection{Completeness of data set}

completeness of data set/benefits of additional sampling (SACs)
- how much of the "actual" data is represented by our data set?




\subsection{Population structure?}


\subsection{Time-scale analysis}

--> what does model support depend on? what does a relatively low model support mean?


% LIMITATIONS
%\todo{add at some point that phylogenetics were not considered because not enough data for fossil species --> or only in discussion? full-evidence analysis would be nice as in Slater et al., 2017 (baleen whales) --> cite Lapparent de Broin, 2006 + 2001: phylogenetic relationships between genera are not definitely established, }


- study would benefit from more sampling! (SAC, model supports paleoTS)

- include more data, not only literature but actually measure shells/other skeletal elements --> maybe gather further data to more reliably estimate body sizes

- include shapes/geometric morphometrics --> volume/body mass?

- paleoTS --> is designed to deal with incomplete data (FOSSIL)

- unbiased random walk on continents --> CL fluctuates more than on islands --> "giant" forms completetely disappeared in comparison to insular species

- what can influence distribution: climate, selective pressures, diet, intra-specific competition (Madagascar)

- I guess: climate affects body size reduction, but extinctions were human-driven
--> Aldabra/Galapagos (whaling industry)
--> mammal megafauna was hunted to extinction by humans




%____________________________________________________

Aldabra tortoise: no evidence of size increase, probably originate from giant continental tortoises (large tortoises are able to float: bouyancy and fasting endurance)
\newpage
%\bibliography{file}
\FloatBarrier
\begin{appendices}
%\section{Appendix}
\section{Sampling Accumulation Curves}

\begin{center}
	\begin{figure}[H]
		\subfloat[Species per locality]{\includegraphics[scale=0.3]{MA_JJ_files/figure-latex/SACSpecies-1.pdf}}
		\subfloat[Species per reference]{\includegraphics[scale=0.3]{MA_JJ_files/figure-latex/SACSpecies-2.pdf}}	\subfloat[Africa]{\includegraphics[scale=0.3]{MA_JJ_files/figure-latex/SACGAfrica-1.pdf}}
		\hfill %
		\subfloat[America]{\includegraphics[scale=0.3]{MA_JJ_files/figure-latex/SACGAmerica-1.pdf}}
		\subfloat[North America]{\includegraphics[scale=0.3]{MA_JJ_files/figure-latex/SACGNAmerica-1.pdf}}
		\subfloat[South America]{\includegraphics[scale=0.3]{MA_JJ_files/figure-latex/SACGSAmerica-1.pdf}}
		\hfill %
		\subfloat[Asia]{\includegraphics[scale=0.3]{MA_JJ_files/figure-latex/SACGAsia-1.pdf}}
		\subfloat[Europe]{\includegraphics[scale=0.3]{MA_JJ_files/figure-latex/SACGEurope-1.pdf}}
		\subfloat[Eurasia]{\includegraphics[scale=0.3]{MA_JJ_files/figure-latex/SACGEurasia-1.pdf}}
		\caption[Additional Sampling Accumulation Curves]{Sampling Accumulation Curves: (a) - (b) Species are not sufficiently sampled, regardless of sampling unit. (c) - (i) Sampling Accumulation Curves on generic level per continent. Only Europe (h) and Eurasia (i) are sufficiently sampled.}
		\label{fig:Hists}
	\end{figure}
\end{center}

\FloatBarrier

\newpage
\section{Histograms}


\begin{center}
	\begin{figure}[H]
		\subfloat[Raw data]{\includegraphics[scale=0.45]{MA_JJ_files/figure-latex/normalDistribution-1.pdf}}
		\subfloat[Logtransformed data]{\includegraphics[scale=0.45]{MA_JJ_files/figure-latex/normalDistribution-2.pdf}}	
		\caption[Testing normal distribution]{Visual test for normal distribution. In case of normally distributed data, the black circles should follow the red line, which is not the case for either raw data (a) nor logtransformed data (b). Therefore, data is not normally distributed.}
		\label{fig:NormDis}
	\end{figure}
\end{center}


\begin{center}
	\begin{figure}[H]
		\subfloat[Fossil vs. modern, continental vs. insular]{\includegraphics[scale=0.45]{MA_JJ_files/figure-latex/HistFMCI-1.pdf}}
		\subfloat[Per continent]{\includegraphics[scale=0.45]{MA_JJ_files/figure-latex/HistCon-1.pdf}}	
		\caption[Additional histograms]{Histograms for several subgroups of the dataset.}
		\label{fig:HistRest}
	\end{figure}
\end{center}
\section{Boxplots}


\begin{figure}[htbp]
	\centering
	\includegraphics{MA_JJ_files/figure-latex/BPFMCI-1.pdf}
	\caption{Boxplots fossil vs.~modern, continental vs.~insular species.}
	\label{BoxFoMCI}
\end{figure}


Wilcoxon Rank Sum Test (unpaired data):

modern continental \textless{} fossil continental (P =
\(4.8532266\times 10^{-8}\))

modern insular \textless{} fossil insular (P = \(0.0018564\))
%Kruskal-Wallis-Test:

%Continent means differ (P = \(1.0833256\times 10^{-6}\)) (still have to
%look into the details\ldots{})

%\begin{figure}[htbp]
%	\centering
%	\includegraphics{MA_JJ_files/figure-latex/Box_FM_timebin.pdf}
%	\caption{Boxplots time bins, continental vs.~insular species.}
%	\label{BoxTBCI}
%\end{figure}


%no longer in R-Script, look in old scripts, if it should stay here!!!

% continental
%Multiple comparison test after Kruskal-Wallis 
%p.value: 0.05 
%Comparisons
%obs.dif critical.dif difference
%Modern-Upper Pleistocene                  15.0333333     41.09142      FALSE
%Modern-Middle Pleistocene                  5.2666667     43.92857      FALSE
%Modern-Lower Pleistocene                   8.7952381     38.93852      FALSE
%Modern-Gelasian                           32.7238095     38.93852      FALSE
%Modern-Piacencian                         20.0333333     41.09142      FALSE
%Modern-Zanclean                           27.0333333     41.09142      FALSE
%Modern-Messinian                          32.8666667     47.87005      FALSE
%Modern-Tortonian                          29.2555556     35.86753      FALSE
%Modern-Serravallian                       20.3666667     41.09142      FALSE
%Modern-Langhian                           30.7238095     38.93852      FALSE
%Modern-Burdigalian/Aquitanian             14.7555556     35.86753      FALSE
%Upper Pleistocene-Middle Pleistocene       9.7666667     51.51082      FALSE
%Upper Pleistocene-Lower Pleistocene        6.2380952     47.32708      FALSE
%Upper Pleistocene-Gelasian                17.6904762     47.32708      FALSE
%Upper Pleistocene-Piacencian               5.0000000     49.11364      FALSE
%Upper Pleistocene-Zanclean                12.0000000     49.11364      FALSE
%Upper Pleistocene-Messinian               17.8333333     54.91071      FALSE
%Upper Pleistocene-Tortonian               14.2222222     44.83441      FALSE
%Upper Pleistocene-Serravallian             5.3333333     49.11364      FALSE
%Upper Pleistocene-Langhian                15.6904762     47.32708      FALSE
%Upper Pleistocene-Burdigalian/Aquitanian   0.2777778     44.83441      FALSE
%Middle Pleistocene-Lower Pleistocene       3.5285714     49.81032      FALSE
%Middle Pleistocene-Gelasian               27.4571429     49.81032      FALSE
%Middle Pleistocene-Piacencian             14.7666667     51.51082      FALSE
%Middle Pleistocene-Zanclean               21.7666667     51.51082      FALSE
%Middle Pleistocene-Messinian              27.6000000     57.06489      FALSE
%Middle Pleistocene-Tortonian              23.9888889     47.44828      FALSE
%Middle Pleistocene-Serravallian           15.1000000     51.51082      FALSE
%Middle Pleistocene-Langhian               25.4571429     49.81032      FALSE
%Middle Pleistocene-Burdigalian/Aquitanian  9.4888889     47.44828      FALSE
%Lower Pleistocene-Gelasian                23.9285714     45.47039      FALSE
%Lower Pleistocene-Piacencian              11.2380952     47.32708      FALSE
%Lower Pleistocene-Zanclean                18.2380952     47.32708      FALSE
%Lower Pleistocene-Messinian               24.0714286     53.31876      FALSE
%Lower Pleistocene-Tortonian               20.4603175     42.86990      FALSE
%Lower Pleistocene-Serravallian            11.5714286     47.32708      FALSE
%Lower Pleistocene-Langhian                21.9285714     45.47039      FALSE
%Lower Pleistocene-Burdigalian/Aquitanian   5.9603175     42.86990      FALSE
%Gelasian-Piacencian                       12.6904762     47.32708      FALSE
%Gelasian-Zanclean                          5.6904762     47.32708      FALSE
%Gelasian-Messinian                         0.1428571     53.31876      FALSE
%Gelasian-Tortonian                         3.4682540     42.86990      FALSE
%Gelasian-Serravallian                     12.3571429     47.32708      FALSE
%Gelasian-Langhian                          2.0000000     45.47039      FALSE
%Gelasian-Burdigalian/Aquitanian           17.9682540     42.86990      FALSE
%Piacencian-Zanclean                        7.0000000     49.11364      FALSE
%Piacencian-Messinian                      12.8333333     54.91071      FALSE
%Piacencian-Tortonian                       9.2222222     44.83441      FALSE
%Piacencian-Serravallian                    0.3333333     49.11364      FALSE
%Piacencian-Langhian                       10.6904762     47.32708      FALSE
%Piacencian-Burdigalian/Aquitanian          5.2777778     44.83441      FALSE
%Zanclean-Messinian                         5.8333333     54.91071      FALSE
%Zanclean-Tortonian                         2.2222222     44.83441      FALSE
%Zanclean-Serravallian                      6.6666667     49.11364      FALSE
%Zanclean-Langhian                          3.6904762     47.32708      FALSE
%Zanclean-Burdigalian/Aquitanian           12.2777778     44.83441      FALSE
%Messinian-Tortonian                        3.6111111     51.11909      FALSE
%Messinian-Serravallian                    12.5000000     54.91071      FALSE
%Messinian-Langhian                         2.1428571     53.31876      FALSE
%Messinian-Burdigalian/Aquitanian          18.1111111     51.11909      FALSE
%Tortonian-Serravallian                     8.8888889     44.83441      FALSE
%Tortonian-Langhian                         1.4682540     42.86990      FALSE
%Tortonian-Burdigalian/Aquitanian          14.5000000     40.10112      FALSE
%Serravallian-Langhian                     10.3571429     47.32708      FALSE
%Serravallian-Burdigalian/Aquitanian        5.6111111     44.83441      FALSE
%Langhian-Burdigalian/Aquitanian           15.9682540     42.86990      FALSE


% insular
%Multiple comparison test after Kruskal-Wallis 
%p.value: 0.05 
%Comparisons
%obs.dif critical.dif difference
%Modern-Upper Pleistocene                  10.0833333     19.92445      FALSE
%Modern-Middle Pleistocene                  9.8333333     22.27622      FALSE
%Modern-Lower Pleistocene                  12.3333333     17.25508      FALSE
%Modern-Gelasian                           12.5000000     22.27622      FALSE
%Modern-Piacencian                          1.8333333     19.92445      FALSE
%Modern-Zanclean                           13.8333333     26.35757      FALSE
%Modern-Messinian                          22.8333333     35.91932      FALSE
%Modern-Tortonian                                 NaN          Inf         NA
%Modern-Serravallian                              NaN          Inf         NA
%Modern-Langhian                                  NaN          Inf         NA
%Modern-Burdigalian/Aquitanian                    NaN          Inf         NA
%Upper Pleistocene-Middle Pleistocene       0.2500000     26.35757      FALSE
%Upper Pleistocene-Lower Pleistocene        2.2500000     22.27622      FALSE
%Upper Pleistocene-Gelasian                 2.4166667     26.35757      FALSE
%Upper Pleistocene-Piacencian               8.2500000     24.40237      FALSE
%Upper Pleistocene-Zanclean                 3.7500000     29.88668      FALSE
%Upper Pleistocene-Messinian               12.7500000     38.58354      FALSE
%Upper Pleistocene-Tortonian                      NaN          Inf         NA
%Upper Pleistocene-Serravallian                   NaN          Inf         NA
%Upper Pleistocene-Langhian                       NaN          Inf         NA
%Upper Pleistocene-Burdigalian/Aquitanian         NaN          Inf         NA
%Middle Pleistocene-Lower Pleistocene       2.5000000     24.40237      FALSE
%Middle Pleistocene-Gelasian                2.6666667     28.17743      FALSE
%Middle Pleistocene-Piacencian              8.0000000     26.35757      FALSE
%Middle Pleistocene-Zanclean                4.0000000     31.50333      FALSE
%Middle Pleistocene-Messinian              13.0000000     39.84891      FALSE
%Middle Pleistocene-Tortonian                     NaN          Inf         NA
%Middle Pleistocene-Serravallian                  NaN          Inf         NA
%Middle Pleistocene-Langhian                      NaN          Inf         NA
%Middle Pleistocene-Burdigalian/Aquitanian        NaN          Inf         NA
%Lower Pleistocene-Gelasian                 0.1666667     24.40237      FALSE
%Lower Pleistocene-Piacencian              10.5000000     22.27622      FALSE
%Lower Pleistocene-Zanclean                 1.5000000     28.17743      FALSE
%Lower Pleistocene-Messinian               10.5000000     37.27524      FALSE
%Lower Pleistocene-Tortonian                      NaN          Inf         NA
%Lower Pleistocene-Serravallian                   NaN          Inf         NA
%Lower Pleistocene-Langhian                       NaN          Inf         NA
%Lower Pleistocene-Burdigalian/Aquitanian         NaN          Inf         NA
%Gelasian-Piacencian                       10.6666667     26.35757      FALSE
%Gelasian-Zanclean                          1.3333333     31.50333      FALSE
%Gelasian-Messinian                        10.3333333     39.84891      FALSE
%Gelasian-Tortonian                               NaN          Inf         NA
%Gelasian-Serravallian                            NaN          Inf         NA
%Gelasian-Langhian                                NaN          Inf         NA
%Gelasian-Burdigalian/Aquitanian                  NaN          Inf         NA
%Piacencian-Zanclean                       12.0000000     29.88668      FALSE
%Piacencian-Messinian                      21.0000000     38.58354      FALSE
%Piacencian-Tortonian                             NaN          Inf         NA
%Piacencian-Serravallian                          NaN          Inf         NA
%Piacencian-Langhian                              NaN          Inf         NA
%Piacencian-Burdigalian/Aquitanian                NaN          Inf         NA
%Zanclean-Messinian                         9.0000000     42.26615      FALSE
%Zanclean-Tortonian                               NaN          Inf         NA
%Zanclean-Serravallian                            NaN          Inf         NA
%Zanclean-Langhian                                NaN          Inf         NA
%Zanclean-Burdigalian/Aquitanian                  NaN          Inf         NA
%Messinian-Tortonian                              NaN          Inf         NA
%Messinian-Serravallian                           NaN          Inf         NA
%Messinian-Langhian                               NaN          Inf         NA
%Messinian-Burdigalian/Aquitanian                 NaN          Inf         NA
%Tortonian-Serravallian                           NaN          Inf         NA
%Tortonian-Langhian                               NaN          Inf         NA
%Tortonian-Burdigalian/Aquitanian                 NaN          Inf         NA
%Serravallian-Langhian                            NaN          Inf         NA
%Serravallian-Burdigalian/Aquitanian              NaN          Inf         NA
%Langhian-Burdigalian/Aquitanian                  NaN          Inf         NA



\section{Random Sampling}


\begin{center}
	\begin{figure}[H]
		\subfloat[Fossil]{\includegraphics[scale=0.45]{MA_JJ_files/figure-latex/RSFM-1.pdf}}
		\subfloat[Modern, insular]{\includegraphics[scale=0.45]{MA_JJ_files/figure-latex/RSMFCI-1.pdf}}	
		\hfill %
		\subfloat[Fossil, continental]{\includegraphics[scale=0.45]{MA_JJ_files/figure-latex/RSMFCI-2.pdf}}
		\subfloat[Continental]{\includegraphics[scale=0.45]{MA_JJ_files/figure-latex/RSCI-1.pdf}}
		\caption[Random sampling]{Random sampling for several subgroups. For (a), (c), and (d) the random sample reflects the real sample, for (b) this is not the case.}
		\label{fig:RSadd}
	\end{figure}
\end{center}



\begin{center}
	\begin{figure}[H]
		\subfloat[America]{\includegraphics[scale=0.3]{MA_JJ_files/figure-latex/RSCon-1.pdf}}
		\subfloat[Europe]{\includegraphics[scale=0.3]{MA_JJ_files/figure-latex/RSCon-2.pdf}}	
		\subfloat[Eurasia]{\includegraphics[scale=0.3]{MA_JJ_files/figure-latex/RSCon-3.pdf}}
		\caption[Random sampling, continents]{Random sampling for different continents. All random samples reflect the real sample.}
		\label{fig:RSadd}
	\end{figure}
\end{center}


\section{Tables}
\begin{longtable}[]{@{}llrr@{}}
	\caption[Species per time bins]{Overview over fossil species per time bin, with sample size and
		mean CL.}
	\label{tab:SpecBins}\tabularnewline
	\toprule
	EpochBins & Taxon & n & meanCL\tabularnewline
	\midrule
	\endfirsthead
	\toprule
	EpochBins & Taxon & n & meanCL\tabularnewline
	\midrule
	\endhead
	Upper Pleistocene & Centrochelys robusta & 1 & 850.0000\tabularnewline
	Upper Pleistocene & Chelonoidis denticulata & 1 &
	616.0000\tabularnewline
	Upper Pleistocene & Chelonoidis lutzae & 1 & 830.0000\tabularnewline
	Upper Pleistocene & Chelonoidis marcanoi & 4 & 672.2500\tabularnewline
	Upper Pleistocene & Chelonoidis monensis & 1 & 500.0000\tabularnewline
	Upper Pleistocene & Chelonoidis sombrerensis & 1 &
	990.0000\tabularnewline
	Upper Pleistocene & Chelonoidis sp. & 3 & 666.6667\tabularnewline
	Upper Pleistocene & Eurotestudo hermanni & 1 & 187.0000\tabularnewline
	Upper Pleistocene & gen. indet. & 1 & 813.0000\tabularnewline
	Upper Pleistocene & Geochelone sp. & 2 & 475.0000\tabularnewline
	Upper Pleistocene & Gopherus agassizi & 1 & 252.0000\tabularnewline
	Upper Pleistocene & Gopherus polyphemus & 20 & 292.9700\tabularnewline
	Upper Pleistocene & Gopherus praecedens & 1 & 360.0000\tabularnewline
	Upper Pleistocene & Hesperotestudo crassiscutata & 6 &
	435.1667\tabularnewline
	Upper Pleistocene & Hesperotestudo incisa & 1 & 232.7600\tabularnewline
	Upper Pleistocene & Hesperotestudo sp. & 2 & 806.5000\tabularnewline
	Upper Pleistocene & Hesperotestudo wilsoni & 1 & 226.0000\tabularnewline
	Upper Pleistocene & Indotestudo elongata & 1 & 270.0000\tabularnewline
	Middle Pleistocene & Centrochelys burchardi & 4 &
	722.5000\tabularnewline
	Middle Pleistocene & Chelonoidis cubensis & 1 & 1139.0000\tabularnewline
	Middle Pleistocene & Eurotestudo aff. hermanni & 2 &
	187.0000\tabularnewline
	Middle Pleistocene & Eurotestudo hermanni & 2 & 204.0500\tabularnewline
	Middle Pleistocene & Geochelone sp. & 1 & 170.0000\tabularnewline
	Middle Pleistocene & Gopherus agassizi & 1 & 445.0000\tabularnewline
	Middle Pleistocene & Gopherus laticaudatus & 1 & 375.0000\tabularnewline
	Middle Pleistocene & Gopherus polyphemus & 31 & 300.4316\tabularnewline
	Middle Pleistocene & Hesperotestudo bermudae & 2 &
	385.0000\tabularnewline
	Middle Pleistocene & Hesperotestudo equicomes & 1 &
	340.0000\tabularnewline
	Middle Pleistocene & Hesperotestudo sp. & 2 & 1650.0000\tabularnewline
	Middle Pleistocene & Testudo kenitrensis & 1 & 132.0000\tabularnewline
	Middle Pleistocene & Testudo lunellensis & 4 & 215.4250\tabularnewline
	Lower Pleistocene & Centrochelys atlantica & 1 & 400.0000\tabularnewline
	Lower Pleistocene & Centrochelys robusta & 3 & 883.3333\tabularnewline
	Lower Pleistocene & Cheirogaster cf.~gymnesica & 1 &
	789.0000\tabularnewline
	Lower Pleistocene & Cheirogaster sp. & 1 & 925.0000\tabularnewline
	Lower Pleistocene & Chelonoidis sp. & 3 & 716.6667\tabularnewline
	Lower Pleistocene & Eurotestudo globosa & 1 & 263.0000\tabularnewline
	Lower Pleistocene & Eurotestudo hermanni & 2 & 205.0000\tabularnewline
	Lower Pleistocene & gen. indet. & 1 & 900.0000\tabularnewline
	Lower Pleistocene & Geochelone sp. & 1 & 340.0000\tabularnewline
	Lower Pleistocene & Gopherus berlandieri & 2 & 225.6500\tabularnewline
	Lower Pleistocene & Gopherus flavomarginatus & 1 &
	450.0000\tabularnewline
	Lower Pleistocene & Gopherus pertenuis & 1 & 1050.0000\tabularnewline
	Lower Pleistocene & Gopherus polyphemus & 3 & 254.4667\tabularnewline
	Lower Pleistocene & Gopherus sp. & 6 & 233.9667\tabularnewline
	Lower Pleistocene & Hesperotestudo crassiscutata & 5 &
	285.6000\tabularnewline
	Lower Pleistocene & Hesperotestudo incisa & 7 & 234.6286\tabularnewline
	Lower Pleistocene & Hesperotestudo mlynarskii & 2 &
	184.2500\tabularnewline
	Lower Pleistocene & Hesperotestudo sp. & 1 & 1500.0000\tabularnewline
	Lower Pleistocene & Hesperotestudo turgida & 1 & 230.0000\tabularnewline
	Lower Pleistocene & Megalochelys sondaari & 2 & 909.0000\tabularnewline
	Lower Pleistocene & Megalochelys sp. & 3 & 1130.4667\tabularnewline
	Lower Pleistocene & Psammobates antiquorum & 1 & 107.8000\tabularnewline
	Lower Pleistocene & Testudo changshanesis & 1 & 330.0000\tabularnewline
	Lower Pleistocene & Testudo graeca & 1 & 195.0000\tabularnewline
	Lower Pleistocene & Testudo hermanni & 2 & 176.5500\tabularnewline
	Lower Pleistocene & Testudo marginata & 3 & 270.0000\tabularnewline
	Lower Pleistocene & Titanochelon gymnesica & 1 &
	1300.0000\tabularnewline
	Gelasian & Centrochelys marocana & 1 & 2050.0000\tabularnewline
	Gelasian & Eurotestudo cf.~hermanni & 1 & 150.0000\tabularnewline
	Gelasian & Gopherus sp. & 15 & 185.7467\tabularnewline
	Gelasian & Hesperotestudo campester & 1 & 1000.0000\tabularnewline
	Gelasian & Hesperotestudo sp. & 1 & 1000.0000\tabularnewline
	Gelasian & Manouria punjabiensis & 1 & 900.0000\tabularnewline
	Gelasian & Megalochelys atlas & 3 & 1683.3333\tabularnewline
	Gelasian & Testudo aff. kenitrensis & 1 & 142.0000\tabularnewline
	Gelasian & Testudo oughlamensis & 1 & 120.0000\tabularnewline
	Gelasian & Testudo ranovi & 1 & 200.0000\tabularnewline
	Gelasian & Testudo sp. & 2 & 192.0000\tabularnewline
	Gelasian & Testudo transcaucasia & 1 & 150.0000\tabularnewline
	Gelasian & Titanochelon aff. schafferi & 1 & 1860.0000\tabularnewline
	Gelasian & Titanochelon sp. & 1 & 1420.0000\tabularnewline
	Piacencian & ``Aldabrachelys'' laetoliensis & 1 &
	1000.0000\tabularnewline
	Piacencian & Aldabrachelys ? sp. & 2 & 1500.0000\tabularnewline
	Piacencian & Centrochelys vulcanica & 1 & 610.0000\tabularnewline
	Piacencian & Chelonoidis alburyorum & 4 & 442.7500\tabularnewline
	Piacencian & Gopherus canyonensis & 1 & 885.5000\tabularnewline
	Piacencian & Hesperotestudo johnstoni & 1 & 235.0000\tabularnewline
	Piacencian & Hesperotestudo oelrichi & 1 & 283.8000\tabularnewline
	Piacencian & Hesperotestudo riggsi & 2 & 180.5000\tabularnewline
	Piacencian & Hesperotestudo sp. & 1 & 176.0000\tabularnewline
	Piacencian & Homopus fenestratus & 1 & 90.0000\tabularnewline
	Piacencian & Megalochelys atlas & 2 & 1600.0000\tabularnewline
	Piacencian & Testudo brevitesta & 2 & 232.5000\tabularnewline
	Piacencian & Testudo pecorinii & 1 & 225.0000\tabularnewline
	Piacencian & Titanochelon sp. & 1 & 520.0000\tabularnewline
	Zanclean & Caudochelys rexroadensis & 2 & 805.5000\tabularnewline
	Zanclean & Centrochelys robusta & 3 & 913.3333\tabularnewline
	Zanclean & Cheirogaster gymnesica & 1 & 739.0000\tabularnewline
	Zanclean & Ergilemys oskarkuhni & 2 & 209.0000\tabularnewline
	Zanclean & Geochelone crassa & 1 & 865.0000\tabularnewline
	Zanclean & Geochelone s. l. & 1 & 1750.0000\tabularnewline
	Zanclean & Geochelone sp. & 2 & 528.0000\tabularnewline
	Zanclean & Geochelone stromeri & 2 & 387.5000\tabularnewline
	Zanclean & Hesperotestudo riggsi & 1 & 195.8000\tabularnewline
	Zanclean & Testudo cf.~graeca & 1 & 185.0000\tabularnewline
	Zanclean & Testudo sp. & 4 & 1675.0000\tabularnewline
	Zanclean & Titanochelon bacharidisi & 4 & 1040.0000\tabularnewline
	Zanclean & Titanochelon perpiniana & 1 & 1140.0000\tabularnewline
	Zanclean & Titanochelon schafferi & 1 & 2500.0000\tabularnewline
	Messinian & Hesperotestudo orthopygia & 2 & 941.0000\tabularnewline
	Messinian & Megalochelys atlas & 2 & 1950.0000\tabularnewline
	Messinian & Testudo amiatae & 1 & 140.0000\tabularnewline
	Messinian & Testudo graeca & 2 & 183.5000\tabularnewline
	Messinian & Testudo sp. & 1 & 200.0000\tabularnewline
	Messinian & Titanochelon bolivari & 1 & 1150.0000\tabularnewline
	Messinian & Titanochelon schafferi & 1 & 1850.0000\tabularnewline
	Tortonian & ``Hadrianus sp.'' & 1 & 1000.0000\tabularnewline
	Tortonian & Cheirogaster richardi & 1 & 1155.0000\tabularnewline
	Tortonian & Cheirogaster sp. & 2 & 1355.0000\tabularnewline
	Tortonian & gen. indet. & 3 & 660.0000\tabularnewline
	Tortonian & Geochelone hesterna & 1 & 278.0000\tabularnewline
	Tortonian & Geochelone sp. & 2 & 973.0000\tabularnewline
	Tortonian & Gopherus ? sp. & 1 & 500.0000\tabularnewline
	Tortonian & Gopherus mohavetus & 5 & 324.8000\tabularnewline
	Tortonian & Hesperotestudo alleni & 1 & 240.9000\tabularnewline
	Tortonian & Hesperotestudo riggsi & 2 & 159.5000\tabularnewline
	Tortonian & Hesperotestudo sp. & 1 & 1200.0000\tabularnewline
	Tortonian & Paleotestudo sp. & 3 & 233.6667\tabularnewline
	Tortonian & Testudo burgenlandica & 2 & 193.5000\tabularnewline
	Tortonian & Testudo catalaunica & 4 & 157.0000\tabularnewline
	Tortonian & Testudo cf.~promarginata & 5 & 250.0000\tabularnewline
	Tortonian & Testudo graeca & 1 & 210.0000\tabularnewline
	Tortonian & Testudo s. s. & 1 & 189.0000\tabularnewline
	Tortonian & Testudo sp. & 7 & 243.1571\tabularnewline
	Tortonian & Titanochelon bolivari & 1 & 1300.0000\tabularnewline
	Tortonian & Titanochelon cf.~bolivari & 1 & 1500.0000\tabularnewline
	Serravallian & Cheirogaster sp. & 2 & 1250.0000\tabularnewline
	Serravallian & gen. indet. & 1 & 270.0000\tabularnewline
	Serravallian & Gopherus ? sp. & 1 & 500.0000\tabularnewline
	Serravallian & Paleotestudo antiqua & 18 & 203.0556\tabularnewline
	Serravallian & Paleotestudo cf.~sp. & 1 & 270.0000\tabularnewline
	Serravallian & Testudo catalaunica & 1 & 232.0000\tabularnewline
	Serravallian & Testudo steinheimensis & 2 & 169.3500\tabularnewline
	Serravallian & Titanochelon bolivari & 1 & 1353.0000\tabularnewline
	Langhian & Caudochelys ducateli & 1 & 339.9000\tabularnewline
	Langhian & Chelonoidis sp. & 3 & 553.3333\tabularnewline
	Langhian & Ergilemys sp. & 1 & 1000.0000\tabularnewline
	Langhian & gen. indet. & 1 & 1000.0000\tabularnewline
	Langhian & Paleotestudo antiqua & 1 & 275.0000\tabularnewline
	Langhian & Paleotestudo cf.~sp. & 1 & 270.0000\tabularnewline
	Langhian & Testudo kalksburgensis & 1 & 275.0000\tabularnewline
	Langhian & Testudo sp. & 1 & 400.0000\tabularnewline
	Langhian & Titanochelon bolivari & 2 & 1175.0000\tabularnewline
	Langhian & Titanochelon cf.~bolivari & 2 & 1450.0000\tabularnewline
	Burdigalian/Aquitanian & Caudochelys williamsi & 1 &
	334.0000\tabularnewline
	Burdigalian/Aquitanian & gen. indet. & 1 & 270.0000\tabularnewline
	Burdigalian/Aquitanian & Geochelone sp. & 2 & 900.0000\tabularnewline
	Burdigalian/Aquitanian & Geochelone tedwhitei & 2 &
	405.0000\tabularnewline
	Burdigalian/Aquitanian & Impregnochelys pachytectis & 1 &
	620.0000\tabularnewline
	Burdigalian/Aquitanian & Mesocherus orangeus & 5 &
	180.0000\tabularnewline
	Burdigalian/Aquitanian & Namibchersus aff. namaquensis & 3 &
	696.6667\tabularnewline
	Burdigalian/Aquitanian & Namibchersus namaquensis & 6 &
	428.8333\tabularnewline
	Burdigalian/Aquitanian & Paleotestudo cf.~antiqua & 1 &
	113.0000\tabularnewline
	Burdigalian/Aquitanian & Paleotestudo sp. & 1 & 179.3000\tabularnewline
	Burdigalian/Aquitanian & Testudo kalksburgensis & 2 &
	227.5000\tabularnewline
	Burdigalian/Aquitanian & Testudo promarginata & 3 &
	281.5667\tabularnewline
	Burdigalian/Aquitanian & Testudo rectogularis & 1 &
	213.0000\tabularnewline
	Burdigalian/Aquitanian & Titanochelon cf.~perpiniana & 1 &
	1001.0000\tabularnewline
	\bottomrule
\end{longtable}

\begin{longtable}[]{@{}lrr@{}}
	\caption[Species overview]{General overview over fossil species, with sample size and mean
		CL}
	\label{tab:Species}\tabularnewline
	\toprule
	Taxon & n & meanCL\tabularnewline
	\midrule
	\endfirsthead
	\toprule
	Taxon & n & meanCL\tabularnewline
	\midrule
	\endhead
	``Aldabrachelys'' laetoliensis & 1 & 1000.0000\tabularnewline
	``Hadrianus sp.'' & 1 & 1000.0000\tabularnewline
	Aldabrachelys ? sp. & 2 & 1500.0000\tabularnewline
	Caudochelys ducateli & 1 & 339.9000\tabularnewline
	Caudochelys rexroadensis & 2 & 805.5000\tabularnewline
	Caudochelys williamsi & 1 & 334.0000\tabularnewline
	Centrochelys atlantica & 1 & 400.0000\tabularnewline
	Centrochelys burchardi & 4 & 722.5000\tabularnewline
	Centrochelys marocana & 1 & 2050.0000\tabularnewline
	Centrochelys robusta & 7 & 891.4286\tabularnewline
	Centrochelys vulcanica & 1 & 610.0000\tabularnewline
	Cheirogaster cf.~gymnesica & 1 & 789.0000\tabularnewline
	Cheirogaster gymnesica & 1 & 739.0000\tabularnewline
	Cheirogaster richardi & 1 & 1155.0000\tabularnewline
	Cheirogaster sp. & 5 & 1227.0000\tabularnewline
	Chelonoidis alburyorum & 4 & 442.7500\tabularnewline
	Chelonoidis cubensis & 1 & 1139.0000\tabularnewline
	Chelonoidis denticulata & 1 & 616.0000\tabularnewline
	Chelonoidis lutzae & 1 & 830.0000\tabularnewline
	Chelonoidis marcanoi & 4 & 672.2500\tabularnewline
	Chelonoidis monensis & 1 & 500.0000\tabularnewline
	Chelonoidis sombrerensis & 1 & 990.0000\tabularnewline
	Chelonoidis sp. & 9 & 645.5556\tabularnewline
	Ergilemys oskarkuhni & 2 & 209.0000\tabularnewline
	Ergilemys sp. & 1 & 1000.0000\tabularnewline
	Eurotestudo aff. hermanni & 2 & 187.0000\tabularnewline
	Eurotestudo cf.~hermanni & 1 & 150.0000\tabularnewline
	Eurotestudo globosa & 1 & 263.0000\tabularnewline
	Eurotestudo hermanni & 5 & 201.0200\tabularnewline
	gen. indet. & 8 & 654.1250\tabularnewline
	Geochelone crassa & 1 & 865.0000\tabularnewline
	Geochelone hesterna & 1 & 278.0000\tabularnewline
	Geochelone s. l. & 1 & 1750.0000\tabularnewline
	Geochelone sp. & 10 & 626.2000\tabularnewline
	Geochelone stromeri & 2 & 387.5000\tabularnewline
	Geochelone tedwhitei & 2 & 405.0000\tabularnewline
	Gopherus ? sp. & 2 & 500.0000\tabularnewline
	Gopherus agassizi & 2 & 348.5000\tabularnewline
	Gopherus berlandieri & 2 & 225.6500\tabularnewline
	Gopherus canyonensis & 1 & 885.5000\tabularnewline
	Gopherus flavomarginatus & 1 & 450.0000\tabularnewline
	Gopherus laticaudatus & 1 & 375.0000\tabularnewline
	Gopherus mohavetus & 5 & 324.8000\tabularnewline
	Gopherus pertenuis & 1 & 1050.0000\tabularnewline
	Gopherus polyphemus & 54 & 295.1144\tabularnewline
	Gopherus praecedens & 1 & 360.0000\tabularnewline
	Gopherus sp. & 21 & 199.5238\tabularnewline
	Hesperotestudo alleni & 1 & 240.9000\tabularnewline
	Hesperotestudo bermudae & 2 & 385.0000\tabularnewline
	Hesperotestudo campester & 1 & 1000.0000\tabularnewline
	Hesperotestudo crassiscutata & 11 & 367.1818\tabularnewline
	Hesperotestudo equicomes & 1 & 340.0000\tabularnewline
	Hesperotestudo incisa & 8 & 234.3950\tabularnewline
	Hesperotestudo johnstoni & 1 & 235.0000\tabularnewline
	Hesperotestudo mlynarskii & 2 & 184.2500\tabularnewline
	Hesperotestudo oelrichi & 1 & 283.8000\tabularnewline
	Hesperotestudo orthopygia & 2 & 941.0000\tabularnewline
	Hesperotestudo riggsi & 5 & 175.1600\tabularnewline
	Hesperotestudo sp. & 8 & 1098.6250\tabularnewline
	Hesperotestudo turgida & 1 & 230.0000\tabularnewline
	Hesperotestudo wilsoni & 1 & 226.0000\tabularnewline
	Homopus fenestratus & 1 & 90.0000\tabularnewline
	Impregnochelys pachytectis & 1 & 620.0000\tabularnewline
	Indotestudo elongata & 1 & 270.0000\tabularnewline
	Manouria punjabiensis & 1 & 900.0000\tabularnewline
	Megalochelys atlas & 7 & 1735.7143\tabularnewline
	Megalochelys sondaari & 2 & 909.0000\tabularnewline
	Megalochelys sp. & 3 & 1130.4667\tabularnewline
	Mesocherus orangeus & 5 & 180.0000\tabularnewline
	Namibchersus aff. namaquensis & 3 & 696.6667\tabularnewline
	Namibchersus namaquensis & 6 & 428.8333\tabularnewline
	Paleotestudo antiqua & 19 & 206.8421\tabularnewline
	Paleotestudo cf.~antiqua & 1 & 113.0000\tabularnewline
	Paleotestudo cf.~sp. & 2 & 270.0000\tabularnewline
	Paleotestudo sp. & 4 & 220.0750\tabularnewline
	Psammobates antiquorum & 1 & 107.8000\tabularnewline
	Testudo aff. kenitrensis & 1 & 142.0000\tabularnewline
	Testudo amiatae & 1 & 140.0000\tabularnewline
	Testudo brevitesta & 2 & 232.5000\tabularnewline
	Testudo burgenlandica & 2 & 193.5000\tabularnewline
	Testudo catalaunica & 5 & 172.0000\tabularnewline
	Testudo cf.~graeca & 1 & 185.0000\tabularnewline
	Testudo cf.~promarginata & 5 & 250.0000\tabularnewline
	Testudo changshanesis & 1 & 330.0000\tabularnewline
	Testudo graeca & 4 & 193.0000\tabularnewline
	Testudo hermanni & 2 & 176.5500\tabularnewline
	Testudo kalksburgensis & 3 & 243.3333\tabularnewline
	Testudo kenitrensis & 1 & 132.0000\tabularnewline
	Testudo lunellensis & 4 & 215.4250\tabularnewline
	Testudo marginata & 3 & 270.0000\tabularnewline
	Testudo oughlamensis & 1 & 120.0000\tabularnewline
	Testudo pecorinii & 1 & 225.0000\tabularnewline
	Testudo promarginata & 3 & 281.5667\tabularnewline
	Testudo ranovi & 1 & 200.0000\tabularnewline
	Testudo rectogularis & 1 & 213.0000\tabularnewline
	Testudo s. s. & 1 & 189.0000\tabularnewline
	Testudo sp. & 15 & 625.7400\tabularnewline
	Testudo steinheimensis & 2 & 169.3500\tabularnewline
	Testudo transcaucasia & 1 & 150.0000\tabularnewline
	Titanochelon aff. schafferi & 1 & 1860.0000\tabularnewline
	Titanochelon bacharidisi & 4 & 1040.0000\tabularnewline
	Titanochelon bolivari & 5 & 1230.6000\tabularnewline
	Titanochelon cf.~bolivari & 3 & 1466.6667\tabularnewline
	Titanochelon cf.~perpiniana & 1 & 1001.0000\tabularnewline
	Titanochelon gymnesica & 1 & 1300.0000\tabularnewline
	Titanochelon perpiniana & 1 & 1140.0000\tabularnewline
	Titanochelon schafferi & 2 & 2175.0000\tabularnewline
	Titanochelon sp. & 2 & 970.0000\tabularnewline
	\bottomrule
\end{longtable}

\begin{longtable}[]{@{}llrr@{}}
	\caption[Genera per time bins]{Overview over genera (modern and fossil) per time bin, with
		sample sizes and mean CL.}
	\label{tab:GenBins}\tabularnewline
	\toprule
	EpochBins & Genus & n & meanCL\tabularnewline
	\midrule
	\endfirsthead
	\toprule
	EpochBins & Genus & n & meanCL\tabularnewline
	\midrule
	\endhead
	Modern & Aldabrachelys & 12 & 974.5833\tabularnewline
	Modern & Astrochelys & 14 & 366.2143\tabularnewline
	Modern & Centrochelys & 3 & 493.3333\tabularnewline
	Modern & Chelonoidis & 45 & 531.5178\tabularnewline
	Modern & Chersina & 15 & 176.2667\tabularnewline
	Modern & Cylindraspis & 5 & 724.0000\tabularnewline
	Modern & Geochelone & 8 & 252.1250\tabularnewline
	Modern & Gopherus & 23 & 302.4839\tabularnewline
	Modern & Hesperotestudo & 1 & 250.0000\tabularnewline
	Modern & Homopus & 7 & 139.2857\tabularnewline
	Modern & Indotestudo & 16 & 242.9875\tabularnewline
	Modern & Kinixys & 15 & 213.0667\tabularnewline
	Modern & Malacochersus & 2 & 166.5000\tabularnewline
	Modern & Manouria & 9 & 380.7778\tabularnewline
	Modern & Psammobates & 17 & 113.4118\tabularnewline
	Modern & Pyxis & 16 & 124.1875\tabularnewline
	Modern & Stigmochelys & 6 & 405.3333\tabularnewline
	Modern & Testudo & 39 & 197.5436\tabularnewline
	Upper Pleistocene & Centrochelys & 1 & 850.0000\tabularnewline
	Upper Pleistocene & Chelonoidis & 11 & 693.1818\tabularnewline
	Upper Pleistocene & Eurotestudo & 1 & 187.0000\tabularnewline
	Upper Pleistocene & gen. & 1 & 813.0000\tabularnewline
	Upper Pleistocene & Geochelone & 2 & 475.0000\tabularnewline
	Upper Pleistocene & Gopherus & 22 & 294.1545\tabularnewline
	Upper Pleistocene & Hesperotestudo & 10 & 468.2760\tabularnewline
	Upper Pleistocene & Indotestudo & 1 & 270.0000\tabularnewline
	Middle Pleistocene & Centrochelys & 4 & 722.5000\tabularnewline
	Middle Pleistocene & Chelonoidis & 1 & 1139.0000\tabularnewline
	Middle Pleistocene & Eurotestudo & 4 & 195.5250\tabularnewline
	Middle Pleistocene & Geochelone & 1 & 170.0000\tabularnewline
	Middle Pleistocene & Gopherus & 33 & 307.0721\tabularnewline
	Middle Pleistocene & Hesperotestudo & 5 & 882.0000\tabularnewline
	Middle Pleistocene & Testudo & 5 & 198.7400\tabularnewline
	Lower Pleistocene & Centrochelys & 4 & 762.5000\tabularnewline
	Lower Pleistocene & Cheirogaster & 2 & 857.0000\tabularnewline
	Lower Pleistocene & Chelonoidis & 3 & 716.6667\tabularnewline
	Lower Pleistocene & Eurotestudo & 4 & 201.5250\tabularnewline
	Lower Pleistocene & gen. & 1 & 900.0000\tabularnewline
	Lower Pleistocene & Geochelone & 1 & 340.0000\tabularnewline
	Lower Pleistocene & Gopherus & 13 & 316.8077\tabularnewline
	Lower Pleistocene & Hesperotestudo & 16 & 323.0562\tabularnewline
	Lower Pleistocene & Megalochelys & 5 & 1041.8800\tabularnewline
	Lower Pleistocene & Psammobates & 1 & 107.8000\tabularnewline
	Lower Pleistocene & Testudo & 6 & 259.1667\tabularnewline
	Lower Pleistocene & Titanochelon & 1 & 1300.0000\tabularnewline
	Gelasian & Centrochelys & 1 & 2050.0000\tabularnewline
	Gelasian & Eurotestudo & 1 & 150.0000\tabularnewline
	Gelasian & Gopherus & 15 & 185.7467\tabularnewline
	Gelasian & Hesperotestudo & 2 & 1000.0000\tabularnewline
	Gelasian & Manouria & 1 & 900.0000\tabularnewline
	Gelasian & Megalochelys & 3 & 1683.3333\tabularnewline
	Gelasian & Testudo & 6 & 166.0000\tabularnewline
	Gelasian & Titanochelon & 2 & 1640.0000\tabularnewline
	Piacencian & Aldabrachelys & 3 & 1333.3333\tabularnewline
	Piacencian & Centrochelys & 1 & 610.0000\tabularnewline
	Piacencian & Chelonoidis & 4 & 442.7500\tabularnewline
	Piacencian & Gopherus & 1 & 885.5000\tabularnewline
	Piacencian & Hesperotestudo & 5 & 211.1600\tabularnewline
	Piacencian & Homopus & 1 & 90.0000\tabularnewline
	Piacencian & Megalochelys & 2 & 1600.0000\tabularnewline
	Piacencian & Testudo & 3 & 230.0000\tabularnewline
	Piacencian & Titanochelon & 1 & 520.0000\tabularnewline
	Zanclean & Caudochelys & 2 & 805.5000\tabularnewline
	Zanclean & Centrochelys & 3 & 913.3333\tabularnewline
	Zanclean & Cheirogaster & 1 & 739.0000\tabularnewline
	Zanclean & Ergilemys & 2 & 209.0000\tabularnewline
	Zanclean & Geochelone & 6 & 741.0000\tabularnewline
	Zanclean & Hesperotestudo & 1 & 195.8000\tabularnewline
	Zanclean & Testudo & 5 & 1377.0000\tabularnewline
	Zanclean & Titanochelon & 6 & 1300.0000\tabularnewline
	Messinian & Hesperotestudo & 2 & 941.0000\tabularnewline
	Messinian & Megalochelys & 2 & 1950.0000\tabularnewline
	Messinian & Testudo & 4 & 176.7500\tabularnewline
	Messinian & Titanochelon & 2 & 1500.0000\tabularnewline
	Tortonian & ``Hadrianus'' & 1 & 1000.0000\tabularnewline
	Tortonian & Cheirogaster & 3 & 1288.3333\tabularnewline
	Tortonian & gen. & 3 & 660.0000\tabularnewline
	Tortonian & Geochelone & 3 & 741.3333\tabularnewline
	Tortonian & Gopherus & 6 & 354.0000\tabularnewline
	Tortonian & Hesperotestudo & 4 & 439.9750\tabularnewline
	Tortonian & Paleotestudo & 3 & 233.6667\tabularnewline
	Tortonian & Testudo & 20 & 218.3050\tabularnewline
	Tortonian & Titanochelon & 2 & 1400.0000\tabularnewline
	Serravallian & Cheirogaster & 2 & 1250.0000\tabularnewline
	Serravallian & gen. & 1 & 270.0000\tabularnewline
	Serravallian & Gopherus & 1 & 500.0000\tabularnewline
	Serravallian & Paleotestudo & 19 & 206.5789\tabularnewline
	Serravallian & Testudo & 3 & 190.2333\tabularnewline
	Serravallian & Titanochelon & 1 & 1353.0000\tabularnewline
	Langhian & Caudochelys & 1 & 339.9000\tabularnewline
	Langhian & Chelonoidis & 3 & 553.3333\tabularnewline
	Langhian & Ergilemys & 1 & 1000.0000\tabularnewline
	Langhian & gen. & 1 & 1000.0000\tabularnewline
	Langhian & Paleotestudo & 2 & 272.5000\tabularnewline
	Langhian & Testudo & 2 & 337.5000\tabularnewline
	Langhian & Titanochelon & 4 & 1312.5000\tabularnewline
	Burdigalian/Aquitanian & Caudochelys & 1 & 334.0000\tabularnewline
	Burdigalian/Aquitanian & gen. & 1 & 270.0000\tabularnewline
	Burdigalian/Aquitanian & Geochelone & 4 & 652.5000\tabularnewline
	Burdigalian/Aquitanian & Impregnochelys & 1 & 620.0000\tabularnewline
	Burdigalian/Aquitanian & Mesocherus & 5 & 180.0000\tabularnewline
	Burdigalian/Aquitanian & Namibchersus & 9 & 518.1111\tabularnewline
	Burdigalian/Aquitanian & Paleotestudo & 2 & 146.1500\tabularnewline
	Burdigalian/Aquitanian & Testudo & 6 & 252.1167\tabularnewline
	Burdigalian/Aquitanian & Titanochelon & 1 & 1001.0000\tabularnewline
	\bottomrule
\end{longtable}

\begin{longtable}[]{@{}lrr@{}}
	\caption[Genera overview]{General overview over genera, with sample sizes and mean
		CL.}
	\label{tab:Genera}\tabularnewline
	\toprule
	Genus & n & meanCL\tabularnewline
	\midrule
	\endfirsthead
	\toprule
	Genus & n & meanCL\tabularnewline
	\midrule
	\endhead
	``Hadrianus'' & 1 & 1000.0000\tabularnewline
	Aldabrachelys & 15 & 1046.3333\tabularnewline
	Astrochelys & 14 & 366.2143\tabularnewline
	Caudochelys & 4 & 571.2250\tabularnewline
	Centrochelys & 17 & 804.1176\tabularnewline
	Cheirogaster & 8 & 1102.2500\tabularnewline
	Chelonoidis & 67 & 571.0940\tabularnewline
	Chersina & 15 & 176.2667\tabularnewline
	Cylindraspis & 5 & 724.0000\tabularnewline
	Ergilemys & 3 & 472.6667\tabularnewline
	Eurotestudo & 10 & 192.5200\tabularnewline
	gen. & 8 & 654.1250\tabularnewline
	Geochelone & 25 & 510.2800\tabularnewline
	Gopherus & 114 & 298.0361\tabularnewline
	Hesperotestudo & 46 & 465.3296\tabularnewline
	Homopus & 8 & 133.1250\tabularnewline
	Impregnochelys & 1 & 620.0000\tabularnewline
	Indotestudo & 17 & 244.5765\tabularnewline
	Kinixys & 15 & 213.0667\tabularnewline
	Malacochersus & 2 & 166.5000\tabularnewline
	Manouria & 10 & 432.7000\tabularnewline
	Megalochelys & 12 & 1446.6167\tabularnewline
	Mesocherus & 5 & 180.0000\tabularnewline
	Namibchersus & 9 & 518.1111\tabularnewline
	Paleotestudo & 26 & 210.1269\tabularnewline
	Psammobates & 18 & 113.1000\tabularnewline
	Pyxis & 16 & 124.1875\tabularnewline
	Stigmochelys & 6 & 405.3333\tabularnewline
	Testudo & 99 & 269.2465\tabularnewline
	Titanochelon & 20 & 1315.2000\tabularnewline
	\bottomrule
\end{longtable}


%__________________________________________________________________
%\subsection{General statistics}

\begin{landscape}
\begin{longtable}[]{@{}rrrrrrrrrrrrl@{}}
	\caption[General statistics]{General statistics of body size data: all, per time bin,
		insular and continental, per continent (all referring to CL: min, max,
		variance, mean, logmean, median, logmedian, skewness, logskewness,
		kurosis, logkurtosis}\tabularnewline
	\toprule
	nCL & min & max & var & mean & logm & med & logmed & skew & logsk & kurt
	& logku & Variable\tabularnewline
	\midrule
	\endfirsthead
	\toprule
	nCL & min & max & var & mean & logm & med & logmed & skew & logsk & kurt
	& logku & Variable\tabularnewline
	\midrule
	\endhead
	616 & 80.00 & 2500 & 164537.80 & 437.2 & 2.5 & 270.5 & 2.4 & 2.14 & 0.69
	& 8.00 & 2.73 & all\tabularnewline
	253 & 80.00 & 1300 & 67485.50 & 330.3 & 2.4 & 242.0 & 2.4 & 1.83 & 0.58
	& 5.87 & 2.69 & Modern\tabularnewline
	49 & 102.44 & 1250 & 69690.66 & 445.9 & 2.6 & 334.7 & 2.5 & 1.20 & 0.24
	& 3.61 & 2.56 & Upper Pleistocene\tabularnewline
	53 & 132.00 & 1800 & 97910.83 & 387.1 & 2.5 & 292.9 & 2.5 & 3.03 & 1.52
	& 12.24 & 5.55 & Middle Pleistocene\tabularnewline
	57 & 107.80 & 2000 & 161948.82 & 463.5 & 2.5 & 263.0 & 2.4 & 1.74 & 0.73
	& 5.76 & 2.40 & Lower Pleistocene\tabularnewline
	31 & 118.90 & 2050 & 411224.51 & 555.2 & 2.5 & 194.9 & 2.3 & 1.31 & 0.93
	& 3.12 & 2.11 & Gelasian\tabularnewline
	21 & 90.00 & 1600 & 270535.82 & 610.6 & 2.6 & 428.0 & 2.6 & 1.00 & 0.14
	& 2.50 & 1.99 & Piacencian\tabularnewline
	26 & 176.00 & 2500 & 476162.71 & 955.2 & 2.9 & 857.5 & 2.9 & 1.11 &
	-0.40 & 3.56 & 2.30 & Zanclean\tabularnewline
	10 & 140.00 & 2100 & 602611.21 & 948.9 & 2.8 & 916.0 & 2.9 & 0.26 &
	-0.22 & 1.49 & 1.29 & Messinian\tabularnewline
	45 & 107.00 & 1540 & 175470.12 & 462.7 & 2.5 & 250.0 & 2.4 & 1.49 & 0.81
	& 3.74 & 2.54 & Tortonian\tabularnewline
	27 & 111.00 & 1500 & 126060.40 & 337.7 & 2.4 & 220.0 & 2.3 & 2.49 & 1.77
	& 7.77 & 5.30 & Serravallian\tabularnewline
	14 & 270.00 & 1600 & 230451.33 & 747.9 & 2.8 & 700.0 & 2.8 & 0.30 & 0.03
	& 1.55 & 1.18 & Langhian\tabularnewline
	30 & 113.00 & 1100 & 76288.76 & 406.8 & 2.5 & 302.4 & 2.5 & 1.27 & 0.45
	& 3.45 & 2.26 & Burdigalian/Aquitanian\tabularnewline
	253 & 80.00 & 1300 & 67485.50 & 330.3 & 2.4 & 242.0 & 2.4 & 1.83 & 0.58
	& 5.87 & 2.69 & Modern\tabularnewline
	363 & 90.00 & 2500 & 219004.66 & 511.7 & 2.6 & 285.6 & 2.5 & 1.83 & 0.68
	& 6.11 & 2.42 & Fossil\tabularnewline
	469 & 81.00 & 2500 & 157808.79 & 392.9 & 2.5 & 250.0 & 2.4 & 2.65 & 1.07
	& 10.57 & 3.74 & continental\tabularnewline
	147 & 80.00 & 2000 & 160834.35 & 578.5 & 2.6 & 500.0 & 2.7 & 1.02 &
	-0.27 & 3.95 & 2.05 & insular\tabularnewline
	157 & 81.00 & 830 & 17009.02 & 244.0 & 2.3 & 221.0 & 2.3 & 1.92 & 0.29 &
	8.09 & 2.98 & modern-con\tabularnewline
	96 & 80.00 & 1300 & 118641.09 & 471.5 & 2.6 & 353.0 & 2.5 & 0.82 & 0.01
	& 2.47 & 1.77 & modern-ins\tabularnewline
	312 & 90.00 & 2500 & 212116.79 & 467.9 & 2.5 & 270.0 & 2.4 & 2.11 & 0.96
	& 7.25 & 2.96 & fossil-con\tabularnewline
	51 & 150.00 & 2000 & 180825.40 & 780.0 & 2.8 & 750.0 & 2.9 & 1.11 &
	-0.40 & 4.02 & 3.18 & fossil-ins\tabularnewline
	142 & 80.00 & 2050 & 112417.26 & 347.7 & 2.4 & 193.5 & 2.3 & 2.10 & 0.68
	& 7.97 & 2.48 & Africa\tabularnewline
	242 & 102.44 & 1800 & 82209.71 & 415.0 & 2.5 & 302.2 & 2.5 & 1.92 & 0.75
	& 6.79 & 2.91 & America\tabularnewline
	59 & 150.00 & 2100 & 323123.20 & 585.5 & 2.6 & 280.0 & 2.4 & 1.43 & 0.85
	& 3.61 & 2.24 & Asia\tabularnewline
	173 & 107.00 & 2500 & 254222.84 & 491.2 & 2.5 & 245.0 & 2.4 & 1.86 &
	0.81 & 6.30 & 2.34 & Europe\tabularnewline
	\bottomrule
\end{longtable}
\end{landscape}

%\begin{landscape}

\tiny{
\begin{longtable}[]{@{}llllrllrlll@{}}
	\caption[Body size data set of fossil \T]{Body size data set of fossil testudinids. Contains information on locality, taxonomy (Genus and Species name), carapace length [mm], age and geographic distribution. Additionaly, it is stated whether carapace length was directly measured (m: exact measurements provided in reference, mf: measured from scaled figure, mo: estimated by original authors) or estimated (e: estimated from fragmentary carapace/plastron, ev: estimated from verbal description, ep: estimated from plastron length, ef: estimated from femur length, eh: estimated from humerus length, ec: estimated from claw phalanges). Further, it is stated on which continent the fossil record was recovered and whether it was continental (n: no) or insular (y: yes). (The references from which the data were obtained can be found in the table on the supplementary CD.)}
	\phantomsection
	\label{tab:DataFossil}\tabularnewline
	\toprule
	& Locality & Genus & Taxon & CL & estimated & Stages & Age & Insular &
	Continent\tabularnewline
	\midrule
	\endfirsthead
	\multicolumn{9}{c}%
	{\tablename\ \thetable\ -- \textit{continued from previous page}}\tabularnewline
	\toprule
	& Locality & Genus & Taxon & CL & estimated & Stages & Age & Insular &
	Continent\tabularnewline
	\midrule
	\endhead
	1 & Laetoli, Tanzania & Aldabrachelys & ``Aldabrachelys'' laetoliensis &
	1000.00 & mo & Piacencian & 2.70300 & n & Africa\tabularnewline
	2 & Sal Island & Centrochelys & Centrochelys atlantica & 400.00 & mo &
	Lower Pleistocene & 1.30000 & y & Africa\tabularnewline
	3 & Ahl al Oughlam (near Casablanca) & Centrochelys & Centrochelys
	marocana & 2050.00 & mo & Gelasian & 2.50000 & n & Africa\tabularnewline
	4 & Kanapoi & Geochelone & Geochelone crassa & 865.00 & mf & Zanclean &
	4.14500 & n & Africa\tabularnewline
	5 & Djebel Krechem & Geochelone & Geochelone sp. & 1446.00 & eh &
	Tortonian & 8.47600 & n & Africa\tabularnewline
	6 & Pellatal Phosphate Member, Varswater Formation, E Quarry
	Langebaanweg & Geochelone & Geochelone stromeri & 350.00 & m & Zanclean
	& 4.46600 & n & Africa\tabularnewline
	7 & Pellatal Phosphate Member, Varswater Formation, E Quarry
	Langebaanweg & Geochelone & Geochelone stromeri & 425.00 & m & Zanclean
	& 4.46600 & n & Africa\tabularnewline
	8 & South Africa & Homopus & Homopus fenestratus & 90.00 & mo &
	Piacencian & 3.05650 & n & Africa\tabularnewline
	9 & Rusinga Island, Lake Victoria, Kenya & Impregnochelys &
	Impregnochelys pachytectis & 620.00 & m & Burdigalian/Aquitanian &
	19.50000 & n & Africa\tabularnewline
	10 & Arrisdrift & Mesocherus & Mesocherus orangeus & 160.00 & mo &
	Burdigalian/Aquitanian & 17.25000 & n & Africa\tabularnewline
	11 & Arrisdrift & Mesocherus & Mesocherus orangeus & 180.00 & mo &
	Burdigalian/Aquitanian & 17.25000 & n & Africa\tabularnewline
	12 & Arrisdrift & Mesocherus & Mesocherus orangeus & 180.00 & mo &
	Burdigalian/Aquitanian & 17.25000 & n & Africa\tabularnewline
	13 & Arrisdrift & Mesocherus & Mesocherus orangeus & 180.00 & mo &
	Burdigalian/Aquitanian & 17.25000 & n & Africa\tabularnewline
	14 & Arrisdrift & Mesocherus & Mesocherus orangeus & 180.00 & mo &
	Burdigalian/Aquitanian & 17.25000 & n & Africa\tabularnewline
	15 & Arrisdrift & Mesocherus & Mesocherus orangeus & 180.00 & mo &
	Burdigalian/Aquitanian & 17.25000 & n & Africa\tabularnewline
	16 & Arrisdrift & Mesocherus & Mesocherus orangeus & 200.00 & mo &
	Burdigalian/Aquitanian & 17.25000 & n & Africa\tabularnewline
	17 & Arrisdrift & Namibchersus & Namibchersus aff. namaquensis & 1100.00
	& mo & Burdigalian/Aquitanian & 17.25000 & n & Africa\tabularnewline
	18 & Arrisdrift & Namibchersus & Namibchersus aff. namaquensis & 440.00
	& mo & Burdigalian/Aquitanian & 17.25000 & n & Africa\tabularnewline
	19 & Arrisdrift & Namibchersus & Namibchersus aff. namaquensis & 550.00
	& mo & Burdigalian/Aquitanian & 17.25000 & n & Africa\tabularnewline
	20 & Auchas & Namibchersus & Namibchersus namaquensis & 254.00 & m &
	Burdigalian/Aquitanian & 18.00000 & n & Africa\tabularnewline
	21 & Elisabethfeld (= Elisabeth Bay) area, northern Sperrgebiet &
	Namibchersus & Namibchersus namaquensis & 264.00 & m &
	Burdigalian/Aquitanian & 19.50000 & n & Africa\tabularnewline
	22 & Elisabethfeld (= Elisabeth Bay) area, northern Sperrgebiet &
	Namibchersus & Namibchersus namaquensis & 300.00 & m &
	Burdigalian/Aquitanian & 19.50000 & n & Africa\tabularnewline
	23 & Auchas & Namibchersus & Namibchersus namaquensis & 470.00 & m &
	Burdigalian/Aquitanian & 18.00000 & n & Africa\tabularnewline
	24 & Auchas & Namibchersus & Namibchersus namaquensis & 470.00 & m &
	Burdigalian/Aquitanian & 18.00000 & n & Africa\tabularnewline
	25 & Auchas & Namibchersus & Namibchersus namaquensis & 815.00 & m &
	Burdigalian/Aquitanian & 18.00000 & n & Africa\tabularnewline
	26 & Drimolon, Sterkfontein, Krugersdorp District, Gauteng Province &
	Psammobates & Psammobates antiquorum & 107.80 & m & Lower Pleistocene &
	1.80000 & n & Africa\tabularnewline
	27 & Ahl al Oughlam (near Casablanca) & Testudo & Testudo aff.
	kenitrensis & 142.00 & mf & Gelasian & 2.50000 & n &
	Africa\tabularnewline
	28 & Kénitra, Guilloux quarry, near Rabat & Testudo & Testudo
	kenitrensis & 132.00 & mo & Middle Pleistocene & 0.45350 & n &
	Africa\tabularnewline
	29 & Ahl al Oughlam (near Casablanca) & Testudo & Testudo oughlamensis &
	120.00 & mo & Gelasian & 2.50000 & n & Africa\tabularnewline
	30 & Ahl al Oughlam (near Casablanca) & Testudo & Testudo sp. & 184.00 &
	mf & Gelasian & 2.50000 & n & Africa\tabularnewline
	31 & Ahl al Oughlam (near Casablanca) & Testudo & Testudo sp. & 200.00 &
	mf & Gelasian & 2.50000 & n & Africa\tabularnewline
	32 & Tha Chang area, Chaloem Pra Kiat district, Nakhon Ratchasima
	Province & Aldabrachelys & Aldabrachelys ? sp. & 1500.00 & mo &
	Piacencian & 3.00000 & n & Asia\tabularnewline
	33 & Tha Chang area, Chaloem Pra Kiat district, Nakhon Ratchasima
	Province & Aldabrachelys & Aldabrachelys ? sp. & 1500.00 & mo &
	Piacencian & 3.00000 & n & Asia\tabularnewline
	34 & Tha Chang area, Chaloem Pra Kiat district, Nakhon Ratchasima
	Province & Aldabrachelys & Aldabrachelys ? sp. & 1500.00 & mo &
	Piacencian & 3.00000 & n & Asia\tabularnewline
	35 & Tha Chang area, Chaloem Pra Kiat district, Nakhon Ratchasima
	Province & Aldabrachelys & Aldabrachelys ? sp. & 1500.00 & mo &
	Piacencian & 3.00000 & n & Asia\tabularnewline
	36 & Altan-Teli main fossiliferous bed (Dzereg valley) & Ergilemys &
	Ergilemys oskarkuhni & 198.00 & m & Zanclean & 3.95000 & n &
	Asia\tabularnewline
	37 & Altan-Teli main fossiliferous bed (Dzereg valley) & Ergilemys &
	Ergilemys oskarkuhni & 220.00 & m & Zanclean & 3.95000 & n &
	Asia\tabularnewline
	38 & Guangxi & gen. & gen. indet. & 900.00 & mo & Lower Pleistocene &
	1.68450 & n & Asia\tabularnewline
	39 & Ghaba & Geochelone & Geochelone sp. & 800.00 & ev &
	Burdigalian/Aquitanian & 16.50000 & n & Asia\tabularnewline
	40 & Lang Rongrien Rockshelter, Krabi, Thailand & Indotestudo &
	Indotestudo elongata & 270.00 & m & Upper Pleistocene & 0.03700 & n &
	Asia\tabularnewline
	41 & Punjab & Manouria & Manouria punjabiensis & 900.00 & mo & Gelasian
	& 2.19050 & n & Asia\tabularnewline
	42 & Sulawesi (Celebes), Indonesia & Megalochelys & Megalochelys atlas &
	1400.00 & mo & Gelasian & 2.00000 & y & Asia\tabularnewline
	43 & Northwest of Naipli & Megalochelys & Megalochelys atlas & 1600.00 &
	mo & Piacencian & 3.09400 & n & Asia\tabularnewline
	44 & Northwest of Naipli & Megalochelys & Megalochelys atlas & 1600.00 &
	mo & Piacencian & 3.09400 & n & Asia\tabularnewline
	45 & Northwest of Naipli & Megalochelys & Megalochelys atlas & 1600.00 &
	mo & Piacencian & 3.09400 & n & Asia\tabularnewline
	46 & Northwest of Naipli & Megalochelys & Megalochelys atlas & 1600.00 &
	mo & Piacencian & 3.09400 & n & Asia\tabularnewline
	47 & Sulawesi (Celebes), Indonesia & Megalochelys & Megalochelys atlas &
	1650.00 & mo & Gelasian & 2.00000 & y & Asia\tabularnewline
	48 & Pauk Twonship & Megalochelys & Megalochelys atlas & 1800.00 & m &
	Messinian & 5.42300 & n & Asia\tabularnewline
	49 & Siwalik & Megalochelys & Megalochelys atlas & 2000.00 & mo &
	Gelasian & 2.19050 & n & Asia\tabularnewline
	50 & Pauk Twonship & Megalochelys & Megalochelys atlas & 2100.00 & mo &
	Messinian & 5.42300 & n & Asia\tabularnewline
	51 & Tres Hermanas, Manila, Luzon & Megalochelys & Megalochelys sondaari
	& 1000.00 & ec & Lower Pleistocene & 1.35000 & y & Asia\tabularnewline
	52 & Tres Hermanas, Manila, Luzon & Megalochelys & Megalochelys sondaari
	& 818.00 & ec & Lower Pleistocene & 1.35000 & y & Asia\tabularnewline
	53 & Flores & Megalochelys & Megalochelys sp. & 1200.00 & ev & Lower
	Pleistocene & 0.90000 & y & Asia\tabularnewline
	54 & Bumiayu, Java Island & Megalochelys & Megalochelys sp. & 191.40 & m
	& Lower Pleistocene & 1.68450 & y & Asia\tabularnewline
	55 & Java Island & Megalochelys & Megalochelys sp. & 2000.00 & m & Lower
	Pleistocene & 1.68450 & y & Asia\tabularnewline
	56 & Zhejiang & Testudo & Testudo changshanesis & 330.00 & mo & Lower
	Pleistocene & 1.68450 & n & Asia\tabularnewline
	57 & Khatlon & Testudo & Testudo ranovi & 200.00 & mo & Gelasian &
	2.19050 & n & Asia\tabularnewline
	58 & Gerogia (Caucasus) & Testudo & Testudo transcaucasia & 150.00 & mo
	& Gelasian & 2.19050 & n & Asia\tabularnewline
	59 & Sawmill Sink, Abaco & Chelonoidis & Chelonoidis alburyorum & 424.00
	& m & Piacencian & 3.20150 & y & America\tabularnewline
	60 & Sawmill Sink, Abaco & Chelonoidis & Chelonoidis alburyorum & 428.00
	& m & Piacencian & 3.20150 & y & America\tabularnewline
	61 & Sawmill Sink, Abaco & Chelonoidis & Chelonoidis alburyorum & 453.00
	& m & Piacencian & 3.20150 & y & America\tabularnewline
	62 & Sawmill Sink, Abaco & Chelonoidis & Chelonoidis alburyorum & 466.00
	& m & Piacencian & 3.20150 & y & America\tabularnewline
	63 & Santa Clara & Chelonoidis & Chelonoidis cubensis & 1139.00 & ef &
	Middle Pleistocene & 0.39350 & y & America\tabularnewline
	64 & Cueva del Papayo, Pedernales & Chelonoidis & Chelonoidis marcanoi &
	530.00 & eh & Upper Pleistocene & 0.06900 & y & America\tabularnewline
	65 & Cueva del Papayo, Pedernales & Chelonoidis & Chelonoidis marcanoi &
	614.00 & eh & Upper Pleistocene & 0.06900 & y & America\tabularnewline
	66 & Cueva del Papayo, Pedernales & Chelonoidis & Chelonoidis marcanoi &
	767.00 & eh & Upper Pleistocene & 0.06900 & y & America\tabularnewline
	67 & Cueva del Papayo, Pedernales & Chelonoidis & Chelonoidis marcanoi &
	778.00 & eh & Upper Pleistocene & 0.06900 & y & America\tabularnewline
	68 & Mona Island & Chelonoidis & Chelonoidis monensis & 500.00 & m &
	Upper Pleistocene & 0.06450 & y & America\tabularnewline
	69 & Sombrero Island & Chelonoidis & Chelonoidis sombrerensis & 990.00 &
	m & Upper Pleistocene & 0.06900 & y & America\tabularnewline
	70 & Navassa Island & Chelonoidis & Chelonoidis sp. & 400.00 & mo &
	Upper Pleistocene & 0.06900 & y & America\tabularnewline
	71 & San Pedro, Curaçao & Chelonoidis & Chelonoidis sp. & 600.00 & mo &
	Lower Pleistocene & 1.35700 & y & America\tabularnewline
	72 & Bayaguana, Los Haitises, San Cristobal & Chelonoidis & Chelonoidis
	sp. & 600.00 & mo & Upper Pleistocene & 0.06900 & y &
	America\tabularnewline
	73 & San Pedro, Curaçao & Chelonoidis & Chelonoidis sp. & 750.00 & mo &
	Lower Pleistocene & 1.35700 & y & America\tabularnewline
	74 & San Pedro, Curaçao & Chelonoidis & Chelonoidis sp. & 800.00 & mo &
	Lower Pleistocene & 1.35700 & y & America\tabularnewline
	75 & Cedazo local fauna, Aguascalientes, Mexico & Geochelone &
	Geochelone sp. & 340.00 & mo & Lower Pleistocene & 1.05000 & n &
	America\tabularnewline
	76 & Cedazo local fauna, Aguascalientes, Mexico & Gopherus & Gopherus
	berlandieri & 195.00 & m & Lower Pleistocene & 1.05000 & n &
	America\tabularnewline
	77 & Cedazo local fauna, Aguascalientes, Mexico & Gopherus & Gopherus
	berlandieri & 256.30 & m & Lower Pleistocene & 1.05000 & n &
	America\tabularnewline
	78 & Cedazo local fauna, Aguascalientes, Mexico & Gopherus & Gopherus
	flavomarginatus & 450.00 & m & Lower Pleistocene & 1.05000 & n &
	America\tabularnewline
	79 & Smith's Parrish, No. 3Verdmont Valley Close & Hesperotestudo &
	Hesperotestudo bermudae & 270.00 & m & Middle Pleistocene & 0.31000 & y
	& America\tabularnewline
	80 & Smith's Parrish, No. 3Verdmont Valley Close & Hesperotestudo &
	Hesperotestudo bermudae & 500.00 & m & Middle Pleistocene & 0.31000 & y
	& America\tabularnewline
	81 & Río Tomayate, Apopa Municipality & Hesperotestudo & Hesperotestudo
	sp. & 1500.00 & mo & Lower Pleistocene & 0.96600 & n &
	America\tabularnewline
	82 & Belomechetskaya & Ergilemys & Ergilemys sp. & 1000.00 & m &
	Langhian & 14.00000 & n & Europe\tabularnewline
	83 & Dmanisi & Testudo & Testudo graeca & 195.00 & mf & Lower
	Pleistocene & 1.77000 & n & Europe\tabularnewline
	84 & Prottes & ``Hadrianus'' & ``Hadrianus sp.'' & 1000.00 & m &
	Tortonian & 8.30000 & n & Europe\tabularnewline
	85 & Adeje, Tenerife & Centrochelys & Centrochelys burchardi & 500.00 &
	mo & Middle Pleistocene & 0.43500 & y & Europe\tabularnewline
	86 & Callao de Fañabé, Tenerife & Centrochelys & Centrochelys burchardi
	& 650.00 & mo & Middle Pleistocene & 0.43500 & y & Europe\tabularnewline
	87 & Adeje, Tenerife & Centrochelys & Centrochelys burchardi & 800.00 &
	m & Middle Pleistocene & 0.43500 & y & Europe\tabularnewline
	88 & Callao de Fañabé, Tenerife & Centrochelys & Centrochelys burchardi
	& 940.00 & mo & Middle Pleistocene & 0.43500 & y & Europe\tabularnewline
	89 & Corrida, Malta & Centrochelys & Centrochelys robusta & 1100.00 & mo
	& Zanclean & 4.91700 & y & Europe\tabularnewline
	90 & Ghar Dalam & Centrochelys & Centrochelys robusta & 1200.00 & ev &
	Lower Pleistocene & 1.30000 & y & Europe\tabularnewline
	91 & Ghar Dalam & Centrochelys & Centrochelys robusta & 600.00 & ev &
	Lower Pleistocene & 1.30000 & y & Europe\tabularnewline
	92 & Mnaidra Gap, Malta & Centrochelys & Centrochelys robusta & 790.00 &
	ef & Zanclean & 4.91700 & y & Europe\tabularnewline
	93 & Corrida, Malta & Centrochelys & Centrochelys robusta & 850.00 & mo
	& Zanclean & 4.91700 & y & Europe\tabularnewline
	94 & Zebbug and Gahr Dalam Cave deposits & Centrochelys & Centrochelys
	robusta & 850.00 & mo & Upper Pleistocene & 0.06600 & y &
	Europe\tabularnewline
	95 & Ghar Dalam & Centrochelys & Centrochelys robusta & 850.00 & ev &
	Lower Pleistocene & 1.30000 & y & Europe\tabularnewline
	96 & Barranco de las Ballenas, Las Palmas, Gran Canaria & Centrochelys &
	Centrochelys vulcanica & 610.00 & mo & Piacencian & 3.09400 & y &
	Europe\tabularnewline
	97 & Pujo d'es Fum, Formentera, Balearic Islands & Cheirogaster &
	Cheirogaster cf.~gymnesica & 789.00 & mo & Lower Pleistocene & 1.80000 &
	y & Europe\tabularnewline
	98 & Punta Nati near Ciutadella, Minorca & Cheirogaster & Cheirogaster
	gymnesica & 739.00 & ef & Zanclean & 4.45000 & y & Europe\tabularnewline
	99 & Hostalets de Piérola, Barcelone province, Cataluña, Vallés-Penedés
	basin & Cheirogaster & Cheirogaster richardi & 1155.00 & mo & Tortonian
	& 10.40000 & n & Europe\tabularnewline
	100 & La Ciesma 1, Aragón & Cheirogaster & Cheirogaster sp. & 1000.00 &
	mo & Serravallian & 12.20000 & n & Europe\tabularnewline
	101 & El Lugarejo (Arévalo), Ávilla, Castilla & Cheirogaster &
	Cheirogaster sp. & 1170.00 & m & Tortonian & 10.25000 & n &
	Europe\tabularnewline
	102 & Chañe, Segovia & Cheirogaster & Cheirogaster sp. & 1500.00 & e &
	Serravallian & 13.80000 & n & Europe\tabularnewline
	103 & Crevillente 2 & Cheirogaster & Cheirogaster sp. & 1540.00 & ef &
	Tortonian & 8.30000 & n & Europe\tabularnewline
	104 & Rock-Cavities, Gibraltar Peninsula & Cheirogaster & Cheirogaster
	sp. & 925.00 & ef & Lower Pleistocene & 0.96500 & y &
	Europe\tabularnewline
	105 & Soave, Zoppega 2 cave, Verona & Eurotestudo & Eurotestudo aff.
	hermanni & 179.30 & mf & Middle Pleistocene & 0.74000 & n &
	Europe\tabularnewline
	106 & Soave, Zoppega 2 cave, Verona & Eurotestudo & Eurotestudo aff.
	hermanni & 194.70 & mf & Middle Pleistocene & 0.74000 & n &
	Europe\tabularnewline
	107 & Monte Tuttavista VII mustelide, Sardinia & Eurotestudo &
	Eurotestudo cf.~hermanni & 150.00 & mo & Gelasian & 2.00000 & y &
	Europe\tabularnewline
	108 & Le Ville, Upper Valdarno & Eurotestudo & Eurotestudo globosa &
	263.00 & m & Lower Pleistocene & 1.80000 & n & Europe\tabularnewline
	109 & Cueva de la Victoria-1 (CV-1), Carthagène, Murcia & Eurotestudo &
	Eurotestudo hermanni & 126.00 & mf & Lower Pleistocene & 1.15000 & n &
	Europe\tabularnewline
	110 & Saint-Estève-Janson, l'Escale Cave (Bouches du Rhône) &
	Eurotestudo & Eurotestudo hermanni & 170.50 & mf & Middle Pleistocene &
	0.60000 & n & Europe\tabularnewline
	111 & Cova del Rinoceront, eastern Garraf Massif, Can'Aymerich quarry,
	Castelldelfs & Eurotestudo & Eurotestudo hermanni & 187.00 & mf & Upper
	Pleistocene & 0.11050 & n & Europe\tabularnewline
	112 & Saint-Estève-Janson, l'Escale Cave (Bouches du Rhône) &
	Eurotestudo & Eurotestudo hermanni & 237.60 & mf & Middle Pleistocene &
	0.60000 & n & Europe\tabularnewline
	113 & Sierra de Quibas, Abanilla, Murcia & Eurotestudo & Eurotestudo
	hermanni & 284.00 & mf & Lower Pleistocene & 1.35000 & n &
	Europe\tabularnewline
	114 & Tarazona de Aragón & gen. & gen. indet. & 1000.00 & mo & Langhian
	& 14.70000 & n & Europe\tabularnewline
	115 & La Ciesma 1, Aragón & gen. & gen. indet. & 270.00 & mo &
	Serravallian & 12.20000 & n & Europe\tabularnewline
	116 & Monteagudo, Aragón & gen. & gen. indet. & 270.00 & mo &
	Burdigalian/Aquitanian & 16.40000 & n & Europe\tabularnewline
	117 & Kohfidisch & gen. & gen. indet. & 440.00 & m & Tortonian & 8.75000
	& n & Europe\tabularnewline
	118 & Kohfidisch & gen. & gen. indet. & 660.00 & m & Tortonian & 8.75000
	& n & Europe\tabularnewline
	119 & Zubbio di Cozzo San Pietro & gen. & gen. indet. & 813.00 & ef &
	Upper Pleistocene & 0.01250 & y & Europe\tabularnewline
	120 & Kohfidisch & gen. & gen. indet. & 880.00 & m & Tortonian & 8.75000
	& n & Europe\tabularnewline
	121 & Jambol & Geochelone & Geochelone s. l. & 1750.00 & mo & Zanclean &
	4.46600 & n & Europe\tabularnewline
	122 & Kirchdorf an der Iller & Geochelone & Geochelone sp. & 1000.00 & m
	& Burdigalian/Aquitanian & 16.65000 & n & Europe\tabularnewline
	123 & Hohenhöwen, Engen, Hegau, southwestern Germany & Paleotestudo &
	Paleotestudo antiqua & 145.00 & mf & Serravallian & 13.00000 & n &
	Europe\tabularnewline
	124 & Hohenhöwen, Engen, Hegau, southwestern Germany & Paleotestudo &
	Paleotestudo antiqua & 152.00 & m & Serravallian & 13.00000 & n &
	Europe\tabularnewline
	125 & Hohenhöwen, Engen, Hegau, southwestern Germany & Paleotestudo &
	Paleotestudo antiqua & 159.50 & m & Serravallian & 13.00000 & n &
	Europe\tabularnewline
	126 & Hohenhöwen, Engen, Hegau, southwestern Germany & Paleotestudo &
	Paleotestudo antiqua & 180.00 & m & Serravallian & 13.00000 & n &
	Europe\tabularnewline
	127 & Gammelsdorf & Paleotestudo & Paleotestudo antiqua & 183.70 & m &
	Serravallian & 12.15000 & n & Europe\tabularnewline
	128 & Hohenhöwen, Engen, Hegau, southwestern Germany & Paleotestudo &
	Paleotestudo antiqua & 185.00 & mf & Serravallian & 13.00000 & n &
	Europe\tabularnewline
	129 & Sansan, Gers (lake) & Paleotestudo & Paleotestudo antiqua & 191.00
	& mf & Serravallian & 13.60000 & n & Europe\tabularnewline
	130 & Hohenhöwen, Engen, Hegau, southwestern Germany & Paleotestudo &
	Paleotestudo antiqua & 195.00 & m & Serravallian & 13.00000 & n &
	Europe\tabularnewline
	131 & Hohenhöwen, Engen, Hegau, southwestern Germany & Paleotestudo &
	Paleotestudo antiqua & 195.00 & mf & Serravallian & 13.00000 & n &
	Europe\tabularnewline
	132 & Gammelsdorf & Paleotestudo & Paleotestudo antiqua & 203.00 & m &
	Serravallian & 12.15000 & n & Europe\tabularnewline
	133 & Hohenhöwen, Engen, Hegau, southwestern Germany & Paleotestudo &
	Paleotestudo antiqua & 206.00 & mf & Serravallian & 13.00000 & n &
	Europe\tabularnewline
	134 & Sansan, Gers (lake) & Paleotestudo & Paleotestudo antiqua & 213.00
	& mf & Serravallian & 13.60000 & n & Europe\tabularnewline
	135 & Hohenhöwen, Engen, Hegau, southwestern Germany & Paleotestudo &
	Paleotestudo antiqua & 220.00 & mf & Serravallian & 13.00000 & n &
	Europe\tabularnewline
	136 & Hohenhöwen, Engen, Hegau, southwestern Germany & Paleotestudo &
	Paleotestudo antiqua & 229.00 & mf & Serravallian & 13.00000 & n &
	Europe\tabularnewline
	137 & Sansan, Gers (lake) & Paleotestudo & Paleotestudo antiqua & 234.00
	& mf & Serravallian & 13.60000 & n & Europe\tabularnewline
	138 & Hohenhöwen, Engen, Hegau, southwestern Germany & Paleotestudo &
	Paleotestudo antiqua & 240.00 & m & Serravallian & 13.00000 & n &
	Europe\tabularnewline
	139 & Sansan, Gers (lake) & Paleotestudo & Paleotestudo antiqua & 240.00
	& mf & Serravallian & 13.60000 & n & Europe\tabularnewline
	140 & Barajas, Madrid & Paleotestudo & Paleotestudo antiqua & 275.00 &
	mf & Langhian & 15.00000 & n & Europe\tabularnewline
	141 & Illescas, Toledo & Paleotestudo & Paleotestudo antiqua & 283.80 &
	mf & Serravallian & 12.50000 & n & Europe\tabularnewline
	142 & Can Mas near El Papiol, Barcelone province, Cataluña,
	Vallés-Penedés basin & Paleotestudo & Paleotestudo cf.~antiqua & 113.00
	& mf & Burdigalian/Aquitanian & 17.30000 & n & Europe\tabularnewline
	143 & El Buste, Aragón & Paleotestudo & Paleotestudo cf.~sp. & 270.00 &
	mo & Serravallian & 12.40000 & n & Europe\tabularnewline
	144 & Tarazona de Aragón & Paleotestudo & Paleotestudo cf.~sp. & 270.00
	& mo & Langhian & 14.70000 & n & Europe\tabularnewline
	145 & Cerro de los Batallones, Madrid & Paleotestudo & Paleotestudo sp.
	& 170.00 & mf & Tortonian & 9.50000 & n & Europe\tabularnewline
	146 & Teiritzberg (T1 = 001/D/C), Korneuburg Basin, Lower Austria &
	Paleotestudo & Paleotestudo sp. & 179.30 & m & Burdigalian/Aquitanian &
	16.55000 & n & Europe\tabularnewline
	147 & Cerro de los Batallones, Madrid & Paleotestudo & Paleotestudo sp.
	& 261.00 & mf & Tortonian & 9.50000 & n & Europe\tabularnewline
	148 & Cerro de los Batallones, Madrid & Paleotestudo & Paleotestudo sp.
	& 270.00 & mf & Tortonian & 9.50000 & n & Europe\tabularnewline
	149 & Torrente Melacce, Cinigiano (GR) & Testudo & Testudo amiatae &
	140.00 & mo & Messinian & 5.81500 & n & Europe\tabularnewline
	150 & Milia, Grevena, W Macedonia & Testudo & Testudo brevitesta &
	165.00 & mf & Piacencian & 2.60000 & n & Europe\tabularnewline
	151 & Milia, Grevena, W Macedonia & Testudo & Testudo brevitesta &
	300.00 & mf & Piacencian & 2.60000 & n & Europe\tabularnewline
	152 & Kohfidisch & Testudo & Testudo burgenlandica & 112.00 & m &
	Tortonian & 8.75000 & n & Europe\tabularnewline
	153 & Kohfidisch & Testudo & Testudo burgenlandica & 275.00 & m &
	Tortonian & 8.75000 & n & Europe\tabularnewline
	154 & Sant Quirze de Terrassa/de Galliners (del Vallès), Barcelona &
	Testudo & Testudo catalaunica & 107.00 & m & Tortonian & 11.50000 & n &
	Europe\tabularnewline
	155 & Castell de Barbera & Testudo & Testudo catalaunica & 165.00 & m &
	Tortonian & 11.50000 & n & Europe\tabularnewline
	156 & Sant Quirze de Terrassa/de Galliners (del Vallès), Barcelona &
	Testudo & Testudo catalaunica & 175.00 & m & Tortonian & 11.50000 & n &
	Europe\tabularnewline
	157 & Sant Quirze de Terrassa/de Galliners (del Vallès), Barcelona &
	Testudo & Testudo catalaunica & 181.00 & m & Tortonian & 11.50000 & n &
	Europe\tabularnewline
	158 & Abocador de Can Mata (els Hostalets de Pierola)(ACM/BDA),
	Vallés-Penedés basin, Cataluña & Testudo & Testudo catalaunica & 232.00
	& m & Serravallian & 12.35000 & n & Europe\tabularnewline
	159 & Megalo Emvolon 1 (MEV), 20 km SW Thessaloniki & Testudo & Testudo
	cf.~graeca & 185.00 & m & Zanclean & 3.90000 & n & Europe\tabularnewline
	160 & Prottes & Testudo & Testudo cf.~promarginata & 250.00 & m &
	Tortonian & 8.30000 & n & Europe\tabularnewline
	161 & Prottes & Testudo & Testudo cf.~promarginata & 250.00 & m &
	Tortonian & 8.30000 & n & Europe\tabularnewline
	162 & Prottes & Testudo & Testudo cf.~promarginata & 250.00 & m &
	Tortonian & 8.30000 & n & Europe\tabularnewline
	163 & Prottes & Testudo & Testudo cf.~promarginata & 250.00 & m &
	Tortonian & 8.30000 & n & Europe\tabularnewline
	164 & Prottes & Testudo & Testudo cf.~promarginata & 250.00 & m &
	Tortonian & 8.30000 & n & Europe\tabularnewline
	165 & Prottes & Testudo & Testudo cf.~promarginata & 250.00 & m &
	Tortonian & 8.30000 & n & Europe\tabularnewline
	166 & Prottes & Testudo & Testudo cf.~promarginata & 250.00 & m &
	Tortonian & 8.30000 & n & Europe\tabularnewline
	167 & Prottes & Testudo & Testudo cf.~promarginata & 250.00 & m &
	Tortonian & 8.30000 & n & Europe\tabularnewline
	168 & Prottes & Testudo & Testudo cf.~promarginata & 250.00 & m &
	Tortonian & 8.30000 & n & Europe\tabularnewline
	169 & Prottes & Testudo & Testudo cf.~promarginata & 250.00 & m &
	Tortonian & 8.30000 & n & Europe\tabularnewline
	170 & Prottes & Testudo & Testudo cf.~promarginata & 250.00 & m &
	Tortonian & 8.30000 & n & Europe\tabularnewline
	171 & Prottes & Testudo & Testudo cf.~promarginata & 250.00 & m &
	Tortonian & 8.30000 & n & Europe\tabularnewline
	172 & Prottes & Testudo & Testudo cf.~promarginata & 250.00 & m &
	Tortonian & 8.30000 & n & Europe\tabularnewline
	173 & Prottes & Testudo & Testudo cf.~promarginata & 250.00 & m &
	Tortonian & 8.30000 & n & Europe\tabularnewline
	174 & Prottes & Testudo & Testudo cf.~promarginata & 250.00 & m &
	Tortonian & 8.30000 & n & Europe\tabularnewline
	175 & Prottes & Testudo & Testudo cf.~promarginata & 250.00 & m &
	Tortonian & 8.30000 & n & Europe\tabularnewline
	176 & Prottes & Testudo & Testudo cf.~promarginata & 250.00 & m &
	Tortonian & 8.30000 & n & Europe\tabularnewline
	177 & Prottes & Testudo & Testudo cf.~promarginata & 250.00 & m &
	Tortonian & 8.30000 & n & Europe\tabularnewline
	178 & Prottes & Testudo & Testudo cf.~promarginata & 250.00 & m &
	Tortonian & 8.30000 & n & Europe\tabularnewline
	179 & Prottes & Testudo & Testudo cf.~promarginata & 250.00 & m &
	Tortonian & 8.30000 & n & Europe\tabularnewline
	180 & Prottes & Testudo & Testudo cf.~promarginata & 250.00 & m &
	Tortonian & 8.30000 & n & Europe\tabularnewline
	181 & Prottes & Testudo & Testudo cf.~promarginata & 250.00 & m &
	Tortonian & 8.30000 & n & Europe\tabularnewline
	182 & Prottes & Testudo & Testudo cf.~promarginata & 250.00 & m &
	Tortonian & 8.30000 & n & Europe\tabularnewline
	183 & Prottes & Testudo & Testudo cf.~promarginata & 250.00 & m &
	Tortonian & 8.30000 & n & Europe\tabularnewline
	184 & Prottes & Testudo & Testudo cf.~promarginata & 250.00 & m &
	Tortonian & 8.30000 & n & Europe\tabularnewline
	185 & Pylea, eastern part of Thessaloniki, western Chalkidiki peninsula
	& Testudo & Testudo graeca & 167.00 & m & Messinian & 5.50000 & n &
	Europe\tabularnewline
	186 & Allatini, eastern part of Thessaloniki, western Chalkidiki
	peninsula & Testudo & Testudo graeca & 200.00 & mf & Messinian & 5.50000
	& n & Europe\tabularnewline
	187 & Platania, Drama basin & Testudo & Testudo graeca & 210.00 & mf &
	Tortonian & 8.45000 & n & Europe\tabularnewline
	188 & Sima del Elefante TE14, Sierra de Atapuerca, Burgos & Eurotestudo
	& Testudo hermanni & 133.10 & mf & Lower Pleistocene & 1.22000 & n &
	Europe\tabularnewline
	189 & Obermaintor, Ebensfeld (Lichtenfels), Franken & Testudo & Testudo
	hermanni & 220.00 & mf & Lower Pleistocene & 1.30000 & n &
	Europe\tabularnewline
	190 & Leithagebirge between Au and Loretto & Testudo & Testudo
	kalksburgensis & 225.00 & mo & Burdigalian/Aquitanian & 18.00000 & n &
	Europe\tabularnewline
	191 & Eggenburg-Schindergraben, Lower Austria & Testudo & Testudo
	kalksburgensis & 230.00 & m & Burdigalian/Aquitanian & 19.96500 & n &
	Europe\tabularnewline
	192 & Wien-Kalksburg & Testudo & Testudo kalksburgensis & 275.00 & m &
	Langhian & 14.50000 & n & Europe\tabularnewline
	193 & Cova de Gràcia, Park Güell, Barcelona & Testudo & Testudo
	lunellensis & 176.00 & mo & Middle Pleistocene & 0.45350 & n &
	Europe\tabularnewline
	194 & Caverna de Gràcia, Güell park, Barcelona & Testudo & Testudo
	lunellensis & 194.00 & mf & Middle Pleistocene & 0.45000 & n &
	Europe\tabularnewline
	195 & Cova de Gràcia, Park Güell, Barcelona & Testudo & Testudo
	lunellensis & 231.00 & ev & Middle Pleistocene & 0.45350 & n &
	Europe\tabularnewline
	196 & Caverna de Gràcia, Güell park, Barcelona & Testudo & Testudo
	lunellensis & 260.70 & mf & Middle Pleistocene & 0.45000 & n &
	Europe\tabularnewline
	197 & Lakonia & Testudo & Testudo marginata & 210.00 & m & Lower
	Pleistocene & 1.72000 & n & Europe\tabularnewline
	198 & Zourida-Höhle & Testudo & Testudo marginata & 290.00 & m & Lower
	Pleistocene & 1.30000 & y & Europe\tabularnewline
	199 & Gerani-Höhle an der Nordküste Kretamin der Nähe von Rethymnon &
	Testudo & Testudo marginata & 310.00 & m & Lower Pleistocene & 1.30000 &
	y & Europe\tabularnewline
	200 & Capo Mannu near San Vero Milis, base of D4 dune, Sardinia &
	Testudo & Testudo pecorinii & 225.00 & m & Piacencian & 3.09400 & y &
	Europe\tabularnewline
	201 & Saint-Gérand-le-Puy, Allier & Testudo & Testudo promarginata &
	230.00 & mf & Burdigalian/Aquitanian & 21.50000 & n &
	Europe\tabularnewline
	202 & Saint-Gérand-le-Puy, Allier & Testudo & Testudo promarginata &
	304.70 & mf & Burdigalian/Aquitanian & 21.50000 & n &
	Europe\tabularnewline
	203 & Neuville-aux-Bois, Loiret & Testudo & Testudo promarginata &
	310.00 & mf & Burdigalian/Aquitanian & 18.00000 & n &
	Europe\tabularnewline
	204 & Sandelzhausen & Testudo & Testudo rectogularis & 213.00 & mo &
	Burdigalian/Aquitanian & 16.37000 & n & Europe\tabularnewline
	205 & Nikiti 2, Chalkidiki, Macedonia & Testudo & Testudo s. s. & 189.00
	& m & Tortonian & 8.00000 & n & Europe\tabularnewline
	206 & Liossati, Kiourka & Testudo & Testudo sp. & 1200.00 & mf &
	Zanclean & 3.96000 & n & Europe\tabularnewline
	207 & Santa-Vittoria d'Alba & Testudo & Testudo sp. & 200.00 & mf &
	Messinian & 6.16500 & n & Europe\tabularnewline
	208 & Holzmannsdorfberg bei St.~Marein & Testudo & Testudo sp. & 232.10
	& m & Tortonian & 10.75000 & n & Europe\tabularnewline
	209 & Prottes & Testudo & Testudo sp. & 245.00 & m & Tortonian & 8.30000
	& n & Europe\tabularnewline
	210 & Prottes & Testudo & Testudo sp. & 245.00 & m & Tortonian & 8.30000
	& n & Europe\tabularnewline
	211 & Prottes & Testudo & Testudo sp. & 245.00 & m & Tortonian & 8.30000
	& n & Europe\tabularnewline
	212 & Prottes & Testudo & Testudo sp. & 245.00 & m & Tortonian & 8.30000
	& n & Europe\tabularnewline
	213 & Prottes & Testudo & Testudo sp. & 245.00 & m & Tortonian & 8.30000
	& n & Europe\tabularnewline
	214 & Prottes & Testudo & Testudo sp. & 245.00 & m & Tortonian & 8.30000
	& n & Europe\tabularnewline
	215 & Prottes & Testudo & Testudo sp. & 245.00 & m & Tortonian & 8.30000
	& n & Europe\tabularnewline
	216 & Prottes & Testudo & Testudo sp. & 245.00 & m & Tortonian & 8.30000
	& n & Europe\tabularnewline
	217 & Prottes & Testudo & Testudo sp. & 245.00 & m & Tortonian & 8.30000
	& n & Europe\tabularnewline
	218 & Prottes & Testudo & Testudo sp. & 245.00 & m & Tortonian & 8.30000
	& n & Europe\tabularnewline
	219 & Prottes & Testudo & Testudo sp. & 245.00 & m & Tortonian & 8.30000
	& n & Europe\tabularnewline
	220 & Prottes & Testudo & Testudo sp. & 245.00 & m & Tortonian & 8.30000
	& n & Europe\tabularnewline
	221 & Prottes & Testudo & Testudo sp. & 245.00 & m & Tortonian & 8.30000
	& n & Europe\tabularnewline
	222 & Prottes & Testudo & Testudo sp. & 245.00 & m & Tortonian & 8.30000
	& n & Europe\tabularnewline
	223 & Prottes & Testudo & Testudo sp. & 245.00 & m & Tortonian & 8.30000
	& n & Europe\tabularnewline
	224 & Prottes & Testudo & Testudo sp. & 245.00 & m & Tortonian & 8.30000
	& n & Europe\tabularnewline
	225 & Prottes & Testudo & Testudo sp. & 245.00 & m & Tortonian & 8.30000
	& n & Europe\tabularnewline
	226 & Prottes & Testudo & Testudo sp. & 245.00 & m & Tortonian & 8.30000
	& n & Europe\tabularnewline
	227 & Prottes & Testudo & Testudo sp. & 245.00 & m & Tortonian & 8.30000
	& n & Europe\tabularnewline
	228 & Prottes & Testudo & Testudo sp. & 245.00 & m & Tortonian & 8.30000
	& n & Europe\tabularnewline
	229 & Prottes & Testudo & Testudo sp. & 245.00 & m & Tortonian & 8.30000
	& n & Europe\tabularnewline
	230 & Prottes & Testudo & Testudo sp. & 245.00 & m & Tortonian & 8.30000
	& n & Europe\tabularnewline
	231 & Prottes & Testudo & Testudo sp. & 245.00 & m & Tortonian & 8.30000
	& n & Europe\tabularnewline
	232 & Prottes & Testudo & Testudo sp. & 245.00 & m & Tortonian & 8.30000
	& n & Europe\tabularnewline
	233 & Prottes & Testudo & Testudo sp. & 245.00 & m & Tortonian & 8.30000
	& n & Europe\tabularnewline
	234 & Prottes & Testudo & Testudo sp. & 245.00 & m & Tortonian & 8.30000
	& n & Europe\tabularnewline
	235 & Prottes & Testudo & Testudo sp. & 245.00 & m & Tortonian & 8.30000
	& n & Europe\tabularnewline
	236 & Prottes & Testudo & Testudo sp. & 245.00 & m & Tortonian & 8.30000
	& n & Europe\tabularnewline
	237 & Prottes & Testudo & Testudo sp. & 245.00 & m & Tortonian & 8.30000
	& n & Europe\tabularnewline
	238 & Prottes & Testudo & Testudo sp. & 245.00 & m & Tortonian & 8.30000
	& n & Europe\tabularnewline
	239 & Prottes & Testudo & Testudo sp. & 245.00 & m & Tortonian & 8.30000
	& n & Europe\tabularnewline
	240 & Prottes & Testudo & Testudo sp. & 245.00 & m & Tortonian & 8.30000
	& n & Europe\tabularnewline
	241 & Prottes & Testudo & Testudo sp. & 245.00 & m & Tortonian & 8.30000
	& n & Europe\tabularnewline
	242 & Prottes & Testudo & Testudo sp. & 245.00 & m & Tortonian & 8.30000
	& n & Europe\tabularnewline
	243 & Prottes & Testudo & Testudo sp. & 245.00 & m & Tortonian & 8.30000
	& n & Europe\tabularnewline
	244 & Prottes & Testudo & Testudo sp. & 245.00 & m & Tortonian & 8.30000
	& n & Europe\tabularnewline
	245 & Megalo Emvolon 1 (MEV), 20 km SW Thessaloniki & Testudo & Testudo
	sp. & 2500.00 & mf & Zanclean & 3.90000 & n & Europe\tabularnewline
	246 & Megalo Emvolon 1 (MEV), 20 km SW Thessaloniki & Testudo & Testudo
	sp. & 2500.00 & mf & Zanclean & 3.90000 & n & Europe\tabularnewline
	247 & Megalo Emvolon 1 (MEV), 20 km SW Thessaloniki & Testudo & Testudo
	sp. & 2500.00 & mf & Zanclean & 3.90000 & n & Europe\tabularnewline
	248 & Megalo Emvolon 1 (MEV), 20 km SW Thessaloniki & Testudo & Testudo
	sp. & 2500.00 & mf & Zanclean & 3.90000 & n & Europe\tabularnewline
	249 & W??e 1 & Testudo & Testudo sp. & 500.00 & mo & Zanclean & 3.90000
	& n & Europe\tabularnewline
	250 & Altenstadt, 7 km S Illertissen & Testudo & Testudo steinheimensis
	& 111.00 & m & Serravallian & 12.15000 & n & Europe\tabularnewline
	251 & Steinheim a. Albuch & Testudo & Testudo steinheimensis & 227.70 &
	mf & Serravallian & 13.00000 & n & Europe\tabularnewline
	252 & Lesbos Island, F-Site & Titanochelon & Titanochelon aff. schafferi
	& 1860.00 & m & Gelasian & 2.00000 & y & Europe\tabularnewline
	253 & Epanomi (EPN II), western Chalkidiki Peninsula, Thessaloniki area
	& Titanochelon & Titanochelon bacharidisi & 1164.00 & m & Zanclean &
	3.95000 & n & Europe\tabularnewline
	254 & Epanomi (EPN I), western Chalkidiki Peninsula, Thessaloniki area &
	Titanochelon & Titanochelon bacharidisi & 1196.00 & m & Zanclean &
	3.95000 & n & Europe\tabularnewline
	255 & Nea Michaniona, western Chalkidiki Peninsula, Thessaloniki area &
	Titanochelon & Titanochelon bacharidisi & 900.00 & mo & Zanclean &
	3.95000 & n & Europe\tabularnewline
	256 & Nea Kallikratia, western Chalkidiki Peninsula, Thessaloniki area &
	Titanochelon & Titanochelon bacharidisi & 900.00 & mo & Zanclean &
	3.95000 & n & Europe\tabularnewline
	257 & Nea Michaniona, western Chalkidiki Peninsula, Thessaloniki area &
	Titanochelon & Titanochelon bacharidisi & 900.00 & mo & Zanclean &
	3.95000 & n & Europe\tabularnewline
	258 & Nea Kallikratia, western Chalkidiki Peninsula, Thessaloniki area &
	Titanochelon & Titanochelon bacharidisi & 900.00 & mo & Zanclean &
	3.95000 & n & Europe\tabularnewline
	259 & Vallecas, Madrid & Titanochelon & Titanochelon bolivari & 1100.00
	& mo & Langhian & 15.00000 & n & Europe\tabularnewline
	260 & Puerto de la Cadena, Murcia & Titanochelon & Titanochelon bolivari
	& 1150.00 & m & Messinian & 6.28900 & n & Europe\tabularnewline
	261 & Alcalá de Henares, Cerro del Viso (Barranco de los Mártires y
	Santos de la Humosa), Madrid & Titanochelon & Titanochelon bolivari &
	1250.00 & mo & Langhian & 15.00000 & n & Europe\tabularnewline
	262 & Cerro de los Batallones, Madrid & Titanochelon & Titanochelon
	bolivari & 1300.00 & mf & Tortonian & 9.50000 & n &
	Europe\tabularnewline
	263 & Cerro del Otero, Palencia & Titanochelon & Titanochelon bolivari &
	1353.00 & mo & Serravallian & 12.50000 & n & Europe\tabularnewline
	264 & Charneco do Lumiar & Titanochelon & Titanochelon cf.~bolivari &
	1300.00 & ev & Langhian & 14.89500 & n & Europe\tabularnewline
	265 & Aveiras de Baixo, Azambuja & Titanochelon & Titanochelon
	cf.~bolivari & 1500.00 & mf & Tortonian & 9.43300 & n &
	Europe\tabularnewline
	266 & Quinta da Farinheira & Titanochelon & Titanochelon cf.~bolivari &
	1600.00 & ef & Langhian & 14.89500 & n & Europe\tabularnewline
	267 & Sandelzhausen unterer Geröllmergel (B) & Titanochelon &
	Titanochelon cf.~perpiniana & 1001.00 & mo & Burdigalian/Aquitanian &
	16.37000 & n & Europe\tabularnewline
	268 & Cala Es Pous near Ciutadella, Minorca & Titanochelon &
	Titanochelon gymnesica & 1300.00 & ef & Lower Pleistocene & 1.30000 & y
	& Europe\tabularnewline
	269 & Serrat-d'en-Vacquer near Perpignan, Pyrénées-Orientales &
	Titanochelon & Titanochelon perpiniana & 1140.00 & m & Zanclean &
	3.90000 & n & Europe\tabularnewline
	270 & Samos 1 & Titanochelon & Titanochelon schafferi & 1850.00 & m &
	Messinian & 6.25000 & y & Europe\tabularnewline
	271 & Pikermi & Titanochelon & Titanochelon schafferi & 2500.00 & mo &
	Zanclean & 4.46600 & n & Europe\tabularnewline
	272 & Fonelas P-1, Guadix Basin & Titanochelon & Titanochelon sp. &
	1420.00 & mo & Gelasian & 1.85000 & n & Europe\tabularnewline
	273 & Cova de Ca Na Reia, Eivissa, Ibiza & Titanochelon & Titanochelon
	sp. & 520.00 & mo & Piacencian & 2.60000 & y & Europe\tabularnewline
	274 & Plum Point, Calvert County, Maryland & Caudochelys & Caudochelys
	ducateli & 339.90 & m & Langhian & 15.00000 & n & America\tabularnewline
	275 & Rexroad local fauna (Fox Canyon locality 3), Meade County, Kansas
	& Caudochelys & Caudochelys rexroadensis & 781.00 & m & Zanclean &
	4.55000 & n & America\tabularnewline
	276 & Rexroad local fauna (Fox Canyon locality 3), Meade County, Kansas
	& Caudochelys & Caudochelys rexroadensis & 830.00 & m & Zanclean &
	4.55000 & n & America\tabularnewline
	277 & Garvin Gullv, 2 mi. north of Navasota, Jl J . Grimes County,
	Texas, Garvin Gullv local fauna & Caudochelys & Caudochelys williamsi &
	334.00 & m & Burdigalian/Aquitanian & 17.75000 & n &
	America\tabularnewline
	278 & Gilliland local fauna, Burnett Ranch, 7 miles W of Vera, Knox
	County, Texas & Geochelone & Geochelone sp. & 170.00 & mf & Middle
	Pleistocene & 0.70000 & n & America\tabularnewline
	279 & Santee, Knox County, Nebraska & Geochelone & Geochelone sp. &
	176.00 & e & Zanclean & 5.00000 & n & America\tabularnewline
	280 & Orange Lake 2 miles south, Marion County, Florida & Geochelone &
	Geochelone sp. & 350.00 & ef & Upper Pleistocene & 0.06900 & n &
	America\tabularnewline
	281 & Ricardo Fauna, Mojave Desert, Kern County, California & Geochelone
	& Geochelone sp. & 500.00 & m & Tortonian & 10.10000 & n &
	America\tabularnewline
	282 & Banana Hole, New Providence Island & Geochelone & Geochelone sp. &
	600.00 & mo & Upper Pleistocene & 0.01250 & y & America\tabularnewline
	283 & Lee Creek Mine, Yorktown Sample, Beaufort County, North Carolina &
	Geochelone & Geochelone sp. & 880.00 & m & Zanclean & 4.50000 & n &
	America\tabularnewline
	284 & Thomas Farm Local Fauna, Gilchrist County, Florida & Geochelone &
	Geochelone tedwhitei & 370.00 & m & Burdigalian/Aquitanian & 18.50000 &
	n & America\tabularnewline
	285 & Thomas Farm Local Fauna, Gilchrist County, Florida & Geochelone &
	Geochelone tedwhitei & 440.00 & m & Burdigalian/Aquitanian & 18.50000 &
	n & America\tabularnewline
	286 & Ricardo Fauna, Mojave Desert, Kern County, California & Gopherus &
	Gopherus ? sp. & 500.00 & m & Tortonian & 10.10000 & n &
	America\tabularnewline
	287 & Iron Canyon Fauna, Mojave Desert, Kern County, California &
	Gopherus & Gopherus ? sp. & 500.00 & m & Serravallian & 11.85000 & n &
	America\tabularnewline
	288 & Sabertooth Camel Maze, Dry Cave (UTEP 5), Eddy County, New Mexico
	& Gopherus & Gopherus agassizi & 252.00 & m & Upper Pleistocene &
	0.02550 & n & America\tabularnewline
	289 & Pecos River near Melena and Acme, 10-15 km NE Roswell, Chaves
	County, New Mexico & Gopherus & Gopherus agassizi & 445.00 & mo & Middle
	Pleistocene & 0.15600 & n & America\tabularnewline
	290 & North Cita Canyon (Middle Stratum), Randall County, Texas &
	Gopherus & Gopherus canyonensis & 885.50 & m & Piacencian & 2.70000 & n
	& America\tabularnewline
	291 & Texas & Gopherus & Gopherus laticaudatus & 375.00 & mo & Middle
	Pleistocene & 0.39635 & n & America\tabularnewline
	292 & Barstow Beds, San Bernardino County, California & Gopherus &
	Gopherus mohavetus & 202.00 & m & Tortonian & 8.47600 & n &
	America\tabularnewline
	293 & Cache Peak fauna, Tehachapi Mountains, Kern County, California &
	Gopherus & Gopherus mohavetus & 315.00 & m & Tortonian & 8.47600 & n &
	America\tabularnewline
	294 & Barstow Beds, San Bernardino County, California & Gopherus &
	Gopherus mohavetus & 334.50 & m & Tortonian & 8.47600 & n &
	America\tabularnewline
	295 & Barstow Beds, San Bernardino County, California & Gopherus &
	Gopherus mohavetus & 360.00 & m & Tortonian & 8.47600 & n &
	America\tabularnewline
	296 & Barstow Beds, San Bernardino County, California & Gopherus &
	Gopherus mohavetus & 412.50 & m & Tortonian & 8.47600 & n &
	America\tabularnewline
	297 & Texas & Gopherus & Gopherus pertenuis & 1050.00 & mo & Lower
	Pleistocene & 1.68450 & n & America\tabularnewline
	298 & Surprise Cave, Alachua, Florida & Gopherus & Gopherus polyphemus &
	102.44 & mo & Upper Pleistocene & 0.06900 & n & America\tabularnewline
	299 & Reddick IA+B, Marion County, Florida & Gopherus & Gopherus
	polyphemus & 155.50 & mo & Upper Pleistocene & 0.06900 & n &
	America\tabularnewline
	300 & Leisey Shell Pit 1A, Hillsborough County, Florida & Gopherus &
	Gopherus polyphemus & 217.90 & mo & Lower Pleistocene & 1.20000 & n &
	America\tabularnewline
	301 & Haile, Alachua County, Florida & Gopherus & Gopherus polyphemus &
	239.80 & mo & Middle Pleistocene & 0.25000 & n & America\tabularnewline
	302 & Surprise Cave, Alachua, Florida & Gopherus & Gopherus polyphemus &
	252.56 & mo & Upper Pleistocene & 0.06900 & n & America\tabularnewline
	303 & Haile, Alachua County, Florida & Gopherus & Gopherus polyphemus &
	253.70 & mo & Middle Pleistocene & 0.25000 & n & America\tabularnewline
	304 & Haile, Alachua County, Florida & Gopherus & Gopherus polyphemus &
	256.44 & mo & Middle Pleistocene & 0.25000 & n & America\tabularnewline
	305 & Haile, Alachua County, Florida & Gopherus & Gopherus polyphemus &
	257.80 & mo & Middle Pleistocene & 0.25000 & n & America\tabularnewline
	306 & Surprise Cave, Alachua, Florida & Gopherus & Gopherus polyphemus &
	258.30 & mo & Upper Pleistocene & 0.06900 & n & America\tabularnewline
	307 & Surprise Cave, Alachua, Florida & Gopherus & Gopherus polyphemus &
	260.11 & mo & Upper Pleistocene & 0.06900 & n & America\tabularnewline
	308 & Coleman 2A & Gopherus & Gopherus polyphemus & 260.50 & mo & Middle
	Pleistocene & 0.40000 & n & America\tabularnewline
	309 & Coleman 2A & Gopherus & Gopherus polyphemus & 260.51 & mo & Middle
	Pleistocene & 0.40000 & n & America\tabularnewline
	310 & Haile, Alachua County, Florida & Gopherus & Gopherus polyphemus &
	267.00 & mo & Middle Pleistocene & 0.25000 & n & America\tabularnewline
	311 & Leisey Shell Pit 1A, Hillsborough County, Florida & Gopherus &
	Gopherus polyphemus & 268.90 & mo & Lower Pleistocene & 1.20000 & n &
	America\tabularnewline
	312 & Haile, Alachua County, Florida & Gopherus & Gopherus polyphemus &
	272.48 & mo & Middle Pleistocene & 0.25000 & n & America\tabularnewline
	313 & Coleman 2A & Gopherus & Gopherus polyphemus & 272.57 & mo & Middle
	Pleistocene & 0.40000 & n & America\tabularnewline
	314 & Surprise Cave, Alachua, Florida & Gopherus & Gopherus polyphemus &
	273.24 & mo & Upper Pleistocene & 0.06900 & n & America\tabularnewline
	315 & Haile, Alachua County, Florida & Gopherus & Gopherus polyphemus &
	274.30 & mo & Middle Pleistocene & 0.25000 & n & America\tabularnewline
	316 & Leisey Shell Pit 1A, Hillsborough County, Florida & Gopherus &
	Gopherus polyphemus & 276.60 & mo & Lower Pleistocene & 1.20000 & n &
	America\tabularnewline
	317 & Surprise Cave, Alachua, Florida & Gopherus & Gopherus polyphemus &
	278.00 & mo & Upper Pleistocene & 0.06900 & n & America\tabularnewline
	318 & Surprise Cave, Alachua, Florida & Gopherus & Gopherus polyphemus &
	279.94 & mo & Upper Pleistocene & 0.06900 & n & America\tabularnewline
	319 & Haile, Alachua County, Florida & Gopherus & Gopherus polyphemus &
	283.00 & mo & Middle Pleistocene & 0.25000 & n & America\tabularnewline
	320 & Haile, Alachua County, Florida & Gopherus & Gopherus polyphemus &
	283.41 & mo & Middle Pleistocene & 0.25000 & n & America\tabularnewline
	321 & Surprise Cave, Alachua, Florida & Gopherus & Gopherus polyphemus &
	284.90 & mo & Upper Pleistocene & 0.06900 & n & America\tabularnewline
	322 & Haile, Alachua County, Florida & Gopherus & Gopherus polyphemus &
	285.20 & mo & Middle Pleistocene & 0.25000 & n & America\tabularnewline
	323 & Coleman 2A & Gopherus & Gopherus polyphemus & 285.60 & mo & Middle
	Pleistocene & 0.40000 & n & America\tabularnewline
	324 & Haile, Alachua County, Florida & Gopherus & Gopherus polyphemus &
	292.00 & mo & Middle Pleistocene & 0.25000 & n & America\tabularnewline
	325 & Haile, Alachua County, Florida & Gopherus & Gopherus polyphemus &
	292.94 & mo & Middle Pleistocene & 0.25000 & n & America\tabularnewline
	326 & Coleman 2A & Gopherus & Gopherus polyphemus & 293.00 & mo & Middle
	Pleistocene & 0.40000 & n & America\tabularnewline
	327 & Coleman 2A & Gopherus & Gopherus polyphemus & 293.57 & mo & Middle
	Pleistocene & 0.40000 & n & America\tabularnewline
	328 & Surprise Cave, Alachua, Florida & Gopherus & Gopherus polyphemus &
	294.16 & mo & Upper Pleistocene & 0.06900 & n & America\tabularnewline
	329 & Coleman 2A & Gopherus & Gopherus polyphemus & 295.90 & mo & Middle
	Pleistocene & 0.40000 & n & America\tabularnewline
	330 & Surprise Cave, Alachua, Florida & Gopherus & Gopherus polyphemus &
	301.97 & mo & Upper Pleistocene & 0.06900 & n & America\tabularnewline
	331 & Surprise Cave, Alachua, Florida & Gopherus & Gopherus polyphemus &
	302.40 & mo & Upper Pleistocene & 0.06900 & n & America\tabularnewline
	332 & Haile, Alachua County, Florida & Gopherus & Gopherus polyphemus &
	302.40 & mo & Middle Pleistocene & 0.25000 & n & America\tabularnewline
	333 & Surprise Cave, Alachua, Florida & Gopherus & Gopherus polyphemus &
	304.20 & mo & Upper Pleistocene & 0.06900 & n & America\tabularnewline
	334 & Coleman 2A & Gopherus & Gopherus polyphemus & 304.70 & mo & Middle
	Pleistocene & 0.40000 & n & America\tabularnewline
	335 & Haile, Alachua County, Florida & Gopherus & Gopherus polyphemus &
	306.00 & mo & Middle Pleistocene & 0.25000 & n & America\tabularnewline
	336 & Haile, Alachua County, Florida & Gopherus & Gopherus polyphemus &
	306.00 & mo & Middle Pleistocene & 0.25000 & n & America\tabularnewline
	337 & Haile, Alachua County, Florida & Gopherus & Gopherus polyphemus &
	306.00 & mo & Middle Pleistocene & 0.25000 & n & America\tabularnewline
	338 & Haile, Alachua County, Florida & Gopherus & Gopherus polyphemus &
	306.00 & mo & Middle Pleistocene & 0.25000 & n & America\tabularnewline
	339 & Coleman 2A & Gopherus & Gopherus polyphemus & 308.20 & mo & Middle
	Pleistocene & 0.40000 & n & America\tabularnewline
	340 & Haile, Alachua County, Florida & Gopherus & Gopherus polyphemus &
	314.60 & mo & Middle Pleistocene & 0.25000 & n & America\tabularnewline
	341 & Haile, Alachua County, Florida & Gopherus & Gopherus polyphemus &
	322.63 & mo & Middle Pleistocene & 0.25000 & n & America\tabularnewline
	342 & Surprise Cave, Alachua, Florida & Gopherus & Gopherus polyphemus &
	324.00 & mo & Upper Pleistocene & 0.06900 & n & America\tabularnewline
	343 & Reddick IA+B, Marion County, Florida & Gopherus & Gopherus
	polyphemus & 327.60 & mo & Upper Pleistocene & 0.06900 & n &
	America\tabularnewline
	344 & Surprise Cave, Alachua, Florida & Gopherus & Gopherus polyphemus &
	334.70 & mo & Upper Pleistocene & 0.06900 & n & America\tabularnewline
	345 & Haile, Alachua County, Florida & Gopherus & Gopherus polyphemus &
	337.30 & mo & Middle Pleistocene & 0.25000 & n & America\tabularnewline
	346 & Coleman 2A & Gopherus & Gopherus polyphemus & 348.70 & mo & Middle
	Pleistocene & 0.40000 & n & America\tabularnewline
	347 & Surprise Cave, Alachua, Florida & Gopherus & Gopherus polyphemus &
	350.00 & mo & Upper Pleistocene & 0.06900 & n & America\tabularnewline
	348 & Coleman 2A & Gopherus & Gopherus polyphemus & 350.83 & mo & Middle
	Pleistocene & 0.40000 & n & America\tabularnewline
	349 & Little Salt Spring, Florida & Gopherus & Gopherus polyphemus &
	352.00 & mo & Upper Pleistocene & 0.01200 & n & America\tabularnewline
	350 & Coleman 2A & Gopherus & Gopherus polyphemus & 353.30 & mo & Middle
	Pleistocene & 0.40000 & n & America\tabularnewline
	351 & Reddick IA+B, Marion County, Florida & Gopherus & Gopherus
	polyphemus & 391.90 & mo & Upper Pleistocene & 0.06900 & n &
	America\tabularnewline
	352 & Surprise Cave, Alachua, Florida & Gopherus & Gopherus polyphemus &
	431.48 & mo & Upper Pleistocene & 0.06900 & n & America\tabularnewline
	353 & Gilliland local fauna, Burnett Ranch, 7 miles W of Vera, Knox
	County, Texas & Gopherus & Gopherus polyphemus & 539.00 & mf & Middle
	Pleistocene & 0.70000 & n & America\tabularnewline
	354 & Melbourne, Brevard County, Florida & Gopherus & Gopherus
	praecedens & 360.00 & mo & Upper Pleistocene & 0.06900 & n &
	America\tabularnewline
	355 & Inglis 1A, Florida & Gopherus & Gopherus sp. & 118.90 & mo &
	Gelasian & 1.90000 & n & America\tabularnewline
	356 & Inglis 1A, Florida & Gopherus & Gopherus sp. & 143.90 & mo &
	Gelasian & 1.90000 & n & America\tabularnewline
	357 & Inglis 1A, Florida & Gopherus & Gopherus sp. & 163.50 & mo &
	Gelasian & 1.90000 & n & America\tabularnewline
	358 & Inglis 1A, Florida & Gopherus & Gopherus sp. & 180.90 & mo &
	Gelasian & 1.90000 & n & America\tabularnewline
	359 & Inglis 1A, Florida & Gopherus & Gopherus sp. & 181.00 & mo &
	Gelasian & 1.90000 & n & America\tabularnewline
	360 & Inglis 1A, Florida & Gopherus & Gopherus sp. & 181.00 & mo &
	Gelasian & 1.90000 & n & America\tabularnewline
	361 & Inglis 1A, Florida & Gopherus & Gopherus sp. & 181.00 & mo &
	Gelasian & 1.90000 & n & America\tabularnewline
	362 & Inglis 1A, Florida & Gopherus & Gopherus sp. & 181.00 & mo &
	Gelasian & 1.90000 & n & America\tabularnewline
	363 & Inglis 1A, Florida & Gopherus & Gopherus sp. & 182.30 & mo &
	Gelasian & 1.90000 & n & America\tabularnewline
	364 & Inglis 1A, Florida & Gopherus & Gopherus sp. & 188.30 & mo &
	Gelasian & 1.90000 & n & America\tabularnewline
	365 & Inglis 1A, Florida & Gopherus & Gopherus sp. & 188.70 & mo &
	Gelasian & 1.90000 & n & America\tabularnewline
	366 & Inglis 1A, Florida & Gopherus & Gopherus sp. & 193.30 & mo &
	Gelasian & 1.90000 & n & America\tabularnewline
	367 & Inglis 1A, Florida & Gopherus & Gopherus sp. & 194.90 & mo &
	Gelasian & 1.90000 & n & America\tabularnewline
	368 & Inglis 1C, Florida & Gopherus & Gopherus sp. & 202.80 & mo & Lower
	Pleistocene & 1.80000 & n & America\tabularnewline
	369 & Inglis 1A, Florida & Gopherus & Gopherus sp. & 204.40 & mo &
	Gelasian & 1.90000 & n & America\tabularnewline
	370 & Inglis 1A, Florida & Gopherus & Gopherus sp. & 209.60 & mo &
	Gelasian & 1.90000 & n & America\tabularnewline
	371 & Inglis 1A, Florida & Gopherus & Gopherus sp. & 218.80 & mo &
	Gelasian & 1.90000 & n & America\tabularnewline
	372 & Inglis 1C, Florida & Gopherus & Gopherus sp. & 224.10 & mo & Lower
	Pleistocene & 1.80000 & n & America\tabularnewline
	373 & Inglis 1C, Florida & Gopherus & Gopherus sp. & 230.10 & mo & Lower
	Pleistocene & 1.80000 & n & America\tabularnewline
	374 & Inglis 1A, Florida & Gopherus & Gopherus sp. & 236.70 & mo &
	Gelasian & 1.90000 & n & America\tabularnewline
	375 & Inglis 1C, Florida & Gopherus & Gopherus sp. & 241.90 & mo & Lower
	Pleistocene & 1.80000 & n & America\tabularnewline
	376 & Inglis 1C, Florida & Gopherus & Gopherus sp. & 245.40 & mo & Lower
	Pleistocene & 1.80000 & n & America\tabularnewline
	377 & Inglis 1C, Florida & Gopherus & Gopherus sp. & 259.50 & mo & Lower
	Pleistocene & 1.80000 & n & America\tabularnewline
	378 & McGehee Farm near Newberry, Alachua County, Florida &
	Hesperotestudo & Hesperotestudo alleni & 240.90 & m & Tortonian &
	10.95000 & n & America\tabularnewline
	379 & Texas & Hesperotestudo & Hesperotestudo campester & 1000.00 & mo &
	Gelasian & 2.19050 & n & America\tabularnewline
	380 & Little Salt Spring, Florida & Hesperotestudo & Hesperotestudo
	crassiscutata & 1250.00 & ev & Upper Pleistocene & 0.01200 & n &
	America\tabularnewline
	381 & Haile, Alachua County, Florida & Hesperotestudo & Hesperotestudo
	crassiscutata & 168.00 & m & Lower Pleistocene & 1.30000 & n &
	America\tabularnewline
	382 & Haile, Alachua County, Florida & Hesperotestudo & Hesperotestudo
	crassiscutata & 180.00 & m & Lower Pleistocene & 1.30000 & n &
	America\tabularnewline
	383 & Reddick IA+B, Marion County, Florida & Hesperotestudo &
	Hesperotestudo crassiscutata & 180.40 & m & Upper Pleistocene & 0.06900
	& n & America\tabularnewline
	384 & Little Salt Spring, Florida & Hesperotestudo & Hesperotestudo
	crassiscutata & 188.00 & mo & Upper Pleistocene & 0.01200 & n &
	America\tabularnewline
	385 & Haile, Alachua County, Florida & Hesperotestudo & Hesperotestudo
	crassiscutata & 192.00 & m & Lower Pleistocene & 1.30000 & n &
	America\tabularnewline
	386 & Reddick IA+B, Marion County, Florida & Hesperotestudo &
	Hesperotestudo crassiscutata & 282.70 & m & Upper Pleistocene & 0.06900
	& n & America\tabularnewline
	387 & Reddick IA+B, Marion County, Florida & Hesperotestudo &
	Hesperotestudo crassiscutata & 284.90 & m & Upper Pleistocene & 0.06900
	& n & America\tabularnewline
	388 & Haile, Alachua County, Florida & Hesperotestudo & Hesperotestudo
	crassiscutata & 327.00 & m & Lower Pleistocene & 1.30000 & n &
	America\tabularnewline
	389 & Little Salt Spring, Florida & Hesperotestudo & Hesperotestudo
	crassiscutata & 425.00 & mo & Upper Pleistocene & 0.01200 & n &
	America\tabularnewline
	390 & Leisey Shell Pit 1A, Hillsborough County, Florida & Hesperotestudo
	& Hesperotestudo crassiscutata & 561.00 & m & Lower Pleistocene &
	1.25000 & n & America\tabularnewline
	391 & Cragin Quarry Local Fauna, Meade County, Kansas & Hesperotestudo &
	Hesperotestudo equicomes & 340.00 & ev & Middle Pleistocene & 0.30000 &
	n & America\tabularnewline
	392 & Haile, Alachua County, Florida & Hesperotestudo & Hesperotestudo
	incisa & 212.00 & m & Lower Pleistocene & 1.30000 & n &
	America\tabularnewline
	393 & Haile, Alachua County, Florida & Hesperotestudo & Hesperotestudo
	incisa & 216.00 & m & Lower Pleistocene & 1.30000 & n &
	America\tabularnewline
	394 & Haile, Alachua County, Florida & Hesperotestudo & Hesperotestudo
	incisa & 224.00 & m & Lower Pleistocene & 1.30000 & n &
	America\tabularnewline
	395 & Haile, Alachua County, Florida & Hesperotestudo & Hesperotestudo
	incisa & 228.00 & m & Lower Pleistocene & 1.30000 & n &
	America\tabularnewline
	396 & Haile, Alachua County, Florida & Hesperotestudo & Hesperotestudo
	incisa & 231.00 & m & Lower Pleistocene & 1.30000 & n &
	America\tabularnewline
	397 & Arredondo IIA, Alachua County, Florida & Hesperotestudo &
	Hesperotestudo incisa & 232.76 & m & Upper Pleistocene & 0.06900 & n &
	America\tabularnewline
	398 & Haile, Alachua County, Florida & Hesperotestudo & Hesperotestudo
	incisa & 241.00 & m & Lower Pleistocene & 1.30000 & n &
	America\tabularnewline
	399 & Haile, Alachua County, Florida & Hesperotestudo & Hesperotestudo
	incisa & 290.40 & m & Lower Pleistocene & 1.30000 & n &
	America\tabularnewline
	400 & Cita Canyon, UCMP V-3721, Harrell Ranch, Randall County, Texas &
	Hesperotestudo & Hesperotestudo johnstoni & 235.00 & m & Piacencian &
	3.35000 & n & America\tabularnewline
	401 & Leisey Shell Pit 1A, Hillsborough County, Florida & Hesperotestudo
	& Hesperotestudo mlynarskii & 165.00 & m & Lower Pleistocene & 1.25000 &
	n & America\tabularnewline
	402 & Leisey Shell Pit 2, Hillsborough County, Florida & Hesperotestudo
	& Hesperotestudo mlynarskii & 203.50 & m & Lower Pleistocene & 1.25000 &
	n & America\tabularnewline
	403 & Sand Draw local fauna, Brown County, Nebraska & Hesperotestudo &
	Hesperotestudo oelrichi & 283.80 & m & Piacencian & 3.00000 & n &
	America\tabularnewline
	404 & UCMP V71137, Turlock Lake 10, Stanislaus County, California &
	Hesperotestudo & Hesperotestudo orthopygia & 1200.00 & mo & Messinian &
	5.50000 & n & America\tabularnewline
	405 & UCMP V81248, Turlock Lake 11, Stanislaus County, California &
	Hesperotestudo & Hesperotestudo orthopygia & 682.00 & mo & Messinian &
	5.50000 & n & America\tabularnewline
	406 & Buis Ranch Local Fauna, Beaver County, Oklahoma & Hesperotestudo &
	Hesperotestudo riggsi & 159.50 & mo & Tortonian & 7.60000 & n &
	America\tabularnewline
	407 & Buis Ranch Local Fauna, Beaver County, Oklahoma & Hesperotestudo &
	Hesperotestudo riggsi & 159.50 & mo & Tortonian & 7.60000 & n &
	America\tabularnewline
	408 & Buis Ranch Local Fauna, Beaver County, Oklahoma & Hesperotestudo &
	Hesperotestudo riggsi & 159.50 & mo & Tortonian & 7.60000 & n &
	America\tabularnewline
	409 & Buis Ranch Local Fauna, Beaver County, Oklahoma & Hesperotestudo &
	Hesperotestudo riggsi & 159.50 & mo & Tortonian & 7.60000 & n &
	America\tabularnewline
	410 & Sawrock Canyon local fauna, Seward County, Kansas & Hesperotestudo
	& Hesperotestudo riggsi & 176.00 & m & Piacencian & 3.00000 & n &
	America\tabularnewline
	411 & Sawrock Canyon local fauna, Seward County, Kansas & Hesperotestudo
	& Hesperotestudo riggsi & 185.00 & m & Piacencian & 3.00000 & n &
	America\tabularnewline
	412 & Rexroad local fauna (Fox Canyon locality 3), Meade County, Kansas
	& Hesperotestudo & Hesperotestudo riggsi & 195.80 & m & Zanclean &
	4.55000 & n & America\tabularnewline
	413 & Caballo Local Fauna, Palomas Basin, Sierra County, New Mexico &
	Hesperotestudo & Hesperotestudo sp. & 1000.00 & mo & Gelasian & 2.00000
	& n & America\tabularnewline
	414 & UCMP V-3952, Ingram Creek site 8, Stanislaus County, California &
	Hesperotestudo & Hesperotestudo sp. & 1200.00 & ev & Tortonian & 9.50000
	& n & America\tabularnewline
	415 & Gilliland local fauna, Burnett Ranch, 7 miles W of Vera, Knox
	County, Texas & Hesperotestudo & Hesperotestudo sp. & 1500.00 & mo &
	Middle Pleistocene & 0.70000 & n & America\tabularnewline
	416 & Cuchillo Negro Creek Local Fauna, Engle Basin, Sierra County, New
	Mexico & Hesperotestudo & Hesperotestudo sp. & 176.00 & mf & Piacencian
	& 3.10000 & n & America\tabularnewline
	417 & Gilliland local fauna, Burnett Ranch, 7 miles W of Vera, Knox
	County, Texas & Hesperotestudo & Hesperotestudo sp. & 1800.00 & mo &
	Middle Pleistocene & 0.70000 & n & America\tabularnewline
	418 & Ingleside Local Fauna, San Patricio County, Texas & Hesperotestudo
	& Hesperotestudo sp. & 639.00 & m & Upper Pleistocene & 0.06000 & n &
	America\tabularnewline
	419 & Ingleside Local Fauna, San Patricio County, Texas & Hesperotestudo
	& Hesperotestudo sp. & 974.00 & ep & Upper Pleistocene & 0.06000 & n &
	America\tabularnewline
	420 & Kansas & Hesperotestudo & Hesperotestudo turgida & 230.00 & mo &
	Lower Pleistocene & 1.68450 & n & America\tabularnewline
	421 & Friesenhahn Cave, Bexar County, Texas & Hesperotestudo &
	Hesperotestudo wilsoni & 226.00 & m & Upper Pleistocene & 0.01800 & n &
	America\tabularnewline
	422 & Atascosa county, Texas & Testudo & Testudo sp. & 400.00 & mo &
	Langhian & 14.18100 & n & America\tabularnewline
	423 & Libertador San Martín north bank Ensenada stream, 15 km E
	Diamante, Entre Rios Province & Chelonoidis & Chelonoidis denticulata &
	616.00 & m & Upper Pleistocene & 0.12000 & n & America\tabularnewline
	424 & Arroyo Toropí, Corrientes & Chelonoidis & Chelonoidis lutzae &
	830.00 & m & Upper Pleistocene & 0.03850 & n & America\tabularnewline
	425 & Quebrada de Ñuapua, Chuquisaca department & Chelonoidis &
	Chelonoidis sp. & 1000.00 & mo & Upper Pleistocene & 0.06900 & n &
	America\tabularnewline
	426 & Beautiful Bone, Alta Guajira Peninsula, Cocinetas basin &
	Chelonoidis & Chelonoidis sp. & 1060.00 & ec & Langhian & 15.90000 & n &
	America\tabularnewline
	427 & Beautiful Bone, Alta Guajira Peninsula, Cocinetas basin &
	Chelonoidis & Chelonoidis sp. & 300.00 & mo & Langhian & 15.90000 & n &
	America\tabularnewline
	428 & Beautiful Bone, Alta Guajira Peninsula, Cocinetas basin &
	Chelonoidis & Chelonoidis sp. & 300.00 & mo & Langhian & 15.90000 & n &
	America\tabularnewline
	429 & Beautiful Bone, Alta Guajira Peninsula, Cocinetas basin &
	Chelonoidis & Chelonoidis sp. & 300.00 & mo & Langhian & 15.90000 & n &
	America\tabularnewline
	430 & Beautiful Bone, Alta Guajira Peninsula, Cocinetas basin &
	Chelonoidis & Chelonoidis sp. & 300.00 & mo & Langhian & 15.90000 & n &
	America\tabularnewline
	431 & San Nicolas, UCMP locality V4536 & Geochelone & Geochelone
	hesterna & 278.00 & m & Tortonian & 8.50000 & n & America\tabularnewline
	\bottomrule
\end{longtable}
}

\end{landscape}

%\begin{landscape}
	
\scriptsize{	
\begin{longtable}[]{@{}llllrrrrrrllll@{}}
	\caption[Body size data set of extant \T]{Body size data set of extant testudinid. Contains information on Genus and Taxon names, collection numbers (Coll.-Nr.) and shell measurements (SCL: straight carapace length, CCL: curved carapace length, SCW: straight carapace width, CCW: curved carapace width, CH: carapace height, PL: plastron length (greatest), PW: plastron width (greatest). Further, it is stated on which continent the fossil record was recovered and whether it was continental (n: no) or insular (y: yes). Finally, the references from which the data were obtained are listed.}
	\label{tab:DataExtant}\tabularnewline
	\toprule
	& Genus & Taxon & CollNr & SCL & CCL & SCW & CCW & CH & PL & PW & Island
	& Con & Reference\tabularnewline
	\midrule
	\endfirsthead
	\multicolumn{13}{c}%
	{\tablename\ \thetable\ -- \textit{continued from previous page}}\tabularnewline
	\toprule
	& Genus & Taxon & CollNr & SCL & CCL & SCW & CCW & CH & PL & PW & Island
	& Con & Reference\tabularnewline
	\midrule
	\endhead
	1 & Kinixys & Kinixys belliana & ZMB 37388 & 162.0 & 16.20 & 22.5 & 15.5
	& 21.5 & 164.0 & 12.6 & n & Africa & freshly measured (MFN
	collection)\tabularnewline
	2 & Aldabrachelys & Aldabrachelys gigantea & ZMB 51996 & 770.0 & 77.00 &
	106.0 & 52.0 & 112.0 & NA & NA & y & Africa & freshly measured (MFN
	collection)\tabularnewline
	3 & Astrochelys & Astrochelys yniphora & - & 426.0 & 42.60 & NA & NA &
	NA & NA & NA & y & Africa & Pedrono, M., \& Smith, L. L. (2013).
	Overview of the natural history of Madagascar's endemic tortoises and
	freshwater turtles: Essential components for effective
	conservation.\tabularnewline
	4 & Centrochelys & Centrochelys sulcata & ZMB 63203 & 215.0 & 21.50 &
	29.5 & 16.5 & 27.0 & 214.0 & 14.8 & n & Africa & freshly measured (MFN
	collection)\tabularnewline
	5 & Malacochersus & Malacochersus tornieri & ZMB 63174 & 153.0 & 15.30 &
	17.0 & 10.5 & 14.0 & 149.0 & 9.8 & n & Africa & freshly measured (MFN
	collection)\tabularnewline
	6 & Astrochelys & Astrochelys radiata & - & 395.0 & 39.50 & NA & NA & NA
	& NA & NA & y & Africa & Pedrono, M., \& Smith, L. L. (2013). Overview
	of the natural history of Madagascar's endemic tortoises and freshwater
	turtles: Essential components for effective conservation.\tabularnewline
	7 & Pyxis & Pyxis arachnoides & ZMB 37616 & 110.0 & 11.00 & 15.0 & 8.0 &
	14.0 & 75.0 & 7.6 & y & Africa & freshly measured (MFN
	collection)\tabularnewline
	8 & Kinixys & Kinixys homeana & ZMB 17747 & 193.0 & 19.30 & 25.0 & 14.0
	& 21.0 & 175.0 & 11.8 & n & Africa & freshly measured (MFN
	collection)\tabularnewline
	9 & Aldabrachelys & Aldabrachelys gigantea & ZMB 47494 & 870.0 & 87.00 &
	116.0 & 57.0 & 110.0 & NA & NA & y & Africa & freshly measured (MFN
	collection)\tabularnewline
	10 & Psammobates & Psammobates tentorius & ZMB 28782 & 111.0 & 11.10 &
	15.0 & 8.5 & 14.0 & 95.0 & 7.9 & n & Africa & freshly measured (MFN
	collection)\tabularnewline
	11 & Psammobates & Psammobates oculifer & ZMB 25439 & 119.0 & 11.90 &
	17.0 & 9.0 & 14.5 & 99.0 & 8.4 & n & Africa & freshly measured (MFN
	collection)\tabularnewline
	12 & Psammobates & Psammobates oculifer & ZMB 37472 & 107.0 & 10.70 &
	15.0 & 8.4 & 13.5 & 106.0 & 8 & n & Africa & freshly measured (MFN
	collection)\tabularnewline
	13 & Astrochelys & Astrochelys yniphora & - & 307.0 & 30.70 & NA & NA &
	NA & NA & NA & y & Africa & Pedrono, M., \& Smith, L. L. (2013).
	Overview of the natural history of Madagascar's endemic tortoises and
	freshwater turtles: Essential components for effective
	conservation.\tabularnewline
	14 & Homopus & Homopus aerolatus & ZMB 229 & 88.0 & 8.80 & 10.5 & 6.9 &
	9.0 & 78.0 & 6.1 & n & Africa & freshly measured (MFN
	collection)\tabularnewline
	15 & Homopus & Homopus signatus & ZMB 63173 & 94.0 & 9.40 & 12.5 & 7.7 &
	11.0 & 82.0 & 5.6 & n & Africa & freshly measured (MFN
	collection)\tabularnewline
	16 & Kinixys & Kinixys belliana & ZMB 63191 & 194.0 & 19.40 & 25.5 &
	12.5 & 19.0 & 173.0 & 12 & n & Africa & freshly measured (MFN
	collection)\tabularnewline
	17 & Astrochelys & Astrochelys radiata & - & 285.0 & 28.50 & NA & NA &
	NA & NA & NA & y & Africa & Pedrono, M., \& Smith, L. L. (2013).
	Overview of the natural history of Madagascar's endemic tortoises and
	freshwater turtles: Essential components for effective
	conservation.\tabularnewline
	18 & Kinixys & Kinixys belliana & ZMB 63192 & 174.0 & 17.40 & 24.5 &
	11.5 & 20.5 & 143.0 & 11.1 & n & Africa & freshly measured (MFN
	collection)\tabularnewline
	19 & Kinixys & Kinixys belliana & ZMB 63193 & 157.0 & 15.70 & 21.0 & 9.9
	& 16.5 & 141.0 & 9.4 & n & Africa & freshly measured (MFN
	collection)\tabularnewline
	20 & Aldabrachelys & Aldabrachelys gigantea & ZMB 37545 & 810.0 & 81.00
	& 110.0 & 52.0 & NA & NA & NA & y & Africa & freshly measured (MFN
	collection)\tabularnewline
	21 & Chersina & Chersina angulata & ZMB 49400 & 162.0 & 16.20 & 21.5 &
	10.9 & 17.5 & 170.0 & 9.2 & n & Africa & freshly measured (MFN
	collection)\tabularnewline
	22 & Chersina & Chersina angulata & ZMB 63181 & 170.0 & 17.00 & 23.0 &
	11.4 & 19.0 & 169.0 & 10 & n & Africa & freshly measured (MFN
	collection)\tabularnewline
	23 & Chersina & Chersina angulata & ZMB 63183 & 120.0 & 12.00 & 17.0 &
	8.6 & 15.5 & 118.0 & 7.3 & n & Africa & freshly measured (MFN
	collection)\tabularnewline
	24 & Chersina & Chersina angulata & ZMB 63182 & 136.0 & 13.60 & 18.0 &
	9.9 & 16.0 & 138.0 & 8 & n & Africa & freshly measured (MFN
	collection)\tabularnewline
	25 & Kinixys & Kinixys erosa & ZMB 63190 & 164.0 & 16.40 & 21.0 & 11.2 &
	16.5 & 163.0 & 10.6 & n & Africa & freshly measured (MFN
	collection)\tabularnewline
	26 & Centrochelys & Centrochelys sulcata & ZMB 37387 & 435.0 & 43.50 &
	54.0 & 29.9 & 53.0 & 405.0 & 29.1 & n & Africa & freshly measured (MFN
	collection)\tabularnewline
	27 & Indotestudo & Indotestudo travancorica & ZMB 37717 & 224.0 & 22.40
	& 28.0 & 15.2 & 23.0 & 200.0 & 15.4 & n & Africa & freshly measured (MFN
	collection)\tabularnewline
	28 & Stigmochelys & Stigmochelys pardalis & ZMB 37344 & 405.0 & 40.50 &
	55.0 & 27.0 & 50.5 & 350.0 & 24.3 & n & Africa & freshly measured (MFN
	collection)\tabularnewline
	29 & Stigmochelys & Stigmochelys pardalis & ZMB 63235 & 315.0 & 31.50 &
	43.5 & 23.4 & 39.0 & 298.0 & 22.1 & n & Africa & freshly measured (MFN
	collection)\tabularnewline
	30 & Stigmochelys & Stigmochelys pardalis & ZMB 37495 & 297.0 & 29.70 &
	41.5 & 21.4 & 36.0 & 271.0 & 19.2 & n & Africa & freshly measured (MFN
	collection)\tabularnewline
	31 & Stigmochelys & Stigmochelys pardalis & ZMB 42400 & 345.0 & 34.50 &
	46.5 & 24.0 & 40.0 & 285.0 & 21.3 & n & Africa & freshly measured (MFN
	collection)\tabularnewline
	32 & Stigmochelys & Stigmochelys pardalis & ZMB 63232 & 350.0 & 35.00 &
	46.0 & 23.9 & 45.0 & 303.0 & 21.1 & n & Africa & freshly measured (MFN
	collection)\tabularnewline
	33 & Psammobates & Psammobates geometricus & ZMB 192 & 92.0 & 9.20 &
	13.5 & 7.1 & 13.0 & 68.0 & 6.3 & n & Africa & freshly measured (MFN
	collection)\tabularnewline
	34 & Chersina & Chersina angulata & - & 181.9 & 18.19 & NA & NA & NA &
	NA & NA & y & Africa & Itescu, Y., Karraker, N. E., Raia, P., Pritchard,
	P. C., \& Meiri, S. (2014). Is the island rule general? Turtles
	disagree. Global Ecology and Biogeography, 23(6),
	689-700.\tabularnewline
	35 & Aldabrachelys & Aldabrachelys gigantea & ZMB 47443 & 800.0 & 80.00
	& 105.0 & 51.5 & 105.0 & NA & NA & y & Africa & freshly measured (MFN
	collection)\tabularnewline
	36 & Astrochelys & Astrochelys yniphora & - & 415.0 & 41.50 & NA & NA &
	NA & NA & NA & y & Africa & Pedrono, M., \& Smith, L. L. (2013).
	Overview of the natural history of Madagascar's endemic tortoises and
	freshwater turtles: Essential components for effective
	conservation.\tabularnewline
	37 & Astrochelys & Astrochelys yniphora & - & 370.0 & 37.00 & NA & NA &
	NA & NA & NA & y & Africa & Pedrono, M., \& Smith, L. L. (2013).
	Overview of the natural history of Madagascar's endemic tortoises and
	freshwater turtles: Essential components for effective
	conservation.\tabularnewline
	38 & Aldabrachelys & Aldabrachelys gigantea & ZMB 51995 & 1030.0 &
	103.00 & 138.0 & NA & NA & NA & NA & y & Africa & freshly measured (MFN
	collection)\tabularnewline
	39 & Aldabrachelys & Aldabrachelys gigantea & ZMB ??? & 720.0 & 72.00 &
	105.5 & 55.0 & 117.0 & NA & NA & y & Africa & freshly measured (MFN
	collection)\tabularnewline
	40 & Cylindraspis & Cylindraspis triserrata & - & 1100.0 & 110.00 & NA &
	NA & NA & NA & NA & y & Africa & Itescu, Y., Karraker, N. E., Raia, P.,
	Pritchard, P. C., \& Meiri, S. (2014). Is the island rule general?
	Turtles disagree. Global Ecology and Biogeography, 23(6),
	689-700.\tabularnewline
	41 & Cylindraspis & Cylindraspis vosmaeri & - & 500.0 & 50.00 & NA & NA
	& NA & NA & NA & y & Africa & Itescu, Y., Karraker, N. E., Raia, P.,
	Pritchard, P. C., \& Meiri, S. (2014). Is the island rule general?
	Turtles disagree. Global Ecology and Biogeography, 23(6),
	689-700.\tabularnewline
	42 & Astrochelys & Astrochelys radiata & - & 334.0 & 33.40 & NA & NA &
	NA & NA & NA & y & Africa & Pedrono, M., \& Smith, L. L. (2013).
	Overview of the natural history of Madagascar's endemic tortoises and
	freshwater turtles: Essential components for effective
	conservation.\tabularnewline
	43 & Astrochelys & Astrochelys radiata & - & 305.0 & 30.50 & NA & NA &
	NA & NA & NA & y & Africa & Pedrono, M., \& Smith, L. L. (2013).
	Overview of the natural history of Madagascar's endemic tortoises and
	freshwater turtles: Essential components for effective
	conservation.\tabularnewline
	44 & Centrochelys & Centrochelys sulcata & - & 830.0 & 83.00 & NA & NA &
	NA & NA & NA & n & Africa & Itescu, Y., Karraker, N. E., Raia, P.,
	Pritchard, P. C., \& Meiri, S. (2014). Is the island rule general?
	Turtles disagree. Global Ecology and Biogeography, 23(6),
	689-700.\tabularnewline
	45 & Psammobates & Psammobates geometricus & ZMB 186 & 105.0 & 10.50 &
	13.5 & 7.4 & 13.0 & 90.0 & 6.9 & n & Africa & freshly measured (MFN
	collection)\tabularnewline
	46 & Astrochelys & Astrochelys radiata & - & 242.0 & 24.20 & NA & NA &
	NA & NA & NA & y & Africa & Pedrono, M., \& Smith, L. L. (2013).
	Overview of the natural history of Madagascar's endemic tortoises and
	freshwater turtles: Essential components for effective
	conservation.\tabularnewline
	47 & Psammobates & Psammobates tentorius & ZMB 37627 & 116.0 & 11.60 &
	15.0 & 9.4 & 14.5 & 117.0 & 8.9 & y & Africa & freshly measured (MFN
	collection)\tabularnewline
	48 & Psammobates & Psammobates tentorius & ZMB 50571 & 95.0 & 9.50 &
	12.0 & 7.3 & 12.0 & 79.0 & 7 & n & Africa & freshly measured (MFN
	collection)\tabularnewline
	49 & Psammobates & Psammobates tentorius & ZMB 14766 & 81.0 & 8.10 &
	10.5 & 6.8 & 10.0 & 67.0 & 5.9 & n & Africa & freshly measured (MFN
	collection)\tabularnewline
	50 & Pyxis & Pyxis planicauda & - & 114.0 & 11.40 & NA & NA & NA & NA &
	NA & y & Africa & Pedrono, M., \& Smith, L. L. (2013). Overview of the
	natural history of Madagascar's endemic tortoises and freshwater
	turtles: Essential components for effective conservation.\tabularnewline
	51 & Pyxis & Pyxis planicauda & - & 134.0 & 13.40 & NA & NA & NA & NA &
	NA & y & Africa & Pedrono, M., \& Smith, L. L. (2013). Overview of the
	natural history of Madagascar's endemic tortoises and freshwater
	turtles: Essential components for effective conservation.\tabularnewline
	52 & Pyxis & Pyxis planicauda & - & 120.0 & 12.00 & NA & NA & NA & NA &
	NA & y & Africa & Pedrono, M., \& Smith, L. L. (2013). Overview of the
	natural history of Madagascar's endemic tortoises and freshwater
	turtles: Essential components for effective conservation.\tabularnewline
	53 & Psammobates & Psammobates oculifer & ZMB 16399 & 111.0 & 11.10 &
	16.0 & 8.8 & 14.0 & 108.0 & 7.9 & n & Africa & freshly measured (MFN
	collection)\tabularnewline
	54 & Psammobates & Psammobates oculifer & ZMB 14772 & 101.0 & 10.10 &
	15.0 & 8.0 & 14.0 & 98.0 & 7.3 & n & Africa & freshly measured (MFN
	collection)\tabularnewline
	55 & Psammobates & Psammobates oculifer & ZMB 24261 & 103.0 & 10.30 &
	14.0 & 8.2 & 13.5 & 100.0 & 7.8 & n & Africa & freshly measured (MFN
	collection)\tabularnewline
	56 & Psammobates & Psammobates oculifer & ZMB 37623 & 105.0 & 10.50 &
	14.5 & 7.9 & 13.5 & 93.0 & 7.4 & n & Africa & freshly measured (MFN
	collection)\tabularnewline
	57 & Kinixys & Kinixys belliana & ZMB 37489 & 180.0 & 18.00 & 24.0 &
	12.0 & 20.5 & 176.0 & 11.8 & n & Africa & freshly measured (MFN
	collection)\tabularnewline
	58 & Pyxis & Pyxis planicauda & - & 160.0 & 16.00 & NA & NA & NA & NA &
	NA & y & Africa & Itescu, Y., Karraker, N. E., Raia, P., Pritchard, P.
	C., \& Meiri, S. (2014). Is the island rule general? Turtles disagree.
	Global Ecology and Biogeography, 23(6), 689-700.\tabularnewline
	59 & Psammobates & Psammobates geometricus & ZMB 50568 & 107.0 & 10.70 &
	15.0 & 7.9 & 14.5 & 79.0 & 7.3 & n & Africa & freshly measured (MFN
	collection)\tabularnewline
	60 & Aldabrachelys & Aldabrachelys gigantea & - & 875.0 & 87.50 & NA &
	NA & NA & NA & NA & y & Africa & Itescu, Y., Karraker, N. E., Raia, P.,
	Pritchard, P. C., \& Meiri, S. (2014). Is the island rule general?
	Turtles disagree. Global Ecology and Biogeography, 23(6),
	689-700.\tabularnewline
	61 & Aldabrachelys & Aldabrachelys gigantea & - & 1190.0 & 119.00 & NA &
	NA & NA & NA & NA & y & Africa & Itescu, Y., Karraker, N. E., Raia, P.,
	Pritchard, P. C., \& Meiri, S. (2014). Is the island rule general?
	Turtles disagree. Global Ecology and Biogeography, 23(6),
	689-700.\tabularnewline
	62 & Chersina & Chersina angulata & - & 202.0 & 20.20 & NA & NA & NA &
	NA & NA & n & Africa & Itescu, Y., Karraker, N. E., Raia, P., Pritchard,
	P. C., \& Meiri, S. (2014). Is the island rule general? Turtles
	disagree. Global Ecology and Biogeography, 23(6),
	689-700.\tabularnewline
	63 & Chersina & Chersina angulata & - & 351.0 & 35.10 & NA & NA & NA &
	NA & NA & y & Africa & Itescu, Y., Karraker, N. E., Raia, P., Pritchard,
	P. C., \& Meiri, S. (2014). Is the island rule general? Turtles
	disagree. Global Ecology and Biogeography, 23(6),
	689-700.\tabularnewline
	64 & Astrochelys & Astrochelys yniphora & - & 446.0 & 44.60 & NA & NA &
	NA & NA & NA & y & Africa & Itescu, Y., Karraker, N. E., Raia, P.,
	Pritchard, P. C., \& Meiri, S. (2014). Is the island rule general?
	Turtles disagree. Global Ecology and Biogeography, 23(6),
	689-700.\tabularnewline
	65 & Chersina & Chersina angulata & ZMB 37393 & 160.0 & 16.00 & 20.0 &
	10.0 & 17.5 & 158.0 & 9.2 & n & Africa & freshly measured (MFN
	collection)\tabularnewline
	66 & Kinixys & Kinixys erosa & ZMB 50198 & 271.0 & 27.10 & 31.5 & 18.5 &
	26.0 & 231.0 & 15.9 & n & Africa & freshly measured (MFN
	collection)\tabularnewline
	67 & Chersina & Chersina angulata & ZMB 37392 & 181.0 & 18.10 & 22.5 &
	11.6 & 19.0 & 177.0 & 9.7 & n & Africa & freshly measured (MFN
	collection)\tabularnewline
	68 & Psammobates & Psammobates oculifer & - & 147.0 & 14.70 & NA & NA &
	NA & NA & NA & n & Africa & Itescu, Y., Karraker, N. E., Raia, P.,
	Pritchard, P. C., \& Meiri, S. (2014). Is the island rule general?
	Turtles disagree. Global Ecology and Biogeography, 23(6),
	689-700.\tabularnewline
	69 & Psammobates & Psammobates tentorius & - & 145.0 & 14.50 & NA & NA &
	NA & NA & NA & n & Africa & Itescu, Y., Karraker, N. E., Raia, P.,
	Pritchard, P. C., \& Meiri, S. (2014). Is the island rule general?
	Turtles disagree. Global Ecology and Biogeography, 23(6),
	689-700.\tabularnewline
	70 & Pyxis & Pyxis arachnoides & - & 150.0 & 15.00 & NA & NA & NA & NA &
	NA & y & Africa & Itescu, Y., Karraker, N. E., Raia, P., Pritchard, P.
	C., \& Meiri, S. (2014). Is the island rule general? Turtles disagree.
	Global Ecology and Biogeography, 23(6), 689-700.\tabularnewline
	71 & Psammobates & Psammobates geometricus & ZMB 185 & 118.0 & 11.80 &
	18.0 & 9.1 & 16.5 & 112.0 & 8.2 & n & Africa & freshly measured (MFN
	collection)\tabularnewline
	72 & Stigmochelys & Stigmochelys pardalis & - & 720.0 & 72.00 & NA & NA
	& NA & NA & NA & n & Africa & Itescu, Y., Karraker, N. E., Raia, P.,
	Pritchard, P. C., \& Meiri, S. (2014). Is the island rule general?
	Turtles disagree. Global Ecology and Biogeography, 23(6),
	689-700.\tabularnewline
	73 & Chersina & Chersina angulata & - & 179.3 & 17.93 & NA & NA & NA &
	NA & NA & n & Africa & Itescu, Y., Karraker, N. E., Raia, P., Pritchard,
	P. C., \& Meiri, S. (2014). Is the island rule general? Turtles
	disagree. Global Ecology and Biogeography, 23(6),
	689-700.\tabularnewline
	74 & Astrochelys & Astrochelys radiata & - & 355.0 & 35.50 & NA & NA &
	NA & NA & NA & y & Africa & Pedrono, M., \& Smith, L. L. (2013).
	Overview of the natural history of Madagascar's endemic tortoises and
	freshwater turtles: Essential components for effective
	conservation.\tabularnewline
	75 & Pyxis & Pyxis planicauda & - & 126.0 & 12.60 & NA & NA & NA & NA &
	NA & y & Africa & Pedrono, M., \& Smith, L. L. (2013). Overview of the
	natural history of Madagascar's endemic tortoises and freshwater
	turtles: Essential components for effective conservation.\tabularnewline
	76 & Testudo & Testudo kleinmanni & - & 144.0 & 14.40 & NA & NA & NA &
	NA & NA & n & Africa & Itescu, Y., Karraker, N. E., Raia, P., Pritchard,
	P. C., \& Meiri, S. (2014). Is the island rule general? Turtles
	disagree. Global Ecology and Biogeography, 23(6),
	689-700.\tabularnewline
	77 & Cylindraspis & Cylindraspis indica & - & 600.0 & 60.00 & NA & NA &
	NA & NA & NA & y & Africa & Itescu, Y., Karraker, N. E., Raia, P.,
	Pritchard, P. C., \& Meiri, S. (2014). Is the island rule general?
	Turtles disagree. Global Ecology and Biogeography, 23(6),
	689-700.\tabularnewline
	78 & Astrochelys & Astrochelys yniphora & - & 361.0 & 36.10 & NA & NA &
	NA & NA & NA & y & Africa & Pedrono, M., \& Smith, L. L. (2013).
	Overview of the natural history of Madagascar's endemic tortoises and
	freshwater turtles: Essential components for effective
	conservation.\tabularnewline
	79 & Astrochelys & Astrochelys yniphora & - & 486.0 & 48.60 & NA & NA &
	NA & NA & NA & y & Africa & Pedrono, M., \& Smith, L. L. (2013).
	Overview of the natural history of Madagascar's endemic tortoises and
	freshwater turtles: Essential components for effective
	conservation.\tabularnewline
	80 & Pyxis & Pyxis planicauda & - & 148.0 & 14.80 & NA & NA & NA & NA &
	NA & y & Africa & Pedrono, M., \& Smith, L. L. (2013). Overview of the
	natural history of Madagascar's endemic tortoises and freshwater
	turtles: Essential components for effective conservation.\tabularnewline
	81 & Pyxis & Pyxis arachnoides & - & 111.0 & 11.10 & NA & NA & NA & NA &
	NA & y & Africa & Pedrono, M., \& Smith, L. L. (2013). Overview of the
	natural history of Madagascar's endemic tortoises and freshwater
	turtles: Essential components for effective conservation.\tabularnewline
	82 & Pyxis & Pyxis arachnoides & - & 110.0 & 11.00 & NA & NA & NA & NA &
	NA & y & Africa & Pedrono, M., \& Smith, L. L. (2013). Overview of the
	natural history of Madagascar's endemic tortoises and freshwater
	turtles: Essential components for effective conservation.\tabularnewline
	83 & Pyxis & Pyxis arachnoides & - & 80.0 & 8.00 & NA & NA & NA & NA &
	NA & y & Africa & Pedrono, M., \& Smith, L. L. (2013). Overview of the
	natural history of Madagascar's endemic tortoises and freshwater
	turtles: Essential components for effective conservation.\tabularnewline
	84 & Kinixys & Kinixys lobatsiana & - & 200.0 & 20.00 & NA & NA & NA &
	NA & NA & n & Africa & Itescu, Y., Karraker, N. E., Raia, P., Pritchard,
	P. C., \& Meiri, S. (2014). Is the island rule general? Turtles
	disagree. Global Ecology and Biogeography, 23(6),
	689-700.\tabularnewline
	85 & Pyxis & Pyxis arachnoides & - & 86.0 & 8.60 & NA & NA & NA & NA &
	NA & y & Africa & Pedrono, M., \& Smith, L. L. (2013). Overview of the
	natural history of Madagascar's endemic tortoises and freshwater
	turtles: Essential components for effective conservation.\tabularnewline
	86 & Pyxis & Pyxis arachnoides & - & 154.0 & 15.40 & NA & NA & NA & NA &
	NA & y & Africa & Pedrono, M., \& Smith, L. L. (2013). Overview of the
	natural history of Madagascar's endemic tortoises and freshwater
	turtles: Essential components for effective conservation.\tabularnewline
	87 & Kinixys & Kinixys homeana & - & 223.0 & 22.30 & NA & NA & NA & NA &
	NA & n & Africa & Itescu, Y., Karraker, N. E., Raia, P., Pritchard, P.
	C., \& Meiri, S. (2014). Is the island rule general? Turtles disagree.
	Global Ecology and Biogeography, 23(6), 689-700.\tabularnewline
	88 & Homopus & Homopus femoralis & - & 168.0 & 16.80 & NA & NA & NA & NA
	& NA & n & Africa & Itescu, Y., Karraker, N. E., Raia, P., Pritchard, P.
	C., \& Meiri, S. (2014). Is the island rule general? Turtles disagree.
	Global Ecology and Biogeography, 23(6), 689-700.\tabularnewline
	89 & Pyxis & Pyxis planicauda & - & 132.0 & 13.20 & NA & NA & NA & NA &
	NA & y & Africa & Pedrono, M., \& Smith, L. L. (2013). Overview of the
	natural history of Madagascar's endemic tortoises and freshwater
	turtles: Essential components for effective conservation.\tabularnewline
	90 & Homopus & Homopus aerolatus & - & 300.0 & 30.00 & NA & NA & NA & NA
	& NA & n & Africa & Itescu, Y., Karraker, N. E., Raia, P., Pritchard, P.
	C., \& Meiri, S. (2014). Is the island rule general? Turtles disagree.
	Global Ecology and Biogeography, 23(6), 689-700.\tabularnewline
	91 & Homopus & Homopus boulengeri & - & 110.0 & 11.00 & NA & NA & NA &
	NA & NA & n & Africa & Itescu, Y., Karraker, N. E., Raia, P., Pritchard,
	P. C., \& Meiri, S. (2014). Is the island rule general? Turtles
	disagree. Global Ecology and Biogeography, 23(6),
	689-700.\tabularnewline
	92 & Kinixys & Kinixys erosa & - & 400.0 & 40.00 & NA & NA & NA & NA &
	NA & n & Africa & Itescu, Y., Karraker, N. E., Raia, P., Pritchard, P.
	C., \& Meiri, S. (2014). Is the island rule general? Turtles disagree.
	Global Ecology and Biogeography, 23(6), 689-700.\tabularnewline
	93 & Chersina & Chersina angulata & ZMB 37479 & 148.0 & 14.80 & 20.0 &
	10.1 & 17.0 & 142.0 & 9.5 & n & Africa & freshly measured (MFN
	collection)\tabularnewline
	94 & Psammobates & Psammobates geometricus & - & 165.0 & 16.50 & NA & NA
	& NA & NA & NA & n & Africa & Itescu, Y., Karraker, N. E., Raia, P.,
	Pritchard, P. C., \& Meiri, S. (2014). Is the island rule general?
	Turtles disagree. Global Ecology and Biogeography, 23(6),
	689-700.\tabularnewline
	95 & Homopus & Homopus solus & - & 109.0 & 10.90 & NA & NA & NA & NA &
	NA & n & Africa & Itescu, Y., Karraker, N. E., Raia, P., Pritchard, P.
	C., \& Meiri, S. (2014). Is the island rule general? Turtles disagree.
	Global Ecology and Biogeography, 23(6), 689-700.\tabularnewline
	96 & Malacochersus & Malacochersus tornieri & - & 180.0 & 18.00 & NA &
	NA & NA & NA & NA & n & Africa & Itescu, Y., Karraker, N. E., Raia, P.,
	Pritchard, P. C., \& Meiri, S. (2014). Is the island rule general?
	Turtles disagree. Global Ecology and Biogeography, 23(6),
	689-700.\tabularnewline
	97 & Chersina & Chersina angulata & - & 153.5 & 15.35 & NA & NA & NA &
	NA & NA & n & Africa & Itescu, Y., Karraker, N. E., Raia, P., Pritchard,
	P. C., \& Meiri, S. (2014). Is the island rule general? Turtles
	disagree. Global Ecology and Biogeography, 23(6),
	689-700.\tabularnewline
	98 & Pyxis & Pyxis arachnoides & - & 144.0 & 14.40 & NA & NA & NA & NA &
	NA & y & Africa & Pedrono, M., \& Smith, L. L. (2013). Overview of the
	natural history of Madagascar's endemic tortoises and freshwater
	turtles: Essential components for effective conservation.\tabularnewline
	99 & Kinixys & Kinixys belliana & - & 230.0 & 23.00 & NA & NA & NA & NA
	& NA & n & Africa & Itescu, Y., Karraker, N. E., Raia, P., Pritchard, P.
	C., \& Meiri, S. (2014). Is the island rule general? Turtles disagree.
	Global Ecology and Biogeography, 23(6), 689-700.\tabularnewline
	100 & Aldabrachelys & Aldabrachelys gigantea & - & 1140.0 & 114.00 & NA
	& NA & NA & NA & NA & y & Africa & Itescu, Y., Karraker, N. E., Raia,
	P., Pritchard, P. C., \& Meiri, S. (2014). Is the island rule general?
	Turtles disagree. Global Ecology and Biogeography, 23(6),
	689-700.\tabularnewline
	101 & Astrochelys & Astrochelys radiata & - & 400.0 & 40.00 & NA & NA &
	NA & NA & NA & y & Africa & Itescu, Y., Karraker, N. E., Raia, P.,
	Pritchard, P. C., \& Meiri, S. (2014). Is the island rule general?
	Turtles disagree. Global Ecology and Biogeography, 23(6),
	689-700.\tabularnewline
	102 & Chersina & Chersina angulata & - & 166.4 & 16.64 & NA & NA & NA &
	NA & NA & n & Africa & Itescu, Y., Karraker, N. E., Raia, P., Pritchard,
	P. C., \& Meiri, S. (2014). Is the island rule general? Turtles
	disagree. Global Ecology and Biogeography, 23(6),
	689-700.\tabularnewline
	103 & Chersina & Chersina angulata & - & 171.6 & 17.16 & NA & NA & NA &
	NA & NA & y & Africa & Itescu, Y., Karraker, N. E., Raia, P., Pritchard,
	P. C., \& Meiri, S. (2014). Is the island rule general? Turtles
	disagree. Global Ecology and Biogeography, 23(6),
	689-700.\tabularnewline
	104 & Cylindraspis & Cylindraspis peltastes & - & 420.0 & 42.00 & NA &
	NA & NA & NA & NA & y & Africa & Itescu, Y., Karraker, N. E., Raia, P.,
	Pritchard, P. C., \& Meiri, S. (2014). Is the island rule general?
	Turtles disagree. Global Ecology and Biogeography, 23(6),
	689-700.\tabularnewline
	105 & Chersina & Chersina angulata & - & 161.3 & 16.13 & NA & NA & NA &
	NA & NA & y & Africa & Itescu, Y., Karraker, N. E., Raia, P., Pritchard,
	P. C., \& Meiri, S. (2014). Is the island rule general? Turtles
	disagree. Global Ecology and Biogeography, 23(6),
	689-700.\tabularnewline
	106 & Homopus & Homopus signatus & - & 106.0 & 10.60 & NA & NA & NA & NA
	& NA & n & Africa & Itescu, Y., Karraker, N. E., Raia, P., Pritchard, P.
	C., \& Meiri, S. (2014). Is the island rule general? Turtles disagree.
	Global Ecology and Biogeography, 23(6), 689-700.\tabularnewline
	107 & Kinixys & Kinixys spekii & - & 220.0 & 22.00 & NA & NA & NA & NA &
	NA & n & Africa & Itescu, Y., Karraker, N. E., Raia, P., Pritchard, P.
	C., \& Meiri, S. (2014). Is the island rule general? Turtles disagree.
	Global Ecology and Biogeography, 23(6), 689-700.\tabularnewline
	108 & Cylindraspis & Cylindraspis inepta & - & 1000.0 & 100.00 & NA & NA
	& NA & NA & NA & y & Africa & Itescu, Y., Karraker, N. E., Raia, P.,
	Pritchard, P. C., \& Meiri, S. (2014). Is the island rule general?
	Turtles disagree. Global Ecology and Biogeography, 23(6),
	689-700.\tabularnewline
	109 & Kinixys & Kinixys natalensis & - & 160.0 & 16.00 & NA & NA & NA &
	NA & NA & n & Africa & Itescu, Y., Karraker, N. E., Raia, P., Pritchard,
	P. C., \& Meiri, S. (2014). Is the island rule general? Turtles
	disagree. Global Ecology and Biogeography, 23(6),
	689-700.\tabularnewline
	110 & Geochelone & Geochelone elegans & ZMB 63222 & 208.0 & 20.80 & 29.5
	& 14.6 & 28.5 & 199.0 & 13.3 & n & Asia & freshly measured (MFN
	collection)\tabularnewline
	111 & Geochelone & Geochelone elegans & ZMB 37523 & 245.0 & 24.50 & 32.0
	& 16.6 & 32.0 & 228.0 & 14.6 & n & Asia & freshly measured (MFN
	collection)\tabularnewline
	112 & Geochelone & Geochelone elegans & ZMB 63220 & 221.0 & 22.10 & 32.0
	& 16.0 & 31.0 & 179.0 & 13.5 & n & Asia & freshly measured (MFN
	collection)\tabularnewline
	113 & Geochelone & Geochelone elegans & ZMB 63221 & 220.0 & 22.00 & 31.0
	& 15.4 & 27.0 & 209.0 & 14 & y & Asia & freshly measured (MFN
	collection)\tabularnewline
	114 & Geochelone & Geochelone elegans & ZMB 63218 & 221.0 & 22.10 & 31.5
	& 15.1 & 30.0 & 203.0 & 13.7 & n & Asia & freshly measured (MFN
	collection)\tabularnewline
	115 & Geochelone & Geochelone platynota & ZMB 6096 & 222.0 & 22.20 &
	29.5 & 15.1 & 27.0 & NA & MA & n & Asia & freshly measured (MFN
	collection)\tabularnewline
	116 & Manouria & Manouria emys & - & 600.0 & 60.00 & NA & NA & NA & NA &
	NA & n & Asia & Itescu, Y., Karraker, N. E., Raia, P., Pritchard, P. C.,
	\& Meiri, S. (2014). Is the island rule general? Turtles disagree.
	Global Ecology and Biogeography, 23(6), 689-700.\tabularnewline
	117 & Indotestudo & Indotestudo forstenii & - & 202.0 & 20.20 & NA & NA
	& NA & NA & NA & y & Asia & Itescu, Y., Karraker, N. E., Raia, P.,
	Pritchard, P. C., \& Meiri, S. (2014). Is the island rule general?
	Turtles disagree. Global Ecology and Biogeography, 23(6),
	689-700.\tabularnewline
	118 & Indotestudo & Indotestudo travancorica & - & 249.7 & 24.97 & NA &
	NA & NA & NA & NA & n & Asia & Itescu, Y., Karraker, N. E., Raia, P.,
	Pritchard, P. C., \& Meiri, S. (2014). Is the island rule general?
	Turtles disagree. Global Ecology and Biogeography, 23(6),
	689-700.\tabularnewline
	119 & Indotestudo & Indotestudo forstenii & - & 309.0 & 30.90 & NA & NA
	& NA & NA & NA & y & Asia & Itescu, Y., Karraker, N. E., Raia, P.,
	Pritchard, P. C., \& Meiri, S. (2014). Is the island rule general?
	Turtles disagree. Global Ecology and Biogeography, 23(6),
	689-700.\tabularnewline
	120 & Indotestudo & Indotestudo elongata & - & 360.0 & 36.00 & NA & NA &
	NA & NA & NA & n & Asia & Itescu, Y., Karraker, N. E., Raia, P.,
	Pritchard, P. C., \& Meiri, S. (2014). Is the island rule general?
	Turtles disagree. Global Ecology and Biogeography, 23(6),
	689-700.\tabularnewline
	121 & Indotestudo & Indotestudo forstenii & - & 199.0 & 19.90 & NA & NA
	& NA & NA & NA & y & Asia & Itescu, Y., Karraker, N. E., Raia, P.,
	Pritchard, P. C., \& Meiri, S. (2014). Is the island rule general?
	Turtles disagree. Global Ecology and Biogeography, 23(6),
	689-700.\tabularnewline
	122 & Indotestudo & Indotestudo elongata & - & 244.2 & 24.42 & NA & NA &
	NA & NA & NA & n & Asia & Itescu, Y., Karraker, N. E., Raia, P.,
	Pritchard, P. C., \& Meiri, S. (2014). Is the island rule general?
	Turtles disagree. Global Ecology and Biogeography, 23(6),
	689-700.\tabularnewline
	123 & Indotestudo & Indotestudo travancorica & - & 244.2 & 24.42 & NA &
	NA & NA & NA & NA & n & Asia & Itescu, Y., Karraker, N. E., Raia, P.,
	Pritchard, P. C., \& Meiri, S. (2014). Is the island rule general?
	Turtles disagree. Global Ecology and Biogeography, 23(6),
	689-700.\tabularnewline
	124 & Manouria & Manouria impressa & ZMB 63172 & 165.0 & 16.50 & 20.0 &
	12.9 & 18.0 & 157.0 & 10.5 & n & Asia & freshly measured (MFN
	collection)\tabularnewline
	125 & Indotestudo & Indotestudo elongata & ZMB 50492 & 276.0 & 27.60 &
	33.0 & 19.4 & 28.5 & 246.0 & 17.1 & n & Asia & freshly measured (MFN
	collection)\tabularnewline
	126 & Indotestudo & Indotestudo elongata & ZMB 63175 & 235.0 & 23.50 &
	30.5 & 16.0 & 29.5 & 202.0 & 14.4 & n & Asia & freshly measured (MFN
	collection)\tabularnewline
	127 & Indotestudo & Indotestudo elongata & ZMB 4174 & 208.0 & 20.80 &
	26.0 & 13.4 & 20.0 & 180.0 & 11.6 & n & Asia & freshly measured (MFN
	collection)\tabularnewline
	128 & Indotestudo & Indotestudo elongata & ZMB 6106 & 166.0 & 16.60 &
	21.0 & 11.3 & 18.0 & 151.0 & 11.3 & n & Asia & freshly measured (MFN
	collection)\tabularnewline
	129 & Manouria & Manouria emys & - & 600.0 & 60.00 & NA & NA & NA & NA &
	NA & n & Asia & Karl, H., \& Staesche, U. (2007). Fossile
	Riesen-Landschildkroten von den Philippinen und ihre palaogeographische
	Bedeutung. Geologisches Jahrbuch Reihe B, 98, 171.\tabularnewline
	130 & Testudo & Testudo graeca & - & 250.0 & 25.00 & NA & NA & NA & NA &
	NA & n & Asia & Itescu, Y., Karraker, N. E., Raia, P., Pritchard, P. C.,
	\& Meiri, S. (2014). Is the island rule general? Turtles disagree.
	Global Ecology and Biogeography, 23(6), 689-700.\tabularnewline
	131 & Testudo & Testudo graeca & - & 280.0 & 28.00 & NA & NA & NA & NA &
	NA & y & Asia & Itescu, Y., Karraker, N. E., Raia, P., Pritchard, P. C.,
	\& Meiri, S. (2014). Is the island rule general? Turtles disagree.
	Global Ecology and Biogeography, 23(6), 689-700.\tabularnewline
	132 & Manouria & Manouria emys & ZMB 49049 & 212.0 & 21.20 & 26.5 & 16.5
	& 25.0 & NA & NA & n & Asia & freshly measured (MFN
	collection)\tabularnewline
	133 & Manouria & Manouria emys & ZMB 37350 & 445.0 & 44.50 & 52.0 & 32.0
	& 50.0 & 455.0 & 29.8 & n & Asia & freshly measured (MFN
	collection)\tabularnewline
	134 & Manouria & Manouria emys & ZMB 37342 & 330.0 & 33.00 & 40.5 & 26.7
	& 37.0 & 330.0 & 23.4 & n & Asia & freshly measured (MFN
	collection)\tabularnewline
	135 & Indotestudo & Indotestudo travancorica & - & 331.0 & 33.10 & NA &
	NA & NA & NA & NA & n & Asia & Itescu, Y., Karraker, N. E., Raia, P.,
	Pritchard, P. C., \& Meiri, S. (2014). Is the island rule general?
	Turtles disagree. Global Ecology and Biogeography, 23(6),
	689-700.\tabularnewline
	136 & Indotestudo & Indotestudo travancorica & - & 219.6 & 21.96 & NA &
	NA & NA & NA & NA & n & Asia & Itescu, Y., Karraker, N. E., Raia, P.,
	Pritchard, P. C., \& Meiri, S. (2014). Is the island rule general?
	Turtles disagree. Global Ecology and Biogeography, 23(6),
	689-700.\tabularnewline
	137 & Indotestudo & Indotestudo forstenii & - & 200.5 & 20.05 & NA & NA
	& NA & NA & NA & y & Asia & Itescu, Y., Karraker, N. E., Raia, P.,
	Pritchard, P. C., \& Meiri, S. (2014). Is the island rule general?
	Turtles disagree. Global Ecology and Biogeography, 23(6),
	689-700.\tabularnewline
	138 & Testudo & Testudo horsfieldii & - & 280.0 & 28.00 & NA & NA & NA &
	NA & NA & n & Asia & Itescu, Y., Karraker, N. E., Raia, P., Pritchard,
	P. C., \& Meiri, S. (2014). Is the island rule general? Turtles
	disagree. Global Ecology and Biogeography, 23(6),
	689-700.\tabularnewline
	139 & Manouria & Manouria impressa & - & 350.0 & 35.00 & NA & NA & NA &
	NA & NA & n & Asia & Itescu, Y., Karraker, N. E., Raia, P., Pritchard,
	P. C., \& Meiri, S. (2014). Is the island rule general? Turtles
	disagree. Global Ecology and Biogeography, 23(6),
	689-700.\tabularnewline
	140 & Geochelone & Geochelone elegans & - & 380.0 & 38.00 & NA & NA & NA
	& NA & NA & n & Asia & Itescu, Y., Karraker, N. E., Raia, P., Pritchard,
	P. C., \& Meiri, S. (2014). Is the island rule general? Turtles
	disagree. Global Ecology and Biogeography, 23(6),
	689-700.\tabularnewline
	141 & Manouria & Manouria impressa & - & 275.0 & 27.50 & NA & NA & NA &
	NA & NA & n & Asia & Karl, H., \& Staesche, U. (2007). Fossile
	Riesen-Landschildkroten von den Philippinen und ihre palaogeographische
	Bedeutung. Geologisches Jahrbuch Reihe B, 98, 171.\tabularnewline
	142 & Indotestudo & Indotestudo elongata & - & 219.6 & 21.96 & NA & NA &
	NA & NA & NA & n & Asia & Itescu, Y., Karraker, N. E., Raia, P.,
	Pritchard, P. C., \& Meiri, S. (2014). Is the island rule general?
	Turtles disagree. Global Ecology and Biogeography, 23(6),
	689-700.\tabularnewline
	143 & Geochelone & Geochelone platynota & - & 300.0 & 30.00 & NA & NA &
	NA & NA & NA & n & Asia & Itescu, Y., Karraker, N. E., Raia, P.,
	Pritchard, P. C., \& Meiri, S. (2014). Is the island rule general?
	Turtles disagree. Global Ecology and Biogeography, 23(6),
	689-700.\tabularnewline
	144 & Testudo & Testudo graeca & - & 300.0 & 30.00 & NA & NA & NA & NA &
	NA & n & Asia & Itescu, Y., Karraker, N. E., Raia, P., Pritchard, P. C.,
	\& Meiri, S. (2014). Is the island rule general? Turtles disagree.
	Global Ecology and Biogeography, 23(6), 689-700.\tabularnewline
	145 & Gopherus & Gopherus flavomarginatus & - & 400.0 & 40.00 & NA & NA
	& NA & NA & NA & n & America & Itescu, Y., Karraker, N. E., Raia, P.,
	Pritchard, P. C., \& Meiri, S. (2014). Is the island rule general?
	Turtles disagree. Global Ecology and Biogeography, 23(6),
	689-700.\tabularnewline
	146 & Gopherus & Gopherus morafkai & - & 299.0 & 29.90 & NA & NA & NA &
	NA & NA & n & America & Itescu, Y., Karraker, N. E., Raia, P.,
	Pritchard, P. C., \& Meiri, S. (2014). Is the island rule general?
	Turtles disagree. Global Ecology and Biogeography, 23(6),
	689-700.\tabularnewline
	147 & Gopherus & Gopherus berlandieri & - & 240.0 & 24.00 & NA & NA & NA
	& NA & NA & n & America & Itescu, Y., Karraker, N. E., Raia, P.,
	Pritchard, P. C., \& Meiri, S. (2014). Is the island rule general?
	Turtles disagree. Global Ecology and Biogeography, 23(6),
	689-700.\tabularnewline
	148 & Testudo & Testudo horsfieldii & ZMB 63259 & 111.0 & 11.10 & 14.0 &
	10.0 & 15.0 & 108.0 & 9.5 & n & Europe & freshly measured (MFN
	collection)\tabularnewline
	149 & Pyxis & Pyxis arachnoides & ZMB 37615 & 108.0 & 10.80 & 15.0 & 7.9
	& 13.0 & 96.0 & 7.1 & n & Europe & freshly measured (MFN
	collection)\tabularnewline
	150 & Testudo & Testudo marginata & - & 241.7 & 24.17 & NA & NA & NA &
	NA & NA & n & Europe & Willemsen, R. E., \& Hailey, A. (2003). Sexual
	dimorphism of body size and shell shape in European tortoises. Journal
	of Zoology, 260(4), 353-365.\tabularnewline
	151 & Testudo & Testudo horsfieldii & ZMB 63258 & 123.0 & 12.30 & 14.5 &
	10.9 & 15.0 & 121.0 & 9.8 & n & Europe & freshly measured (MFN
	collection)\tabularnewline
	152 & Testudo & Testudo hermanni & - & 183.3 & 18.33 & NA & NA & NA & NA
	& NA & y & Europe & Itescu, Y., Karraker, N. E., Raia, P., Pritchard, P.
	C., \& Meiri, S. (2014). Is the island rule general? Turtles disagree.
	Global Ecology and Biogeography, 23(6), 689-700.\tabularnewline
	153 & Testudo & Testudo hermanni & - & 176.9 & 17.69 & NA & NA & NA & NA
	& NA & n & Europe & Willemsen, R. E., \& Hailey, A. (2003). Sexual
	dimorphism of body size and shell shape in European tortoises. Journal
	of Zoology, 260(4), 353-365.\tabularnewline
	154 & Testudo & Testudo horsfieldii & ZMB 63257 & 114.0 & 11.40 & 14.5 &
	10.2 & 14.0 & 110.0 & 9.9 & n & Europe & freshly measured (MFN
	collection)\tabularnewline
	155 & Testudo & Testudo marginata & - & 246.7 & 24.67 & NA & NA & NA &
	NA & NA & n & Europe & Willemsen, R. E., \& Hailey, A. (2003). Sexual
	dimorphism of body size and shell shape in European tortoises. Journal
	of Zoology, 260(4), 353-365.\tabularnewline
	156 & Testudo & Testudo hermanni & - & 196.0 & 19.60 & NA & NA & NA & NA
	& NA & n & Europe & Willemsen, R. E., \& Hailey, A. (2003). Sexual
	dimorphism of body size and shell shape in European tortoises. Journal
	of Zoology, 260(4), 353-365.\tabularnewline
	157 & Testudo & Testudo hermanni & - & 143.5 & 14.35 & NA & NA & NA & NA
	& NA & y & Europe & Itescu, Y., Karraker, N. E., Raia, P., Pritchard, P.
	C., \& Meiri, S. (2014). Is the island rule general? Turtles disagree.
	Global Ecology and Biogeography, 23(6), 689-700.\tabularnewline
	158 & Testudo & Testudo graeca & - & 194.6 & 19.46 & NA & NA & NA & NA &
	NA & n & Europe & Willemsen, R. E., \& Hailey, A. (2003). Sexual
	dimorphism of body size and shell shape in European tortoises. Journal
	of Zoology, 260(4), 353-365.\tabularnewline
	159 & Testudo & Testudo hermanni & - & 200.0 & 20.00 & NA & NA & NA & NA
	& NA & y & Europe & Itescu, Y., Karraker, N. E., Raia, P., Pritchard, P.
	C., \& Meiri, S. (2014). Is the island rule general? Turtles disagree.
	Global Ecology and Biogeography, 23(6), 689-700.\tabularnewline
	160 & Testudo & Testudo hermanni & - & 250.0 & 25.00 & NA & NA & NA & NA
	& NA & n & Europe & Itescu, Y., Karraker, N. E., Raia, P., Pritchard, P.
	C., \& Meiri, S. (2014). Is the island rule general? Turtles disagree.
	Global Ecology and Biogeography, 23(6), 689-700.\tabularnewline
	161 & Testudo & Testudo marginata & - & 246.0 & 24.60 & NA & NA & NA &
	NA & NA & n & Europe & Itescu, Y., Karraker, N. E., Raia, P., Pritchard,
	P. C., \& Meiri, S. (2014). Is the island rule general? Turtles
	disagree. Global Ecology and Biogeography, 23(6),
	689-700.\tabularnewline
	162 & Testudo & Testudo marginata & - & 242.5 & 24.25 & NA & NA & NA &
	NA & NA & y & Europe & Itescu, Y., Karraker, N. E., Raia, P., Pritchard,
	P. C., \& Meiri, S. (2014). Is the island rule general? Turtles
	disagree. Global Ecology and Biogeography, 23(6),
	689-700.\tabularnewline
	163 & Testudo & Testudo marginata & - & 246.0 & 24.60 & NA & NA & NA &
	NA & NA & n & Europe & Itescu, Y., Karraker, N. E., Raia, P., Pritchard,
	P. C., \& Meiri, S. (2014). Is the island rule general? Turtles
	disagree. Global Ecology and Biogeography, 23(6),
	689-700.\tabularnewline
	164 & Testudo & Testudo hermanni & - & 147.0 & 14.70 & NA & NA & NA & NA
	& NA & n & Europe & Itescu, Y., Karraker, N. E., Raia, P., Pritchard, P.
	C., \& Meiri, S. (2014). Is the island rule general? Turtles disagree.
	Global Ecology and Biogeography, 23(6), 689-700.\tabularnewline
	165 & Testudo & Testudo marginata & - & 290.0 & 29.00 & NA & NA & NA &
	NA & NA & n & Europe & Itescu, Y., Karraker, N. E., Raia, P., Pritchard,
	P. C., \& Meiri, S. (2014). Is the island rule general? Turtles
	disagree. Global Ecology and Biogeography, 23(6),
	689-700.\tabularnewline
	166 & Testudo & Testudo marginata & - & 250.0 & 25.00 & NA & NA & NA &
	NA & NA & y & Europe & Itescu, Y., Karraker, N. E., Raia, P., Pritchard,
	P. C., \& Meiri, S. (2014). Is the island rule general? Turtles
	disagree. Global Ecology and Biogeography, 23(6),
	689-700.\tabularnewline
	167 & Testudo & Testudo hermanni & - & 145.9 & 14.59 & NA & NA & NA & NA
	& NA & y & Europe & Itescu, Y., Karraker, N. E., Raia, P., Pritchard, P.
	C., \& Meiri, S. (2014). Is the island rule general? Turtles disagree.
	Global Ecology and Biogeography, 23(6), 689-700.\tabularnewline
	168 & Testudo & Testudo graeca & - & 178.2 & 17.82 & NA & NA & NA & NA &
	NA & n & Europe & Willemsen, R. E., \& Hailey, A. (2003). Sexual
	dimorphism of body size and shell shape in European tortoises. Journal
	of Zoology, 260(4), 353-365.\tabularnewline
	169 & Testudo & Testudo marginata & - & 400.0 & 40.00 & NA & NA & NA &
	NA & NA & n & Europe & Itescu, Y., Karraker, N. E., Raia, P., Pritchard,
	P. C., \& Meiri, S. (2014). Is the island rule general? Turtles
	disagree. Global Ecology and Biogeography, 23(6),
	689-700.\tabularnewline
	170 & Testudo & Testudo horsfieldii & ZMB 63255 & 136.0 & 13.60 & 18.0 &
	13.0 & 16.5 & 129.0 & 12.2 & n & Europe & freshly measured (MFN
	collection)\tabularnewline
	171 & Testudo & Testudo horsfieldii & ZMB 63256 & 132.0 & 13.20 & 17.0 &
	12.4 & 17.0 & 133.0 & 11.3 & n & Europe & freshly measured (MFN
	collection)\tabularnewline
	172 & Testudo & Testudo hermanni & - & 168.3 & 16.83 & NA & NA & NA & NA
	& NA & y & Europe & Itescu, Y., Karraker, N. E., Raia, P., Pritchard, P.
	C., \& Meiri, S. (2014). Is the island rule general? Turtles disagree.
	Global Ecology and Biogeography, 23(6), 689-700.\tabularnewline
	173 & Testudo & Testudo hermanni & - & 160.0 & 16.00 & NA & NA & NA & NA
	& NA & y & Europe & Itescu, Y., Karraker, N. E., Raia, P., Pritchard, P.
	C., \& Meiri, S. (2014). Is the island rule general? Turtles disagree.
	Global Ecology and Biogeography, 23(6), 689-700.\tabularnewline
	174 & Testudo & Testudo hermanni & - & 154.0 & 15.40 & NA & NA & NA & NA
	& NA & n & Europe & Itescu, Y., Karraker, N. E., Raia, P., Pritchard, P.
	C., \& Meiri, S. (2014). Is the island rule general? Turtles disagree.
	Global Ecology and Biogeography, 23(6), 689-700.\tabularnewline
	175 & Testudo & Testudo hermanni & - & 138.5 & 13.85 & NA & NA & NA & NA
	& NA & n & Europe & Itescu, Y., Karraker, N. E., Raia, P., Pritchard, P.
	C., \& Meiri, S. (2014). Is the island rule general? Turtles disagree.
	Global Ecology and Biogeography, 23(6), 689-700.\tabularnewline
	176 & Testudo & Testudo hermanni & - & 173.0 & 17.30 & NA & NA & NA & NA
	& NA & y & Europe & Itescu, Y., Karraker, N. E., Raia, P., Pritchard, P.
	C., \& Meiri, S. (2014). Is the island rule general? Turtles disagree.
	Global Ecology and Biogeography, 23(6), 689-700.\tabularnewline
	177 & Testudo & Testudo marginata & - & 242.5 & 24.25 & NA & NA & NA &
	NA & NA & y & Europe & Itescu, Y., Karraker, N. E., Raia, P., Pritchard,
	P. C., \& Meiri, S. (2014). Is the island rule general? Turtles
	disagree. Global Ecology and Biogeography, 23(6),
	689-700.\tabularnewline
	178 & Testudo & Testudo hermanni & - & 195.0 & 19.50 & NA & NA & NA & NA
	& NA & y & Europe & Itescu, Y., Karraker, N. E., Raia, P., Pritchard, P.
	C., \& Meiri, S. (2014). Is the island rule general? Turtles disagree.
	Global Ecology and Biogeography, 23(6), 689-700.\tabularnewline
	179 & Testudo & Testudo hermanni & - & 157.0 & 15.70 & NA & NA & NA & NA
	& NA & y & Europe & Itescu, Y., Karraker, N. E., Raia, P., Pritchard, P.
	C., \& Meiri, S. (2014). Is the island rule general? Turtles disagree.
	Global Ecology and Biogeography, 23(6), 689-700.\tabularnewline
	180 & Testudo & Testudo hermanni & - & 176.6 & 17.66 & NA & NA & NA & NA
	& NA & y & Europe & Itescu, Y., Karraker, N. E., Raia, P., Pritchard, P.
	C., \& Meiri, S. (2014). Is the island rule general? Turtles disagree.
	Global Ecology and Biogeography, 23(6), 689-700.\tabularnewline
	181 & Testudo & Testudo hermanni & - & 130.0 & 13.00 & NA & NA & NA & NA
	& NA & n & Europe & Itescu, Y., Karraker, N. E., Raia, P., Pritchard, P.
	C., \& Meiri, S. (2014). Is the island rule general? Turtles disagree.
	Global Ecology and Biogeography, 23(6), 689-700.\tabularnewline
	182 & Testudo & Testudo hermanni & - & 161.0 & 16.10 & NA & NA & NA & NA
	& NA & n & Europe & Itescu, Y., Karraker, N. E., Raia, P., Pritchard, P.
	C., \& Meiri, S. (2014). Is the island rule general? Turtles disagree.
	Global Ecology and Biogeography, 23(6), 689-700.\tabularnewline
	183 & Gopherus & Gopherus polyphemus & - & 300.0 & 30.00 & NA & NA & NA
	& NA & NA & y & America & Itescu, Y., Karraker, N. E., Raia, P.,
	Pritchard, P. C., \& Meiri, S. (2014). Is the island rule general?
	Turtles disagree. Global Ecology and Biogeography, 23(6),
	689-700.\tabularnewline
	184 & Gopherus & Gopherus sp. & MVZ 210020 & NA & NA & NA & NA & NA &
	219.6 & NA & n & America & Biewer J., Sankey J., Hutchison H., Garber
	D., 2016: A fossil giant tortoise from the Mehrten Formation of Northern
	California. PaleoBios 33: 1-13 or Brattstrom B.H., 1961: Some new fossil
	tortoises from western North America with remarks on the zoogeography
	and paleoecology of tortoises. Journal of Paleontology 35(3):
	543-560\tabularnewline
	185 & Gopherus & Gopherus sp. & MVZ 210003 & NA & NA & NA & NA & NA &
	192.1 & NA & n & America & Biewer J., Sankey J., Hutchison H., Garber
	D., 2016: A fossil giant tortoise from the Mehrten Formation of Northern
	California. PaleoBios 33: 1-13 or Brattstrom B.H., 1961: Some new fossil
	tortoises from western North America with remarks on the zoogeography
	and paleoecology of tortoises. Journal of Paleontology 35(3):
	543-560\tabularnewline
	186 & Gopherus & Gopherus polyphemus & - & 268.8 & 26.88 & NA & NA & NA
	& NA & NA & y & America & Itescu, Y., Karraker, N. E., Raia, P.,
	Pritchard, P. C., \& Meiri, S. (2014). Is the island rule general?
	Turtles disagree. Global Ecology and Biogeography, 23(6),
	689-700.\tabularnewline
	187 & Gopherus & Gopherus sp. & MVZ 120004 & NA & NA & NA & NA & NA &
	196.7 & NA & n & America & Biewer J., Sankey J., Hutchison H., Garber
	D., 2016: A fossil giant tortoise from the Mehrten Formation of Northern
	California. PaleoBios 33: 1-13 or Brattstrom B.H., 1961: Some new fossil
	tortoises from western North America with remarks on the zoogeography
	and paleoecology of tortoises. Journal of Paleontology 35(3):
	543-560\tabularnewline
	188 & Gopherus & Gopherus sp. & MVZ 210009 & NA & NA & NA & NA & NA &
	232.8 & NA & n & America & Biewer J., Sankey J., Hutchison H., Garber
	D., 2016: A fossil giant tortoise from the Mehrten Formation of Northern
	California. PaleoBios 33: 1-13 or Brattstrom B.H., 1961: Some new fossil
	tortoises from western North America with remarks on the zoogeography
	and paleoecology of tortoises. Journal of Paleontology 35(3):
	543-560\tabularnewline
	189 & Gopherus & Gopherus sp. & MVZ 210010 & NA & NA & NA & NA & NA &
	240.1 & NA & n & America & Biewer J., Sankey J., Hutchison H., Garber
	D., 2016: A fossil giant tortoise from the Mehrten Formation of Northern
	California. PaleoBios 33: 1-13 or Brattstrom B.H., 1961: Some new fossil
	tortoises from western North America with remarks on the zoogeography
	and paleoecology of tortoises. Journal of Paleontology 35(3):
	543-560\tabularnewline
	190 & Gopherus & Gopherus agassizii & - & 400.0 & 40.00 & NA & NA & NA &
	NA & NA & n & America & Itescu, Y., Karraker, N. E., Raia, P.,
	Pritchard, P. C., \& Meiri, S. (2014). Is the island rule general?
	Turtles disagree. Global Ecology and Biogeography, 23(6),
	689-700.\tabularnewline
	191 & Gopherus & Gopherus flavomarginatus & KU 39415 & 303.0 & 30.30 &
	NA & 23.2 & NA & NA & NA & n & America & Legler, 1959\tabularnewline
	192 & Gopherus & Gopherus polyphemus & - & 308.0 & 30.80 & NA & NA & NA
	& NA & NA & n & America & Itescu, Y., Karraker, N. E., Raia, P.,
	Pritchard, P. C., \& Meiri, S. (2014). Is the island rule general?
	Turtles disagree. Global Ecology and Biogeography, 23(6),
	689-700.\tabularnewline
	193 & Gopherus & Gopherus polyphemus & - & 303.0 & 30.30 & NA & NA & NA
	& NA & NA & y & America & Itescu, Y., Karraker, N. E., Raia, P.,
	Pritchard, P. C., \& Meiri, S. (2014). Is the island rule general?
	Turtles disagree. Global Ecology and Biogeography, 23(6),
	689-700.\tabularnewline
	194 & Gopherus & Gopherus polyphemus & - & 387.0 & 38.70 & NA & NA & NA
	& NA & NA & n & America & Itescu, Y., Karraker, N. E., Raia, P.,
	Pritchard, P. C., \& Meiri, S. (2014). Is the island rule general?
	Turtles disagree. Global Ecology and Biogeography, 23(6),
	689-700.\tabularnewline
	195 & Gopherus & Gopherus polyphemus & - & 342.0 & 34.20 & NA & NA & NA
	& NA & NA & n & America & Itescu, Y., Karraker, N. E., Raia, P.,
	Pritchard, P. C., \& Meiri, S. (2014). Is the island rule general?
	Turtles disagree. Global Ecology and Biogeography, 23(6),
	689-700.\tabularnewline
	196 & Gopherus & Gopherus flavomarginatus & USNM 61253 & 222.0 & 22.20 &
	NA & 16.6 & NA & 212.0 & NA & n & America & Legler, 1959\tabularnewline
	197 & Gopherus & Gopherus flavomarginatus & USNM 61254 & 371.0 & 37.10 &
	NA & 29.2 & NA & 358.0 & NA & n & America & Legler, 1959\tabularnewline
	198 & Gopherus & Gopherus polyphemus & - & 238.9 & 23.89 & NA & NA & NA
	& NA & NA & n & America & Itescu, Y., Karraker, N. E., Raia, P.,
	Pritchard, P. C., \& Meiri, S. (2014). Is the island rule general?
	Turtles disagree. Global Ecology and Biogeography, 23(6),
	689-700.\tabularnewline
	199 & Gopherus & Gopherus flavomarginatus & USNM 60976 & 246.0 & 24.60 &
	NA & 21.2 & NA & 252.0 & NA & n & America & Legler, 1959\tabularnewline
	200 & Gopherus & Gopherus flavomarginatus & IU 42953 & 281.0 & 28.10 &
	NA & 22.0 & NA & NA & NA & n & America & Legler, 1959\tabularnewline
	201 & Gopherus & Gopherus flavomarginatus & IU 42954 & 278.0 & 27.80 &
	NA & 21.4 & NA & NA & NA & n & America & Legler, 1959\tabularnewline
	202 & Chelonoidis & Chelonoidis nigra & USNM 51069 & 588.0 & 58.80 &
	68.3 & 44.5 & NA & 506.0 & NA & y & America & Franz, R., \& Franz, S. E.
	(2009). A new fossil land tortoise in the genus Chelonoidis (Testudines:
	Testudinidae) from the Northern Bahamas: with an osteological assessment
	of other neotropical tortoises. University of Florida.\tabularnewline
	203 & Chelonoidis & Chelonoidis nigra & USNM1 102904 & 610.0 & 61.00 &
	67.5 & 44.4 & NA & 515.0 & NA & y & America & Franz, R., \& Franz, S. E.
	(2009). A new fossil land tortoise in the genus Chelonoidis (Testudines:
	Testudinidae) from the Northern Bahamas: with an osteological assessment
	of other neotropical tortoises. University of Florida.\tabularnewline
	204 & Chelonoidis & Chelonoidis carbonaria & - & 593.0 & 59.30 & NA & NA
	& NA & NA & NA & n & America & Itescu, Y., Karraker, N. E., Raia, P.,
	Pritchard, P. C., \& Meiri, S. (2014). Is the island rule general?
	Turtles disagree. Global Ecology and Biogeography, 23(6),
	689-700.\tabularnewline
	205 & Chelonoidis & Chelonoidis abingdonii & - & 980.0 & 98.00 & NA & NA
	& NA & NA & NA & y & America & Itescu, Y., Karraker, N. E., Raia, P.,
	Pritchard, P. C., \& Meiri, S. (2014). Is the island rule general?
	Turtles disagree. Global Ecology and Biogeography, 23(6),
	689-700.\tabularnewline
	206 & Chelonoidis & Chelonoidis denticulata & - & 333.4 & 33.34 & NA &
	NA & NA & NA & NA & n & America & Itescu, Y., Karraker, N. E., Raia, P.,
	Pritchard, P. C., \& Meiri, S. (2014). Is the island rule general?
	Turtles disagree. Global Ecology and Biogeography, 23(6),
	689-700.\tabularnewline
	207 & Chelonoidis & Chelonoidis chilensis & UF33604 & 169.0 & 16.90 &
	21.5 & 13.2 & NA & 161.0 & NA & n & America & Franz, R., \& Franz, S. E.
	(2009). A new fossil land tortoise in the genus Chelonoidis (Testudines:
	Testudinidae) from the Northern Bahamas: with an osteological assessment
	of other neotropical tortoises. University of Florida.\tabularnewline
	208 & Chelonoidis & Chelonoidis chilensis & UF33618 & 186.0 & 18.60 &
	25.0 & 14.7 & NA & 169.0 & NA & n & America & Franz, R., \& Franz, S. E.
	(2009). A new fossil land tortoise in the genus Chelonoidis (Testudines:
	Testudinidae) from the Northern Bahamas: with an osteological assessment
	of other neotropical tortoises. University of Florida.\tabularnewline
	209 & Chelonoidis & Chelonoidis nigra & - & 717.0 & 71.70 & NA & NA & NA
	& NA & NA & y & America & Itescu, Y., Karraker, N. E., Raia, P.,
	Pritchard, P. C., \& Meiri, S. (2014). Is the island rule general?
	Turtles disagree. Global Ecology and Biogeography, 23(6),
	689-700.\tabularnewline
	210 & Chelonoidis & Chelonoidis chilensis & UF33617 & 169.0 & 16.90 &
	22.8 & 14.6 & NA & 162.0 & NA & n & America & Franz, R., \& Franz, S. E.
	(2009). A new fossil land tortoise in the genus Chelonoidis (Testudines:
	Testudinidae) from the Northern Bahamas: with an osteological assessment
	of other neotropical tortoises. University of Florida.\tabularnewline
	211 & Chelonoidis & Chelonoidis carbonaria & UF27384 & 242.0 & 24.20 &
	31.7 & 15.5 & NA & 219.0 & NA & n & America & Franz, R., \& Franz, S. E.
	(2009). A new fossil land tortoise in the genus Chelonoidis (Testudines:
	Testudinidae) from the Northern Bahamas: with an osteological assessment
	of other neotropical tortoises. University of Florida.\tabularnewline
	212 & Chelonoidis & Chelonoidis carbonaria & UF33597 & 253.0 & 25.30 &
	31.7 & 15.3 & NA & 215.0 & NA & n & America & Franz, R., \& Franz, S. E.
	(2009). A new fossil land tortoise in the genus Chelonoidis (Testudines:
	Testudinidae) from the Northern Bahamas: with an osteological assessment
	of other neotropical tortoises. University of Florida.\tabularnewline
	213 & Chelonoidis & Chelonoidis nigra & USNM1 222494 & 595.0 & 59.50 &
	68.0 & 43.6 & NA & 533.0 & NA & y & America & Franz, R., \& Franz, S. E.
	(2009). A new fossil land tortoise in the genus Chelonoidis (Testudines:
	Testudinidae) from the Northern Bahamas: with an osteological assessment
	of other neotropical tortoises. University of Florida.\tabularnewline
	214 & Chelonoidis & Chelonoidis carbonaria & - & 333.4 & 33.34 & NA & NA
	& NA & NA & NA & n & America & Itescu, Y., Karraker, N. E., Raia, P.,
	Pritchard, P. C., \& Meiri, S. (2014). Is the island rule general?
	Turtles disagree. Global Ecology and Biogeography, 23(6),
	689-700.\tabularnewline
	215 & Chelonoidis & Chelonoidis carbonaria & UF5259 & 226.0 & 22.60 &
	28.7 & 12.9 & NA & 198.0 & NA & n & America & Franz, R., \& Franz, S. E.
	(2009). A new fossil land tortoise in the genus Chelonoidis (Testudines:
	Testudinidae) from the Northern Bahamas: with an osteological assessment
	of other neotropical tortoises. University of Florida.\tabularnewline
	216 & Chelonoidis & Chelonoidis becki & - & 1050.0 & 105.00 & NA & NA &
	NA & NA & NA & y & America & Itescu, Y., Karraker, N. E., Raia, P.,
	Pritchard, P. C., \& Meiri, S. (2014). Is the island rule general?
	Turtles disagree. Global Ecology and Biogeography, 23(6),
	689-700.\tabularnewline
	217 & Chelonoidis & Chelonoidis denticulata & UF33661 & 333.0 & 33.30 &
	38.0 & 21.4 & NA & 305.0 & NA & n & America & Franz, R., \& Franz, S. E.
	(2009). A new fossil land tortoise in the genus Chelonoidis (Testudines:
	Testudinidae) from the Northern Bahamas: with an osteological assessment
	of other neotropical tortoises. University of Florida.\tabularnewline
	218 & Chelonoidis & Chelonoidis denticulata & UF61931 & 317.0 & 31.70 &
	41.2 & 18.5 & NA & 291.0 & NA & n & America & Franz, R., \& Franz, S. E.
	(2009). A new fossil land tortoise in the genus Chelonoidis (Testudines:
	Testudinidae) from the Northern Bahamas: with an osteological assessment
	of other neotropical tortoises. University of Florida.\tabularnewline
	219 & Chelonoidis & Chelonoidis denticulata & UF33670 & 365.0 & 36.50 &
	47.0 & 22.0 & NA & 326.0 & NA & n & America & Franz, R., \& Franz, S. E.
	(2009). A new fossil land tortoise in the genus Chelonoidis (Testudines:
	Testudinidae) from the Northern Bahamas: with an osteological assessment
	of other neotropical tortoises. University of Florida.\tabularnewline
	220 & Chelonoidis & Chelonoidis chilensis & UF33603 & 183.0 & 18.30 &
	23.4 & 14.5 & NA & 166.0 & NA & n & America & Franz, R., \& Franz, S. E.
	(2009). A new fossil land tortoise in the genus Chelonoidis (Testudines:
	Testudinidae) from the Northern Bahamas: with an osteological assessment
	of other neotropical tortoises. University of Florida.\tabularnewline
	221 & Chelonoidis & Chelonoidis nigra & - & 731.3 & 73.13 & NA & NA & NA
	& NA & NA & y & America & Itescu, Y., Karraker, N. E., Raia, P.,
	Pritchard, P. C., \& Meiri, S. (2014). Is the island rule general?
	Turtles disagree. Global Ecology and Biogeography, 23(6),
	689-700.\tabularnewline
	222 & Chelonoidis & Chelonoidis chilensis & - & 200.0 & 20.00 & NA & NA
	& NA & NA & NA & n & America & Itescu, Y., Karraker, N. E., Raia, P.,
	Pritchard, P. C., \& Meiri, S. (2014). Is the island rule general?
	Turtles disagree. Global Ecology and Biogeography, 23(6),
	689-700.\tabularnewline
	223 & Chelonoidis & Chelonoidis carbonaria & UF48278 & 247.0 & 24.70 &
	33.9 & 15.5 & NA & 214.0 & NA & n & America & Franz, R., \& Franz, S. E.
	(2009). A new fossil land tortoise in the genus Chelonoidis (Testudines:
	Testudinidae) from the Northern Bahamas: with an osteological assessment
	of other neotropical tortoises. University of Florida.\tabularnewline
	224 & Chelonoidis & Chelonoidis carbonaria & - & 296.5 & 29.65 & NA & NA
	& NA & NA & NA & n & America & Itescu, Y., Karraker, N. E., Raia, P.,
	Pritchard, P. C., \& Meiri, S. (2014). Is the island rule general?
	Turtles disagree. Global Ecology and Biogeography, 23(6),
	689-700.\tabularnewline
	225 & Chelonoidis & Chelonoidis carbonaria & - & 290.0 & 29.00 & NA & NA
	& NA & NA & NA & y & America & Itescu, Y., Karraker, N. E., Raia, P.,
	Pritchard, P. C., \& Meiri, S. (2014). Is the island rule general?
	Turtles disagree. Global Ecology and Biogeography, 23(6),
	689-700.\tabularnewline
	226 & Chelonoidis & Chelonoidis carbonaria & UF33596 & 189.0 & 18.90 &
	24.7 & 12.1 & NA & 174.0 & NA & n & America & Franz, R., \& Franz, S. E.
	(2009). A new fossil land tortoise in the genus Chelonoidis (Testudines:
	Testudinidae) from the Northern Bahamas: with an osteological assessment
	of other neotropical tortoises. University of Florida.\tabularnewline
	227 & Chelonoidis & Chelonoidis nigra & - & 745.7 & 74.57 & NA & NA & NA
	& NA & NA & y & America & Itescu, Y., Karraker, N. E., Raia, P.,
	Pritchard, P. C., \& Meiri, S. (2014). Is the island rule general?
	Turtles disagree. Global Ecology and Biogeography, 23(6),
	689-700.\tabularnewline
	228 & Chelonoidis & Chelonoidis chathamensis & - & 890.0 & 89.00 & NA &
	NA & NA & NA & NA & y & America & Itescu, Y., Karraker, N. E., Raia, P.,
	Pritchard, P. C., \& Meiri, S. (2014). Is the island rule general?
	Turtles disagree. Global Ecology and Biogeography, 23(6),
	689-700.\tabularnewline
	229 & Chelonoidis & Chelonoidis denticulata & UF19242 & 466.0 & 46.60 &
	59.7 & 26.5 & NA & 410.0 & NA & n & America & Franz, R., \& Franz, S. E.
	(2009). A new fossil land tortoise in the genus Chelonoidis (Testudines:
	Testudinidae) from the Northern Bahamas: with an osteological assessment
	of other neotropical tortoises. University of Florida.\tabularnewline
	230 & Chelonoidis & Chelonoidis denticulata & UF23231 & 377.0 & 37.70 &
	47.1 & 23.8 & NA & 334.0 & NA & n & America & Franz, R., \& Franz, S. E.
	(2009). A new fossil land tortoise in the genus Chelonoidis (Testudines:
	Testudinidae) from the Northern Bahamas: with an osteological assessment
	of other neotropical tortoises. University of Florida.\tabularnewline
	231 & Chelonoidis & Chelonoidis denticulata & - & 820.0 & 82.00 & NA &
	NA & NA & NA & NA & n & America & Itescu, Y., Karraker, N. E., Raia, P.,
	Pritchard, P. C., \& Meiri, S. (2014). Is the island rule general?
	Turtles disagree. Global Ecology and Biogeography, 23(6),
	689-700.\tabularnewline
	232 & Chelonoidis & Chelonoidis duncanensis & - & 840.0 & 84.00 & NA &
	NA & NA & NA & NA & y & America & Itescu, Y., Karraker, N. E., Raia, P.,
	Pritchard, P. C., \& Meiri, S. (2014). Is the island rule general?
	Turtles disagree. Global Ecology and Biogeography, 23(6),
	689-700.\tabularnewline
	233 & Chelonoidis & Chelonoidis chilensis & - & 222.0 & 22.20 & NA & NA
	& NA & NA & NA & n & America & Itescu, Y., Karraker, N. E., Raia, P.,
	Pritchard, P. C., \& Meiri, S. (2014). Is the island rule general?
	Turtles disagree. Global Ecology and Biogeography, 23(6),
	689-700.\tabularnewline
	234 & Chelonoidis & Chelonoidis chilensis & UF33600 & 157.0 & 15.70 &
	20.8 & 11.9 & NA & 145.0 & NA & n & America & Franz, R., \& Franz, S. E.
	(2009). A new fossil land tortoise in the genus Chelonoidis (Testudines:
	Testudinidae) from the Northern Bahamas: with an osteological assessment
	of other neotropical tortoises. University of Florida.\tabularnewline
	235 & Chelonoidis & Chelonoidis phantastica & - & 860.0 & 86.00 & NA &
	NA & NA & NA & NA & y & America & Itescu, Y., Karraker, N. E., Raia, P.,
	Pritchard, P. C., \& Meiri, S. (2014). Is the island rule general?
	Turtles disagree. Global Ecology and Biogeography, 23(6),
	689-700.\tabularnewline
	236 & Chelonoidis & Chelonoidis vicina & - & 1250.0 & 125.00 & NA & NA &
	NA & NA & NA & y & America & Itescu, Y., Karraker, N. E., Raia, P.,
	Pritchard, P. C., \& Meiri, S. (2014). Is the island rule general?
	Turtles disagree. Global Ecology and Biogeography, 23(6),
	689-700.\tabularnewline
	237 & Chelonoidis & Chelonoidis hoodensis & - & 813.0 & 81.30 & NA & NA
	& NA & NA & NA & y & America & Itescu, Y., Karraker, N. E., Raia, P.,
	Pritchard, P. C., \& Meiri, S. (2014). Is the island rule general?
	Turtles disagree. Global Ecology and Biogeography, 23(6),
	689-700.\tabularnewline
	238 & Chelonoidis & Chelonoidis nigra & - & 1300.0 & 130.00 & NA & NA &
	NA & NA & NA & y & America & Itescu, Y., Karraker, N. E., Raia, P.,
	Pritchard, P. C., \& Meiri, S. (2014). Is the island rule general?
	Turtles disagree. Global Ecology and Biogeography, 23(6),
	689-700.\tabularnewline
	239 & Chelonoidis & Chelonoidis darwini & - & 965.0 & 96.50 & NA & NA &
	NA & NA & NA & y & America & Itescu, Y., Karraker, N. E., Raia, P.,
	Pritchard, P. C., \& Meiri, S. (2014). Is the island rule general?
	Turtles disagree. Global Ecology and Biogeography, 23(6),
	689-700.\tabularnewline
	240 & Chelonoidis & Chelonoidis chilensis & - & 450.0 & 45.00 & NA & NA
	& NA & NA & NA & n & America & Itescu, Y., Karraker, N. E., Raia, P.,
	Pritchard, P. C., \& Meiri, S. (2014). Is the island rule general?
	Turtles disagree. Global Ecology and Biogeography, 23(6),
	689-700.\tabularnewline
	\bottomrule
\end{longtable}
}

	
\end{landscape}
\end{appendices}
\newpage
\section*{Acknowledgements}
\begin{flushleft}

I want to thank everyone who supported me during my master thesis.
\vspace{1cm}

First of all, I am very grateful to Prof. Johannes Müller for making sure I could work on a topic that I am passionate about. Thank you for offering just the right amount of guidance whenever I needed it while allowing me to learn how to work independently at the same time.
\vspace{0.75cm}
Further, I would like to thank PD Dr. Mark-Oliver Rödel for agreeing to evaluate this thesis and helpful literature suggestions.
\vspace{0.75cm}
I wish to acknowledge the help I received from Dr. Catalina Pimiento, who kindly offered help with the analyses, readily shared her R-Scripts with me and was always open for questions. Thank you very much for your support!
\vspace{0.75cm}
I am thankful to Frank Tillack for access to and help with handling and measuring collection material.
\vspace{0.75cm}

I would also like to thank Irena Rostalski and Roberto Rozzi for assistance with foreign literature.

\vspace{0.75cm}
Special thanks goes to Sonja Rothkugel, who was always helpful with R and advice on statistics and method in general. Further, thank you for reading parts of this thesis in advance and giving constructive feedback.
\vspace{0.75cm}
I also want to thank the following for critically reading a first draft of this thesis: Phillip Pinder, Viviane Kremling, Aaron Czycholl and Falk Mielke. Your help is very much appreciated!
\vspace{0.75cm}
Reading first draft: Phillip, Aaron, Falk, Vivi, Maria
\vspace{0.75cm}
Lastly, I want to thank my parents and family for their unwavering support during my studies!


\end{flushleft}{
\section*{Eigenständigkeitserklärung}
\vspace{2cm}
Hiermit versichere ich, dass ich die vorliegende Masterarbeit erstmalig einreiche, selbständig verfasst und keine anderen als die angegebenen Quellen und Hilfsmittel verwendet habe.
\vspace{4cm}
\begin{flushright}
	\_\_\_\_\_\_\_\_\_\_\_\_\_\_\_\_\_\_\_\_\_\_\_\_\_\_\_\_\_
\end{flushright}
Berlin, den 15. September 2017





\end{document}